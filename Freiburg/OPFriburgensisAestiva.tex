\documentclass[fontsize=9pt,paper=A6,twoside,BCOR=1mm,DIV=22,headinclude]{scrarticle}
\usepackage{breviarium}
\renewcommand\A{\Ant}
\begin{document}
\titulum{Officia Propria}{Archidiœcesis Friburgensis}{Pars Æstiva}
\begin{multicols}{2}
\mensii{Festa Junii}
\dieii{Die 5 Junii}{S. Bonifatii}{Episcopi et Martyris}{Duplex II classis}

{\setstretch{0.96}
\rubric{Omnia ut in Communi unius Martyris, præter ea, quæ in Breviario hac die habentur propria.}

\rubric{In I Vesperis fit Commemoratio præcedentis.}

\rubric{In I Nocturno Lectiones \black{A Miléto.}}

\rubric{In II Vesperis fit Commemoratio sequentis.}


\mens{Festa Julii}

}


\dieii{Die 4 Julii}{S. Udalrici}{Episcopi et Confessoris}{Duplex}

{\setstretch{0.96}
\V Amávit eum Dóminus, et ornávit eum.
\R Stolam glóriæ índuit eum.

\red{Ad Magnif. Ant.} Sacérdos et Póntifex, * et virtútum ópifex, pastor bone in pópulo, ora pro nobis Dóminum.

\pars{Oratio}
\y{D}{eus}, qui cónspicis, quia ex nulla nostra virtúte subsístimus: concéde propítius; ut, intercessióne beáti Udalríci Confessóris tui atque Pontíficis, contra ómnia advérsa muniámur. Per Dóminum.

}

\rubric{Et fit Commemoratio præcedentis:}

\A Dum esset Summus Póntifex, terréna non métuit, sed ad cæléstia regna gloriósus migrávit.

\V Justum dedúxit Dóminus per vias rectas.
\R Et osténdit illi regnum Dei.

\pars{Oratio}
\yb{D}{eus}, qui beátum Leónem Pontíficem sanctórum tuórum méritis coæquásti: concéde propítius; ut, qui commemoratiónis ejus festa percólimus, vitæ quoque imitémur exémpla. \red{(}Per Dóminum.\red{)}

\rubric{Deinde Commemoratio Octavæ Ss. Petri et Pauli Apostolorum:}

\A Petrus Apóstolus, et Paulus Doctor géntium, ipsi nos docuérunt legem tuam Dómine.

\V Constítues eos príncipes super omnem terram.
\R Mémores erunt nóminis tui Dómine.

\pars{Oratio}
\yb{D}{eus}, qui hodiérnam diem Apostolórum tuórum Petri et Pauli martýrio consecrásti: da Ecclésiæ tuæ, eórum in ómnibus sequi præcéptum; per quos religiónis sumpsit exórdium. Per Dóminum.

\nocturn{In II Nocturno}
\lectio{Lectio iv}
\Y{U}{dalrícus}, excélsa prosápia Alamannórum patre Hupáldo, cómite de Díllingen et Khyburg, et matre Thietbúrga natus, puer mónachis sancti Galli tráditur educándus. Ibi stúdio literrárum et virtútum zelo, præsértim castimóniæ, ádeo excélluit, ut co\'ævi „sánctulum“ illum jam abínde c\'œperint vocitáre. In pátriam revérsus, cognáto suo Adalberóni, epíscopo Augustáno, fidéliter desérviit et post ejus mortem bonórum suórum patrimoniálium per quatuórdecim fere annos curam gessit. Mórtuo Hiltíno, Adalberónis successóre, Udalrícus ab Henríco primo rege epíscopus ecclésiæ Augustánæ præfícitur.

\R Invéni David servum meum, óleo sancto meo unxi eum: \red{*} Manus enim mea auxiliábitur ei.
\V Nihil profíciet inimícus in eo, et fílius iniquitátis non nocébit ei. \red{M}anus.

\lectio{Lectio v}
\y{F}{actus} ígitur epíscopus, totum se contemplatióni dedit et sacræ lectióni: ínterim nihilóminus, ut pastor bonus, summa cura ac sollicitúdine óvibus suis invigilábat. Quas, tum s\'æpjus visitándo, tum quibúsvis ingruéntibus perículis confirmándo, ádeo numquam deséruit, ut irrumpéntibus aliquándo Húngaris fórtiter resisténdo precésque férvidas ad Deum ex ánimo fundéndo cives suos e summo discrímine felíciter liberáret.

\R Pósui adjutórium super poténtem, et exaltávi eléctum de plebe mea: \red{*} Manus enim mea auxiliábitur ei.
\V Invéni David servum meum, óleo sancto meo unxi eum. \red{M}anus.

\lectio{Lectio vi}
\y{C}{omponéndis} discídiis et inimícis reconciliándis óperam semper navávit egrégjam; Ottónem C\'æsarem et Ljudólphum ejus fílium, gravíssimis inter se dissidéntes ódiis, reconciliávit. Páuperes quotídie mensæ suæ adhibébat convívas. Abstinéntia fuit admirábili; in oratióne étiam ad multam sæpe noctem assíduus, lecto plúmeo numquam est usus. Longum est commemoráre, quam multa ædificárit templa, quántaque per eum Deus contúlerit dona. Atque in his piis stúdiis ac labóribus per quinquagínta annos episcopátus múnere digne et sancte functus óbiit quinto Nonas Júlias, anno Incarnatiónis Domínicæ nongentésimo septuagésimo tértio. Vigínti annis post a sýnodo Lateranénsi, præsidénte Joánne Papa quintodécimo, inter Sanctos relátus est, quæ fuit prima omníno inter omnes canonizatiónes solémnes. Hujus relíquiæ Augústæ Vindelicórum in ecclésia sanctæ Afræ, póstea ejus nómine nuncupáta, religiosíssime servátur, ubi gloriósa ejus mérita miráculis corúscant.

\R Iste est, qui ante Deum magnas virtútes operátus est, et omnis terra doctrína ejus repléta est: \red{*} Ipse intercédat pro peccátis ómnium populórum.
\V Iste est, qui contémpsit vitam mundi, et pervénit ad cæléstia regna. \red{I}pse. \red{G}lória Patri. \red{I}pse.

\rubric{In III Nocturno Homilia in Evangelium \black{Homo péregre,} de Communi Confessoris Pontificis 1 loco.}

\hora{Ad Laudes}
\V Justum dedúxit Dóminus per vias rectas.
\R Et osténdit illi regnum Dei.

\red{Ad Bened. Ant.} Euge serve bone * et fidélis, quia in pauca fuísti fidélis, supra multa te contítuam, dicit Dóminus.

\pars{Oratio}
\y{D}{eus}, qui cónspicis, quia ex nulla nostra virtúte subsístimus: concéde propítius; ut, intercessióne beáti Udalríci Confessóris tui atque Pontíficis, contra ómnia advérsa muniámur. Per Dóminum.

\rubric{Et fit Com. Octavæ:}

\A Gloriósi Príncipes terræ, quómodo in vita sua dilexérunt se, ita et in morte non sunt separáti.

\V In omnem terram exívit sonus eórum.
\R Et in fines orbis terræ verba eórum.

\pars{Oratio}
\yb{D}{eus}, qui hodiérnam diem Apostolórum tuórum Petri et Pauli martýrio consecrásti: da Ecclésiæ tuæ, eórum in ómnibus sequi præcéptum; per quos religiónis sumpsit exórdium. Per Dóminum.

\rubric{Vesperæ a Capitulo de sequenti, Com. præcedentis:}

\A Amávit eum Dóminus, et ornávit eum: stolam glóriæ índuit eum, et ad portas paradísi coronávit eum.

\V Justum \red{et Oratio} Deus, qui cónspicis, \red{ut supra.}

\rubric{Deinde Com. Octavæ:}

\A Petrus Apóstolus, et Paulus Doctor géntium, ipsi nos docuérunt legem tuam Dómine.

\V Constítues eos príncipes super omnem terram.
\R Mémores erunt nóminis tui Dómine.

\red{Oratio} Deus, qui hodiérnam, \red{ut supra.}

\die{Die 8 Julii}{Ss. Kiliani \red{Ep.} et Sociorum}{Martyrum}{Duplex}

{\setstretch{0.99}
\V Lætámini in Dómino et exsultáte, justi.
\R Et gloriámini, omnes recti corde.

\red{Ad Magnif. Ant.} Istórum est enim * regnum cælórum, qui contempsérunt vitam mundi, et pervenérunt ad pr\'æmia regni, et lavérunt stolas suas in sánguine Agni.

\pars{Oratio}
\y{D}{eus}, qui sanctórum Mártyrum Kiliáni et Sociórum ejus méntibus in passióne constántiam inspirásti: concéde propítius; ut sicut illi devíctis persecutóribus a te coronári meruérunt, ita nos eórum interventióne tua semper protectióne lætémur. Per Dóminum.

\rubric{Et fit Com. præcedentis:}

\A Isti sunt viri sancti, facti amíci Dei, divínæ veritátis præcónio gloriósi: linguæ eórum claves cæli factæ sunt.

\V Sacerdótes tui induántur justítiam. 
\R Et sancti tui exsúltent.

\pars{Oratio}
\yb{O}{mnípotens} sempitérne Deus, qui Slavóniæ gentes per beátos Confessóres tuos atque Pontífices Cyrillum et Methódium ad agnitiónem tui nóminis veníre tribuísti: præsta; ut, quorum festivitáte gloriámur, eórum consórtio copulémur. \red{(}Per Dóminum.\red{)}

\rubric{Deinde Com. S. Elisabeth Reginæ, Viduæ:}

\A Et nunc, reges, intellígite: erudímini, qui judicátis terram.

\V Ora pro nobis, beáta Elísabeth.
\R Ut digni efficiámur promissiónibus Chisti.

\pars{Oratio}

\yb{C}{lementíssime} Deus, qui beátam Elísabeth regínam, inter céteras egregias dotes, béllici furóris sedándi prærogativa decorásti: da nobis ejus intercessióne post mortális vitæ, quam supplíciter pétimus, pacem, ad ætérna gáudia perveníre. Per Dóminum.


\nocturn{In II Nocturno}
\lectio{Lectio iv}
\Y{K}{iliánus} ex Scótia nobílibus paréntibus ortus contémptis hujus mundi illécebris, ad monastérium, tamquam salútis portum tutíssimum, confúgit, in quo assíduis jejúniis, vigíliis et précibus, ac rerum divinárum contemplatióne máximam nóminis celebritátem ac sanctitátis laudem consecútus, et a frátribus monastérii régimen assúmere coáctus, accítis quibúsdam sóciis inde discéssit, ac Herbípolim, Francóniæ urbem, pervénit, ibíque ignótus inter incrédulæ gentis feritátem aliquámdiu vitam egit, parátus pro illíus salúte Christíque glória sánguinem fúndere. Proféctus póstea Romam, accépta a Summo Pontífice prædicándi licéntia, ad ducem Herbipolénsem, Gisbértum nómine, revértitur episcopáli auctus dignitáte, eum ad fidem christiánam prædicatióne sua convértit, et cum multis áliis ipsi subjéctis baptizávit.

\R Sancti tui, Dómine mirábile consecúti sunt iter, serviéntes præceptis tuis, ut inveniréntur ill\'æsi in aquis válidis: \red{*} Terra appáruit árida: et in Mari Rubro via sine impediménto.
\V Quóniam percússit petram, et fluxérunt aquæ, et torréntes inundavérunt. \red{T}erra.

\lectio{Lectio v}
\y{V}{idens} autem vir sanctus ducem, quantúmvis in fide instrúctus esset, periclitári tamen salúte, quod germáni sui uxórem Geylánam martimónio sibi copulátam contra evangélicæ legis doctrínam retinéret, blando collóquio familiárius admónito presuásit, ut eam a toro separátam dimíttere cogitáret. Hæc cum ad aures ímpiæ mulíeris perveníssent, furóre succénsa Kiliáni necem die noctúque pertráctans, ei insídias paráre cœpit, quas ille cognóscens, cum sóciis Colománno presbýtero, et Totnáno diácono oratiónibus et jejúnio vacáre non désiit.

\R Vérbera carníficum non timuérunt sancti Dei, moriéntes pro Christi nómine: \red{*} Ut herédes fíerent in domo Dómini.
\V Tradidérunt córpora sua propter Deum ad supplícia. \red{U}t.

}

\lectio{Lectio vi}
\y{C}{um} ígitur diu optátum martýrii diem alácriter expectárent, irruéntibus dum divínas laudes perágerent sicáriis ad hoc destinátis, ímpio gládio trucidáti sunt; córpora eórum mox in eódem loco cum Pontificálibus induméntis et sacris libris et cruce, ne quis posset indícium necis deprehéndere, defóssa sunt. Unde Geylána a dæmónio vexáta ad ætérnas pœnas ivit, unus lictórum in furórem versus se déntibus laniávit, alter a d\'æmone in rábiem actus gládio se perémit. Ejus ac sociórum córpora póstmodum crebérrimis coruscántia miráculis a sanctis Bonifátio archiepíscopo et Burchárdo ejus loci epíscopo, consílio et præcépto sancti Zacharíæ Papáe, anno Dómini septingentésimo quinquagésimo de túmulo, quo indecénter pósita fúerant, subleváta sunt, et púplico cúltui honorífice exhíbita.

\R Tamquam aurum in fornáce probávit eléctos Dóminus, et quasi holocáusti hóstiam accépit illos: et in témpore erit respéctus illórum: \red{*} Quóniam donum et pax est eléctis Dei.
\V Qui confídunt in illum, intélligent veritátem: et fidéles in dilectióne acquiéscent illi. \red{Q}uóniam. \red{G}lória Patri. \red{Q}uóniam.

\rubric{In III Nocturno Homilia in Ev. \black{Cum audiéritis pr\'ælia,} de Communi plurimorum Mm. 1 loco.}

\pro{Pro S. Elisabeth:}
\lectio{Lectio ix}
\yb{E}{lísabeth} Aragóniæ régibus orta est anno Christi millésimo ducentésimo septuagésimo primo. Natális ejus lætítia perniciósas avi patrísque dissensiónes in concórdiam convértit, ex quo statim pátuit, quam felix regum regnorúmque esset futúra pacátrix. In castigándo córpore, in précibus assídue recitándis, in caritátis opéribus exercéndis admirábilis fuit. Dionýsio Lusitániæ regi in matrimónium trádita, non minórem excoléndis virtútibus quam líberis educándis óperam dabat, viro placére studens, sed magis Deo. Monastéria, collégia et templa non modo exstrúxit, sed étiam magnífice dotávit. In regum discórdiis componéndis admirábilis fuit, in privátis publicísque mortálium sublevándis calamitátibus indeféssa et miráculis clara. Defúncto rege Dionýsio, cum hábitum Seráphici órdinis induísset, quidquid sibi carum aut pretiósum supérerat, pro regis ánima templo Compostelláno óbtulit, et in sacros ac pios usus convértit. Dénique reges duos, fílium et génerum, pacificatúra, morbo ex itínere contrácto, a Vírgine Deípara visitáta, sanctíssime óbiit. Eam, miráculis claram, Urbánus octávus inter Sanctos adscrípsit.

\red{T}e Deum laudámus.

\hora{Ad Laudes}

\V Exsultábunt Sancti in glória.
\R Lætabúntur in cubílibus suis.

\red{Ad Bened. Ant.} Vestri capílli cápitis * omnes numeráti sunt : nolíte timére : multis passéribus melióres estis vos.

\pars{Oratio}
\y{D}{eus}, qui sanctórum Mártyrum Kiliáni et Sociórum ejus méntibus in passióne constántiam inspirásti: concéde propítius; ut sicut illi devíctis persecutóribus a te coronári meruérunt, ita nos eórum interventióne tua semper protectióne lætémur. Per Dóminum.

\rubric{Et fit Com. S. Elisabeth:}

\A Tu glória Jerúsalem, tu lætítia Isra\"el, tu honorificéntia pópuli tui.

\V Méritis et précibus beátæ Elísabeth.
\R Propítius esto, Dómine, pópulo tuo.

%\columnbreak
\pars{Oratio}

\yb{C}{lementíssime} Deus, qui beátam Elísabeth regínam, inter céteras egregias dotes, béllici furóris sedándi prærogativa decorásti: da nobis ejus intercessióne post mortális vitæ, quam supplíciter pétimus, pacem, ad ætérna gáudia perveníre. Per Dóminum.

\hora{In II Vesperis}

\V Exsultábunt Sancti in glória.
\R Lætabúntur in cubílibus suis.

\red{Ad Magnif. Ant.} Gaudent in cælis * ánimæ Sanctórum, qui Christi vestígia sunt secúti: et quia pro ejus amóre sánguinem suum fudérunt, ídeo cum Christo exsúltant sine fine.

\rubric{Et fit Com. S. Elisabeth:}

\A Elísabeth pacis et pátriæ mater, in cælo triúmphans, dona nobis pacem.

\V Ora pro nobis, beáta Elísabeth.
\R Ut digni efficiámur promissiónibus Christi.

\red{Oratio} Clementíssime, \red{ut supra.}



\die{Die 11 Julii}{S. Udalrici Monachi}{Confessoris}{Semiduplex \black{(m. t. v.)}}

\V Amávit eum Dóminus, et ornávit eum.
\R Stolam glóriæ índuit eum.

\red{Ad Magnif. Ant.} Similábo eum * viro sapiénti, qui ædificávit domum suam supra petram.

{\setstretch{0.98}
\pars{Oratio}
\y{D}{eus}, qui nos ádmones, exémplo beáti Udalríci temporália despícere et ad ætérna festináre: da fámulis tuis; ut, quæ tibi plácita cognóvimus, ipso pro nobis intercedénte, implére valeámus. Per Dóminum.

\rubric{Et fit Com. præcedentis:}

\A Gaudent in cælis ánimæ sanctórum, qui Christi vestígia sunt secúti: et quia pro ejus amóre sánguinem suum fudérunt, ídeo cum Christo exsúltant sine fine.

\V Exsultábunt sancti in glória.
\R Lætabúntur in cubílibus suis.

\pars{Oratio}
\yb{P}{ræsta}, quǽsumus, omnípotens Deus: ut, qui gloriósos Mártyres fortes in sua confessióne cognóvimus, pios apud te in nostra intercessióne sentiámus. \red{(}Per Dóminum.\red{)}

\rubric{Deinde Com. S. Pii I Papæ et Mart.:}

\A Iste Sanctus pro lege Dei sui certávit usque ad mortem, et a verbis impiórum non tímuit: fundátus enim erat supra firmam petram.

}

\V Glória et honóre coronásti eum, Dómine.
\R Et constituísti eum super ópera mánuum tuárum.

\pars{Oratio}
\yb{I}{nfirmitátem} nostram réspice, omnípotens Deus: et quia pondus própriæ actiónis gravat, beáti Pii Mártyris tui atque Pontíficis intercéssio gloriósa nos prótegat. Per Dóminum.

\nocturn{In II Nocturno}
\lectio{Lectio iv}
\Y{U}{dalrícus} ex illústri Bojariórum gente, patre Bernóldo, Henríco tértio Imperatóri acceptíssimo, Ratisbónæ natus, in liberálibus stúdiis et disciplínis sédulo fuit institútus. Adolescéntior factus et aulam Imperatóris ingréssus, virtútum ómnium exémplar céteris áulicis se pr\'æbuit. Mox a pátruo Frisingénsi epíscopo ab aula evocátus, clericáli milítiæ adscríptus et diáconus factus, ob spectátam morum integritátem Frisingénsi ecclésiæ præpósitus est renuntiátus. Sed non ita multo post, desidério sacra Palæstínæ loca viséndi incénsus, longíssimam illam peregrinatiónem non sine magnis labóribus vit\'æque discrímine suscépit perfecítque. Ratisbónam revérsus, de solitária vita capessénda construendóque monastério cogitáre cœpit; sed fama permótus sancti Hugónis abbátis Cluniacénsis, suis bonis máxima ex parte paupéribus distribútis, salutatísque Apostolórum limínibus Cluníacum proféctus est, ut se illi in monástica vita instituéndum tráderet.

\R Honéstum fecit illum Dóminus, et custodívit eum ab inimícis, et a seductóribus tutávit illum: \red{*} Et dedit illi claritátem ætérnam.
\V Justum dedúxit Dóminus per vias rectas, et osténdit illi regnum Dei. \red{E}t.


\lectio{Lectio v}
\y{A}b Hugóne in monastérium admíssus, statim virtútibus ómnibus, imprímis christiána ánimi demissióne, ádeo inclaréscere cœpit, ut paulo post Hugo eúndem ad sacerdótium promovéndum curáverit, sibi a sacris et a consíliis esse volúerit, ac monachórum confessiónibus excipiéndis præfécerit. Hoc múnere admirábili prudéntia, comitáte et caritáte defungebátur. Sævíssimas carnis tentatiónes mira fortitúdine devícit, illátas sibi injúrias patientíssime tulit. Marciníacum, ut sacris virgínibus præésset, missus, óculi dolóre corréptus, Cluníacum revérti coáctus est; sed prístinæ incolumitáti restitútus, Hugónis jussu primo monastérium in monte Rótgeri a fundaméntis exstrúxit, mox Gruningénse monastérium, jam pridem in Brisgóvia fundátum, regéndum suscépit. Verum in hoc monastério non diu súbstitit Udalrícus, quod illud laicórum frequéntiæ nímium patére cérneret: quare in locum magis remótum, Cella dictum, in Silva nigra monastérium tránstulit. Hujus monastérii ecclésiam in honórem Apostolórum Perti et Pauli étiam ædificávit, áltero in Bollschwilénsi pago pro sacris virgínibus exstrúcto ascetério.

\R Amávit eum Dóminus, et ornávit eum, \red{*} Et ad portas paradísi coronávit eum.
\V Induit eum Dóminus lorícam fídei, et ornávit eum. \red{E}t.

\lectio{Lectio vi}
\y{M}ónachos sanctíssimis légibus atque institútis regébat, sed præsértim abstinéntia, vigíliis, júgibus lácrimis continénti ferventissimáque oratióne et aliárum virtútum exercitatióne ad pietátem excitábat: quare brevi ejúsdem sanctimóniæ fama lóngjus latiúsque diffúsa est. Ad Guliélmum étiam abbátem Hirsaugiénsem divértit, ab eodémque hospítio recéptus, ejus rogátu de consuetudínibus Cluniacénsibus libros mira sapiéntia et pietáte refértos conscrípsit. Dénique virtútibus et miráculis clarus, cum oculórum usum pénitus amisísset, eámque calamitátem biénnio íntegro æquíssimo ánimo pertulísset, in extrémum íncidit morbum, atque ad ea verba: Sancti per fidem vicérunt regna, quæ sibi ánimam agénti jússerat recitári, spíritum Deo réddidit in eódem monastério sanctórum Petri et Pauli, ubi ejus sacrum corpus cónditum fuit, et quod deínde a sancto Udalríco nomen accépit.

\R Iste homo perfécit ómnia quæ locútus est ei Deus, et dixit ad eum: Ingrédere in réquiem meam: \red{*} Quia te vidi justum coram me ex ómnibus géntibus.
\V Iste est, qui contémpsit vitam mundi, et pervénit ad cæléstia regna. \red{Q}uia. \red{G}lóri Patri. \red{Q}uia.

\rubric{In III Nocturno Homilia in Evangel. \black{Sint lumbi,} de Communi Conf. non Pont. 1 loco.}

\pro{Pro S. Pio I:}
\lectio{Lectio ix}
\yb{P}{jus}, hujus nóminis primus, Aquilejénsis, Ruffini fílius, ex presbytero sanctæ Romanæ Ecclésiæ Summus Pontifex creátus est, Antonino Pio et Marco Aurelio imperatóribus augustis.
Quinque ordinatiónibus, mense Decembri, episcopos duódecim, octódecim presbyteros creávit. Exstant nonnulla ab eo præcláre instituta, præsertim ut Resurréctio Dómini nonnísi die Dominico celebrarétur. Pudentis domum in ecclésiam mutávit, eamque ob præstantiam supra ceteros titulos, útpote Romani Pontificis mansiónem, título Pastoris dicávit; et in qua sæpe rem sacram fecit, et multos ad fidem conversos baptizávit ac in fidelium númerum adscripsit. Dum vero boni pastoris munus obiret, fuso pro suis ovibus et summo pastore Christo sánguine, martyrio coronátus est quinto Idus Julii, ac sepúltus in Vaticano.

\red{T}e Deum laudámus.

{\setstretch{1.005}
\hora{Ad Laudes}

\V Justum dedúxit Dóminus per vias rectas.
\R Et osténdit illi regnum Dei.

\red{Ad Bened. Ant.} Euge, serve bone * et fidélis, quia in pauca fuísti fidélis, supra multa te constítuam, intra in gáudium Dómini tui.

\pars{Oratio}
\y{D}{eus}, qui nos ádmones, exémplo beáti Udalríci temporália despícere et ad ætérna festináre: da fámulis tuis; ut, quæ tibi plácita cognóvimus, ipso pro nobis intercedénte, implére valeámus. Per Dóminum.

}

{\setstretch{0.98}
\rubric{Et fit Com. S. Pii I:}

\A Qui odit ánimam suam in hoc mundo, in vitam ætérnam custódit eam.

\V Justus ut palma florébit.
\R Sicut cedrus Líbani multiplicábitur.

\pars{Oratio}
\yb{I}{nfirmitátem} nostram réspice, omnípotens Deus: et quia pondus própriæ actiónis gravat, beáti Pii Mártyris tui atque Pontíficis intercéssio gloriósa nos prótegat. Per Dóminum.

\rubric{Vesperæ de sequenti, Com. præcedentis:}

\A Hic vir despíciens mundum et terréna triúmphans, divítias cælo cóndidit ore manu.

\V Justum dedúxit. \red{et Oratio} Deus, qui nos, \red{ut supra.}

\rubric{Deinde Com. Ss. Naboris et Felicis Mm.:}

\A Istórum est enim regnum cælórum, qui contempsérunt vitam mundi, et pervenérunt ad pr\'æmia regni, et lavérunt stolas suas in sánguine Agni.

\V Lætámini in Dómino, et exsultáte justi.
\R Et gloriámini omnes recti corde.

}

\pars{Oratio}
\yb{P}{ræsta}, quǽsumus, Dómine: ut, sicut nos sanctórum Mártyrum tuórum Náboris et Felícis natalícia celebránda non déserunt; ita júgiter suffrágiis comiténtur. Per Dóminum.

%\thirdline


\die{Die 15 Julii}{S. Henrici}{Imperatoris, Confessoris}{Duplex majus \mtv}

\rubric{Omnia ut in Breviario eadem die, præter ritum.}

\rubric{In I Vesperis fit Com. præcedentis}

\rubric{In II Vesperis fit Com. sequentis}

%\end{multicols}

%\vspace{8em}

%\ornamentv
%\pagebreak
%\begin{multicols}

\privdie{Die 24 Julii}{Beati Bernardi}{Marchionis Baden., Confessoris \rubric{In Magno Ducatu Baden.:} Patroni principalis}{Duplex I classis \mtv}

\rubric{Omnia de Communi Conf. non Pont., præter sequentia.}

\V Amávit eum Dóminus, et ornávit eum.
\R Stolam glóriæ índuit eum.

\red{Ad Magnif. Ant.} Similabo eum * viro sapienti, qui ædificávit domum suam supra petram.

\pars{Oratio}
\y{D}{eus}, qui beátum Bernárdum Confessórem tuum æternitátis láurea decorásti: concéde propítius; ut, qui commemoratiónis ejus festa percólimus, vitæ quoque imitémur exémpla. Per Dóminum.

\rubric{In I Nocturno Lectiones \black{Beátus vir,} de Communi Conf. non Pont. 1 loco.}

{\setstretch{0.99}
\nocturn{In II Nocturno}\label{Bernardi-NII}
\lectio{Lectio iv}
\Y{B}{ernárdus}, in Margraviátu Badénsi natus, paréntes hábuit Jacóbum primum, marchiónem Badénsem, et Catharínam, Cároli ducis Lotharíngiæ fíliam, quorum ópera sanctísque institútis erudítus, jam a ténera ætáte Deum timére dídicit, eíque religiónis áctibus famulári. Jacóbo vita functo ditiónum régimen suscépit, quod munus dignitatémque sic sustínuit, ut illíus quique prudéntiam, æquitátem ac justítiam celebrárent. Sed cum virginiátem serváre ánimo statuísset, Magdalénæ, Cároli séptimi Galliárum regis fíliæ, præclaríssimas núptias recusávit subject\'æque sibi ditiónis régimen Cárolo fratri sponte cessit, ut a terrénis curis solútus, libérius Deo divinísque rebus vacáret.

\R Honéstum fecit illum Dóminus, et custodívit eum ab inimícis, et a seductóribus tutávit illum:
\red{*} Et dedit illi claritátem ætérnam.
\V Justum dedúxit Dóminus per vias rectas, et osténdit illi regnum Dei. \red{E}t.

\lectio{Lectio v}
\y{I}n cæléstium meditatióne assíduus, brevi christiánam perfectiónem adéptus est; suóque exémplo monstrávit, in ipso étiam aulæ splendóre sólida excelléntis virtútis jaci posse fundaménta. Sui rerúmque terrenárum contémptor, humilitátem, modéstiam et castitátem appríme cóluit. Corpus vero jejúniis, vigíliis aliísque pœnárum genéribus veheménter afflíxit: cilício enim utebátur ad carnes hórrido, equínis crínibus intertéxto, quo étiam indútus post mortem invéntus fuit. Quotídie Missæ sacrifício devotíssime intérerat, quotídie, priúsquam cúbitum iret, consciéntiæ suæ exomologésin faciébat. Erga Deíparam Vírginem singulári ferebátur pietátis sensu: erga Christi páuperes caritáte afficiebátur máxima, quorum ádeo amans erat, ut ob insígnem commiseratiónem páuperum pater dictus fúerit.

\R Amávit eum Dóminus, et ornávit eum: stolam glóriæ índuit eum, \red{*} Et ad portas paradísi coronávit eum.
\V Índuit eum Dóminus lorícam fídei, et ornávit eum. \red{E}t.

\lectio{Lectio vi}
\y{C}{athólicæ} religiónis propagándæ stúdio incénsus ad Friderícum tértium se cóntulit, ut contra Turcas bellum sacrum indíci posset. Præses cæsáreus per Itáliam ab eo deléctus, et ad Calíxtum tértium ad fœdus ineúndum orátor designátus, in Sabáudiam ad Ludovícum ducem perréxit, eúmque sacro bello sociávit. Romam inde proficíscens Montiscalérii in cœnóbio Fratrum minórum conventuálium sancti Francísci spirituálibus exercítiis óperam dare constítuit et, cum postrémum diem sibi imminére divínitus intellexísset, sacraméntis ecclésiæ munítus, in hospítio prope dictum cœnóbium, in crucis suavíssimo ampléxu ánimam Deo réddidit, Idibus Júlii, anno millésimo quadringentésimo quinquagésimo octávo. Ob plúrima mirácula, quæ illíus mortem jam in die depositiónis consecúta sunt, cultu público statim coli cœpit: quem quidem cultum ab immemorábili témpore ad ætátem nostram prodúctum Clemens décimus quartus Póntifex máximus rite probávit.

}

{\setstretch{0.98}
\R Iste homo perfécit ómnia quæ locútus est ei Deus, et dixit ad eum: Ingrédere in réquiem meam: \red{*} Quia te vidi justum coram me ex ómnibus géntibus.
\V Iste est, qui contémpsit vitam mundi, et pervénit ad cæléstia regna. \red{Q}uia. \red{G}lória Patri. \red{Q}uia.

\rubric{In III Nocturno Homilia in Evangel. \black{Sint lumbi,} de Communi Conf. non Pont. 1 loco.}

\hora{Ad Laudes}

\VRCii

\BC

\pars{Oratio}
\y{D}{eus}, qui beátum Bernárdum Confessórem tuum æternitátis láurea decorásti: concéde propítius; ut, qui commemoratiónis ejus festa percólimus, vitæ quoque imitémur exémpla. Per Dóminum.

\hora{In II Vesperis}

\V Justum dedúxit \red{ut supra.}

\MiiC

\rubric{Et fit Com. sequentis tantum:}

\A Tradent enim vos in concíliis, et in synagógis suis flagellábunt vos, et ante reges et prǽsides ducémini propter me in testimónium illis, et Géntibus.

\V In omnem terram exívit sonus eórum.
\R Et in fines orbis terræ verba eórum.

\pars{Oratio}
\yb{E}{sto}, Dómine, plebi tuæ sanctificátor et custos: ut, Apóstoli tui Jacóbi muníta præsídiis, et conversatióne tibi pláceat, et secúra mente desérviat. Per Dóminum.

}



\die{Eadem die 24 Julii\\\rubric{In territorio Hohenzollerano:}}{Beati Bernardi}{Marchionis Baden., Confessoris, Patroni minus principalis}{Duplex majus \mtv}
% change header here
\chead{\trim{Die 24 Julii}{Beati Bernardi}}

\V Amávit eum Dóminus, et ornávit eum.
\R Stolam glóriæ índuit eum.

\red{Ad Magnif. Ant.} Similabo eum * viro sapienti, qui ædificávit domum suam supra petram.

\pars{Oratio}
\y{D}{eus}, qui beátum Bernárdum Confessórem tuum æternitátis láurea decorásti: concéde propítius; ut, qui commemoratiónis ejus festa percólimus, vitæ quoque imitémur exémpla. Per Dóminum.

\rubric{Et fit Com. præcedentis:}

\A Qui vult veníre post me, ábneget semetípsum, et tollat crucem suam, et sequátur me.

\V Justus ut palma florébit.
\R Sicut cedrus Líbani multiplicábitur.

\pars{Oratio}
\yb{D}{eus}, fidélium remunerátor animárum, qui hunc diem beáti Apollináris Sacerdótis tui martýrio consecrásti: tríbue nobis, quǽsumus, fámulis tuis; ut, cujus venerándam celebrámus festivitátem, précibus ejus indulgéntiam consequámur. Per Dóminum.

\rubric{Deinde Com. S. Christinæ Virginis et Martyris:}

\A Veni, Sponsa Christi, áccipe corónam, quam tibi Dóminus præparávit in ætérnum.

\V Diffúsa est grátia in lábiis tuis.
\R Proptérea benedíxit te Deus in ætérnum.

\pars{Oratio}
\yb{I}{ndulgéntiam} nobis, quǽsumus, Dómine, beáta Christina Virgo et Martyr implóret: quæ tibi grata semper éxstitit, et merito castitátis, et tuæ professióne virtútis. Per Dóminum.

\rubric{Lectiones II Nocturni, ut supra. \black{\pageref{Bernardi-NII}}}

\rubric{In III Nocturno Homilia in Ev. \black{Sint lumbi vestri,} de Communi Conf. non Pont. 1 loco.}

\pro{Pro Vigilia S. Jacobi Ap.:}
\scriptura{Léctio sancti Evangélii secúndum Joánnem}
\lectiocap{Lectio ix}{Cap. 15, 12-16}
\y{I}{n} illo témpore: Dixit Jesus discípulis suis: Hoc est præcéptum meum, ut diligátis ínvicem, sicut diléxi vos. Et réliqua.

\scriptura{Homilía sancti Gregórii Papæ}
\ex{Homilia 27 in Evangelia}
\Y{C}{um} cuncta sacra elóquia Domínicis plena sint præcéptis, quid est quod de dilectióne, quasi de singulári mandáto, Dóminus dicit: Hoc est præcéptum meum, ut diligátis ínvicem: nisi quia omne mandátum de sola dilectióne est: et ómnia unum præcéptum sunt? quia quidquid præcípitur, in sola caritáte solidátur. Ut enim multi árboris rami ex una radíce pródeunt: sic multæ virtútes ex una caritáte generántur. Nec habet áliquid viriditátis ramus boni óperis, si non manet in radíce caritátis.

\Te

\hora{Ad Laudes}

\V Justum dedúxit Dóminus per vias rectas.
\R Et osténdit illi regnum Dei.

\red{Ad Bened. Ant.} Euge, serve bone * et fidélis, quia in pauca fuísti fidélis, supra multa te constítuam, intra in gáudium Dómini tui.

\pars{Oratio}
\y{D}{eus}, qui beátum Bernárdum Confessórem tuum æternitátis láurea decorásti: concéde propítius; ut, qui commemoratiónis ejus festa percólimus, vitæ quoque imitémur exémpla. Per Dóminum.

\rubric{Et fit Com. Vigiliæ:}

\rubric{Ant. et \V de Feria currenti.}

\pars{Oratio}
\yb{D}{a} qu\'æsumus, omnípotens Deus: ut beáti Jacóbi, Apóstoli tui, quam prævenímus, veneránda solémnitas, et devotiónem nobis áugeat et salútem.

\rubric{Deinde Com. S. Christinæ:}

\A Símile est regnum cælórum hómini negotiatóri quærénti bonas margarítas: invénta una pretiósa, dedit ómnia sua, et comparávit eam.

\V Diffúsa est grátia in lábiis tuis.
\R Proptérea benedíxit te Deus in ætérnum.

\pars{Oratio}
\yb{I}{ndulgéntiam} nobis, quǽsumus, Dómine, beáta Christína Virgo et Martyr implóret: quæ tibi gráta semper éxstitit, et mérito castitátis, et tuæ professióne virtútis. Per Dóminum.

\rubric{Vesperæ de sequenti, Com. præcedentis:}

\red{Ad Magnif. Ant.} Hic vir despíciens mundum * et terréna, triúmphans, divítias cælo cóndidit ore, manu.

\V Justum dedúxit \red{et Oratio} Deus, qui beátum, \red{ut supra.}


\mens{Festa Augusti}


\dieii{Die 27 Augusti}{S. Gebhardi}{Episcopi Constantiensis, Confessoris}{Duplex}

\M Sacérdos et Póntifex, * et virtútum ópifex, pastor bone in pópulo, ora pro nobis Dóminum.

\V Amávit eum Dóminus, et ornávit eum.
\R Stolam glóriæ índuit eum.

\pars{Oratio}
\y{E}{xáudi} qu\'æsumus, Dómine, preces nostras, quas in beáti Gebhárdi Confessóris tui atque Pontíficis solemnitáte deférimus: et qui tibi digne méruit famulári, ejus intercedéntibus méritis ab ómnibus nos absólve peccátis. Per Dóminum.

\rubric{Et fit Com. S. Josephi Calasanctii Conf.:}

\A Simliábo eum viro sapiénti, qui ædificávit domum suam supra petram.

\V Justum dedúxit Dóminus per vias recatas.
\R Et osténdit illi regnum Dei.

\pars{Oratio}
\yb{D}{eus}, qui per sanctum Ioséphum Confessórem tuum, ad erudiéndam spíritu intellegéntiæ ac pietátis juventútem, novum Ecclésiæ tuæ subsídium providére dignátus es: præsta, quǽsumus; nos, ejus exémplo et intercessióne, ita fácere et docére, ut prǽmia consequámur ætérna. Per Dóminum.

\nocturn{In II Nocturno}
\lectio{Lectio iv}
\Y{G}{ebhárdus}, hoc nómine secúndus Constantiénsis Epíscopus, Comes Brigantínus, paréntes hábuit Uttónem et Thietbúrgam, ex præclaríssimo Alamannórum génere procreátos; puerítiam lítteris grammaticálibus addiscéndis consúmpsit; adoléscens, mirum est, quanta ánimi puritáte atque innocéntia lúbricam illam ætátem transégerit; apud cives eam probitátis integritatísque opiniónem adéptus est, ut Gaminólpho Constantíensi Antístite, qui sancto Conrádo succésserat, ex hac vita subláto, summo ómnium consénsu atque lætítia, anno incarnatiónis Domínicæ nongentésimo octogésimo, Ottónis secúndi octávo, epíscopus creátus sit.

\R Invéni David servum meum, óleo sancto meo unxi eum: \red{*} Manus enim mea auxiliábitur ei.
\V Nihil profíciet inimícus in eo, et fílius iniquitátis non nocébit ei. \red{M}anus.

\lectio{Lectio v}
\y{F}{actus} epíscopus, humilitátem, castitátem aliásque virtútes perpétuo diléxit: Cathólicam fidem et disciplínam ecclesiásticam acérrime deféndit: satis amplum ac lautum patrimónium, quod ei pii paréntes relíquerant, paupéribus distríbuit, illud Dómini secum ánimo volvens: Si vis perféctus esse, vade et vende ómnia, quæ habes, et da paupéribus, et habébis thesáurum in cælis. Monastérium beáti Gregórii Papæ nómine, anno post Christum natum nongentésimo octogésimo tértio, magnificentíssime exstrúxit. Collégium prætérea monachórum, qui diu noctúque Deo deservírent, instítuit, quod multis pr\'ædiis attribútis locupletávit, ac munéribus amplíssimis exornávit.

\R Pósui adjutórium super poténtem, et exaltávi eléctum de plebe mea: \red{*} Manus enim mea auxiliábitur ei.
\V Invéni David servum meum, óleo sancto meo unxi eum. \red{M}anus.

\lectio{Lectio vi}
\y{B}eátus ítaque vir, Ecclésiam sibi commíssam cum séxdecim annos sanctíssime administrásset, tandem Dei benignitáte, datis præcláris sanctitátis suæ testimóniis, cumulátus méritis quiévit in Dómino, sexto Kaléndas Septémbris, anno a Vírginis partu nongentésimo nonagésimo sexto. Incredíbile quidem dictu est, quantum mórtuus apud omnes sui desidérium relíquerit. Ejus corpus honorífico sepúlchro tráditum est.

\R Iste est, qui ante Deum magnas virtútes operátus est, et omnis terra doctrína ejus repléta est: \red{*} Ipse intercédat pro peccátis ómnium populórum.
\V Iste est, qui contémpsit vitam mundi, et pervénit ad cæléstia regna. \red{I}pse. \red{G}lória Patri. \red{I}pse.

\nocturn{In III Nocturno}
\scriptura{Léctio sancti Evangélii secúndum Matth\'æum}
\lectiocap{Lectio vii}{Cap. 24, 42-47}
\y{I}{n} illo témpore: Dixit Jesus discípulis suis: Vigiláte, quia nescítis qua hora Dóminus vester ventúrus sit. Et réliqua.

\scriptura{Homilía sancti Hilárii Epíscopi}
\ex{Comment. in Matth. can. 26 in fine}
\Y{U}{t} ignorántiam illam diéi ómnibus táciti, non sine útilis siléntii ratióne esse scirémus, vigiláre nos Dóminus propter advéntum furis admónuit, et oratiónum assiduitáte deténtos, ómnibus præceptórum suórum opéribus inhærére. Furem enim esse osténdit zábulum, ad detrahénda ex nobis spólia pervígilem, et córporum nostrórum dómibus insidiántem: ut ea, incuriósis nobis, et somno déditis, consiliórum suórum atque illecebrárum jáculis perfódiat. Parátos ígitur esse nos cónvenit, quia diéi ignorátio inténtam sollicitúdinem suspénsæ expectatiónis exágitet.

\R Amávit eum Dóminus, et ornávit eum: stolam glóriæ índuit eum, \red{*} Et ad portas paradísi coronávit eum.
\V Índuit eum Dóminus lorícam fídei, et ornávit eum. \red{E}t.

\lectiocap{Lectio viii}{Can. 27}
\y{Q}{uisnam} est fidélis servus et prudens, quem constítuit Dóminus super famíliam suam? Quamquam in commúne nos ad indeféssam vigilántiæ curam adhortétur: speciálem tamen pópuli principibus, id est, epíscopis, in exspectatióne adventúque suo sollicitúdinem mandat. Hunc enim servum fidélem atque prudéntem, præpósitum famíliæ signíficat, cómmoda atque utilitátes commíssi sibi pópuli curántem. Qui si dicto áudiens, et præcéptis obédiens erit, id est, si doctrínæ opportunitáte et veritáte infírma confirmet, disrúpta consólidet, depraváta convértat, et verbum vitæ in æternitátis cibum aléndæ famíliæ dispéndat, atque hæc agens, hisque ímmorans deprehendátur: glóriam a Dómino tamquam dispensátor fidélis, et víllicus útilis consequétur, et super ómnia bona constituétur; id est, in Dei glória collocábitur, quia nihil sit ultra, quod mélius sit.

\R Sint lumbi vestri præcíncti, et lucérnæ ardéntes in mánibus vestris: \red{*} Et vos símiles homínibus exspectántibus dóminum suum, quando revertátur a núptiis.
\V Vigiláte ergo, quia nescítis qua hora Dóminus vester ventúrus sit. \red{E}t. \red{G}lória Patri. \red{E}t.

\pro{Pro S. Josepho Calasanctio:}
\lectio{Lectio ix}
\yb{I}{oséphus} Calasánctius, Petráltæ in Aragónia natus, adhuc párvulus æquáles ad se convocátos mystériis fídei et sacris précibus erudiébat. Sacérdos ex voto factus, summa vitæ asperitáte, vigíliis et ieiúniis corpus afflígens, in oratióne et rerum cæléstium contemplatióne dies noctésque versabátur. Cum divínitus accepísset, se ad informándos intellegéntiæ ac pietátis spíritu adolescéntes præcípue páuperes, destinári; órdinem Clericórum regulárium páuperum Matris Dei scholárum piárum fundávit, qui peculiárem curam de púeris erudiéndis ex próprio institúto profiteréntur. Innúmeris proptérea labóribus atque ærúmnis invícto ánimo tolerátis, secúndum et nonagésimum annum agens, óbiit in Dómino, octávo kaléndas septémbris, anno millésimo sexcentésimo quadragésimo octávo.

\Te 

\hora{Ad Laudes}

\V Justum dedúxit Dóminus per vias recatas.
\R Et osténdit illi regnum Dei.

\B Euge, serve bone * et fidélis, quia in pauca fuísti fidélis, supra multa te constítuam, dicit Dóminus.

\pars{Oratio}
\y{E}{xáudi} qu\'æsumus, Dómine, preces nostras, quas in beáti Gebhárdi Confessóris tui atque Pontíficis solemnitáte deférimus: et qui tibi digne méruit famulári, ejus intercedéntibus méritis ab ómnibus nos absólve peccátis. Per Dóminum.

\rubric{Et fit Com. S. Josephi Calasanctii Conf.:}

\A Euge, serve bone * et fidélis, quia in pauca fuísti fidélis, supra multa te constítuam, intra in gáudium Dómini tui.

\V Amávit eum Dóminus, et ornávit eum.
\R Stolam glóriæ índuit eum.

\pars{Oratio}
\yb{D}{eus}, qui per sanctum Ioséphum Confessórem tuum, ad erudiéndam spíritu intellegéntiæ ac pietátis juventútem, novum Ecclésiæ tuæ subsídium providére dignátus es: præsta, quǽsumus; nos, ejus exémplo et intercessióne, ita fácere et docére, ut prǽmia consequámur ætérna. Per Dóminum.

\rubric{Vesperæ a Capitulo de sequenti, Com. præcedentis:}

\A Sacérdos et Póntifex, * et virtútum ópifex, pastor bone in pópulo, ora pro nobis Dóminum.

\V Justum dedúxit. \red{et Oratio} Exáudi qu\'æsumus, \red{ut supra.}

\rubric{Deinde Com. S. Josephi Calasanctii:}

\A Hic vir despíciens mundum * et terréna, triúmphans, divítias cælo cóndidit ore, manu.

\V Os justi meditábitur sapiéntiam.
\R Et lingua ejus loquétur judícium.

\red{Oratio} Deus qui, \red{ut supra.}

\rubric{Postea Com. S. Hermetis Mart.:}

\A Iste Sanctus pro lege Dei sui certávit usque ad mortem, et a verbis impiórum non tímuit: fundátus enim erat supra firmam petram.

\V Glória et honóre coronásti eum, Dómine.
\R Et constituísti eum super ópera mánuum tuárum.

\columnbreak
\pars{Oratio}
\yb{D}{eus}, qui beátum Hermétem Mártyrem tuum virtúte constántiæ in passióne roborásti: ex ejus nobis imitatióne tríbue; pro amóre tuo próspera mundi despícere, et nulla ejus advérsa formidáre. Per Dóminum.

\end{multicols}

\ornamentvi

\end{document}
