\documentclass[fontsize=9pt,paper=A6,twoside,BCOR=1mm,DIV=22,headinclude]{scrarticle}
\usepackage{breviarium}
\begin{document}
\renewcommand{\section}{}
\renewcommand\A{\Ant}
\titulum{Officia Propria}{Archidiœcesis Friburgensis}{Pars Autumnalis}
\begin{multicols}{2}
	{\setstretch{0.98}
\mensii{Festa Septembris}
\dieii{Die 17 Septembris\\\rubric{In Archidiœcesi:}}{S. Lamberti}{Episcopi et Martyris}{Duplex}
\chead{\trim{Die 17 Septembris}{S. Lamberti}}

\V Glória et honóre coronásti eum, Dómine.
\R Et constituísti eum super ópera mánuum tuárum.

\M Iste Sanctus * pro lege Dei sui certávit usque ad mortem, et a verbis impiórum non tímuit: fundátus enim erat supra firmam petram.

\pars{Oratio}
\y{D}{eus}, qui nos beáti Lambérti Mártyris tui atque Pontíficis ánnua solemnitáte lætíficas: concćede propítius; ut cujus natalítia cólimus, de ejúsdem étiam protectióne gaudéamus. Per Dóminum.

\rubric{Et fit Commemoratio præcedentis:}

\Ant Gaudent in cælis ánimæ Sanctórum, qui Christi vestígia sunt secúti: et quia pro ejus amóre sánguinem suum fudérunt, ídeo cum Christo exsúltant sine fine.

\V Exsultábunt Sancti in glória.
\R Lætabúntur in cubílibus suis.

\pars{Oratio}
\yb{B}{eatórum} Mártyrum paritérque Pontíficum Cornélii et Cypriáni nos, quǽsumus, Dómine, festa tueántur: et eórum comméndet orátio veneránda. Per Dóminum.

\rubric{Deinde Com. Imperssionis sacrorum Stigmatum S. Francisci Conf.:}

\Ant Similábo eum viro sapiénti, qui ædificávit domum suam supra petram.

\V Signasti, Domine, servum tuum Franciscum.
\R Signis redemptionis nostræ.

}

\pars{Oratio}
\yb{D}{ómine} Jesu Christe, qui, frigescénte mundo, ad inflammándum corda nostra tui amóris igne, in carne beatíssimi Francísci passiónis tuæ sacra Stígmata renovásti: concéde propítius; ut ejus méritis et précibus crucem iúgiter ferámus, et dignos fructus pæniténtiæ faciámus: Qui vivis.

\rubric{In I Nocturno, si dicendæ non sint de Scriptura occurenti, Lectiones \black{A Miléto,} de Communi unius Mart. 1 loco.}

\nocturn{In II Nocturno}
\lectio{Lectio iv}
\Y{L}{ambértus}, Trajecténsis ecclésiæ epíscopus, Apricómitis et Herisplíndis piórum paréntum fílius, præceptóre imprímis usus est sancto Landoáldo archipresbýtero. Deínde beátum Theodárdum epíscopum máximis virtútibus præstántem sequens, eídem anno Christi sexcentésimo quinquagésimo nono in episcopátu succéssit. Clodov\'æo Francórum regi ob suas exímias virtútes usque ádeo accéptus fuit, ut et ipsum præcípuo honóre ac veneratióne prosequerétur, ipsúmque et patrem et apostólicum virum appelláre et nomináre consuéverit. Childeríco item regi non minus gratus fuit, quippe qui ejus pectus Spíritus Sancti templum et linguam veritátis instruméntum díceret.

\R Honéstum fecit illum Dóminus, et custodívit eum ab inimícis, et a seductóribus tutávit illum: \red{*} Et dedit illi claritátem ætérnam.
\V Descendítque cum illo in fóveam, et in vínculis non derelíquit eum. \red{E}t.

\lectio{Lectio v}
\y{V}{erum} Childeríco rege occíso Ebroínus tyránnus, qui major domus sub Theodoríco s\'æpjus Gálliam perturbáverat, sanctum antístitem véluti justítiæ public\'æque libertátis defensórem episcopátu éxpulit; qui Stabulénse monastérium petens ibi támdiu fuit, quoad Ebroíno crudéliter interfécto a Pipíno de Herstéllo, Pipíni Brabántiæ ducis ex Begga fília nepóte, summa totíus pópuli gratulatióne ab exsílio revocátur.

\R Desidérium ánimæ ejus tribuísti ei Dómine, \red{*} Et voluntáte labiórum ejus non fraudásti eum.
\V Quóniam prævenísti eum in benedictiónibus dulcédinis: posuísti in cápite ejus corónam de lápide pretióso. \red{E}t.

\lectio{Lectio vi}
\y{Q}{ui} cum boni pastóris offícium exséquitur, et príncipem ob adultérium cum Alpáide zelo religiónis arguísset, noctu ante altáre sanctórum Cosmæ et Damiáni Mártyrum oratióni vacans, Dodónis adúlteræ fratris ópera, innocens a nocéntibus, una cum Petro et Andoléto consanguíneis interfícitur anno post Christum natum sexcentésimo nonagésimo octávo, episcopátus quadragésimo. Ejus relíquiæ primum Trajéctum ad ecclésiam sancti Petri, deínde Leódium delátæ et collocátæ sunt in templo ejus memóriæ dicáto, ubi multis claruére miráculis. Mártyris demum caput anno salútis humánæ milésimo centésimo nonagésimo Rudólphus dux Zaringénsis, Leodiénsis Epíscopus, Fribúrgum Brisgóviæ honorífice tránstulit.

\R Stola jucunditátis índuit eum Dóminus: \red{*} Et corónam pulchritúdinis pósuit super caput ejus.
\V Cibávit illum Dóminus pane vitæ et intelléctus: et aqua sapiéntiæ salutáris potávit illum. \red{E}t. \red{G}lória Patri. \red{E}t.

\nocturn{In III Nocturno}
\scriptura{Léctio sancti Evangélii secúndum Joánnem}
\lectiocap{Lectio vii}{Cap. 10, 11-16}
\y{I}{n} illo témpore: Dixit Jesus Pharis\'æis: Ego sum pastor bonus. Bonus pastor ánimam suam dat pro óvibus suis. Et réliqua.

\scriptura{Homilía sancti Joánnis Chrysóstomi}
\ex{Homilia 59 in Joann.}

\Y{M}{agnum} quiddam, dilectíssimi, magnum, inquam, est ecclésiæ prælátio, et quæ multa índiget sapiéntia et fortitúdine, qualem Christus propósuit: ut ánimam pro óvibus ponámus et numquam illas deserámus: ut lupo generóse resistámus. Hæc enim inter pastórem et mercenárium est differéntia. Alter própriæ contémptis óvibus: alter sua contémpta óvium semper salúti invígilat. Pastóris ergo exémplo demonstráto, deceptóres duos méminit, furem mactántem et rapiéntem oves, et mercenárium permitténtem neque defendéntem commíssas.

\R Coróna áurea super caput ejus,
\red{*} Expréssa signo sanctitátis, glória honóris, et opus fortitúdinis.
\V Quóniam prævenísti eum in benedictiónibus dulcédinis, posuísti in cápite ejus corónam de lápide pretióso.
\red{E}xpréssa.

{\setstretch{0.99}
\lectio{Lectio viii}
\y{Q}{uod} superióri témpore Ezéchiel his verbis isectátus est: Væ pastŕibus Israël: nonne pascébant semetípsos? nonne greges pascúntur a pastóribus? Sed illi contrárium faciébant, quod máximæ malítiæ genus est et plurimórum causa malórum. Idcírco inquit: Neque quod abjéctum erat, reducébant, neque quod períerat, quærébant, neque confráctum alligábant, neque infírmum consolidábant: quóniam se, non gregem, pascébant. Idem et Paulus áliis verbis signíficat: Omnes, quæ sua sunt quærunt, non quæ Jesu Christi.

\R Hic est vere Martyr, qui pro Christi nómine sánguinem suum fudit:
\red{*} Qui minas júdicum non tímuit, nec terrénæ dignitátis glóriam quæsívit, sed ad cæléstia regna pervénit.
\V. Justum dedúxit Dóminus per vias rectas, et osténdit illi regnum Dei.
\red{Q}ui.
\red{G}lória Patri.
\red{Q}ui.

\rubric{In Feriis Quatuor Temporum IX Lectio de Homilia Feriæ et fit ejus Com. in Laudibus; alias}

\pro{Pro Impressione sacrorum Stigmatum S. Francisci:}
\lectio{Lectio ix}
\yb{F}{rancíscus} singulári privilégio retroáctis s\'æculis non concésso insignítus appáruit, cum sacris Stigmátibus decorátus descéndit de monte, secum ferens Crucifíxi effígiem, non in tábulis lapídeis vel lígneis manu figurátam artíficis, sed in cárneis membris descríptam dígito Dei vivi. Quóniam sacraméntum Regis Seráphicus vir abscóndere bonum esse óptime norat, secréti regális cónscjus, signácula illa sacra pro viribus occultábat. Verum, quia Dei est ad glóriam suam magna reveláre quæ facit, Dóminus ipse, qui signácula illa secréte imprésserat, mirácula quædam apérte per ipsa monstrávit, ut illórum occúlta et mira vis Stígmatum manifésta patéret claritáte signórum. Porro rem admirábilem, ac tantópere testátam, atque in pontifíciis diplomátibus, præcípuis láudibus et favóribus exaltátam, Benedíctus Papa undécimus anniversária solemnitáte celebrári vóluit; quam póst ea Paulus quintus Póntifex máximus, ut corda fidélium in Christi crucifixi accenderéntur amórem, ad univérsam Ecclésiam propagávit.

\Te

\hora{Ad Laudes}

\V Justus ut palma florébit.
\R Sicut cedrus Líbani multiplicábitur.

}

\B Qui odit ánimam suam in hoc mundo, in vitam ætérnam custódit eam.

\pars{Oratio}
\y{D}{eus}, qui nos beáti Lambérti Mártyris tui atque Pontíficis ánnua solemnitáte lætíficas: concćede propítius; ut cujus natalítia cólimus, de ejúsdem étiam protectióne gaudéamus. Per Dóminum.

\rubric{Et fit Com. Impressionis sacrorum Stigmatum S. Francisci:}

\A Euge serve bone * et fidélis, quia in pauca fuísti fidélis, supra multa te contítuam, dicit Dóminus.

\V Justum dedúxit Dóminus per vias rectas.
\R Et osténdit illi regnum Dei.

\pars{Oratio}

\yb{D}ómine Jesu Christe, qui, frigescénte mundo, ad inflammándum corda nostra tui amóris igne, in carne beatíssimi Francísci passiónis tuæ sacra Stígmata renovásti: concéde propítius; ut ejus méritis et précibus crucem iúgiter ferámus, et dignos fructus pæniténtiæ faciámus: Qui vivis.

\rubric{Vesperæ a Capitulo de sequenti, Com. præcedentis:}

\A Qui vult veníre post me, ábneget semetípsum, et tollat crucem suam, et sequátur me.

\V Justus ut palma \red{et Oratio} Deus, qui, \red{ut supra.}

\rubric{Deinde Com. Impressionis sacrorum Stigmatum S. Francisci:}

\A Hic vir despíciens mundum et terréna triúmphans, divítias cælo cóndidit ore manu.

\V Signásti, Dómine, servum tuum Francíscum.
\R Signis redemptiónis nostræ.

\red{Oratio} Dómine Jesu, \red{ut supra.}

%\end{multicols}

%\ornamentv

%\pagebreak

%\begin{multicols}

\privdie{Eadem die 17 Septembris\\\rubric{In urbe Friburgensi:}}{Ss. Lamberti Episcopi et Alexandri}{Martyrum, Patronum urbis principalium}{Duplex I classis cum Octava communi}
% change header here
\chead{\trim{Die 17 Septembris}{Ss. Lamberti et Alexandri}}

{\setstretch{0.96}

\rubric{Omnia de Communi plurim. Mm., præter sequentia:}

\V Lætámini in Dómino et exsultáte, justi.
\R Et gloriámini, omnes recti corde.

\red{Ad Magnif. Ant.} Istórum~est enim * regnum cælórum, qui con\-tem\-psérunt vitam mundi, et pervenérunt ad pr\'æmia regni, et la\-vérunt stolas suas in sánguine Agni.

\vspace{-0.3em}
\pars{Oratio}
\vspace{-0.3em}
\y{D}{a} nobis, qu\'æsumus Dómine, beatórum Mártyrum tu\-ó\-rum Lambérti et Alexándri solemnitátibus gloriári: ut eórum semper et patrocíniis sublevémur, et fidem cóngrua devotióne sectémur. Per Dóminum.

\rubric{In I Nocturno Lectiones \black{Fratres: Debitóres,} de eodem Communi.}

\nocturn{In II Nocturno}
\lectio{Lectio iv}

}

\Y{L}{ambértus}, Trajecténsis ecclésiæ epíscopus, Apricómitis et Herisplíndis piórum paréntum fílius, præceptóre imprímis usus est sancto Landoáldo archipresbýtero. Deínde beátum Theodárdum epíscopum máximis virtútibus præstántem sequens, eídem anno Christi sexcentésimo quinquagésimo nono in episcopátu succéssit. Clodov\'æo Francórum regi ob suas exímias virtútes usque ádeo accéptus fuit, ut et ipsum præcípuo honóre ac veneratióne prosequerétur, ipsúmque et patrem et apostólicum virum appelláre et nomináre consuéverit. Childeríco item regi non minus gratus fuit, quippe qui ejus pectus Spíritus Sancti templum et linguam veritátis instruméntum díceret.
Verum Childeríco rege occíso Ebroínus tyránnus, qui major domus sub Theodoríco s\'æpjus Gálliam perturbáverat, sanctum antístitem véluti justítiæ public\'æque libertátis defensórem episcopátu éxpulit; qui Stabulénse monastérium petens ibi támdiu fuit, quoad Ebroíno crudéliter interfécto a Pipíno de Herstéllo, Pipíni Brabántiæ ducis ex Begga fília nepóte, summa totíus pópuli gratulatióne ab exsílio revocátur.

{\setstretch{0.98}
\R Sancti tui, Dómine, mirábile consecúti sunt iter, serviéntes præcéptis tuis, ut inveniréntur illǽsi in aquis válidis:
\red{*} Terra appáruit árida: et in Mari Rubro via sine impediménto.
\V Quóniam percússit petram, et fluxérunt aquæ, et torréntes inundavérunt.
\red{T}erra.

\lectio{Lectio v}
\y{Q}{ui} cum boni pastóris offícium exséquitur, et príncipem ob adultérium cum Alpáide zelo religiónis arguísset, noctu ante altáre sanctórum Cosmæ et Damiáni Mártyrum oratióni vacans, Dodónis adúlteræ fratris ópera, innocens a nocéntibus, una cum Petro et Andoléto consanguíneis interfícitur anno post Christum natum sexcentésimo nonagésimo octávo, episcopátus quadragésimo. Ejus relíquiæ primum Trajéctum ad ecclésiam sancti Petri, deínde Leódium delátæ et collocátæ sunt in templo ejus memóriæ dicáto, ubi multis claruére miráculis. Mártyris demum caput anno salútis humánæ milésimo centésimo nonagésimo Rudólphus dux Zaringénsis, Leodiénsis Epíscopus, Fribúrgum Brisgóviæ honorífice tránstulit.

\R Vérbera carníficum non timuérunt sancti Dei, moriéntes pro Christi nómine: \red{*} Ut herédes fíerent in domo Dómini.
\V Tradidérunt córpora sua propter Deum ad supplícia. \red{U}t.

\lectio{Lectio vi}
\Y{A}{lexánder} Romæ gloriósa martýrii palma coronátus est, ejúsque corpus in subterráneis illíus urbis cœmetériis ex more depósitum, ínibi per plura s\'æcula quiévit. Occurrénte vero jubilári anno millésimo sexcentésimo quinquagésimo, de mandáto Innocéntii Papæ décimi, sacræ athlétæ Christi relíquiæ e cryprta sunt etráctæ et per Martínum Cardinálem ejúsdem Summi Pontíficis Vicárium Generálem dono tráditæ Friburgénsis cœnóbii præpósito, qui tunc generáli Ordinis sui capítulo Romæ intervénerat, cum facultáte illas ex Urbe transferéndi, et in quavis ecclésia vel sacéllo públicæ fidélium veneratióni exponéndi. Pia cum sárcina domum redux vir religiósus sacras sancti Alexándri relíquias rite recógnitas magistrátui et cívibus pátriæ urbis dono dedit. Qui sacrum corpus, véstibus auro intéxtis, margarítis gemmísque insértis indútum et lóculo pretióso inclúsum, solémni supplicatióne in ecclésiam majórem beátæ Maríæ Vírgini in cælos assúmptæ sacram asportárunt, ibíque in sacéllo sanctæ Annæ dicáto collocavérunt, atque ipsum sanctum Alexándrum Mártyrem cum beatíssima Virgine et sancto Lambérto Epíscopo et Mártyre inter præcípuos civitátis Friburgénsis Patrónos adscripsérunt. Cujus translatiónis die décima séptima Septémbris anni millésimi sexcentésimi quinquagésimi primi peráctæ memória deínceps ad nostra usque témpora quotánnis solémniter celebrátur.

\R Tamquam aurum in fornáce probávit eléctos Dóminus, et quasi holocáusti hóstiam accépit illos: et in témpore erit respéctus illórum: \red{*} Quóniam donum et pax est eléctis Dei.
\V Qui confídunt in illum, intélligent veritátem: et fidéles in dilectióne acquiéscent illi. \red{Q}uóniam. \red{G}lória Patri. \red{Q}uóniam.

\nocturn{In III Nocturno}
\scriptura{Léctio sancti Evangélii secúndum Lucam}
\lectiocap{Lectio vii}{Cap. 21, 9-19}
\y{I}{n} illo témpore: Dixit Jesus discípulis suis: Cum audiéritis pr\'ælia, et seditiónes, nolíte terréri: opórtet primum hæc fíeri, sed nondum statim finis. Et réliqua.

\scriptura{De Homilía sancti Gregórii Papæ}
\ex{Homilia 35 in Evangelia}
\Y{O}{mnia}, quæ ad usum vitæ accépimus, ad usum convértimus culpæ: sed cuncta, quæ ad usum pravitátis infléximus, ad usum nobis vertúntur ultiónis. Tranqullitátem quippe humánæ pacis ad usum vértimus vanæ securitátis: peregrinatiónem terræ pro habitatióne diléximus pátriæ: salútem córporum redégimus in usum vitiórum: ipsa seréna blandiménta áëris ad amórem nobis servíre coégimus terrénæ delectatiónis. Jure ígitur restat, ut simul nos ómnia fériant, quæ simul ómnia vítiis nostris male subácta serviébant: ut quot prius in mundo incólumes habúimus gáudia, tot de ipso póstmodum cogámur sentíre torménta.

\R Propter testaméntum Dó\-mi\-ni, et leges patérnas, Sancti Dei perstitérunt in amóre fraternitátis:
\red{*} Quia unus fuit semper spíritus in eis, et una fides.
\V Ecce quam bonum et quam jucúndum habitáre fratres in unum. \red{Q}uia.

\lectio{Lectio viii}
\y{Q}{uia} autem cuncta hæc non de iniustítia ferréntis sunt, sed de mérito mundi patiéntis, facta pravórum hóminum præmittúntur, cum dícitur: Sed ante hæc ómnia incípient vobis manus suas injícere. Ac si apérte dicat: Prius corda hóminum, et post eleménta turbántur: ut cum rerum ordo confúnditur, ex qua jam retributióne véniat, demonstréur. Nam quamvis finis mundi ex ipso suo órdine péndeat, perversióres tamen quosque invéniens, quia digne ruínis illíus opprimántur, onnotéscit.

\R Sancti mei, qui in carne pósiti, certámen habuístis: \red{*} Mercédem labóris ego reddam vobis.
\V Veníte, benedícti Patris mei, percípite regnum.
\red{M}ercédem. \red{G}loria Patri. \red{M}ercédem.

\rubric{In Feriis Quatuor Temporum IX Lectio de Homilia Feriæ, et fit ejus Com. in Laudibus; alias}

\lectio{Lectio ix}
\y{S}{ed} audítis tot terróribus, turbári póterant infirmórum corda; atque ídeo consolátio adjúngitur, cum prótinus subinfértur: Pónite ergo in córdibus vestris, non præmeditári quemádmodum respondeátis. Ac si apérte membris suis infirmántibus dicat: Nolíte terréri, nolíte pertiméscere: vos ad certámen accéditis, sed ego pr\'ælior: vos verba éditis, sed ego sum qui loquor.

\Te

\hora{Ad Laudes}

\V Exsultábunt Sancti in glória.
\R Lætabúntur in cubílibus suis.

\red{Ad Bened. Ant.} Vestri capílli cápitis * omnes numeráti sunt : nolíte timére : multis passéribus melióres estis vos.

\pars{Oratio}
\y{D}{a} nobis, qu\'æsumus Dómine, beatórum Mártyrum tu\-ó\-rum Lambérti et Alexándri solemnitátibus gloriári: ut eórum semper et patrocíniis sublevémur, et fidem cóngrua devotióne sectémur. Per Dóminum.

\rubric{Et fit Com. Impressionis sacrorum Stigmatum S. Francisci:}

\A Euge serve bone * et fidélis, quia in pauca fuísti fidélis, supra multa te contítuam, dicit Dóminus.

\V Justum dedúxit Dóminus per vias rectas.
\R Et osténdit illi regnum Dei.

\pars{Oratio}
\yb{D}ómine Jesu Christe, qui, frigescénte mundo, ad inflammándum corda nostra tui amóris igne, in carne beatíssimi Francísci passiónis tuæ sacra Stígmata renovásti: concéde propítius; ut ejus méritis et précibus crucem iúgiter ferámus, et dignos fructus pæniténtiæ faciámus: Qui vivis.

\hora{In II Vesperis}

\V Exsultábunt Sancti in glória.
\R Lætabúntur in cubílibus suis.

\red{Ad Magnif. Ant.} Gaudent in cælis * ánimæ Sanctórum, qui Christi vestígia sunt secúti: et quia pro ejus amóre sánguinem suum fudérunt, ídeo cum Christo exsúltant sine fine.

\red{Oratio} Da nobis, \red{ut supra.}

\rubric{Et fit Com. sequentis:}

\A Mórtuus sum, et vita mea est abscóndita cum Christo in Deo.

\V Amávit eum Dóminus, et ornávit eum.
\R Stolam glóriæ índuit eum.

\pars{Oratio}
\yb{D}{eus}, qui ad unigénitum Fílium tuum exaltátum a terra ómnia tráhere disposuísti: pérfice propítius; ut méritis et exémplo seráphici Confessóris tui Joséphi supra terrénas omnes cupiditátes eleváti, ad eum perveníre mereámur: \red{(}Qui tecum vivit et regnat.\red{)}

\rubric{Deinde Com. Impressionis sacrorum Stigmatum S. Francisci:}

\A Hic vir despíciens mundum et terréna triúmphans, divítias cælo cóndidit ore manu.

\V Signásti, Dómine, servum tuum Francíscum.
\R Signis redemptiónis nostræ.

\red{Oratio} Dómine Jesu, \red{ut supra.}

}


\die{\rubric{In urbe Friburgensi:}}{Pro Commemoratione Octavæ}{}{}
\chead{\trim{}{Pro Commemoratione Octavæ}}
{\setstretch{0.98}
\hora{In I Vesperis}

\A Istórum est enim regnum cælórum, qui contempsérunt vitam mundi, et pervenérunt ad pr\'æmia regni, et lavérunt stolas suas in sánguine Agni.

\V Lætámini in Dómino, et exsultáte justi.
\R Et gloriámini omnes recti corde.

\pars{Oratio}
\y{D}{a} nobis, qu\'æsumus Dómine, beatórum Mártyrum tu\-ó\-rum Lambérti et Alexándri solemnitátibus gloriári: ut eórum semper et patrocíniis sublevémur, et fidem cóngrua devotióne sectémur. Per Dóminum.

\hora{Ad Laudes}
\A Vestri capílli cápitis omnes numeráti sunt : nolíte timére : multis passéribus melióres estis vos.

\V Exsultábunt Sancti in glória.
\R Lætabúntur in cubílibus suis.

\hora{In II Vesperis}

\Ant Gaudent in cælis ánimæ Sanctórum, qui Christi vestígia sunt secúti: et quia pro ejus amóre sánguinem suum fudérunt, ídeo cum Christo exsúltant sine fine.

\V Exsultábunt Sancti.

}

\columnbreak
\dieii{22 Septembris}{S. Landelini}{Martyris}{Duplex}

{\setstretch{0.97}
\M Iste Sanctus pro lege Dei sui certávit usque ad mortem, et a verbis impiórum non tímuit: fundátus enim erat supra firmam petram.

\V Glória et honóre coronásti eum, Dómine.
\R Et constituísti eum super ópera mánuum tuárum.

\pars{Oratio}
\y{D}{eus}, pro cujus nómine beátus Landelínus martyr occúbuit: præsta qu\'æsumus; ut ipsíus exémplo ánimas nostras in patiéntia possidéntes, coronári cum ipso mereámur in cælis. \red{(}Per Dóminum.\red{)}

\rubric{Et fit Com. S. Thomæ de Villanova Ep. et Conf.:}

\Ant Dispérsit dedit paupéribus: justítia ejus manet in sǽculum sǽculi.

\V Amávit eum Dóminus, et ornávit eum.
\R Stolam glóriæ índuit eum.

\pars{Oratio}
\yb{D}{eus}, qui beátum Thomam Pontíficem insígnis in páuperes misericórdiæ virtúte decorásti: quǽsumus; ut, ejus intercessióne, in omnes, qui te deprecántur, divítias misericórdiæ tuæ benígnus effúndas. Per Dóminum.

}

\rubric{In I Nocturno, si dicendæ non sint de Scriptura occurenti, Lectiones \black{Fratres: Debitóres,} de Communi plurimorum Martyrum cum Responsoriis de Communi unius Martyris.}

\nocturn{In II Nocturno}
\lectio{Lectio iv}
\Y{L}{andelínus}, claris in Hibérnia paréntibus natus, vitæ sanctióris desidério mítio s\'æculi séptimi e pátria sponte discéssit, ut manéret in solitúdine. Peragrátis provínciis multis, in Ortenávia, incúlta tum et ínvia ac latrocíniis multis dumtáxat famósa, súbstitit, ubi prope rívulum, cui Undízio nomen est, tugúrium sibi constrúxit, in quo oratióni ac rerum divinárum contemplatióni vacans innócuam tranquillámque degit vitam.

\R Honéstum fecit illum Dóminus, et custodívit eum ab inimícis, et a seductóribus tutávit illum: \red{*} Et dedit illi claritátem ætérnam.
\V Descendítque cum illo in fóveam, et in vínculis non derelíquit eum. \red{E}t.

\lectio{Lectio v}
\y{I}{n} hac erémo Landelínus juxta Apóstoli verba, quæ sursum sunt sapiébat, non quæ super terram: mórtuus enim erat, et vita illíus abscóndita cum Christo in Deo. Pax hæc brevi turbáta fuit: síquidem beátus solitárius a quodam venatóre feras fortuíto persequénte detéctus fuit, qui sub vili hábitu latrónem delitéscere súspicans, convítiis primo sanctum óbruit. Postrémo nec précibus nec frontis modéstia, aut ánimi patiéntia viri Dei commótus, homo ferox furóre pércitus eúndem pro inimíco deprecántem ímpia manu trucidávit.

\R Desidérium ánimæ ejus tribuísti ei Dómine, \red{*} Et voluntáte labiórum ejus non fraudásti eum.
\V Quóniam prævenísti eum in benedictiónibus dulcédinis: posuísti in cápite ejus corónam de lápide pretióso. \red{E}t.

\lectio{Lectio vi}
\y{P}{retiósa} in conspéctu Dómini mors mártyris ubi primum pópulis circumjacéntibus innótuit, pii fidéles ejus corpus reverénter terræ mandárunt. Sepúlchrum vero ejus tanta cœpit frequentári devotióne, ut Widegérnus Argentinénsis episcopus inítio s\'æculi octávi ibídem ecclésiam fundáverit et cellam monachórum institúerit, quæ póstea ab Ettóne ejus successóre cum relíquiis beáti Landelíni ex antíquo loco in propínquum transláta, multum aucta, latis dotáta pr\'ædiis et ex ipsíus nómine monastérium Ettoniánum dicta est. In loco autem, ubi sanctus occubuísse asseverátur, limpidíssimi manant fontes, curándis morbis público testimónio máxime salutáres.

\R Stola jucunditátis índuit eum Dóminus: \red{*} Et corónam pulchritúdinis pósuit super caput ejus.
\V Cibávit illum Dóminus pane vitæ et intelléctus: et aqua sapiéntiæ salutáris potávit illum. \red{E}t. \red{G}lória Patri. \red{E}t.

\nocturn{In III Nocturno}
\scriptura{Léctio sancti Evangélii secúndum Matth\'æum}
\lectiocap{Lectio vii}{Cap. 16, 24-27}
\y{I}{n} illo témpore: Dixit Jesus discípulis suis: Si quis vult post me veníre, ábneget semetípsum, et tollat crucem suam, et sequátur me. Et réliqua.

\scriptura{Homilía sancti Augustíni Epíscopi}
\ex{Sermo 96 de Script.}
\Y{P}{rima} hóminis perdítio fuit amor sui: si enim se non amáret et Deum sibi præpóneret, Deo esse semper súbditus vellet; non autem converterétur ad negligéndam voluntátem illíus et ad faciéndam voluntátem suam. Hoc est enim amáre se, velle fácere voluntátem suam. Præpóne his voluntátem Dei: disce amáre te, non amándo te. Quid est: neget se? non præsúmat de se, séntiat se hóminem et respíciat dictum prophéticum: Maledíctus omnis, qui spem suam ponit in hómine. Subdúcat se sibi, sed non deórsum versus: subdúcat se sibi, ut h\'æreat Deo. Quidquid boni habet, illi tríbuat, a quo factus est: quidquid mali habet, ipse sibi fecit. Deus, quod in illo malum est, non fecit. Perdat quod fecit, qui inde defécit.

\R Coróna áurea super caput ejus,
\red{*} Expréssa signo sanctitátis, glória honóris, et opus fortitúdinis.
\V Quóniam prævenísti eum in benedictiónibus dulcédinis, posuísti in cápite ejus corónam de lápide pretióso.
\red{E}xpréssa.

\lectio{Lectio viii}
\y{Q}{uid} est: tollat crucem suam? Ferat quidquid moléstum est; sic me sequátur. Cum enim c\'œperit me móribus et præcéptis meis sequi, multos habébit contradictóres, multos habébit prhibitóres, multos habébit dissuasóres, et hoc de ipsis quasi comítibus Christi: cum Christo ambulábant, qui cæcos clamáre prohibébant. Sive ergo minas, sive blandiménta, sive quáslibet prohibitiónes, si sequi vis, in crucem verte, tólera, porta; noli succúmbere. Vidéntur his verbis Dómini exhortáta martýria; si persecútio est, nonne pro Christi debent cuncta contémni? Amátur mundus; sed præponátur a quo factus est mundus; sed suávior est a quo factus est mundus. Malus est mundus; et bonus est a quo factus est mundus.

\R Hic est vere Martyr, qui pro Christi nómine sánguinem suum fudit:
\red{*} Qui minas júdicum non tímuit, nec terrénæ dignitátis glóriam quæsívit, sed ad cæléstia regna pervénit.
\V. Justum dedúxit Dóminus per vias rectas, et osténdit illi regnum Dei.
\red{Q}ui.
\red{G}lória Patri.
\red{Q}ui.

\rubric{In Feriis Quattuor Temporum IX Lectio de Homilia Feriæ et fit ejus Com. in Laudibus; alias}

\pro{Pro S. Thoma de Villanova:}
\lectio{Lectio ix}
\yb{T}{homas}, in óppido Fontispláni Toletánæ diœcéseos in Hispánia natus, a bonis paréntibus, ineúnte vita, pietátem et singulárem in páuperes misericórdiam accépit, cujus toto vitæ suæ decúrsu præclára dedit exémpla. Puer enim, ut nudos operíret, própriis véstibus non semel seípsum éxuit; adoléscens, post patris óbitum univérsam hereditátem egénis virgínibus aléndis dicávit. Cum, theologíæ cursu confécto, divíno instínctu, Eremitárum sancti Augustíni institútum ampléxus esset, virtútibus ómnibus ornátus, caritáte præsértim erga páuperes et peccatóres excélluit, quos e vitiórum cœno edúcere satégit. Præcípue vero ejus misericórdia elúxit, cum, ad regéndam Valentínam ecclésiam ex obediéntia vocátus, vigilantíssimi pastóris vices explévit et amplos ecclésiæ réditus in egénos dispérsit, ne léctulo quidem sibi relícto. Obdormívit in Dómino, sexto idus septémbris, annos natus octo supra sexagínta.

\red{T}e Deum laudámus.

\hora{Ad Laudes}

\V Justus ut palma florébit.
\R Sicut cedrus Líbani multiplicábitur.

\B Qui odit ánimam suam in hoc mundo, in vitam ætérnam custódit eam.

\pars{Oratio}
\y{D}{eus}, pro cujus nómine beátus Landelínus martyr occúbuit: præsta qu\'æsumus; ut ipsíus exémplo ánimas nostras in patiéntia possidéntes, coronári cum ipso mereámur in cælis. \red{(}Per Dóminum.\red{)}

\rubric{Et fit Com. S. Thomæ de Villanova:}

\A Eleemósynas illíus enarrábit omnis ecclésia sanctórum.

\V Justum dedúxit Dóminus per vias rectas.
\R Et osténdit illi regnum Dei.

\pars{Oratio}
\yb{D}{eus}, qui beátum Thomam Pontíficem insígnis in páuperes misericórdiæ virtúte decorásti: quǽsumus; ut, ejus intercessióne, in omnes, qui te deprecántur, divítias misericórdiæ tuæ benígnus effúndas. \red{(}Per Dóminum.\red{)}

\rubric{Et post Commemorationem Feriæ in Quatuor Temporibus (præmissa in urbe Friburgensi Commemoratione Octavæ, ut supra) fit Com. Ss. Mauritii et Socior. Mm.:}

\Ant Vestri capílli cápitis omnes numeráti sunt: nolíte timére: multis passéribus melióres estis vos.

\V Exsultábunt Sancti in glória.
\R Lætabúntur in cubílibus suis.

\pars{Oratio}
\yb{A}{nnue}, quǽsumus, omnípotens Deus: ut sanctórum Mártyrum tuórum Maurítii et Sociórum ejus nos lætíficet festíva solémnitas; ut, quorum suffrágiis nítimur, eórum natalíciis gloriémur. Per Dóminum.

\hora{In II Vesperis}

\M Qui vult veníre post me, * ábneget semetípsum, et tollat crucem suam, et sequátur me.

\V Justus ut \red{et Oratio} Deus, pro cujus, \red{ut supra.}

\rubric{Et fit Com. sequentis:}

\Ant Iste Sanctus pro lege Dei sui certávit usque ad mortem, et a verbis impiórum non tímuit: fundátus enim erat supra firmam petram.

\V Glória et honóre coronásti eum, Dómine.
\R Et constituísti eum super ópera mánuum tuárum.

\pars{Oratio} 
\yb{D}{eus}, qui nos beáti Lini Mártyris tui atque Pontíficis ánnua solemnitáte lætíficas: concéde propítius; ut cujus natalítia cólimus, de ejúsdem étiam protectióne gaudeámus. \red{(}Per Dóminum.\red{)}

\rubric{Deinde Com. S. Thomæ de Villanova:}

\Ant Dispérsit dedit paupéribus: justítia ejus manet in sǽculum sǽculi.

\V Justum dedúxit \red{et Oratio} Deus, qui beátum, \red{ut supra.}

\rubric{Postea (præmissa in urbe Friburgensi Com. Octavæ, ut supra) Com. S. Theclæ Virg. et Mart.:}

\Ant Veni, Sponsa Christi, áccipe corónam, quam tibi Dóminus præparávit in ætérnum.

\V Spécie tua et pulchritúdine tua.
\R Inténde, próspere procéde, et regna.

\pars{Oratio}
\yb{D}{a}, quǽsumus, omnípotens Deus: ut, qui beátæ Theclæ Vírginis et Mártyris tuæ natalícia cólimus; et ánnua solemnitáte lætémur, et tantæ fídei proficiámus exémplo. Per Dóminum.



\die{Die 24 Septembris \\\rubric{In urbe Friburgensi:}}{In Octava Ss. Lamberti \red{Episcopi} et Alexandri}{Martyrum}{Duplex majus}
\chead{\trim{Die 24 Septembris}{In Octava Ss. Lamberti et Alexandri}}

\rubric{Antiphonæ et Psalmi ad omnes Horas et Versus Nocturnorum de occurrenti Hebdomadæ die, ut in Psalterio; reliqua ut in Festo præter Lectiones, quæ in I Nocturno dicuntur de Scriptura occurrenti cum suis Responsoriis de Tempore, in II et III assignantur propriæ.}

\V Lætámini in Dómino, et exsultáte justi.
\R Et gloriámini omnes recti corde.

\M Istórum est enim * regnum cælórum, qui contempsérunt vitam mundi, et pervenérunt ad pr\'æmia regni, et lavérunt stolas suas in sánguine Agni.

\pars{Oratio}
\y{D}{a} nobis, qu\'æsumus Dómine, beatórum Mártyrum tu\-ó\-rum Lambérti et Alexándri solemnitátibus gloriári: ut eórum semper et patrocíniis sublevémur, et fidem cóngrua devotióne sectémur. Per Dóminum.

\rubric{Et fit Com. præcedentis:}

\A Qui vult veníre post me, ábneget semetípsum, et tollat crucem suam, et sequátur me.

\V Justus ut palma florébit.
\R Sicut cedrus Líbani multiplicábitur.

\red{Oratio} Deus, qui nos, \red{ut supra.}

\rubric{Deinde Commemoratio B. Mariæ Virg. de Mercede (duplex majus):}
\A Sancta María, succúre míseris, juva pusillánimes, réfove flébiles, ora pro pópulo, intérveni pro clero, intercéde pro devóto femíneo sexu: séntiant omnes tuum juvámen, quicúmque célebrant tuam sanctam festivitátem.

\VRBMVi

\pars{Oratio}
\yb{D}{eus}, qui per gloriosíssimam Fílii tui Matrem, ad liberándos Christi fidéles a potestáte paganórum nova Ecclésiam tuam prole amplificáre dignátus es: præsta, quǽsumus; ut, quam pie venerámur tanti operis institutrícem, ejus páriter méritis et intercessióne, a peccátis ómnibus et captivitáte dǽmonis liberémur. Per eúndem Dóminum.

\rubric{In II Nocturno Lectiones: \black{Sermo sancti Joánnis Chrysóstomi: Nemo est, qui nésciat,} ut in Communi plurim. Mm. 2 loco.}

\nocturn{In III Nocturno}
\scriptura{Lectio sancti Evangélii secúndum Lucam}
\lectiocap{Lectio vii}{Cap. 21,9-19}
\y{I}{n} illo témpore: Dixit Jesus discípulis suis: Cum audiéritis pr\'ælia, et seditiónes, nolíte terréri: opórtet primum hæc fíeri, sed nondum statim finis. Et réliqua.

\scriptura{Homilía sancti Augustíni Epíscopi}
\ex{Sermo 14 Lib. 50 Homil.}
\Y{D}{óminus} noster Jesus Christus téstibus idest martýribus suis, pro humána fragilitáte sollícitis, ne forte eum confiténdo atque moriéndo perírent, magnam securitátem dedit, dicens: Capíllus de cápite vestro non períbit. Times ergo ne péreas, cujus capíllus non períbit? Si sic custodiúntur supérflua tua, in quanta securitáte est ánima tua? Non perit capíllus, qui cum tondétur, non sentis; et perit ánima, per quam sentis? Sane multa dura eos passúros esse prædíxit, ut prædicéndo fáceret paratióres, diceréntque illi: Parátum cor meum.

\RVMmvii

\lectio{Lectio viii}
\y{Q}{uid} est: Parátum cor meum; nisi, Paráta volúntas mea? Parátam ergo habent mártyres voluntátem in martýrio, sed præparátur volúntas a Dómino. Illis autem in malis duris atque ásperis futúris commémorans adjécit: In patiéntia vestra possidébitis ánimas vestras. In vestra patiéntia, inquit: non enim esset patiéntia tua, si non ibi esset et volúntas tua. In vestra patiéntia: sed unde vestra? Nostrum est, quod a nobis habétur; nostrum est, et quod nobis donátur.

\RVMmviii

{\setstretch{0.98}
\pro{Pro B. Maria V. de Mercede:}
\lectio{Lectio ix}
\yb{Q}{uo} témpore innumeri fidéles sub immani servitute, cum periculo christianæ fidei abjurandæ, amittendæ salútis ætérnæ, infeliciter detinebántur, beáta Virgo María, sancto Petro Nolasco, beato Raymundo de Péñafort et Jacobo Aragoniæ regi noctu appárens, acceptíssimum sibi ac unigénito suo Fílio fore dixit, si suum in honórem instituerétur ordo religiosórum, quibus cura incúmberet captívos e Turcárum tyránnide liberándi. Quare, collátis inter se consíliis et consentiéntibus ánimis, in honórem ejúsdem Vírginis Matris órdinem institúere aggréssi sunt, sub invocatióne sanctæ Maríæ de Mercéde redemptiónis captivórum, sodálibus quarto voto astríctis manéndi in pignus sub paganórum potestáte, si pro Christianórum liberatióne opus esset. Quibus rex ipse arma sua régia in péctore deferre concessit, et a Gregório nono illud tam præcellentis caritátis institútum confirmándum curávit. Ut autem tanti beneficii et institutiónis, débitæ Deo et Vírgini Matri referántur gratiæ, Sedes apostólica hanc peculiárem festivitátem celebrári indúlsit.

\Te

\hora{Ad Laudes}
\V Exsultábunt Sancti in glória.
\R Lætabúntur in cubílibus suis.

\B Vestri capílli cápitis * omnes numeráti sunt : nolíte timére : multis passéribus melióres estis vos.

\pars{Oratio}
\y{D}{a} nobis, qu\'æsumus Dómine, beatórum Mártyrum tu\-ó\-rum Lambérti et Alexándri solemnitátibus gloriári: ut eórum semper et patrocíniis sublevémur, et fidem cóngrua devotióne sectémur. Per Dóminum.

\rubric{Et fit Com. B. Mariæ Virg. de Mercede:}

\AiiBMV

\VRBMVii

\pars{Oratio}
\yb{D}{eus}, qui per gloriosíssimam Fílii tui Matrem, ad liberándos Christi fidéles a potestáte paganórum nova Ecclésiam tuam prole amplificáre dignátus es: præsta, quǽsumus; ut, quam pie venerámur tanti operis institutrícem, ejus páriter méritis et intercessióne, a peccátis ómnibus et captivitáte dǽmonis liberémur. Per eúndem Dóminum.

\hora{In II Vesperis}

\M Gaudent in cælis * ánimæ Sanctórum, qui Christi vestígia sunt secúti: et quia pro ejus amóre sánguinem suum fudérunt, ídeo cum Christo exsúltant sine fine.

\V Exsultábunt Sancti.

\rubric{Et fit Com. B. Mariæ Virg. de Mercede:}

\AiiiBMV

\VRBMViii

\red{Oratio} Deus, qui per, \red{ut supra.}

}

\die{Die 28 Septembris}{S. Liobæ}{Virginis}{\\Duplex}

{\setstretch{0.97}

\VRVi

\MiV

\pars{Oratio}
\y{E}{xáudi} nos, Deus, salutáris noster: ut, sicut de beátæ Líobæ Vírginis tuæ festivitáte gaudémus, ita piæ devotiónis erudiámur afféctu. Per Dóminum nostrum.

}

\rubric{Et fit Com. præcedentis:}

\AiiiMm

\VRMmiii

\pars{Oratio}
\yb{P}{ræsta}, qu\'æsumus omnípotens Deus: ut, qui sanctórum Mártyrum tuórum Cosmæ et Damiáni natalítia cólimus, a cunctis malis imminéntibus, eórum intercessiónibus liberémur.

\rubric{Deinde Com. S. Wenceslai Ducis, Mart. (semiduplex):}

\AiM

\VRMi

\pars{Oratio}
\yb{D}{eus}, qui beátum Wencesláum per martýrii palmam a terréno principátu ad cæléstem glóriam transtulísti: ejus précibus nos ab omni adversitáte custódi; et ejúsdem tríbue gaudére consórtio. Per Dóminum.

\nocturn{In II Nocturno}
\lectio{Lectio iv}
\Y{L}{íoba}, sancti Bonifátii Epíscopi et Mártyris Germanórum Apóstoli consanguínea, in Anglia paréntibus ætáte jam provéctis nata, Tettæ insígnis sanctimóniæ vírgini educánda ac divínis stúdiis imbuénda tráditur. Eo témpore, quo cæléstis vitæ stúdiis Líoba in monastério Winburnénsi in Anglia flóruit, sanctus Bonifátius, in Germánia fidem Christi felicíssimo cum evéntu pr\'ædicans, legátos cum epístolis diréxit in pátriam ad Tettam Abbatíssam déprecans, ut ad auxílium legatiónis a Sede apostólica sibi injúnctæ, transmítteret hanc Líobam vírginem, fama sanctitátis, doctrínæ et virtútum per longínqua terrárum spátia celebrátam; quam cum advenísset, vir Dei summa cum veneratióne suscépit, ac spiriálem vírginum matrem esse decrévit; quaprópter illam monastérii in loco, qui vocátur Episcópium ad Túberam, a se erécti Abbatíssam constítuit, ubi non parvus ancillárum Dei númerus colléctus est, quæ ad exémplar beátæ magístræ cæléstis disciplínæ stúdiis instituebántur.

\RVViv

\lectio{Lectio v}
\y{E}{rat} Líoba magnárum virtútum fémina, quæ omni fervóre conténdit, ut semetípsam Deo irreprehensíbilem exhibéret et cunctis sibi obtemperántibus in verbo et conversatióne forma salútis exsísteret. Omnibus sine personárum acceptióne se affábilem ac benévolam exhíbuit. Erat aspéctu angélica, sermóne jucúnda, ingénio clara, consílio próvida, spe patientíssima, caritáte profúsa. Et cum lætam semper fáciem præférret, numquam hilaritáte nímia resolúta est in risum. Cibos autem et potum, cum áliis summa humanitát exhibéret, ipsa parcíssime sumpsit. Lectióni sacræ tanta diligéntia incúbuit, ut nisi oratióni vacáret, aut aliménto vel somno corpúsculum refíceret, numquam divína página de mánibus ejus abscéderet. Plúribus miráculis Dóminus ancíllæ suæ sanctitátem vóluit esse testatam; nam moniális vírginis diri críminis accusátæ innocéntiam sancta Líoba prodigióse manifestári a Deo ímpetrat, exstínguit inćendium, sedat tempestátem invocatióne sanctíssimæ Dei Genitrícis, cœnóbii sui mónacham hæmorrhoidáli morbo laborántem et morti jam próximam sanat.

\RVVv

\lectio{Lectio vi}
\y{B}{eáta} autem virgo Líoba non terram hereditáre cúpiens sed cælum, ad finem christiánæ perfectiónis ferventíssime tendébat. Quare disseminabátur de ea semper et ubíque fama laudábilis, et odóre sanctitátis ac sapiéntiæ cunctórum in se traxit amórem. Erat enim ómnibus, qui eam nóverant, et ipsis quoque régibus honorábilis. Nam Pipínus rex Francórum et fílii ejus omni eam veneratióne coluérunt; ac præcípue Cárolus, qui religiósam Dei vírginem ad se frequénter invitátam magna cum reveréntia suscépit et dignis munéribus honorávit. Sic et regína Hildegárdis puro eam venerabátur afféctu voluítque, ut assídue secum manéret; sed Líoba ut venéni póculum palatínum detestabátur tumúltum. Et quia erat in Scriptúris eruditíssima atque in consílio prudentíssima, nonnúmquam et ipsi Epíscopi verbum vitæ cum ea conferébat et institúta ecclesiástica sæpe tractábant. Tandem beáta Líoba, cum plúrimas ánimas sanctitátis suæ exémplis Dómino lucráta in omni ferme Germánia signórum glória et doctrínæ fama coruscáret, accépto Córporis et Sánguinis Christi viático, in pace quiévit quarto Kaléndas Octóbris, anno salútis septingentésimo septuagésimo nono. Corpus vero illíus honorífice ad Fuldénse cœnóbium translátum, in Ecclésia sancti Epíscopi et Mártyris Bonifácii dignis cum láudibus tumulátum, ibídem miráculis cláruit.

\RVVvi

\rubric{In III Nocturno Lectiones de Homilia in Evingel. \black{Símile erit,} ut in Communi Virg. 1 loco.}

\columnbreak
{\setstretch{0.98}
\pro{Pro S. Wenceslao:}
\lectio{Lectio ix}
\yb{W}{encesláus}, Bohémiæ dux, Wratisláo patre Christiáno, Drahomíra matre gentíli natus, ab ávia Ludmílla fémina sanctíssima pie educátus, omni virtútum génere insígnis, summo stúdio virginitátem per omnem vitam servávit illibátam. Mater, per nefáriam Ludmíllæ necem regni administratiónem assecúta, ímpie cum iunióre fílio Bolesláo vivens, concitávit in se prócerum indignatiónem; quare, ímpii regíminis pertǽsi, utriúsque excússo jugo, Wencesláum in urbe Pragénsi regem salutárunt. Qui regnum pietáte magis quam império gubernávit, in egénis et afflíctis sublevándis solers et assíduus. Summa religióne sacerdótes venerátus, suis ipse mánibus tríticum serébat et vinum exprimébat, in Missæ sacrifício adhibénda. Cum vero ab imperatóre régiis insígnibus decorátus fuísset, ab ímpio fratre, matris suásu, orans in ecclésia interféctus est. Sanguis ejus per paríetes aspérsus adhuc conspícitur.

\Te

\hora{Ad Laudes}

\VRVii 

\BV 

\pars{Oratio}
\y{E}{xáudi} nos, Deus, salutáris noster: ut, sicut de beátæ Líobæ Vírginis tuæ festivitáte gaudémus, ita piæ devotiónis erudiámur afféctu. Per Dóminum nostrum.

}

\rubric{Et fit Com. S. Wenceslai:}

\AiiM

\VRMii 

\pars{Oratio}
\yb{D}{eus}, qui beátum Wencesláum per martýrii palmam a terréno principátu ad cæléstem glóriam transtulísti: ejus précibus nos ab omni adversitáte custódi; et ejúsdem tríbue gaudére consórtio. Per Dóminum.

\rubric{Vesperæ de sequenti, Com. præcedentis tantum:}

\AiiiV 

\V Diffúsa est. \red{et Oratio} Exáudi nos, \red{ut supra.}



\mensii{Festa Octobris}
\dieii{Die 15 Octobris\\\rubric{In Ecclesiis Cathedrali et non consecratis:}}{S. Teresiæ}{Virginis}{Duplex}
\chead{\trim{Die 15 Octobris}{S. Teresiæ}}

\rubric{Ut in Breviario hac die.}

\rubric{Vesperæ a Capitulo de sequenti, Com. præcedentis, ut infra.}

%\end{multicols}

%\ornamentv

%\pagebreak

%\begin{multicols}

\privdie{Eadem die 15 Octobris\\\rubric{In Ecclesiis consecratis:}}{In Dedicatione propriæ Ecclesiæ}{}{Duplex I classis cum Octava communi}
\chead{\trim{Die 15 Octobris}{In Dedicatione propriæ Ecclesiæ}}

\rubric{Omnia de Communi Dedicationis Ecclesiæ.}

\rubric{Ad Laudes tantum fit Com. S. Teresiæ Virg.:}

\AiiV 

\VRVii 

\pars{Oratio}
\yb{E}{xáudi} nos, Deus, salutáris noster: ut, sicut de beátæ Terésiæ Vírginis tuæ festivitáte gaudémus; ita cæléstis ejus doctrínæ pábulo nutriámur, et piæ devotiónis erudiámur afféctu. Per Dóminum nostrum.

\rubric{In II Vesperis Com. sequentis ut infra.}

\columnbreak
\dieii{\rubric{In Ecclesiis consecratis:}}{Pro Commemoratione Octavæ}{}{}
\chead{\trim{}{Pro Commemoratione Octavæ}}

\hora{In I Vesperis}

\A Sanctificávit Dóminus tabernáculum suum: quia hæc est domus Dei, in qua invocábitur nomen ejus, de quo scriptum est: Et erit nomen meum ibi, dicit Dóminus.

\V Hæc est domus Dómini fírmiter ædificáta.
\R Bene fundáta est supra firmam petram.

\pars{Oratio}
\y{D}{eus}, qui nobis per síngulos annos hujus sancti templi tui consecratiónis réparas diem, et sacris semper mystériis repræséntas incólumes: exáudi preces pópuli tui, et præsta; ut quisquis hoc templum benefícia petitúrus ingréditur, cuncta se impetrásse lætétur. Per Dóminum.

\hora{Ad Laudes}

\A Zach\'æe, festínans descénde, quia hódie in domo tua opórtet me manére: at ille festínans descéndit, et suscépit illum gaudens in domum suam. Hódie huic dómui salus a Deo facta est, allelúja.

\V Hæc est domus Dómini fírmiter ædificáta.
\R Bene fundáta est supra firmam petram.

\hora{In II Vesperis}

\A O quam metuéndus est locus iste: vere non est hic áljud, nisi domus Dei, et porta cæli.

\V Domum tuam, Dómine, decet sanctitúdo.
\R In longitúdinem diérum.



\die{Die 16 Octobris}{S. Galli}{Abbatis}{\\Duplex}

\VRCi 

\MiC

\pars{Oratio}
\y{I}{ntercéssio} nos, qu\'æsumus Dómine, beáti Galli Abbátis comméndet: ut quod nostris méritis non valémus, ejus patrocínio assequámur. Per Dóminum.

\rubric{In Ecclesiis Cathedrali et non consecratis fit Com. S. Teresiæ Virg.}

\AiiiV 

\VRViii 

\red{Oratio} Exáudi nos, \red{ut supra.}

\rubric{Et fit Com. S. Hedwigis.}

\AiNV

\VRNVi

\yb{D}{eus}, qui beátam Hedwígem a sǽculi pompa ad húmilem tuæ crucis sequélam toto corde transíre docuísti: concéde; ut ejus méritis et exémplo discámus peritúras mundi calcáre delícias, et in ampléxu tuæ crucis ómnia nobis adversántia superáre: Qui vivis.

\columnbreak
\nocturn{In II Nocturno}
\lectio{Lectio iv}
\Y{G}{allus} Abbas nobílibus apud Scotos natálibus ortus, in Hibérnia in monastério Benchor adolescéntulus vitam páuperum cum elegísset, humilitáti et obediéntiæ perfect\'æque subjectióni semper stúduit. Deínde vero cum non morum tantum, sed et ingénii laude ac sanctárum Scripturárum sciéntia usúque appríme floréret, sacérdos Dei unctus, prædicándi Evangélii causa abbátem Columbánum secútus in Británniam índeque Gálliam trajécit, ubi Sigebérti regis voluntáte Luxovißensem solitúdinem dum incóleret, ad Christi fidem complúres, multos étiam ad monásticæ vitæ institúta permóvit. Sed cum regis Theodoríci concubinátum redargúere Columbánus abbas non desísteret, Brunichíldis regínæ instínctu Luxóvio pulsus, accépta a Theodebérto Austrasianórum rege potestáte, in Alamánnia ad lacum Turicínum primum, deínde vero apud Brigántium óppidum, reconciliáto sanctæ Auréliæ templo, consédit.

\RVCiv

\lectio{Lectio v}
\y{V}{erum} quóniam Gallus idololatríæ scelus ubique convellébat, fana ac simulácra dissipábat, gentílium ódia concitávit. Quibus étiam cedéndum ratus Columbánus, in Itáliam ipse ad Agilúlphum Longobardórum regem proféctus, Gallum fébribus deténtum una cum Magnoáldo et Theodóro mónachis in Alamánnia relíquit. Gallus vero sexagésimum quintum círciter annum agens mox ut conváluit, erémum cum jam dictis discípulis ingréssus, triduáno eam jejúnio, oratiónibus ac lácrimis consecrávit, et monastérium cóndidit. In quo plures fratres monásticis disciplínis divinarúmque Scripturárum intelligéntia imbuit, gentes circumpósitas Jesu Christi fidem sectári dócuit. Corpus suum inédia, frígore, cilício dénique et caténa castigávit, ita secréto ut de caténa ejus \'ærea cilicióve, quibus ad hujúsmodi exercítia utebátur, nec discípuli ejus, dum víveret, rescírent. Fridibúrgam Alamánniæ ducis fíliam, Sigebérti regis sponsam, a sævíssimo dæmónio cum liberásset, ducis opulentíssima dona ómnia commínuit, et apud Arbónam in páuperes erogávit. Episcopátum Constantiénsem, Abbatíam Luxoviénsem suscípere recusávit. Columbánum in Itália vitæ múnere sanctíssime perfúnctum, per visiónem in Alamánnia præcognóvit.

\RVCv 

\columnbreak
\lectio{Lectio vi}
\y{D}{énique} cum pópulos multos ab idolórum cultúra avocásset, ipsóque die sancti Michaélis Archángeli apud Arbónam post Missárum solémnia Evangélium prædicásset, febre in jam dicto óppido íterum corréptus, circa annum Dómini sexcentésimum quadragésimum sextum, Theodóri Papæ primi quartum, ætátis suæ anno nonagésimo quinto, die vero décimo séptimo ante Kaléndas Novémbris inter discipulórum manus exspirávit. Sed cum illic nullo modo potuísset humári, indomitórum equórum ductu, ardéntibus contínuo ad fèretrum céreis, in sacram erémum perlátus, in oratório suo Joánnis epíscopi Constantiénsis, sui olim discípuli, fratrúmque mánibus sepúltus, miráculis cláruit.

\RVCvi

\nocturn{In III Nocturno}
\scriptura{Léctio sancti Evangélii secúndum Matth\'æum}
\lectiocap{Lectio vii}{Cap. 19, 27-29}
\y{I}{n} illo témpore: Dixit Petrus ad Jesum: Ecce nos relíquimus ómnia, et secúti sumus te: quid ergo erit nobis? Et réliqua.

\scriptura{Homilía sancti Petri Damiáni}
\ex{Sermo de S. Benedicto}
\Y{Q}{uid} ait: Ecce nos relíquimus ómnia, et secúti sumus te? Solémne verbum, magna promíssio, opus sanctum dignum benedictióne, relínquere ómnia et sequi Christum. Hæc sunt verba voluntáriæ persuasória paupertátis, quæ monastéria genuérunt, quæ claustra mónachis, anachorétis silvas coiósius replevérunt. Hæc enim sunt, de quibus psallit eccĺésia: Propter verba labiórum tuórum ego custodívi vias duras, perceptúra réquiem pro labóre, pro paupertáte divítias, pro tribulatióne mercédem.

\RVCvii

\lectio{Lectio viii}
\y{E}{cce}, ínquit, nos relíquimus ómnia, non solum facultátes mundi, sed et ánimi quoque cupiditátes. Neque enim relinquit ómnia, qui retínuit vel se ípsum; immo vero nihil prodest, sine seípso cétera reliquísse: quandóquidem nullum áljud onus est grávjus hómini quam homo ipse. Quis enim tyránnus crudélio, quæ s\'ævior potéstas hómini, quam hóminis ipsíus volúntas? Numquam sub ea reqiéscere, numquam sedére licet: et quo ámplius te ad obediéndum sibi nóverit fatigári, eo magis urget, ínstigat et ónerat pietátis ímmemor, misericórdiam nésciens. 

\RVCviii

\pro{Pro S. Hedwigi:}
\lectio{Lectio ix}
\yb{H}{edwígis}, régiis clara natálibus, sanctæ Elísabeth, fíliæ regis Hungáriæ matértera, duodénnis Henríco Polóniæ ducis núptui trádita, prolem inde suscéptam in Dei timóre erudívit. Ut autem commódjus Deo vacáret, ex pari voto et consénsu unánimi, ad separatiónem thori virum indúxit. Quo defúncto, ipsa in monastério Trebnicénsi Cisterciénsem sumpsit hábitum; in eóque, contemplatióni inténta, divínis Offíciis et Missárum solémniis assídua assístere in delíciis hábuit. Sublimióribus florens virtútibus, arctíssima pœniténtia, consiliórum gravitáte animíque candóre, in exímium religiósæ pietátis evásit exémplum. Omnibus se ultro subícere atque vilióra múnia subíre, paupéribus étiam flexu genu ministráre, leprosórum pedes ablúere et osculári, ipsi familiáre erat. Mira ejus patiéntia animíque constántia, præcípue in morte Henríci, ducis Silésiæ, sui fílii in bello a Tártaris cæsi, enítuit. Miraculórum glória, præcípue post óbitum, claram, Clemens quartus Sanctórum número eam adscrípsit.
%\y{H}{oc} enim próprium própria volúntas habet, ut quo obédiens obediéntio fúerit sibi, eo ámplius cum crudelióribus vínculis internéctat. Sola dilígitur, cum sola digna sit ódio, princípium iniquitátis, mortis infúsio, destrúctio magna virtútum. Veníte ergo, qui laborátis et oneráti estis, ad ónerum levatórem, et ei voce et ópere respondéte: Ecce nos relíquimus ómnia, et secúti sumus te.

\Te 

{\setstretch{1.01}
\hora{Ad Laudes}

\VRCii

\BC

\pars{Oratio}
\y{I}{ntercéssio} nos, qu\'æsumus Dómine, beáti Galli Abbátis comméndet: ut quod nostris méritis non valémus, ejus patrocínio assequámur. Per Dóminum.

\rubric{Et fit Com. S. Hedwigis.}

\AiiNV

\VRNVii

\yb{D}{eus}, qui beátam Hedwígem a sǽculi pompa ad húmilem tuæ crucis sequélam toto corde transíre docuísti: concéde; ut ejus méritis et exémplo discámus peritúras mundi calcáre delícias, et in ampléxu tuæ crucis ómnia nobis adversántia superáre: Qui vivis.

\rubric{In Ecclesiis consecratis fit Com. Octavæ, ut supra.}

\rubric{Vesperæ a Capitulo de sequenti, Com. præcedentis et S. Hedwigis:}
\AiiiC

\V Justum dedúxit \red{et Oratio} Intercéssio nos, \red{ut supra.}

\AiiiNV

\V Diffúsa est \red{et Oratio} Deus, qui, \red{ut supra.}

\rubric{In Ecclesiis consecratis fit Com. Octavæ, ut supra.}

}


\die{Die 20 Octobris}{S. Wendelini}{Abbatis}{Duplex}

{\setstretch{0.97}
\MiC 

\VRCi 

\pars{Oratio}
\y{I}{ntercéssio} nos, qu\'æsumus Dómine, beáti Wendelíni Abbátis comméndet: ut quod nóstris méritis non valémus, ejus patrocínio assequámur. Per Dóminum.

\rubric{Et fit Com. præcendtis:}

\AiiiC 

\VRCiii 

\pars{Oratio}

\yb{D}{eus}, qui beátum Petrum Confessórem tuum admirábilis pœniténtiæ et altíssimæ contemplatiónis múnere illustráre dignátus es: da nobis, quǽsumus; ut, ejus suffragántibus méritis, carne mortificáti, facílius cæléstia capiámus. \red{(}Per Dóminum.\red{)}

\rubric{Deinde Com. S. Joannis Cantii Conf.}

\AiiC 

\VRCiv

\pars{Oratio}

\yb{D}{a}, quǽsumus, omnípotens Deus: ut, sancti Joánnis Confessóris exémplo in sciéntia Sanctórum proficiéntes, atque áliis misericórdiam exhibéntes; ejus méritis, indulgéntiam apud te consequámur. Per Dóminum.

\rubric{Postea in Ecclesiis consecratis Commemoratio Octavæ, ut supra.}

}

\nocturn{In II Nocturno}
\lectio{Lectio iv}
\Y{W}{endelínus} (qui et Wendalínus) natióne Scotus et régia (ut ferunt) orígine procreátus, cum humilitátem christiánam in abjécta vita natálium splendóri præférret, patris digréssus solo ad illam Trevirórum regiónem pervénit, quam nunc sancti Wendelíni (ut ab illo nuncupátur) óppidum tenet: ubi sese viro nóbili in servitútem últimam subúlci addíxit, in qua cúltui divíno sacr\'æque religiónis exercítiis inter pervigília, inediámque et assíduas noctu diúque prodúctas obsecratiónes, magna sanctitátis fama vacávit, cum omne, quod a labóre suppetébat ótium, ad Dei laudes precandíque stúdium convérteret.

\RVCiv 

\lectio{Lectio v}
\y{V}{erum} postrémo perfectióris vitæ sub aliórum disciplína transigéndæ amóre inflammátus, in Tholojénse cœnóbium concéssit, ubi latiórem nactus in aliórum consórtio et sub aliórum império humilitátis exercéndæ campum, quæ seórsim meditátus erat abjectiónis exémpla, in societáte aliórum usu exercitióque confirmávit. In superióribus Dei persónam auctoritatémque revéritus, mandáta imperántium ceu divínam voluntátem amplectebátur; æquálibus ad ómnia caritátis offícia promptíssimus; nulli non demísse insérviens prætérquam córpori próprio, quod insígni abstinéntia macerábat.

\RVCv 

\lectio{Lectio vi}
\y{H}{is} aliísque virtútum exercítiis ad mortem suque déditus, Abbátis múnere monastério invítus præfícitur, in quo non domésticis tantum, verum et extérnis insígni sanctitátis poinióne venerándus appáruit. Quo factum est, at postquam duodécimo Kaléndas Novémbris e vivis sublátus est, in ejus memóriam et honórem óppidum, hodiédum suo nómine insignítum, cum illústri Basílica eo loco exstruerétur, quo corpus Sancti Nonis Júlii translátum est. Quod insígni olim miraculórum frequéntia, et pópuli undequáque religiónis grátia eódem confluéntis multitúdine celebrátum fuit.

\RVCvi 

\nocturn{In III Nocturno}
\scriptura{Léctio sancti Evangélii secúndum Matth\'æum}
\lectiocap{Lectio vii}{Cap. 19, 27-29}
\y{I}{n} illo témpore: Dixit Petrus ad Jesum: Ecce nos relíquimus ómnia, et secúti sumus te: quid ergo erit nobis? Et réliqua.

\scriptura{Homilía sancti Bedæ Venerábilis Presbýteri}
\ex{Homilia in natali S. Benedicti}
\Y{D}{uo} sunt órdines electórum in judício futúri: unus judicántium cum Dómino, de quibus hoc loco mémorat, qui reliquérunt ómnia et secúti sunt illum: álius judicandórum a Dómino, qui non quidem ómnia sua páriter reliquérunt, sed de his tamen quæ habébant, quotidiánas dare eleemósynas Christi paupéribus curábant; unde et auditúri sunt in judício: Veníte, benedícti Patris mei, possidéte præparátum vobis regnum a constitutióne mundi. Esurívi enim, et dedístis mihi manducáre: sitívi, et dedístis mihi bíbere.

\RVCvii 

{\setstretch{0.98}
\lectio{Lectio viii}
\y{S}{ed} et reprobórum duos ibi futúros órdines Dómino narránte compérimus: unum eórum, qui fídei christiánæ mystériis initiáti, ópera fídei exercére contémnunt, quibus in judício testátur: Discédite a me maledícti in ignem ætérnum, qui præparátus est diábolo et ángelis ejus: esurívi enim, et non dedístis mihi manducáre. Alterum eórum, qui fidem et mystéria Christi vel numquam suscepére, vel suscéptam per apostásiam deseruére, de quibus dicit: Qui autem non credit, jam judicátus est; quia non credit in nómine unigéniti Fílii Dei.

\RVCviii 

\pro{Pro S. Joanne Cantio:}
\lectio{Lectio ix}
}

\yb{J}{oánnes}, in óppido Kenty Cracoviénsis diœcésis, a quo Cántii cognómen duxit, Stanisláo et Anna piis et honéstis paréntibus natus, morum suavitáte et innocéntia, ab ipsa infántia spem fecit máximæ virtútis. Sacérdos factus, stúdium auxit christiánæ perfectiónis. Ilkusiénsem paróchjam annis áliquot egrégie administrávit. Quidquid témporis a stúdio supérerat, partim salúti proximórum sacris præsértim conciónibus curándæ, partim oratióni dábat. Quater ad Apostolórum límina, pedes et viária onústus sárcina, venit, tum ut Sedem apostólicam honoráret, tum, ut sui (sic enim aiébat) purgatórii pœnas expósita illic cotídie peccatórum vénia redímeret. Virginálem pudicítiam vigilantíssime custodívit, et ante óbitum per annos trigínta círciter et quinque ab esu cárnium abstínuit. Prídie Nativitátis Christi volávit in cælum. A Cleménte Papa décimo tértio fastis Sanctórum adscríptus, inter primários Polóniæ ac Lithuániæ patrónos cólitur.

\Te 

\hora{Ad Laudes}

\VRCii 

\BC 

\pars{Oratio}

\y{I}{ntercéssio} nos, qu\'æsumus Dómine, beáti Wendelíni Abbátis comméndet: ut quod nóstris méritis non valémus, ejus patrocínio assequámur. Per Dóminum.

\rubric{Et fit Com. S. Joannis Cantii:}

\AiC 

\VRCi 

\pars{Oratio}

\yb{D}{a}, quǽsumus, omnípotens Deus: ut, sancti Joánnis Confessóris exémplo in sciéntia Sanctórum proficiéntes, atque áliis misericórdiam exhibéntes; ejus méritis, indulgéntiam apud te consequámur. Per Dóminum.

\rubric{Deinde in Ecclesiis consecratis Com. Octavæ, ut supra.}

\hora{In II Vesperis}

\MiiC

\V Justum dedúxit, \red{et Oratio} Intercéssio nos, \red{ut supra.}

\rubric{Et fit, in Ecclesiis Cathedrali et non consecratis, Com. sequentis:}

{\setstretch{0.98}
\AiC 

\VRCi 

\pars{Oratio}

\yb{D}{eus}, qui nos beáti Hilariónis Abbátis ánnua solemnitáte lætíficas: concéde propítius; ut, cujus natalítia cólimus, étiam actiónes imitémur. \red{(}Per Dóminum.\red{)}

\rubric{Deinde Com. S. Joannis Cantii:}

\AiiC 

\VRCiv 

\red{Oratio} Da, qu\'æsumus, \red{ut supra.}

\rubric{Postea Com. Ss. Ursulæ et Sociarum Vv. et Mm.:}

\AiV 

\VRVi 

\yb{D}{a} nobis, quǽsumus, Dómine, Deus noster, sanctárum Vírginum et Mártyrum tuárum Ursulæ et Sociárum ejus palmas incessábili devotióne venerári: ut, quas digna mente non póssumus celebráre, humílibus saltem frequentémus obséquiis. Per Dóminum.

}

\halfline

\rubric{In Ecclesiis consecratis Com. sequentis diei infra Octavam (Ant. et \V e I Vesperis) ut supra.}

\rubric{Deinde Com. S. Joannis Cantii:}

\AiiC 

\VRCi 

\red{Oratio} Da, qu\'æsumus, \red{ut supra.}

\rubric{Postea Com. S. Hilarionis Abbatis:}

\AiC 

\VRCiv 

\red{Oratio} Deus, qui nos, \red{ut supra.}

\rubric{Denique Ss. Ursulæ et Sociarum Vv. et Mm., ut supra.}



\die{Die 21 Octobris}{De VII die infra Octav. Dedicationis propriæ Ecclesiæ}{}{Semiduplex}

{\setstretch{0.98}
\rubric{Omnia de Psalterio et Communi, juxta Rubricas.}

\pro{Pro S. Hilarione:}
\lectio{Lectio ix}
\yb{H}{ilarion}, ortus Tabáthæ in Palæstina ex paréntibus infidelibus, Alexandríam missus studiórum causa, ibi morum et ingenii laude flóruit; ac, Jesu Christi suscepta religióne, in fide et caritate mirabíliter profecit. Frequens enim erat in ecclésia, assiduus in jejúnio et oratióne; omnes voluptátum illécebras et terrenárum rerum cupiditates contemnebat. Cum autem Antonii nomen in Ægypto celeberrimum esset, ejus vidéndi studio in solitúdinem conténdit; apud quem duobus mensibus omnem ejus vitæ ratiónem didicit. Domum reversus, mórtuis paréntibus, facultates suas paupéribus dilargitus est; necdum quintum décimum annum egréssus, rediit in solitúdinem, ubi, exstructa exígua casa, quæ vix ipsum cáperet, humi cubábat. Nec vero saccum, quo semel amíctus est, umquam aut lavit aut mutávit, cum supervacáneum esse diceret, mundítias in cilício quærere. In sanctárum Litterárum lectióne et meditatióne multus erat. Paucas ficus et succum herbárum ad victum adhibebat; nec illis ante solis occásum vescebátur. Continéntia et humilitate fuit incredibili. Quibus aliisque virtútibus varias horribilesque tentatiónes diaboli superávit, et innumerábiles dæmones in multis orbis terræ partibus ex hóminum corpóribus ejécit. Qui, octogesimum annum agens, multis ædificátis monastériis, et clarus miraculis, in morbum incidit; cujus vi cum extremo pene spíritu conflictarétur, dicebat: Egredere, quid times? egredere, ánima mea, quid dubitas? septuagínta prope annis servísti Christo, et mortem times? Quibus in verbis spíritum exhalávit.

\Te 

\rubric{Ad Laudes fit Com. S. Hilarionis:}

\AiiC 

\VRCii 

\yb{I}{ntercéssio} nos, quǽsumus, Dómine, beáti Hilariónis Abbátis comméndet: ut, quod nostris méritis non valémus, ejus patrocínio assequámur. \red{(}Per Dóminum.\red{)}

\rubric{Deinde Com. Ss. Ursulæ et Soc.}

\AiiV 

\VRVii 

\pars{Oratio}
\yb{D}{a} nobis, quǽsumus, Dómine, Deus noster, sanctárum Vírginum et Mártyrum tuárum Ursulæ et Sociárum ejus palmas incessábili devotióne venerári: ut, quas digna mente non póssumus celebráre, humílibus saltem frequentémus obséquiis. Per Dóminum.

\rubric{Vesperæ de sequenti.}

\rubric{In Ecclesiis Cathedrali et non consecratis S. Hilarionis Abbatis simplex, ut in Breviario.}

}



\die{Die 22 Octobris}{In Octava Dedicationis propriæ Ecclesiæ}{}{Duplex majus}

\rubric{Omnia de Psalterio et Communi, juxta Rubricas.}

\rubric{Quando dies Octava occurrerit in Dominica, Sabbato in I Vesperis Dominicæ, Commemoratio Octavæ facienda est per Ant. et \V e II Vesperis.}

\rubric{Pro utentibus Breviario antiquo Lectiones III Nocturni e Feria IV infra Octavam.}

\rubric{In Eccles. Cathedrali et non consecratis hac die agitur de Feria currenti.}




\mens{Festa Novembris}
\dieii{Die 3 Novembris}{S. Pirmini}{Episcopi et Confessoris}{Duplex}

{\setstretch{0.97}
\rubric{Quodsi hoc Festum occurrerit in Feria secunda, de eo nihil fit, quia celebratur Commemoratio Omnium Fidelium Defunctorum.}

\VRCPi 

\MiCP 

\vspace{-.2em}
\pars{Oratio}
\vspace{-.2em}
\y{D}{eus}, qui in corde sancti Pirmíni Confessóris tui atque Pontíficis ignem tui amóris accendísti: da, ut eódem caritátis ardóre ab omni labe peccáti mundémur. Per Dóminum.

\rubric{Et fit Com. Octavæ Omnium Sanctorum:}

\A Ángeli, Archángeli, Throni et Dominatiónes, Principátus et Potestátes, Virtútes cælórum, Chérubim atque Séraphim, Patriárchæ et Prophétæ, sancti legis Doctóres, Apóstoli, omnes Christi Mártyres, sancti Confessóres, Vírgines Dómini, Anachorítæ, Sanctíque omnes, intercédite pro nobis.

\VRMmi 

\vspace{-.2em}
\pars{Oratio}
\vspace{-.2em}
\yb{O}{mnípotens} sempitérne Deus, qui nos ómnium Sanctórum tuórum mérita sub una tribuísti celebritáte venerári: quǽsumus; ut desiderátam nobis tuæ propitiatiónis abundántiam, multiplicátis intercessóribus, largiáris. Per Dóminum.

}

{\setstretch{0.99}
\nocturn{In II Nocturno}
\lectio{Lectio iv}
\Y{D}{e} pátria, ortu et juveníli ætáte sancti Pirmíni nihíl omníno nóscimus. Pro certo vero scimus sanctum virum a Deo suscitátum fuísse ad instaurándam in plúrimis monastériis collápsam disciplínam, nováque cœnóbia tuéndæ promovend\'æque evangélicæ perfectiónis grátia instituénda. Episcopáli unctióne insignítus, a nóbili quodam viro ad veram fidem propagándam prope Rhenum vocátus, primo límina Apostolórum pétiit, ut a Summo Pontífice ad regiónis illíus pópulos mitterétur. Roma regréssus Gálliæ epíscopos in sýnodum tunc congregátos ádiit, ipsis obténtam verbi divíni prædicándi potestátem nuntiatúrus, et ab eis étiam in eórum diœcésibus fidem et sanctitátem verbo suo promovéndi licéntiam impetratúrus. A quibus benigníssime excéptus, Helvétiam proféctus est, ibíque sicut et in Suévia, Bavária, Francónia et Palatinátu verbum Dei cum máximo succéssu nuntiávit, multósque non solum ad fidem et sanctitátem, sed ad arctam quidem perfectiónis evangélicæ exercitatiónem perdúxit. Ideo favénte semper nóbili viro, qui eum vocáverat, insígne in insula Augia prope Constántiam in fínibus Suéviæ cœnóbium cóndidit, quod deínde Dives-Augia fuit vocátum.

\RVCPiv 

\lectio{Lectio v}
\y{A}{} Theobáldo Suéviæ duce propter amicítiam Cároli Martélli Francórum Príncipis, qua gaudébat, cœnóbio suo pulsus, sibi in gubérnio monastérii sanctum et sapiéntem Heddónem, qui póstea ad pontifíciam sedem Argentinénsem fuit evéctus, suffécit et in Alsátiam proféctus est. Ibi ab Eberhárdo, Adalbérti Alsátiæ ducis fílio, ad quem usque Pirmíni fama pervénerat, vocátus ipso suadénte in domíniis ejus áljud illústre Murbacénse cœnóbium fundávit, quod póstea Eberhárdus líberis orbátus própriis suis bonis consentiéntibus uxóre Emeltrúde et fratre Luitfrído Alsátiæ duce, ditávit.

\RVCPv 

\lectio{Lectio vi}
\y{I}{gne} divínæ caritátis consúmptus, glóriæ Dei et evangélicæ perfectiónis stúdio elátus, nullam sibi dabat réquiem. Transácto in abbatía Murbacénsi anno uno, in sui locum monastério præfécit Románum, et abbatías diœcésis Argentinénsis multásque álias longe latéque reformávit, nullis médiis usus nisi jejúniis aliísque carnis castigatiónibus, oratióne, lácrimis et caritátis misericordi\'æque suávi unctióne. Multa in Gállia, Germánia, Helvétia nova cœnóbia eréxit, inter quæ emínuit monastérium Hornbacénse in diœcési Meténsi, in quod sénio et labóribus fractus demum secéssit. Ibi ínnocens simul ac p\'œnitens pie obdormívit in Dómino, tértio Nonas Novémbris circa annum post Christum natum septingentésimum quinquagésimum quartum.

\RVCPvi 

\nocturn{In III Nocturno}
\scriptura{Léctio sancti Evangélii secúndum Matth\'æum}
\lectiocap{Lectio vii}{Cap. 19, 27-29}
\y{I}{n} illo témpore: Dixit Petrus ad Jesum: Ecce nos relíquimus ómnia, et secúti sumus te: quid ergo erit nobis? Et réliqua.

\scriptura{Homilía sancti Bernárdi Abbátis}
\ex{Decl. de Evang. hujus verbis}
\Y{A}{rbitror} verba lectiónis hujus ea esse, de quibus ad immortálem Sponsum a fínibus terræ clamat Ecclésia: Propter verba labiórum tuórum ego custodívi vias duras. Hæc nempe sunt verba, quæ contémptum mundi in univérso mundo et voluntáriam persuasére homínibus paupertátem. Hæc sunt, quæ mónachis claustra replent, desérta anachorétis. Hæc inquam, sunt verba, quæ Ægýptum spóliant et óptima quæque ejus vasa dirípiunt. Hic sermo vivus et éfficax, convértens ánimas felíci æmulatióne sanctitátis et veritátis promissióne fidéli.

\RVCPvii

\lectio{Lectio viii}
\y{N}{am} et mundus transit et concupiscéntia ejus, et relínquere hæc magis éxpedit quam relínqui. Ecce inquit, relíquimus ómnia, et secúti sumus te; nimírum quia exsultávit ut gigas ad curréndam vjam, nec curréntem sequi póteras onerátus. Sed nec inútilis commutátio pro eo, qui super ómnia est, ómnia reliquísse; nam et simul cum eo donántur ómnia, et ubi apprehénderis eum, erit unus ipse ómnia in ómnibus, qui pro ipso ómnia reliquérunt. Omnia sane díxerim nec tantum possessiónes, sed étiam cupiditátes, et eas máxime.

\RVCPviii

\lectio{Lectio ix}
\y{P}{lus} enim concupiscéntia mundi quam substántia nocet. Et hæc fugiendárum causa divitiárum præcípua est, quod aut vix aut numquam sine amóre váleant possidéri. Limósa síquidem et glutinósa nimis non modo extérior, verum étiam intérior substántia nostra vidétur, et fácile cor humánum ómnibus, quæ frequéntat, adh\'æret. Age ergo qui relínquere univérsa dispónis, te quoque inter relinquénda numeráre meménto. Immo vero máxime et principáliter ábnega temetípsum, si delíberas sequi eum, qui exinanívit propter te semetípsum. Pone gravíssimam sárcinam, pone asináriam molam, terrénam molem; pone illa quinque non hóminum plane juga, sed boum, quæ tibi insipiénter emísti. Alióqui sequi sponsum et veníre ad núptias spirituáles, quinária hac pressus et oppréssus córporis sensualitáte, non póteris; sed et si novíssime véneris et pulsáveris, mínime profécto aperiétur tibi, sed respondébitur de intus, quod non sit de bobus et ásinis ceterísque juméntis insipiéntibus cura Deo.

\Te 

}

\hora{Ad Laudes}

\VRCPii 

\BCP 

\pars{Oratio}
\y{D}{eus}, qui in corde sancti Pirmíni Confessóris tui atque Pontíficis ignem tui amóris accendísti: da, ut eódem caritátis ardóre ab omni labe peccáti mundémur. Per Dóminum.

\rubric{Et fit Com. Octavæ Omnium Sanctorum:}

\A Te gloriósus Apostolórum chorus, te Prophetárum laudábilis númerus, te Mártyrum candidátus laudat exércitus, te omnes Sancti et elécti voce confiténtur unánimes, beáta Trínitas, unus Deus.

\VRMmii 

\pars{Oratio}
\yb{O}{mnípotens} sempitérne Deus, qui nos ómnium Sanctórum tuórum mérita sub una tribuísti celebritáte venerári: quǽsumus; ut desiderátam nobis tuæ propitiatiónis abundántiam, multiplicátis intercessóribus, largiáris. Per Dóminum.

\rubric{Vesperæ a Capitulo de sequenti, Com. præcedentis:}

\AiiiCP 

\V Justum dedúxit, \red{et Oratio} Deus, qui, \red{ut supra.}

\rubric{Deinde Com. Octavæ Omnium Sanctorum:}

\A O quam gloriósum est regnum in quo cum Christo gaudent omnes Sancti, amícti stolis albis sequúntur Agnum quocúmque íerit.

\V Exsultábunt sancti, \red{et Oratio} Omnípotens, \red{ut supra.}



\die{Die 5 Novembris}{Sacrarum Reliquiarum}{quæ~in~Ecclesiis Diœcesis asservantur}{Duplex majus}

\rubric{Capitula, Hymni et reliqua, quæ de Psalterio Feriæ non sunt sumenda, dicuntur de Communi plurimorum Martyrum, præter sequentia.}

\VRMmi

\MiMm 

\columnbreak

\pars{Oratio}
\y{A}{uge} in nobis, Dómine, resurrectiónis fidem, qui in Sanctórum tuórum relíquiis mirabília operáris: et fac nos immortális glóriæ partícipes, cujus in eórum cinéribus pígnora venerámur. Per Dóminum.

\rubric{Et fit Commemoratio præcedentis:}

\AiiiCP 

\VRCPiii

\pars{Oratio}
\yb{E}{cclésiam} tuam, Dómine, sancti Cároli Confessóris tui atque Pontíficis contínua protectióne custódi: ut, sicut illum pastorális sollicitúdo gloriósum réddidit; ita nos ejus intercéssio in tuo semper fáciat amóre fervéntes. \red{(}Per Dóminum.\red{)}

{\setstretch{0.98}
\rubric{Deinde Commemoratio Octavæ Omnium Sanctorum:}

\A O quam gloriósum est regnum in quo cum Christo gaudent omnes Sancti, amícti stolis albis sequúntur Agnum quocúmque íerit.

\VRMmiii 

\pars{Oratio}
\yb{O}{mnípotens} sempitérne Deus, qui nos ómnium Sanctórum tuórum mérita sub una tribuísti celebritáte venerári: quǽsumus; ut desiderátam nobis tuæ propitiatiónis abundántiam, multiplicátis intercessóribus, largiáris. Per Dóminum.

\rubric{In I Nocturno Lectiones de Scriptura occurrenti cum suis Responsoriis de Tempore.}

\nocturn{In II Nocturno}
\scriptura{Ex Tractátu sancti Joánnis Damascéni de fide orthodóxa}
\ex{Liber 4, cap. 15}
\lectio{Lectio iv}
\Y{C}{hristus} Dóminus Sanctórum relíquias velut salutáres fontes pr\'æbuit, ex quibus plúrima ad nos benefícia manant, suavissimúmque unguéntum prófluit. Nec vero quisquam his fidem détrahat. Nam si aqua in desérto ex dura et áspera rupe, atque ex ásini maxílla, ad sedándam Samsónis sitim, Deo ita volénte, prosíliit, cur incredíbile videátur ex Mártyrum relíquiis unguéntum suáve scaturíre? Mínime certe iis, quibus Dei poténtia et honor, quo Sanctos suos áfficit, perspécta sunt et exploráta.

\RVMmiv

\lectio{Lectio v}
\y{I}{n} lege quidem, quisquis mórtuum tetígerat, immúndus censebátur. Hi verum in mortuórum número mínime sunt habéndi. Ex quo enim ille qui ipsa Vita est, et vitæ Auctor, inter mórtuos deputátus est: eos qui cum spe resurrectiónis fidéque in ipsum obdormiérunt, nequáquam mórtuos appellámus. Qui enim mórtuum corpus mirácula édere queat? Quanam ígitur ratióne, horum ópera d\'æmones expellúntur, morbi profligántur, ægróti sanántur, cæci visum recípiunt, leprósi mundántur, tentatiónes discutiúntur ac mæróres, omne dénique datum óptimum, iis qui fide non dúbia póstulant, per eos descéndit a Patre lúminum? Quid labóris non suscípias, ut patrónum nanciscáris, qui te mortáli regi ófferat, et pro te ad eum verba fáciat? An non ígitur ii honorándi, qui totíus géneris humáni patróni sunt, Deóque pro nobis súpplices preces ádhibent?

\RVMmv 

\lectio{Lectio vi}
\y{H}{onorándi} certe: et quidem ita ut in eórum nómine templa Deo exstruámus, dona offerámus, memórias eórum colámus, atque in iis spirituáliter oblectémur: ea útique lætítia, quæ illis arrídeat a quibus invitámur, ne, dum demeréri illos studémus, offendámus pótius et irritémus. Quibus enim rebus Deus cólitur, iísdem servi quoque ipsíus oblectántur: quibus autem Deus offénditur, iísdem étiam mílites offendúntur. Quocírca in psalmis et hymnis et cánticis spirituálibus, in compunctióne quoque, et egenórum miseratióne, quibus et Deus potíssimum cólitur, nos qui fidéles sumus, cólere Sanctos opórtet. Státuas eis, et imágines quæ videántur, erigámus: immo virtútes eórum imitándo hoc consequámur, ut vivæ ipsórum státuæ imaginésque evadámus.

\RVMmvi 


\rubric{In III Nocturno Homilia in Evangelium: \black{Descéndens Jesus,} de Communi plurimorum Martyrum 2 loco.}
}

\hora{Ad Laudes}

\VRMmii 

\BMm

\pars{Oratio}
\y{A}{uge} in nobis, Dómine, resurrectiónis fidem, qui in Sanctórum tuórum relíquiis mirabília operáris: et fac nos immortális glóriæ partícipes, cujus in eórum cinéribus pígnora venerámur. Per Dóminum.

\rubric{Et fit Commemoatio Octavæ Omnium Sanctorum}

\A Te gloriósus Apostolórum chorus, te Prophetárum laudábilis númerus, te Mártyrum candidátus laudat exércitus, te omnes Sancti et elécti voce confiténtur unánimes, beáta Trínitas, unus Deus.

\VRMmi

\pars{Oratio}
\yb{O}{mnípotens} sempitérne Deus, qui nos ómnium Sanctórum tuórum mérita sub una tribuísti celebritáte venerári: quǽsumus; ut desiderátam nobis tuæ propitiatiónis abundántiam, multiplicátis intercessóribus, largiáris. Per Dóminum.

\hora{In II Vesperis}

\VRMmiii 

\MiiMm

\rubric{Deinde Com. Octavæ Omnium Sanctorum:}

\A O quam gloriósum est regnum in quo cum Christo gaudent omnes Sancti, amícti stolis albis sequúntur Agnum quocúmque íerit.

\V Lætámini, \red{et Oratio} Omnípotens, \red{ut supra.}



\privdie{Die 26 Novembris}{S. Conradi}{Episcopi Constant., Conf., Patroni primarii Archid.}{Duplex I classis cum Octava communi}

\rubric{Omnia de Communi Conf. Pont., præter sequentia.}

\VRCPi

\MiCP

\pars{Oratio}
\y{S}{ancti} Pontíficis et Confessóris tui Conrádi solémnia celebrántes, te, Dómine, supplíciter obsecrámus: ut ipsum apud tuam cleméntiam sentiámus habére patrónum, quem nobis tua grátia providísti salútis ætérnæ minístrum. Per Dóminum.

\rubric{In I Nocturno Lectiones \black{Fidélis sermo,} de eodem Communi.}

\nocturn{In II Nocturno}
\lectio{Lectio iv}
\Y{C}{onrádus} Henríci cómitis Altorffénsis fílius Nothíngo Constantiénsi epíscopo puer septénnis pie educándus litterísque erudiéndus tráditur. Quo magístro cum in sciéntia et christiánæ perfectiónis stúdio mirábiles fecísset progréssus, Constantiénsis ecclésiæ præpósitus primum creátur. Deínde Nothíngo defúncto ob doctrínam et morum íntegritátem epíscopus commúnibus suffrágiis est designátus. Quod munus cum diu recusásset, sancti Udalríci epíscopi Augustáni sibi amicíssimi rogátu et précibus victus post triduánum jejúnium suscépit.

\RVCPiv

{\setstretch{0.98}
\lectio{Lectio v}
\y{E}{piscopátum} magna cum laude et sanctitátis opinióne administrávit. Religiónis causa ter ad sepúlchrum Dómini Hierosólymam peregrinátus est. Constántiam plúribus templis et ædifíciis illustrávit. Duódecim páuperes exstrúcto xenodochío aléndos perpétuo curávit. Tribus templis sancti Joánnis, sancti Pauli et sancti Maurítii ædificátis proventúque ánnuo iis assignáto, Canonicórum étiam númerum ádditis reddítibus auxit.

\RVCPv 

\lectio{Lectio vi}
\y{M}{ultis} virtútibus pr\'ædictus máxime oratiónis stúdium et jejúnii observántiam cóluit. Tanta ipsíus in Deum erat fidúcia, ut quodam die festo Paschæ perácta consecratióne aráneam in cálicem delápsam animadvértens, irreveréntiam sacrosáncti sánguinis Christi omnem pénitus vitatúrus, deglutíre non sit véritus; quæ póstea ómnibus vidéntibus discumbénti ex ore, non sine admiratióne spectántium dénuo exívit. Sanctum Gebhárdum adhuc púerum sibi post Gaminólphum in episcopátu successúrum prædíxit, quam prædictiónem póstea evéntus comprobávit. Cumque Constantiénsi ecclésiæ quadragínta duóbus annis sanctíssime præfuísset, virtútum et miraculórum laude conspícuus obdormívit in Dómino Ottóne secúndo imperánte, sexto Kaléndas Decémbris anno nongentésimo septuagésimo quinto. Sepúltus est in ecclésia sancti Maurítii, et a Calíxto Papa secúndo quinto Kaléndas Aprílis Sanctórum número adscríptus.

\RVCPvi

\rubric{In III Nocturno Homilia in Evangel. \black{Vigiláte,} de eodem Communi 2 loco.}

}

\hora{Ad Laudes}

\VRCPii

\BCP

\pars{Oratio}
\y{S}{ancti} Pontíficis et Confessóris tui Conrádi solémnia celebrántes, te, Dómine, supplíciter obsecrámus: ut ipsum apud tuam cleméntiam sentiámus habére patrónum, quem nobis tua grátia providísti salútis ætérnæ minístrum. Per Dóminum.

\rubric{Et fit Com. S. Silvestri Abb. ad Laudes tantum.}

\AiiC 

\VRCi 

\columnbreak
\pars{Oratio}
\yb{C}{lementíssime} Deus, qui sanctum Silvéstrum Abbátem, sǽculi hujus vanitátem in apérto túmulo pie meditántem, ad erémum vocáre, et præcláris vitæ méritis decoráre dignátus es: te súpplices exorámus; ut, ejus exémplo terréna despiciéntes, tui consórtio perfruámur ætérno. Per Dóminum.

\hora{In II Vesperis}

\VRCPiii 

\MiiCP

\rubric{Infra Octavam Antiphonæ et Psalmi ad omnes Horas et Versus Nocturnorum de occurenti Hebdomadæ die, ut in Psalterio; reliqua ut in Festo præter Lectiones, quæ in I Nocturno dicuntur de Scriptura occurenti cum suis Responsoriis de Tempore, in II et III singulis diebus assignantur propriæ.}


\columnbreak
\dieii{}{Pro Commemoratione Octavæ S. Conradi}{}{}
\hora{In I Vesperis}

\AiCP 

\VRCPi 

\pars{Oratio}
\y{S}{ancti} Pontíficis et Confessóris tui Conrádi solémnia celebrántes, te, Dómine, supplíciter obsecrámus: ut ipsum apud tuam cleméntiam sentiámus habére patrónum, quem nobis tua grátia providísti salútis ætérnæ minístrum. Per Dóminum.

\hora{Ad Laudes}

\AiiCP

\VRCPii 

 
\hora{In II Vesperis}

\AiiiCP

\V Justum dedúxit.


\die{Die 27 Novembris}{De II die infra Octavam S. Conradi}{Episcopi et Confessoris}{Semiduplex}
\nocturn{In II Nocturno}
\scriptura{Sermo sancti Ambrósii Epíscopi}
\ex{Serm. de Deposit. S. Eusebii. Int. opp. S. Ambrosii (in Append.).}
\lectio{Lectio iv}
\Y{D}{epositiónem} sancti Conrádi hódie celebrámus. Quid est deposítio? Non illa útique, quæ sepeliéndis in terra memrórum relíquiis clericórum mánibus procurátur; seu illa, qua homo vínculis carnálibus absolútus, liber itúrus ad cælum, terrénum corpus expónit. Ipsa plane est deposítio, in qua concupíscere abjícimus, cessámus delínquere, peccáre desínimus: et totum quidquid salúti est grave, quasi abjécta sárcina óneris expónimus: nam ídeo hæc dies pro celebritáte máxima procurátur, quia vere est summa festívitas mórtuum esse vítiis, soli vívere justítiæ.

\RVCPiv 

\lectio{Lectio v}
\y{U}{nde} et depositiónis ipsa dies natális dícitur, quod delictórum cárcere liberáti libertáti náscimur Salvatóris. Sed videámus, sanctus Conrádus cum qua glória ad hanc diem depositiónis advénerit. Voláre non potest, nisi quod purum, leve atque subtíle est, cujus nec sincéritas intemperántia retardátur nec alácritas nec velócitas mole gravátur. Gravári autem volátum dico non tam mole membrórum, quam delictórum: unde puto étiam in ipsis ávibus ídeo velócjus colúmbam pene præ ómnibus volitáre, quod alacritátem innocéntia comitétur.

\RVCPv 

\lectio{Lectio vi}
\y{D}{énique} sanctus David cum puritáte mentis voláre concupísceret, non alteríus animántis nisi colúmbæ optávit alas dicens: Quis dabit mihi pennas sicut colúmbæ, et volábo et requiéscam? Intelligébat enim, quod ad altióra facílius penetrátur simplicitáte mentis quam levitáte pennárum. Volávit ergo hac die sanctus Conrádus. Æstimémus jam innocéntiam de volátu, et puritátem ejus de ipsa elatióne judicémus; tamquam colúmba enim in domo Dei degens, assúmpsit spirituáles pennas et requiévit in monte.

\RVCPvi 


\nocturn{In III Nocturno}
\scriptura{Léctio sancti Evangélii secúndum Matt\'æum}
\lectiocap{Lectio vii}{Cap. 24, 42-47}
{\setstretch{0.96}
\y{I}{n} illo témpore: Dixit Jesus discípulis suis: Vigiláte, quia nescítis qua hora Dóminus vester ventúrus sit. Et réliqua.

\scriptura{Homilía sancti Fulgéntii Epíscopi}
\ex{Sermo 1 de Dispensatoribus}
\Y{D}{omínicus} sermo, quem debémus omnes non solum studióse, verum étiam sapiénter audíre, cui nos opórtet humíliter ac delectabíliter obedíre, moderatiónis suæ tenens ubique tempériem, ut nec óvibus desint pábula nec pastóribus aliménta: quædam vero speciáliter solia pr\'æcipit nobis, quædam vero generáliter et nobis et vobis. Nobis namque id est servis, quos pater ille famílias rerum ómnium Dóminus ad hoc in sua magna domo constítuit, ut pópulo ejus verbum grátiæ ministrémus, speciáliter injúngitur sanctæ prædicatiónis offícium; generáliter vero nobis et vobis salutáris indícitur obediéntia mandatórium.

\RVCPvii 

\lectio{Lectio viii}
\y{S}{ervórum} ígitur, quos pópulo suo præpósuit Dóminus, speciáliter volens offícium demonstráre ait, quod ex Evangélio modo audívimus: Quis putas est fidélis dispensátor et prudens, quem constítuit Dóminus super famíliam suam, ut det illis in témpore trítici mensúram? Beátus ille servus, quem cum vénerit Dóminus, invénerit ita faciéntem. Quis est iste Dóminus, fratres? Christus sine dúbio, qui suis discípulis ait: Vos vocátis me Magíster et Dómine, et bene dícitus: sum étenim. 

\RVCPviii 

\lectio{Lectio ix}
\y{Q}{uæ} est étiam hujus Dómini família? Nimírum illa est, quam ipse Dóminus de manu inimíci redémit et suo domínio mancipávit. Hæc família est sancta Cathólica Ecclésia, quæ per orbem terræ copiósa fertilitáte diffúnditur, et redémptam se pretióso sui Dómini sánguine gloriátur. Fílius enim hóminis, sict ipse ait, non venit ministrári, sed ministráre, et dare ánimam suam redemptiónem pro multis. Ipse est étiam pastor bonus, qui ánimam suam pósuit pro óvibus suis. Grex ergo boni pastóris ipsa est família Redemptóris. 

\Te 

}



\die{Die 28 Novembris}{De III die infra Octavam S. Conradi}{Episcopi et Confessoris}{Semiduplex}
\nocturn{In II Nocturno}
\scriptura{Sermo sancti Gregórii Nysséni}
\ex{In funere magni Meletii, circa medium}
{\setstretch{0.98}
\lectio{Lectio iv}
\Y{C}{um} primum bene moráta, ac modésta Ecclésia beátum Conrádum vidit, vidit fáciem ad imáginem Dei vere formátam, vidit dilectiónem fontis modo scaturiéntem, vidit grátiam lábiis circumfúsam, ánimi demissiónis summum gradum, post quem ámplius quidquam cogitári non potest. Vidit mansuetúdinem atque cleméntiam, qualis in Davíde fuit: qualis in Salomóne, intelligéntiam atque prudéntiam: qualis in Móyse, bonitátem: qualis in Samuéle, perfectiónem: qualis in Josépho, continéntiam pudicitiámque: qualis in Daniéle, sapiéntiam: quemádmodum magnus Elías zelo fídei pr\'æditum: sicut sublímis Joánnes, integritáte córporis ornátum: sícuti Paulus, inexsuperábili pr\'æditum dilectióne. Vidit tot bonórum circa unam ánimam concúrsum. Amóre beáto vulneráta est, casto bonóque sponsum suum amóre atque benevoléntia prosecúta diléxit.

\RVCPiv 

\lectio{Lectio v}
\y{S}{ed} priúsqam cupiditátem expléret, ántequam desidérium recreáret atque sedáret, amóris vi fervens, tentatiónibus athlétam ad certámina vocántibus, sola relícta est. Atque ille quidem in certamínibus pro pietáte suscéptis desudábat; hæc vero durábat in castitáte matrimónium consérvans. Non ablátus est a nobis sponsus; in médio nostrum stat, etiámsi nos non videámus; in ádytis ac penetrálibus sacérdos est, in interióribus veli, quo præcúrsor pro nobis ingréssus est Christus, relíquit carnis teguméntum. Non ámplius signo et umbræ cæléstium servit, sed in ipsam rerum imáginem intuétur: non ámplius per spéculum, atque per transénnam et ænígma, sed ipsa fácie cum fácie colláta intercédit apud Deum.

\RVCPv 

\lectio{Lectio vi}
\y{I}{ntercédit} autem pro nobis et pópuli errátis. Depósuit túnicas pellíceas: neque enim tálibus túnicis opus habent, qui in paradíso degunt: sed habet induménta, quæ puritáte vitæ suæ contéxuit, iísque sese exornávit. Honoráta ac pretiósa coram Dómino talis viri mors est; immo vero non mors, sed ruptúra membrórum est: Dirupísti enim, inquit, víncula mea. Dimíssus est Símeon, liberátus est a vínculis córporis: láqueus contrítus est, et avícula avolávit. Ingréssus est terram promissiónis, in monte cum Deo philosophátur. Solvit ánimæ calceaméntum, ut pura planta mentis terram sanctam, ubi conspícitur Deus, conscénderet.

\RVCPvi 

\nocturn{In III Nocturno}
\scriptura{Léctio sancti Evangélii secúndum Matt\'æum}
\lectiocap{Lectio vii}{Cap. 24, 42-47}
\y{I}{n} illo témpore: Dixit Jesus discípulis suis: Vigiláte, quia nescítis qua hora Dóminus vester ventúrus sit. Et réliqua.

\scriptura{Homilía sancti Fulgéntii Epíscopi}
\ex{Sermo de Confessoribus}
\Y{D}{ispensátor} vero, quis sit, quem opórtet esse fidélem páriter et prudéntem, Paulus nobis osténdit Apóstolus, qui de se suísque sóciis loquens ait: Sic nos exístimet homo, ut minístros Christi, et dispensatóres mysteriórum Dei. Ne quis autem vestrum solos Apóstolos dispensatóres factos exístimet, neglectóque milítiæ spiritális offício, servus piger, infidéliter imprudentérque dormítet, ipse beátus Apóstolus epíscopos quoque dispensatóres esse osténdens, ait: Opórtet enim epíscopum sine crímine esse, sicut Dei dispensatórem.

\RVCPvii 

\lectio{Lectio viii}
\y{S}{ervi} autem patrisfamílias sumus, dispensatóres Dómini sumus, mensúram trítici, quam vobis erogámus, accépimus. Quæ vero sit ista mensúra trítici, si quærámus, ipsam quoque nobis beátus Paulus Apóstolus osténdit, dicens:
Unicuíque sicut Deus divísit mensúram fídei. Quam ergo mensúram trítici Christus núncupat, ipsam mensúram fídei Paulus appéllat: ut agnoscámus, non áljud esse spiritále tríticum, quam christiánæ fídei venerábile Sacraméntum.

\RVCPviii 

\lectio{Lectio ix}
\y{H}{ujus} trítici mensúram vobis in nómine Dómini damus, quóties illumináti dono grátiæ spiritális, secúndum régulam veræ fídei disputámus: et eámdem trítici mensúram per domínicos dispensatóres accípitis, cum quotídie per Dei fámulos verbum veritátis audítis. De ipsa ergo trítici mensúra loquámur, ex ipsa, sicut Deus divísit, univérsi pascámur; inde aliménta bonæ vitæ pr\'æmia perveníre possímus: in illum credéntes, in illum sperántes, illum præ ómnibus diligéntes, qui seípsum nobis et aliméntum præstat, ne deficiámus in via: et pr\'æmium servat, ut gaudeámus in pátria.

\Te 

\rubric{In II Vesperis fit Com. S. Saturnini Mart.:}

\AiM 

\VRMi 

\pars{Oratio}
\yb{D}{eus}, qui nos beáti Saturníni Mártyris tui concédis natalítio pérfrui: ejus nos tríbue méritis adjuvári. Per Dóminum.

}


\die{Die 29 Novembris}{De IV die infra Octavam S. Conradi}{Episcopi et Confessoris}{Semiduplex}
{\setstretch{0.99}
\nocturn{In II Nocturno}
\scriptura{Sermo sancti Joánnis Chrysóstomi}
\ex{In cap. 15 Epist. ad Romanos, Serm. 29, circa finem}
\lectio{Lectio iv}
\Y{C}{hristum} díligens, et gregem illíus útique díligit. Et Móysen tunc primum super pópulum Judæórum pósuit, quando, qua esset in illum benevoléntia, reípsa jam declarávat. David quque símili modo regno inaugurátus est, posteáquam apparúerat, quam amíco esset in pópulum afféctu. Ita quippe júvenis adhuc, pópuli causa dóluit ac zelávit, ut et ánimam suam expóneret, cum vidélicet bárbarum illum e médio tollébat. Quod vero dicébat: Quid dábitur ei, qui alienígenam hunc interfécerit? non ídeo dicébat, quod mercédem exígeret, sed quo sibi crederétur, et in pugnam cum illo committerétur; nam cum adépta jam victória ad regem esset ingréssus, nihil de mercéde méminit.

\RVCPiv 

\lectio{Lectio v}
\y{E}{t} Sámuel quoque benígnus erat, et amátor pópuli: unde et dicébat: Absit autem hoc a me peccátum, ut cessem pro vobis oráre Dóminum. Ita et Paulus, immo non ita, sed multo plus ómnibus ardébat erga súbditos. Unde et discípulos ita erga se animávit, ut díceret: Si possíbile fuísset, óculos vestros eruissétis, mihíque dedissétis. Et Christus óptimi Pastóris régulam próferens dixit: Bonus pastor ánimam suam ponit pro óvibus. Sunt enim sanctórum ánimæ veheménter mites, et hóminum amántes, non solum erga domésticos sed aliénos, ita ut hanc suam mansuetúdinem étiam ad animántia bruta exténdant. Proptérea et sápiens quíspiam dixit: Justus miserétur animárum jumentórum suórum. Si jumentórum, multo magis hóminum.

\RVCPv 

\y{V}{erum} quóniam pécorum mentiónem feci, perpendámus et óvium pastóres illos, qui in Cappádocum regióne sunt, quália et quanta pro pécorum suórum custódia patiántur. Illi sæpenúmero univérsum tríduum nive óbruti perdúrant. Dicúntur autem, et qui in Líbya sunt, non minóra mala ferre, dum íntegros menses diffícilemillam solitúdinem, pessimarúmque bestiárum plenam vagándo circúmeunt. Si tantum erga pécora diligéntiæ impéndunt pastóres illi: quam, quæso, excusatiónem habébimus, quibus rationáles ánimæ concréditæ sunt, quod profúndum hunc somnum dormímus? An ignorámus gregis hujus dignitátem? an illíus grátia Dóminus tuus innúmera non fecit? an non postrémo et sánguinem suum fudit? Tu vero réquiem quæris? Et quid póterit pejus esse pastóribus istis?

\RVCPvi 

\nocturn{In III Nocturno}
\scriptura{Léctio sancti Evangélii secúndum Matt\'æum}
\lectiocap{Lectio vii}{Cap. 24, 42-47}
\y{I}{n} illo témpore: Dixit Jesus discípulis suis: Vigiláte, quia nescítis qua hora Dóminus vester ventúrus sit. Et réliqua.

}

{\setstretch{0.97}
\scriptura{Homilía sancti Joánnis Chrysóstomi}
\ex{Homil. 78 in Matth. ante med.}
\Y{P}{roptérea} hæc dicit Dóminus discípulis, ut vígilent, ut júgiter paráti sint, quia qua non exístimant hora, ventúrus est. Sollícitos ígitur facit et curiósos, ne umquam virtútem négligant. Tale quid autem est, quod dícitur: Si præscírent hómines, quando moritúri sint, diligéntiam suam circa illam horam osténderent. Ne ígitur in illo solúmmodo témpore, sed contínue diligéntes sint, nec generálem, nec singulárem horam prædícit, ut semper exspctándo, semper vígilent: hac enim ratióne términum vitæ uniuscujúsque occúluit. 

\RVCPvii 

\lectio{Lectio viii}

}

\y{D}{eínde} líquido, ádeo ut numquam liquídjus, Dóminum se ipsum appellávit. Hæc autem étiam ad erubescéntiam deídiæ mihi dicta vidéntur: majórem enim diligéntiam pecuniárum conservandárum habent furem exspectántes, quam vos salútis ánimæ. Vígilant enim tunc, ne áliquid sibi surripiátur: vos vero, quamvis certo sciátis ventúrum Dóminum, non perseverátis tamen, neque ita vigilátis, ut possítis non impræparáti ex hac vita discédere. Quaprópter cum pernície dormiéntium dies ille ventúrus est.


\RVCPviii 

\pro{Pro Vigilia S. Andreæ Ap.:}
\scriptura{Léctio sancti Evangélii secúndum Joánnem}
\lectiocap{Lectio ix}{Cap. 1, 35-51}
\yb{I}{n} illo témpore: Stabat Joannes, et ex discipulis ejus duo. Et respiciens Jesum ambulantem, dicit: Ecce Agnus Dei. Et réliqua.

\scriptura{Homilía sancti Augustíni Epíscopi}
\ex{Tract. 7 in Joann., post init.}
\yb{Q}{uia} talis erat Joánnes amícus sponsi, non quærébat glóriam suam, sed testimónium perhibébat veritáti: numquid vóluit apud se remanére discípulos suos, ut non sequeréntur Dóminum? Magis ipse osténdit discípulis suis quem sequeréntur: habébant enim illum tamquam agnum. Et ille: Quid me atténditis: Ego non sum agnus. Ecce Agnus Dei. De quo et supérius díxerat: Ecce Agnus Dei. Et quid nobis prodest Agnus Dei? Ecce, ait, qui tollit peccátum mundi. Secúti sunt illum, hoc audíto, duo qui erant cum Joánne.

\Te

\rubric{Ad Laudes fit Com. Vigiliæ S. Andreæ:}

\rubric{Antiphona et \V de Feria currenti.}

\pars{Oratio}
\yb{Q}{uæsumus} omnipotens Deus: ut beatus Andreas Apostolus, cujus prævenimus festivitatem, tuum pro nobis imploret auxilium; ut a nostris reatibus absoluti, a cunctis étiam periculis eruamur. Per Dóminum.

\rubric{Deinde Com. S. Saturnini:}

\AiiM 

\VRMii 

\pars{Oratio}
\yb{D}{eus}, qui nos beáti Saturníni Mártyris tui concédis natalítio pérfrui: ejus nos tríbue méritis adjuvári. Per Dóminum.

\rubric{Vesperæ de sequenti sine Commemoratione.}



\mens{Festa Decembris}

\dieii{Die 1 Decembris}{De VI die infra Octavam S. Conradi}{Episcopi et Confessoris}{Semiduplex}
\nocturn{In II Nocturno}

\rubric{Lectiones de Sermone S. Maximi \black{Ad sancti ac beatíssimi Patris nostri Conrádi,} ut in Breviario de Communi Conf. Pont. 1 loco.}

\nocturn{In III Nocturno}
\scriptura{Léctio sancti Evangélii secúndum Matt\'æum}
\lectiocap{Lectio vii}{Cap. 24, 42-47}
\y{I}{n} illo témpore: Dixit Jesus discípulis suis: Vigiláte, quia nescítis qua hora Dóminus vester ventúrus sit. Et réliqua.

\scriptura{Homilía sancti Joánnis Chrysóstomi}
\ex{Unde supra}
\Y{H}{oc} autem in loco, ne rursus curiósius perscruténtur, et adhuc, ut, quod rárius est, pluris osténdat.
Expénde autem qualem hæc verba ignorántiam significáre possunt, síquidem a se constitátum super famíliam suam ignoráret. Pretérea beatíficat quidem eum, (Beátus enim, inquit, servus ille); quis vero sit, siléntio trádidit: quis enim est, aut quómodo constítuet Dóminus super familiam suam, quem invéniet sic faciéntem? Hæc non de pecúniis solúmmodo, verum étiam de verbis, de virtúte, de omni dispensatióne síngulo hómini concéssa, dicta sunt.


\RVCPvii

\lectio{Lectio viii}
\y{H}{æc} parábola vel ad geréntes gubernantésque rempúblicam accommodári potest. Docet enim unumquémque ad commúnem utilitátem ómnia conférre, quæ sua sunt, sive sapiéntiam, sive principátum, sive quidquid aljud, non ad detriméntum conservórum, nec ad perditiónem suam. Quaprópter utrúmque ab eo flágitat, prudéntiam scílicet
atque fidem. Delícta enim ab améntia oríginem habent.

\RVCPviii

\lectio{Lectio ix}
\y{F}{idélem} igitur appéllat, quóniam nihil ex rebus dómini sibi attríbuit, nec incássum quidquam expéndit; prudéntem autem, quia dispensáre nóverit: opportúne síquidem utráque re nobis opus est, ne scílicet, quæ dómini sunt, ad próprios usus rapiámus, et ut opportúne ómnia dispensémus.

\Te

\rubric{Vesperæ de sequenti, Com. præcedentis}

\end{multicols}

\ornamentvi

\end{document}
