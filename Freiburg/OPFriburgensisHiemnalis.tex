\documentclass[fontsize=9pt,paper=A6,twoside,BCOR=1mm,DIV=22,headinclude]{scrarticle}
\usepackage{breviarium}
\renewcommand\A\Ant
\begin{document}
\titulum{Officia Propria}{Archidiœcesis Friburgensis}{Pars Hiemnalis}
\begin{multicols}{2}
\mensii{Festa Novembris}
\dieii{}{Pro Commemoratione Octavæ S. Conradi}{}{}
\hora{In I Vesperis}

\AiCP 

\VRCPi 

\pars{Oratio}
\y{S}{ancti} Pontíficis et Confessóris tui Conrádi solémnia celebrántes, te, Dómine, supplíciter obsecrámus: ut ipsum apud tuam cleméntiam sentiámus habére patrónum, quem nobis tua grátia providísti salútis ætérnæ minístrum. Per Dóminum.

\columnbreak
\hora{Ad Laudes}

\AiiCP

\VRCPii 

\hora{In II Vesperis}

\AiiiCP

\V Justum dedúxit.


\die{Die 28 Novembris}{De III die infra Octavam S. Conradi}{Episcopi et Confessoris}{Semiduplex}
{\setstretch{0.98}
\rubric{Infra Octavam et in die Octava Antiphonæ et Psalmi ad omnes Horas et Versus Nocturnorum de occurrenti Hebdomadæ die ut in Psalterio; reliqua ut in Festo præter Lectiones, quæ in I Nocturno dicuntur de Scriptura occurrenti cum suis Responsoriis de Tempore, in II et III Nocturno pro singulis diebus assignantur propriæ.}

\nocturn{In II Nocturno}
\structura{Sermo sancti Gregórii Nysséni}
\ex{In funere magni Meletii, circa medium}
\lectio{Lectio iv}
\Y{C}{um} primum bene moráta, ac modésta Ecclésia beátum Conrádum vidit, vidit fáciem ad imáginem Dei vere formátam, vidit dilectiónem fontis modo scaturiéntem, vidit grátiam lábiis circumfúsam, ánimi demissiónis summum gradum, post quem ámplius quidquam cogitári non potest. Vidit mansuetúdinem atque cleméntiam, qualis in Davíde fuit: qualis in Salomóne, intelligéntiam atque prudéntiam: qualis in Móyse, bonitátem: qualis in Samuéle, perfectiónem: qualis in Josépho, continéntiam pudicitiámque: qualis in Daniéle, sapiéntiam: quemádmodum magnus Elías zelo fídei pr\'æditum: sicut sublímis Joánnes, integritáte córporis ornátum: sícuti Paulus, inexsuperábili pr\'æditum dilectióne. Vidit tot bonórum circa unam ánimam concúrsum. Amóre beáto vulneráta est, casto bonóque sponsum suum amóre atque benevoléntia prosecúta diléxit.

\RVCPiv 

\lectio{Lectio v}
\y{S}{ed} priúsqam cupiditátem expléret, ántequam desidérium recreáret atque sedáret, amóris vi fervens, tentatiónibus athlétam ad certámina vocántibus, sola relícta est. Atque ille quidem in certamínibus pro pietáte suscéptis desudábat; hæc vero durábat in castitáte matrimónium consérvans. Non ablátus est a nobis sponsus; in médio nostrum stat, etiámsi nos non videámus; in ádytis ac penetrálibus sacérdos est, in interióribus veli, quo præcúrsor pro nobis ingréssus est Christus, relíquit carnis teguméntum. Non ámplius signo et umbræ cæléstium servit, sed in ipsam rerum imáginem intuétur: non ámplius per spéculum, atque per transénnam et ænígma, sed ipsa fácie cum fácie colláta intercédit apud Deum.

\RVCPv 

\lectio{Lectio vi}
\y{I}{ntercédit} autem pro nobis et pópuli errátis. Depósuit túnicas pellíceas: neque enim tálibus túnicis opus habent, qui in paradíso degunt: sed habet induménta, quæ puritáte vitæ suæ contéxuit, iísque sese exornávit. Honoráta ac pretiósa coram Dómino talis viri mors est; immo vero non mors, sed ruptúra membrórum est: Dirupísti enim, inquit, víncula mea. Dimíssus est Símeon, liberátus est a vínculis córporis: láqueus contrítus est, et avícula avolávit. Ingréssus est terram promissiónis, in monte cum Deo philosophátur. Solvit ánimæ calceaméntum, ut pura planta mentis terram sanctam, ubi conspícitur Deus, conscénderet.

\RVCPvi 

\nocturn{In III Nocturno}
\scriptura{Léctio sancti Evangélii secúndum Matt\'æum}
\lectiocap{Lectio vii}{Cap. 24, 42-47}
\y{I}{n} illo témpore: Dixit Jesus discípulis suis: Vigiláte, quia nescítis qua hora Dóminus vester ventúrus sit. Et réliqua.

}

{\setstretch{0.98}
\scriptura{Homilía sancti Fulgéntii Epíscopi}
\ex{Sermo de Confessoribus}
\Y{D}{ispensátor} vero, quis sit, quem opórtet esse fidélem páriter et prudéntem, Paulus nobis osténdit Apóstolus, qui de se suísque sóciis loquens ait: Sic nos exístimet homo, ut minístros Christi, et dispensatóres mysteriórum Dei. Ne quis autem vestrum solos Apóstolos dispensatóres factos exístimet, neglectóque milítiæ spiritális offício, servus piger, infidéliter imprudentérque dormítet, ipse beátus Apóstolus epíscopos quoque dispensatóres esse osténdens, ait: Opórtet enim epíscopum sine crímine esse, sicut Dei dispensatórem.

\RVCPvii 

\lectio{Lectio viii}
\y{S}{ervi} autem patrisfamílias sumus, dispensatóres Dómini sumus, mensúram trítici, quam vobis erogámus, accépimus. Quæ vero sit ista mensúra trítici, si quærámus, ipsam quoque nobis beátus Paulus Apóstolus osténdit, dicens:
Unicuíque sicut Deus divísit mensúram fídei. Quam ergo mensúram trítici Christus núncupat, ipsam mensúram fídei Paulus appéllat: ut agnoscámus, non áljud esse spiritále tríticum, quam christiánæ fídei venerábile Sacraméntum.

\RVCPviii 

\lectio{Lectio ix}
\y{H}{ujus} trítici mensúram vobis in nómine Dómini damus, quóties illumináti dono grátiæ spiritális, secúndum régulam veræ fídei disputámus: et eámdem trítici mensúram per domínicos dispensatóres accípitis, cum quotídie per Dei fámulos verbum veritátis audítis. De ipsa ergo trítici mensúra loquámur, ex ipsa, sicut Deus divísit, univérsi pascámur; inde aliménta bonæ vitæ pr\'æmia perveníre possímus: in illum credéntes, in illum sperántes, illum præ ómnibus diligéntes, qui seípsum nobis et aliméntum præstat, ne deficiámus in via: et pr\'æmium servat, ut gaudeámus in pátria.

\Te 

\rubric{Ad Laudes fit Commemoratio Feriæ.}

\rubric{In Vesperis fit Commemoratio Feriæ.}

\rubric{Deinde Com. S. Saturnini Mart.:}

\AiM 

\VRMi 

\pars{Oratio}
\yb{D}{eus}, qui nos beáti Saturníni Mártyris tui concédis natalítio pérfrui: ejus nos tríbue méritis adjuvári. Per Dóminum.

}

\die{Die 29 Novembris}{De IV die infra Octavam S. Conradi}{Episcopi et Confessoris}{Semiduplex}
\nocturn{In II Nocturno}
\scriptura{Sermo sancti Joánnis Chrysóstomi}
\ex{In cap. 15 Epist. ad Romanos, Serm. 29, circa finem}
\lectio{Lectio iv}
\Y{C}{hristum} díligens, et gregem illíus útique díligit. Et Móysen tunc primum super pópulum Judæórum pósuit, quando, qua esset in illum benevoléntia, reípsa jam declarávat. David quque símili modo regno inaugurátus est, posteáquam apparúerat, quam amíco esset in pópulum afféctu. Ita quippe júvenis adhuc, pópuli causa dóluit ac zelávit, ut et ánimam suam expóneret, cum vidélicet bárbarum illum e médio tollébat. Quod vero dicébat: Quid dábitur ei, qui alienígenam hunc interfécerit? non ídeo dicébat, quod mercédem exígeret, sed quo sibi crederétur, et in pugnam cum illo committerétur; nam cum adépta jam victória ad regem esset ingréssus, nihil de mercéde méminit.

\RVCPiv 


\lectio{Lectio v}
\y{E}{t} Sámuel quoque benígnus erat, et amátor pópuli: unde et dicébat: Absit autem hoc a me peccátum, ut cessem pro vobis oráre Dóminum. Ita et Paulus, immo non ita, sed multo plus ómnibus ardébat erga súbditos. Unde et discípulos ita erga se animávit, ut díceret: Si possíbile fuísset, óculos vestros eruissétis, mihíque dedissétis. Et Christus óptimi Pastóris régulam próferens dixit: Bonus pastor ánimam suam ponit pro óvibus. Sunt enim sanctórum ánimæ veheménter mites, et hóminum amántes, non solum erga domésticos sed aliénos, ita ut hanc suam mansuetúdinem étiam ad animántia bruta exténdant. Proptérea et sápiens quíspiam dixit: Justus miserétur animárum jumentórum suórum. Si jumentórum, multo magis hóminum.

\RVCPv 

\lectio{Lectio vi}
\y{V}{erum} quóniam pécorum mentiónem feci, perpendámus et óvium pastóres illos, qui in Cappádocum regióne sunt, quália et quanta pro pécorum suórum custódia patiántur. Illi sæpenúmero univérsum tríduum nive óbruti perdúrant. Dicúntur autem, et qui in Líbya sunt, non minóra mala ferre, dum íntegros menses diffícilemillam solitúdinem, pessimarúmque bestiárum plenam vagándo circúmeunt. Si tantum erga pécora diligéntiæ impéndunt pastóres illi: quam, quæso, excusatiónem habébimus, quibus rationáles ánimæ concréditæ sunt, quod profúndum hunc somnum dormímus? An ignorámus gregis hujus dignitátem? an illíus grátia Dóminus tuus innúmera non fecit? an non postrémo et sánguinem suum fudit? Tu vero réquiem quæris? Et quid póterit pejus esse pastóribus istis?

\RVCPvi 

\nocturn{In III Nocturno}
\scriptura{Léctio sancti Evangélii secúndum Matt\'æum}
\lectiocap{Lectio vii}{Cap. 24, 42-47}
\y{I}{n} illo témpore: Dixit Jesus discípulis suis: Vigiláte, quia nescítis qua hora Dóminus vester ventúrus sit. Et réliqua.

\scriptura{Homilía sancti Joánnis Chrysóstomi}
\ex{Homil. 78 in Matth. ante med.}
\Y{P}{roptérea} hæc dicit Dóminus discípulis, ut vígilent, ut júgiter paráti sint, quia qua non exístimant hora, ventúrus est. Sollícitos ígitur facit et curiósos, ne umquam virtútem négligant. Tale quid autem est, quod dícitur: Si præscírent hómines, quando moritúri sint, diligéntiam suam circa illam horam osténderent. Ne ígitur in illo solúmmodo témpore, sed contínue diligéntes sint, nec generálem, nec singulárem horam prædícit, ut semper exspctándo, semper vígilent: hac enim ratióne términum vitæ uniuscujúsque occúluit. 

\RVCPvii 

\lectio{Lectio viii}
\y{D}{eínde} líquido, ádeo ut numquam liquídjus, Dóminum se ipsum appellávit. Hæc autem étiam ad erubescéntiam deídiæ mihi dicta vidéntur: majórem enim diligéntiam pecuniárum conservandárum habent furem exspectántes, quam vos salútis ánimæ. Vígilant enim tunc, ne áliquid sibi surripiátur: vos vero, quamvis certo sciátis ventúrum Dóminum, non perseverátis tamen, neque ita vigilátis, ut possítis non impræparáti ex hac vita discédere. Quaprópter cum pernície dormiéntium dies ille ventúrus est.

\RVCPviii 

\lectio{Lectio ix}
\y{N}{am} quemádmodum non læderétur ille furto, si furem sciret ventúrum; sic et vos, quóniam scitis Dóminum ventúrum, paratióres esse debétis. Céterum quóniam judícii mentiónem fecit, ad Doctóres étiam vertit oratiónem, de pœna et méritis dísserens. Et primo bonis collátis ad peccatóres éxitum facit, commodíssime oratiónem conclúdens. Quis ergo, inquit, fidélis servus et prudens, quem constítuit Dóminus super famíliam suam, ut det illis cibum in témpore?

\Te

\rubric{Ad Laudes fit Com. Feriæ:}

\rubric{Deinde Com. S. Saturnini:}

\AiiM 

\VRMii 

\pars{Oratio}
\yb{D}{eus}, qui nos beáti Saturníni Mártyris tui concédis natalítio pérfrui: ejus nos tríbue méritis adjuvári. Per Dóminum.

\rubric{Vesperæ de sequenti, Com. Feriæ tantum.}


\mens{Festa Decembris}

\dieii{Die 1 Decembris}{De VI die infra Octavam S. Conradi}{Episcopi et Confessoris}{Semiduplex}
\nocturn{In II Nocturno}

\rubric{Lectiones de Sermone S. Maximi \black{Ad sancti ac beatíssimi Patris nostri Conrádi,} ut in Breviario de Communi Conf. Pont. 1 loco.}

\nocturn{In III Nocturno}
\scriptura{Léctio sancti Evangélii secúndum Matt\'æum}
\lectiocap{Lectio vii}{Cap. 24, 42-47}
\y{I}{n} illo témpore: Dixit Jesus discípulis suis: Vigiláte, quia nescítis qua hora Dóminus vester ventúrus sit. Et réliqua.

\scriptura{Homilía sancti Joánnis Chrysóstomi}
\ex{Unde supra}
\Y{H}{oc} autem in loco, ne rursus curiósius perscruténtur, et adhuc, ut, quod rárius est, pluris osténdat.
Expénde autem qualem hæc verba ignorántiam significáre possunt, síquidem a se constitátum super famíliam suam ignoráret. Pretérea beatíficat quidem eum, (Beátus enim, inquit, servus ille); quis vero sit, siléntio trádidit: quis enim est, aut quómodo constítuet Dóminus super familiam suam, quem invéniet sic faciéntem? Hæc non de pecúniis solúmmodo, verum étiam de verbis, de virtúte, de omni dispensatióne síngulo hómini concéssa, dicta sunt.

\RVCPvii

\lectio{Lectio viii}
\y{H}{æc} parábola vel ad geréntes gubernantésque rempúblicam accommodári potest. Docet enim unumquémque ad commúnem utilitátem ómnia conférre, quæ sua sunt, sive sapiéntiam, sive principátum, sive quidquid aljud, non ad detriméntum conservórum, nec ad perditiónem suam. Quaprópter utrúmque ab eo flágitat, prudéntiam scílicet
atque fidem. Delícta enim ab améntia oríginem habent.

\RVCPviii

{\setstretch{0.98}
\lectio{Lectio ix}
\y{F}{idélem} igitur appéllat, quóniam nihil ex rebus dómini sibi attríbuit, nec incássum quidquam expéndit; prudéntem autem, quia dispensáre nóverit: opportúne síquidem utráque re nobis opus est, ne scílicet, quæ dómini sunt, ad próprios usus rapiámus, et ut opportúne ómnia dispensémus.

\Te

\rubric{Ad Laudes fit Com. Feriæ.}

\rubric{Vesperæ de sequenti, Com. præcedentis, ut supra, et Feriæ.}
}


\die{Die 3 Decembris}{In Octava S. Conradi}{Episcopi et Confessoris}{Duplex majus}

{\setstretch{0.98}

\rubric{In I Vesperis fit Com. præcedentis:}

\AiiiV

\VRViii

\pars{Oratio}
\yb{D}{eus}, ómnium largítor bonórum, qui in fámula tua Bibiána cum virginitátis flore martýrii palmam coniunxísti: mentes nostras ejus intercessióne tibi caritáte coniúnge; ut, amótis perículis, prǽmia consequámur ætérna.

\rubric{Deinde Com. S. Francisci Xaverii Conf.:}

\AiiC 

\VRCi 

\pars{Oratio}
\yb{D}{eus}, qui Indiárum gentes beáti Francísci prædicatióne et miráculis Ecclésiæ tuæ aggregáre voluísti: concéde propítius; ut cujus gloriósa mérita venerámur, virtútum quoque imitémur exémpla.

\rubric{Postea Com. Feriæ.}

\nocturn{In II Nocturno}
\scriptura{Sermo sancti Gregórii Papæ}
\ex{Part. 2 Pastoralis, cap. 1}
\lectio{Lectio iv}
\Y{T}{antum} debet actiónem pópuli àctio transcéndere pr\'æsulis, quantum distáre solet a grege vita pastóris. Opórtet namque, ut metíri se sollícite stúdeat, quanta tenéndæ rectitúdinis necessitáte constríngitur, sub cujus æstimatióne pópulus grex vocátur. Sit ergo necésse est cogitatióne mundus, actióne præcípuus, discrétus in siléntio, útilis in verbo, síngulis compassióne próximus, præ cunctis contemplatióne suspénsus, bene agéntibus per humilitátem sócjus, contra delinquéntium vítia per zelum justítiæ eréctus, infernórum curam in exteriórum occupatióne non mínuens, exteriórum providéntiam in internórum sollicitúdine non relínquens.

\RVCPiv 

\lectio{Lectio v}
\ex{Part. 2 Pastor. cap. 9 et 10}
\y{C}{onsiderándum} quoque est, quia cum curam pópuli eléctus præsul súscipit, quasi ad ægrum médicus accédit. Si ergo adhuc in ejus córpore passiónes vivunt, qua præsumptióne percússum medéri próperat, qui in fácie vulnus portat? Ille modis ómnibus debet ad exémplum bene vivéndi pértrahi, qui cunctis carnis passiónibus móriens, jam spiritáliter vivit, qui próspera mundi postpónit, qui nulla advérsa pertiméscit, qui sola intérna desíderat: cujus intentióni bene cóngruens, nec omníno per imbecillitátem corpus, nec valde per contumáciam repúgnat spíritus: qui ad aliéna cupiénda non dúcitur, sed própria largítur.

\RVCPv 

\lectio{Lectio vi}
\ex{Ibid. cap. 8}
\y{U}{nde} ipsum quoque episcopátus offícium boni óperis expressióne definítur, cum dícitur: Si quis episcopátum desíderat bonum opus desíderat. Ipse ergo sibi testis est, quia episcopátum non áppetit, qui non per hunc boni óperis ministérium, sed honóris glóriam quærit. Sacrum quippe offícium non solum non díligit omníno, sed nescit, qui ad culmen regíminis anhélans, in occúlta meditatióne cogitatiónis, ceterórum subjectióne páscitur, laude própria lætátur, ad honórem cor élevat, rerum affluéntium abundántia exsúltat. Mundi ergo lucrum qu\'æritur sub ejus honóris spécie, quo mundi déstrui lucra debúerant.

\RVCPvi

\nocturn{In III Nocturno}
\scriptura{Léctio sancti Evangélii secúndum Matt\'æum}
\lectiocap{Lectio vii}{Cap. 24, 42-47}
\y{I}{n} illo témpore: Dixit Jesus discípulis suis: Vigiláte, quia nescítis qua hora Dóminus vester ventúrus sit. Et réliqua.

\scriptura{Homilía sancti Ambrósii Epíscopi}
\ex{De Fide, lib. 5 cap. 8}
\Y{Q}{u\'ærimus}, qua ratióne Dóminus designáre moménta nolúerit. Si quærámus, non ignorántiæ inveniémus esse, sed sapiéntiæ. Nobis enim scire non próderat, ut dum certa futúri judícii moménta nescímus, semper tamquam in excúbiis constitúti, et in quadam virtútis spécula collocáti, peccándi consuetúdinem declinémus, ne nos inter vitia dies Dómini deprehéndat.


\RVCPvii 

\lectio{Lectio viii}
\y{N}{on} enim prodest sciro, sed metúere quod futúrum est; scriptum est enim: Noli alta sápere, sed time. Nam si diem designásset exprésse, uni ætáti hóminum, quæ próxima erat judício, viderétur disciplínam præscripsísse vivéndi superióris témporis; aut justus esset remíssior, aut peccátor secúrior. Namque adúlter, nisi quotidiánam pœnam métuat, non potest adulterándi cupiditátem desínere.

\RVCPviii 

\pro{Pro S. Francisco Xaverio:}
\lectio{Lectio ix}
\yb{F}{rancíscus}, in Xavério diœcésis Pampelonénsis nobílibus paréntibus natus, Parísiis sancto Ignátio sese cómitem et discípulum iungit, et brevi mira vitæ austeritáte et rerum divinárum assídua contemplatióne cláruit. A Paulo tértio apostólicus núntius pro Indiis creátus, províncias innúmeras pédibus semper, et sæpe nudis, peragrávit. Fidem Iapóniæ et sex áliis regiónibus invéxit. Multa centéna hóminum míllia ad Christum in Indiis convértit; magnósque príncipes regésque complúres sacro fonte expiávit. Ea tamen erat humilitáte, ut sancto Ignátio, præpósito suo, fléxis génibus scríberet. Ejus dilatándi Evangélii ardórem multitúdine et excelléntia miraculórum Dóminus roborávit. Demum in Sanciáno Sinárum ínsula, die secúnda decémbris, óbiit plenus méritis laboribúsque conféctus. Eum Gregórius décimus quintus inter Sanctos rétulit, Pjus autem décimus sodalitáti et óperi Propagándæ Fídei cæléstem patrónum constítuit.

\Te

\rubric{Ad Laudes fit Com. S. Francisci Xaverii:}

\AiC 

\VRCii 

\pars{Oratio}
\yb{D}{eus}, qui Indiárum gentes beáti Francísci prædicatióne et miráculis Ecclésiæ tuæ aggregáre voluísti: concéde propítius; ut cujus gloriósa mérita venerámur, virtútum quoque imitémur exémpla.

\rubric{Postea Com. Feriæ.}

\rubric{In II Vesperis Com. sequentis:}

\A O Doctor óptime, Ecclésiæ sanctæ lumen, beáte Petre Chrysóloge, divínæ legis amátor, deprecáre pro nobis Fílium Dei.

\VRCi 

\pars{Oratio}
\yb{D}{eus}, qui beátum Petrum Chrysólogum Doctórem egrégium, divínitus præmonstrátum, ad regéndam et instruéndam Ecclésiam tuam éligi voluísti: præsta, quǽsumus; ut, quem Doctórem vitæ habúimus in terris, intercessórem habére mereámur in cælis.

\rubric{Deinde Com. S. Francisci Xaverii:}

\AiiiC 

\VRCiv

\red{Oratio} Deus, qui Indiárum, \red{ut supra.}

\rubric{Postea Com. Feriæ.}

\rubric{Denique Com. S. Barbaræ Virg. et Mart.}

\AiV 

\VRVi 

\yb{D}{eus}, qui inter cétera poténtiæ tuæ mirácula étiam in sexu frágili victóriam martýrii contulísti: concéde propítius; ut, qui beátæ Bárbaræ Vírginis et Mártyris tuæ natalícia cólimus, per ejus ad te exémpla gradiámur.
Per Dóminum.

}



\mens{Festa Januarii}
\dieii{Die 28 Januarii}{S. Meinradi}{Martyris}{Duplex}

\VRMi 

\MiM

\pars{Oratio}
\y{P}{ræsta}, qu\'æsumus omnípotens Deus: ut, qui beáti Meinrádi Mártyris tui natalítia cólimus, intercessióne ejus in tui nóminis amóre roborémur. Per Dóminum.

\rubric{Et fit Com. præcedentis:}

\A O Doctor óptime, Ecclésiæ sanctæ lumen, beáte Joánnes Chrysóstome, divínæ legis amátor, deprecáre pro nobis Fílium Dei.

\VRCPiii

\pars{Oratio}
\yb{E}{cclésiam} tuam, quǽsumus, Dómine, grátia cæléstis amplíficet: quam beáti Joánnis Chrysóstomi Confessóris tui atque Pontíficis illustráre voluísti gloriósis méritis et doctrínis.

\rubric{Deinde Com. S. Petri Nolasci:}

\AiC 

\VRCi 

\pars{Oratio}
\yb{D}{eus}, qui in tuæ caritátis exémplum ad fidélium redemptiónem sanctum Petrum Ecclésiam tuam nova prole fecundáre divínitus docuísti: ipsíus nobis intercessióne concéde; a peccáti servitúte solútis, in cælésti pátria perpétua libertáte gaudére: \red{(}Qui vivis.\red{)}

\rubric{Deinde Com. S. Agnetis Virg. et Mart. secundo:}

\A Stans a dextris ejus Agnus nive candidior, Christus sibi sponsam et Mártyrem consecrávit.

\VRVi 

\pars{Oratio}
\yb{D}{eus}, qui nos ánnua beátæ Agnetis Vírginis et Mártyris tuæ solemnitáte lætíficas: da, quǽsumus; ut, quam venerámur officio, étiam piæ conversatiónis sequámur exémplo. Per Dóminum.

\nocturn{In II Nocturno}
\lectio{Lectio iv}
\Y{M}{einrádus} ex nóbili génere in Alamánnia natus, adhuc puer a Berthóldo patre, Sulgóvie cómite, in sacrum Augiæ monastérium delátus traditúsque est Erlebáldo propínquo, exímia pietáte mónacho. Cum mox plúrimum in virtútibus et sciéntiis profecísset, annos natus vigínti quinque présbyter et deínde mónachus factus, céteros æmulatóres longe anteívit. Probáta ígitur ejus in virtútibus christiánis perfectióne, stúdio Bollingénsis ascetérii præféctus fuit; quo loco anachoréticæ vitæ desidério mox flagráre cœpit.

\RVCiv 

\lectio{Lectio v}
\y{I}{gitur} exploráta aliquándo vicíni saltus commoditáte, habitáque Abbátis vénia, Missáli, Breviário, sancti Benedícti régula et Cassiáni opúsculis secum sumptis, in Ezelíum concéssit montem, lácui Turicíno imminéntem, in quo jejúniis, vigíliis, córporis afflictatiónibus ásperum septénnium egit. Cum vero frequéntem hóminum ad admirábilis sanctitátis ejus experiéndam utilitátem accedéntium concúrsum gravem ádmodum experirétur, ad hunc vitándum simúlque ad tranquillitátem consequéndam, in vastíssimam silva obscúræ solitúdinem sese ábdidit, et céllula ac oratório in accómmoda planítie constrúctis, vitæ réliquum admiránda sanctitáte perégit. Ibi septénnis Puérulus augustíssima spécie e sacrário procédere cumque eo oráre, et quæ non licet hómini arcána loqui, ei communicáre est visus.

\RVCv 

\lectio{Lectio vi}
\y{V}{igínti} sex annis in tam vasta erémo a viro sancto sanctíssime absúmptis, duo nefárii hómines advérsus eum conjuráti in solitúdinem adventárunt, et quamvis humaníssime ab eódem excépti, deque eorúmdem propósito ab ipso edócti, impie nihilóminus diréque cæsum miserabíliter strangulárunt, duodécimo Kaléndas Februárii, anno Redemptiónis octingentésimo sexagésimo primo. Quam grata autem Deo fúerit hæc víctima, nonnúlla mox mirácula comprobárunt. Céreus enim ad Mártyris caput appósitus c\'ælitus illúxit, et tota solitúdo suavíssimo odóre perfúsa est; et hómines profáni corvórum, quos sanctus olim alúerat, insectatióne ac vellicatióne próditi, propédiem vivi igne combústi sunt. Aliquot vero post annis nóbile monastérium in martýrii loco eréctum est, quod hódie Mártyris memória, et imprímis ædícula Vírginis Deíparæ mirífice consecráta, ac præterea grátiis crebrísque ejúsdem Vírginis miráculis toti mundo est venerábile.

\RVCvi 

\nocturn{In III Nocturno}
\scriptura{Léctio sancti Evangélii secúndum Matt\'æum}
\lectiocap{Lectio vii}{Cap. 24, 42-47}
\y{I}{n} illo témpore: Dixit Jesus discípulis suis: Si quis vult post me veníre, ábneget semetípsum, et tollat crucem suam, et sequátur me. Et réliqua.

\scriptura{Homilía sancti Gregórii Papæ.}
\ex{Homilia 32. in Evang.}
\Y{Q}{ia} Dóminus ac Redémptor noster novus homo venit in mundum, nova præcépta dedit mundo. Vitæ étenim nostræ véteri in vítiis enutrítæ contrarietátem oppósuit novitátis suæ. Quid enim vetus, quid carnális homo nóverat, nisi sua retinére, aliéna rápere, si posset; concupíscere, si non posset? Sed cæléstis médicus síngulis quibúsque vítiis obviántia ádhibet medicaménta. Nam sicut arte medicínæ cálida frígidis, frígida cálidis curántur: ita Dóminus noster contrária oppósuit medicaménta peccátis, ut lúbricis continéntiam, tenácibus largitátem, iracúndis mansuetúdinem, elátis præcíperet humilitátem.

\RVMvii

\lectio{Lectio viii}
\y{C}{erte} cum se sequéntibus nova mandáta propóneret, dixit: Nisi quis renuntiáverit ómnibus quæ póssidet, non potest meus esse discípulus. Ac si apérte dicat: Qui per vitam véterem aliéna concupíscitis, per novæ conversatiónis stádium et vestra largímini. Quid vero in hac lectióne dicat, audiámus: Qui vult post me veníre, ábneget semetípsum. Ibi dícitur, ut abnegémus nostra: hic dícitur, ut abnegémus nos. Et fortásse laboriósum non est hómini relínquere sua: sed valde laboriósum est relínquere semetípsum. Minus quippe est abnegáre quod habet: valde autem multum est abnegáre quod est.

\RMVviii

\pro{Pro S. Petro Nolasci}
\lectio{Lectio ix}
\yb{P}{etrus} Noláscus, Recáudi prope Carcasónam in Gállia nóbili génere natus, adoléscens paréntibus orbátus, Albigénsium hǽresim éxsecrans, divéndito património, in Hispániam secéssit, ubi noctu oránti beáta Virgo appárens, Fílio suo sibíque acceptíssimum fore suggéssit, si Ordo religiosórum instituerétur pro captívis ab infidélium tyránnide liberándis. Quare una cum sancto Raymúndo de Péñafort et Iacóbo primo, rege Aragóniæ, de eádem re a Dei Genetríce ipsa nocte præmónitis, religiónem beátæ Maríæ de Mercéde redemptiónis captivórum instítuit; sodálibus quarto voto obstríctis, manéndi in pignus sub paganórum potestáte, si pro Christiánis liberándis opus esset. Angeli Custódis ac Deíparæ Vírginis apparitiónibus sæpe recreátus, cum ad bonam senectútem pervenísset, piíssime óbiit média nocte Vigíliæ Nativitátis Dómini, anno millésimo ducentésimo quinquagésimo sexto.

\Te 

\hora{Ad Laudes}

\VRMii 

\BM

\pars{Oratio}
\y{P}{ræsta}, qu\'æsumus omnípotens Deus: ut, qui beáti Meinrádi Mártyris tui natalítia cólimus, intercessióne ejus in tui nóminis amóre roborémur. Per Dóminum.


\rubric{Et fit Com. S. Petri Nolasci:}

\AiiC 

\VRCii 

\pars{Oratio}
\yb{D}{eus}, qui in tuæ caritátis exémplum ad fidélium redemptiónem sanctum Petrum Ecclésiam tuam nova prole fecundáre divínitus docuísti: ipsíus nobis intercessióne concéde; a peccáti servitúte solútis, in cælésti pátria perpétua libertáte gaudére: \red{(}Qui vivis.\red{)}


\rubric{Deinde Com. S. Agnetis Virg. et Mart. secundo:}

\A Ecce, quod concupívi, jam vídeo: quod sperávi, jam téneo: ipsi sum iuncta in cælis, quem in terris pósita, tota devotióne diléxi.

\VRVii

\pars{Oratio}
\yb{D}{eus}, qui nos ánnua beátæ Agnetis Vírginis et Mártyris tuæ solemnitáte lætíficas: da, quǽsumus; ut, quam venerámur officio, étiam piæ conversatiónis sequámur exémplo. Per Dóminum.

\rubric{Vesperæ a Capitulo de sequenti, Com. præcedentis:}

\AiiiM

\V Justus ut, \red{et Oratio} Præsta, qu\'æsumus, \red{ut supra.}

\rubric{Et fit Com. S. Petri Nolasci:}

\AiiiC 

\V Justum dedúxit, \red{et Oratio} Deus, qui in, \red{ut supra.}




\mens{Festa Martii}
\dieii{Die 2 Martii}{B. Henrici Susonis}{Confessoris}{\\Duplex \mtv}

\VRCi 

\MiC 

\pars{Oratio}
\y{D}{eus}, qui beátum Henrícum Confessórem tuum córporis mortificatióne et caritáte mirábilem effecísti: concéde; ut Christum crucifíxum corde dilígere, et ópere in nobis exprímere valeámus. Per eúndem Dóminum nostrum.

\rubric{Et in Quadragesima fit Com. Feriæ.}

{\setstretch{0.96}
\rubric{In I Nocturno, si sumendæ sint de Communi, Lectiones \black{Beátus vir,} 1 loco.}

\nocturn{In II Nocturno}
\lectio{Lectio iv}
\Y{H}{enrícus} Suso, Joánnes a Suévia dictus, ubi ex illústri família ortus, cognómen non a patre Montése, sed pótius Sunon a Matre, ceu piíssima fémina retínuit. Juste sanctéque educátus ádeo in virtútibus profécit, ut décimo tértio ætátis anno ad altióra a Deo vocátus, Prædicatórum Ordini Constántiæ nomen déderit. Humánæ vero fragilitáti in religióne páululum indúlgens, per quinquénnium ita se hábuit, ut leves communésque culpas parvipénderet, donec eum Deus, volens divítias misericórdiæ suæ osténdere, a terrénis afféctibus expedítum ad se, pénitus attráxit: propter eáque Henrícus intérnis certamínibus et afflictatiónibus, quibus retardabátur a via perfectiónis, magno ánimo superátis, cæléstibus solum ínhians, Dei præséntiam júgiter contemplári constítuit.

}

{\setstretch{0.99}
\RVCiv 

\lectio{Lectio v}
\y{C}{hristiánæ} mortificatiónis exémplar efféctus, Christi passiónem in semetípso exprímere miris abstinéntiis. corporísque cruciátibus conátus est; nam cilícium et caténam férream ácubus armátam diu noctúque gestávit, et in ejúsdem passiónis memóriam lígneam crucem clavis férreis hórridam inter scápulas ad carnem sibi aptávit. Super nudam tábulam, ac sine ullo operiménto dormíre solébat; multis prætérea vigíliis, flagellatiónibus aliísque pœniténtiæ opéribus corpus suum, tamquam infensíssimum hostem, quotídie in servitútem redégit. Divíno amóre veheménter accénsus, nomen Jesu sinístro péctoris láteri stylo férreo impréssit; sed Deus, qui ad sublimióra illum vocáverat, córporis cruciátibus hisce amarióres spiritus afflictatiónes adjúnxit.

\RVCv 

\lectio{Lectio vi}
\y{Q}{ibus} angústiis, necnon hóminum moléstiis ac persecutiónibus Henrícus úndique jactátus, in Dei bonitáte confídens, firma spe, humilitáte et patiéntia de advérsis triumphávit. In Ordinis observatiónibus erat sollícitus, orándi habens stúdium indeféssum, ac siléntium ita servans, ut numquam illud frégerit. Erga Deíparam Vírginem ferventíssimo ferebátur obséquio. Salútem zelans animárum, innúmeros peccatóres ad pœniténtiam redúxit, præcípue vero dum magnam Germánis partem, Suévjam imprímis et Alsátiam, verbum Dei contínuo pr\'ædicans, perlustrávit. Móritur post multos labóres plurimáque patiéntiæ documénta, plenus diérum ac virtútum, in cœnóbio Ulménsi, ubi multis miráculis a Deo claruísse perhibétur. Hujus sancti viri fama longe latéque diffúsa, cultum ecclesiásticum íllico ad præséntem usque diem ipsi conciliávit, quo rite probáto primo Gregórius décimus sextus Ordini Prædicatórum, deínde Leo décimus tértius Archidiœcési Friburgénsi Offícium Missámque in ejus honórem benígne concéssit.

\RVCvi 

\rubric{In III Nocturno Homilia in Evangel. \black{Confiteor tibi Pater,} de Communi Abbatum 2 loco.}

\rubric{In Quadragesima IX Lectio de Homilia Feriæ.}

\hora{Ad Laudes}

\VRCii 

\BC 

\pars{Oratio}
\y{D}{eus}, qui beátum Henrícum Confessórem tuum córporis mortificatióne et caritáte mirábilem effecísti: concéde; ut Christum crucifíxum corde dilígere, et ópere in nobis exprímere valeámus. Per eúndem Dóminum nostrum.

\rubric{Et in Quadragesima fit Com. Feriæ.}

\rubric{Vesperæ a Capitulo de sequenti, Com. præcedentis, et in Quadragesima Feriæ.}

}



\die{Die 3 Martii}{S. Cunegundis}{Imperatricis, Virginis}{Duplex}
{\setstretch{0.99}

\VRVi 

\MiV 

\pars{Oratio}
\y{D}{eus}, qui beátæ Cunegúndi Vírgini tuæ terréna despícere, et ad cæléstis impérii culmen anheláre tribuísti: concéde, qu\'æsumus; ut ejus imitatióne calcátis mundi illécebris, ad gáudia ætérna secúri perveníre valeámus. Per Dóminum.

\rubric{Et fit Commem. præcedentis:}

\MiiC 

\V Justum dedúxit, \red{et Oratio} Deus, qui, \red{ut supra.}

\rubric{Et in Quadragesima fit Com. Feriæ.}

\rubric{In Quadragesima Lectiones \black{De Virgínibus,} de Communi 1 loco.}

\nocturn{In II Nocturno}
\lectio{Lectio iv}
\Y{C}{unegúndis} Augústa, Germanórum nobilíssima, ab ineúnte ætáte præclára exhíbuit sanctitátis indícia, ac Deíparæ Vírginis, cui se totam addíxerat, patrocínio mundi illécebras, et fluxas carnis voluptátes ea morum severitáte coércuit, ut virginitátem quoque perpétuam cólere, atque angélicam in terris vitam ágere proposúerit. Verum paréntum jussis, qui illam in matrimónium despónderant, obtemperáre coácta, illud a cælésti sponso, qui páscitur inter lília, précibus efflagitávit, ut quem suscíperet virum, virginitátis suæ custódem inveníret. Henríco ítaque Baváriæ duci, cognoménto Pio, póstea Romanórum imperatóri, in matrimónium trádita, raro exémplo virginitátem ipso annuénte illibátam servávit: in cujus testimónium, cum aliquándo, instigánte humáni géneris hoste, suspício quædam contra ipsam fuísset exórta, nudis pédibus super ignítos vómeres incédens ill\'æsa, egrégium suæ innocéntiæ pr\'æbuit arguméntum, quod sanctus ipse Henrícus suprémo mortis témpori appropínquans, abundántius declarávit; accítis namque impérii procéribus, aliísque princípibus viris, ac sanctæ Vírginis consanguíneis: qualem mihi, inquit, eam assignástis, talem reddo: vírginem dedístis, vírginem reddo.

\RVViv

\lectio{Lectio v}
\y{S}{ingulári} erga ecclésias pietáte pr\'ædita, sanctum Henrícum cónjugem suum ad eas construéndas et locupletándas exhortatiónibus incitávit, ópibus adjúvit. Bambergénsem ítaque episcopátum própriis facultátibus fundátum, beáto Petro Apostolórum Príncipi, Roman\'æque Ecclésiæ vectigálem fecérunt, plúribus áliis templis et cœnóbiis cæsaréa plane munificéntia locupletátis. Cum pene ómnia, quæ habébat in terris, Christi ecclésiis, ac paupéribus distribuísset, seípsam quoque defúncto viro in Jesu Christi obséquium impéndit, ac sancti Patris Benedícti régulam proféssa, et regnum mundi et omnem ornátum s\'æculi ingénti ánimi demissióne contémnens, sanctum religiósæ vitæ propósitum usque ad mortem exactíssime custodívit.

\RVVv 

\lectio{Lectio vi}
\y{J}{ugi} ítaque virtútum exercítio, cum annos quíndecim in cœnóbio bonórum óperum méritis cumuláta transegísset, ad cæléstis Agni núptias, quas únice optáverat, invitáta sancto fine quiévit. Fámulæ suæ sanctitátem Deus íllico póstmodum crebris miráculis demonstrávit; nam ad ejus túmulum cæci visum, claudi gressum, surdi audítum, muti loquélam, aliísque multis ac diutúrnis affécti languóribus salútem retulérunt; sed illud inprímis nóbile, quod pulvis de sepúlchro ejus assúmptus sæpe in fruméntum convérti repértus sit. Quibus rite probátis, ab Innocéntio tértio Pontífice Máximo sanctárum vírginum número est adscrípta.

\RVVvi 

\rubric{In III Nocturno Homilia in Evangel. \black{Símile erit regnum cælórum decem virgínibus,} de Communi 1 loco.}

\rubric{In Quadragesima IX Lectio de Homilia Feriæ}
}

\hora{Ad Laudes}

\VRVii 

\BV 

\pars{Oratio}
\y{D}{eus}, qui beátæ Cunegúndi Vírgini tuæ terréna despícere, et ad cæléstis impérii culmen anheláre tribuísti: concéde, qu\'æsumus; ut ejus imitatióne calcátis mundi illécebris, ad gáudia ætérna secúri perveníre valeámus. Per Dóminum.

\rubric{Et in Quadragesima fit Com. Feriæ.}

\hora{In II Vesperis}

\VRVii 

\MiiV 

\rubric{Et fit Com. sequentis:}

\AiC 

\VRCi 

\pars{Oratio}
\yb{D}{eus}, qui inter regáles delícias et mundi illécebras sanctum Casimírum virtúte constántiæ roborásti: quǽsumus; ut ejus intercessióne fidéles tui terréna despíciant, et ad cæléstia semper aspírent.

\rubric{Deinde post Com. Feriæ in Quadragesima Com. S. Lucii I Papæ et Mart.:}

\AiM 

\VRMi 

\pars{Oratio}
\yb{D}{eus} qui nos beáti Lucii Martyris tui atque Pontíficis annua solemnitáte lætíficas: concede propítius; ut cujus natalícia cólimus, de ejúsdem étiam protectione gaudeámus. Per Dóminum.



\dieii{Die 6 Martii}{S. Fridolini}{Confessoris}{Duplex}

{\setstretch{0.97}

\VRCi 

\MiC 

\pars{Oratio}
\y{D}{eus}, qui perfécte coram te ambulántibus protéctor es et merces magna nimis: da nobis beáti Fridolíni Abbátis intercessióne et exémplo temporália despícere, et ad te tota mentis intentióne festináre. Per Dóminum.

\rubric{Et fit Com. Ss. Perpetuæ et Felicitatis Mm.:}

\A Istárum est enim regnum cælórum, qui contempsérunt vitam mundi, et pervenérunt ad prǽmia regni, et lavérunt stolas suas in sánguine Agni.

\V Glória et honóre coronásti eas, Dómine.
\R Et constituísti eas super ópera mánuum tuárum.

\pars{Oratio}
\yb{D}{a} nobis, quǽsumus, Dómine, Deus noster, sanctárum Mártyrum tuárum Perpétuæ et Felicitátis palmas incessábili devotióne venerári: ut, quas digna mente non póssumus celebráre, humílibus saltem frequentémus obséquiis. \red{(}Per Dóminum.\red{)}

\rubric{Deinde in Quadragesima Com. Feriæ.}

\rubric{In I Nocturno, si sumendæ sint de Communi, Lectiones \black{Beátus vir,} 1 loco.}

}

{\setstretch{0.97}
\nocturn{In II Nocturno}
\lectio{Lectio iv}
\Y{F}{ridolínus}, in Hibérnia nóbili génere ortus, jam a juventúte sciéntia ac pietáte elúxit. Sacerdótio insignítus et máxime egrégiis ánimi dótibus pr\'æditus, verbum Dei cum tanta eloquéntia et cum tanto ómnium appláusu prædicábat, ut præcípuis honóribus propter suam artis oratóriæ perítiam afficerétur. Ast celebritátem humilitáti postpónens, divítias et omnem s\'æculi pompam spernens, opes patérnas inter cognátos, páuperes et ecclésias divísit, sicque sponte pauper factus, pátriam relíquit et in Gálliam sese cóntulit.

\RVCiv 

\lectio{Lectio v}
\y{I}{bi} váriis in locis cum ingénti zelo verbum Dei annuntiándo, multos in virtúte firmávit, multos ad pœniténtiam vocávit, plúrimos ad veri Dei notítiam et cultum perdúxit. Tandem in Pictávii subúrbiis cónstitit, ubi monastérii sancti Hilárii eléctus fuit Abbas. In rudéribus ecclésiæ suæ ossa magni Hilárii detéxit, éaque in ecclésia, quam delápsam una cum monastério instaurávit, honorífice recóndidit.

}
{\setstretch{0.98}

\RVCv 

\lectio{Lectio vi}
\y{C}{aritátis} igne consúmptus Pictávium relíquit, plures Gálliæ províncias lustrávit, quasi tuba exáltans vocem suam, verbi Dei indeféssus præco. Vária in váriis locis templa et monastéria struxit. In Alsátia quoque et ípsamet civitáte Argentína, quam máxime Dei glóriæ et animárum salúti operátus fídei propagátor, demum in Séckingen, Rheni ínsula prope Basiléam, in fínibus Suévie , consédit, ubi in honórem sancti Hilárii templum et duplex monastérium, álterum pro mónachis et álterum pro moniálibus, constítuit, ea sapientíssime rexit, et ibi circa annum quingentésimum trigésimum octávum post Christum natum sanctíssime in Dómino obdormívit.

\RVCvi 

\nocturn{In III Nocturno}
\scriptura{Léctio sancti Evangélii secúndum Matt\'æum}
\lectiocap{Lectio vii}{Cap. 24, 42-47}
\y{I}{n} illo témpore: Dixit Petrus ad Jesum: Ecce nos relíquimus ómnia et secúti sumus te; quid ergo erit nobis? Et réliqua.

\scriptura{Homilía sancti Hilárii Epíscopi}
\ex{Can. 20.}
\Y{A}{póstoli} dictis Dómini hæc reddunt: reliquísse se ómnia et cum ipso esse. Quibus Dóminus, cum séderit in majestátis suæ sede, sessúros super sedes duódecim ac tótidem tribus Israël judicatúros spopóndit, omnibúsque qui univérsa relíquerint propter nomen ejus, fructum céntupli pr\'æmii reservátum, multos autem ex novíssimis primos futúros et ex primis futúros novíssimos. Multa sunt, quæ non sinunt nos símplici intelléctu dicta evangélica suscípere.

\RVCvii 

\lectio{Lectio viii}
\y{I}{nterpósitis} nonnúllis rebus, quæ ex natúra humáni sensus sibi contrária sunt, ratiónem qu\'ærere cæléstis intelligéntiæ admonémur. Apóstoli dicunt et sequi se Christum, et se ómnia reliquísse. Quómodo ígitur fiunt tristes, et quómodo métuunt, dicéntes salvum esse néminem posse? Namque et ab áliis fíeri póterat, si quid fecíssent ipsi. Deínde cum fecíssent, quare metus, vel unde suscéptus est? Additur étiam in responsióne Dómini, hæc apud hómines impossibília, possibília apud Deum. Numquid apud hómines impossibília erant, quæ et Apóstoli se fecísse gloriántur, et fecísse eos Dóminus agnóscit?

}

\RVCviii 

\rubric{In Quadragesima IX Lectio de Homilia Feriæ, alias}

\pro{Pro Ss. Perpetua et Felicitate:}

\lectio{Lectio ix}
\yb{P}{erpétua} et Felícitas, in persecutióne Sevéri imperatóris, in Africa, una cum Revocáto, Saturníno et Secúndolo comprehénsæ sunt et in tenebricósum cárcerem detrúsæ, quibus ultra adiúnctus est Sátyrus. Ibi, cum adhuc catechúmenæ essent, baptizátæ sunt. Tum ad béstias damnántur; cumque a Felicitáte, in partus labóribus dolénti, quǽreret quidam e custódibus, quid in amphitheátro esset factúra, illa respóndit: Modo ego pátior; illic autem álius erit in me, qui patiétur pro me, quia et ego pro illo passúra sum. Itaque in amphitheátrum, toto inspectánte pópulo, prodúctæ, primum flagéllis cædúntur; tum a ferocíssima vacca aliquámdiu jactátæ, plagis concísæ et in terram elísæ sunt; demum cum sóciis, qui a váriis béstiis vexáti fúerant, die séptima Mártii, gladiórum íctibus conficiúntur.

\Te

\hora{Ad Laudes}

\VRCii 

\BC 

\pars{Oratio}
\y{D}{eus}, qui perfécte coram te ambulántibus protéctor es et merces magna nimis: da nobis beáti Fridolíni Abbátis intercessióne et exémplo temporália despícere, et ad te tota mentis intentióne festináre. Per Dóminum.

\rubric{Et fit Com. Ss. Perpetuæ et Felicitatis Mm.:}

\A Istárum est enim regnum cælórum, qui contempsérunt vitam mundi, et pervenérunt ad prǽmia regni, et lavérunt stolas suas in sánguine Agni.

\V Glória et honóre coronásti eas, Dómine.
\R Et constituísti eas super ópera mánuum tuárum.

\pars{Oratio}
\yb{D}{a} nobis, quǽsumus, Dómine, Deus noster, sanctárum Mártyrum tuárum Perpétuæ et Felicitátis palmas incessábili devotióne venerári: ut, quas digna mente non póssumus celebráre, humílibus saltem frequentémus obséquiis. \red{(}Per Dóminum.\red{)}

\rubric{Deinde in Quadragesima Com. Feriæ.}

\rubric{Vesperæ a Capitulo de sequenti, Com. præcedentis:}

\AiiiC 

\V Justum dedúxit \red{et Oratio} Deus, qui perfécte, \red{ut supra.}

\rubric{Deinde Com. Ss. Perpetuæ et Felicitatis, ut supra ad Laudes}

\rubric{Postea in Quadrages. Com. Feriæ.}

\end{multicols}

\ornamentvi

\end{document}
