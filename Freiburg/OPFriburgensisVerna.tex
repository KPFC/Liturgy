\documentclass[fontsize=9pt,paper=A6,twoside,BCOR=1mm,DIV=22,headinclude]{scrarticle}
\usepackage{breviarium}
\renewcommand\A\Ant
\begin{document}
\titulum{Officia Propria}{Archidiœcesis Friburgensis}{Pars Verna}
\begin{multicols}{2}
\mensii{Festa Martii}
\dieii{Die 2 Martii}{B. Henrici Susonis}{Confessoris}{\\Duplex \mtv}

{\setstretch{0.99}
\VRCi 

\MiC 

\pars{Oratio}
\y{D}{eus}, qui beátum Henrícum Confessórem tuum córporis mortificatióne et caritáte mirábilem effecísti: concéde; ut Christum crucifíxum corde dilígere, et ópere in nobis exprímere valeámus. Per eúndem Dóminum.

\rubric{Et fit Com. Feriæ.}

\rubric{In I Nocturno Lectiones \black{Beátus vir,} de Communi Conf. non Pont. 1 loco.}

\nocturn{In II Nocturno}
\lectio{Lectio iv}
\Y{H}{enrícus} Suso, Joánnes a Suévia dictus, ubi ex illústri família ortus, cognómen non a patre Montése, sed pótius Sunon a Matre, ceu piíssima fémina retínuit. Juste sanctéque educátus ádeo in virtútibus profécit, ut décimo tértio ætátis anno ad altióra a Deo vocátus, Prædicatórum Ordini Constántiæ nomen déderit. Humánæ vero fragilitáti in religióne páululum indúlgens, per quinquénnium ita se hábuit, ut leves communésque culpas parvipénderet, donec eum Deus, volens divítias misericórdiæ suæ osténdere, a terrénis afféctibus expedítum ad se, pénitus attráxit: propter eáque Henrícus intérnis certamínibus et afflictatiónibus, quibus retardabátur a via perfectiónis, magno ánimo superátis, cæléstibus solum ínhians, Dei præséntiam júgiter contemplári constítuit.

\RVCiv 

\lectio{Lectio v}
\y{C}{hristiánæ} mortificatiónis exémplar efféctus, Christi passiónem in semetípso exprímere miris abstinéntiis. corporísque cruciátibus conátus est; nam cilícium et caténam férream ácubus armátam diu noctúque gestávit, et in ejúsdem passiónis memóriam lígneam crucem clavis férreis hórridam inter scápulas ad carnem sibi aptávit. Super nudam tábulam, ac sine ullo operiménto dormíre solébat; multis prætérea vigíliis, flagellatiónibus aliísque pœniténtiæ opéribus corpus suum, tamquam infensíssimum hostem, quotídie in servitútem redégit. Divíno amóre veheménter accénsus, nomen Jesu sinístro péctoris láteri stylo férreo impréssit; sed Deus, qui ad sublimióra illum vocáverat, córporis cruciátibus hisce amarióres spiritus afflictatiónes adjúnxit.

\RVCv 

\lectio{Lectio vi}
\y{Q}{ibus} angústiis, necnon hóminum moléstiis ac persecutiónibus Henrícus úndique jactátus, in Dei bonitáte confídens, firma spe, humilitáte et patiéntia de advérsis triumphávit. In Ordinis observatiónibus erat sollícitus, orándi habens stúdium indeféssum, ac siléntium ita servans, ut numquam illud frégerit. Erga Deíparam Vírginem ferventíssimo ferebátur obséquio. Salútem zelans animárum, innúmeros peccatóres ad pœniténtiam redúxit, præcípue vero dum magnam Germánis partem, Suévjam imprímis et Alsátiam, verbum Dei contínuo pr\'ædicans, perlustrávit. Móritur post multos labóres plurimáque patiéntiæ documénta, plenus diérum ac virtútum, in cœnóbio Ulménsi, ubi multis miráculis a Deo claruísse perhibétur. Hujus sancti viri fama longe latéque diffúsa, cultum ecclesiásticum íllico ad præséntem usque diem ipsi conciliávit, quo rite probáto primo Gregórius décimus sextus Ordini Prædicatórum, deínde Leo décimus tértius Archidiœcési Friburgénsi Offícium Missámque in ejus honórem benígne concéssit.

\RVCvi 

\rubric{In III Nocturno Homilia in Evangel. \black{Confiteor tibi Pater,} de Communi Abbatum 2 loco.}

\rubric{IX Lectio de Homilia Feriæ.}

\hora{Ad Laudes}

\VRCii 

\BC 

\pars{Oratio}
\y{D}{eus}, qui beátum Henrícum Confessórem tuum córporis mortificatióne et caritáte mirábilem effecísti: concéde; ut Christum crucifíxum corde dilígere, et ópere in nobis exprímere valeámus. Per eúndem Dóminum.

\rubric{Et fit Com. Feriæ.}

\rubric{Vesperæ a Capitulo de sequenti, Com. præcedentis et Feriæ.}

}


\die{Die 3 Martii}{S. Cunegundis}{Imperatricis, Virginis}{Duplex}

{\setstretch{0.99}
\VRVi 

\MiV 

\pars{Oratio}
\y{D}{eus}, qui beátæ Cunegúndi Vírgini tuæ terréna despícere, et ad cæléstis impérii culmen anheláre tribuísti: concéde, qu\'æsumus; ut ejus imitatióne calcátis mundi illécebris, ad gáudia ætérna secúri perveníre valeámus. Per Dóminum nostrum.

\rubric{Et fit Commem. præcedentis:}

\MiiC 

\V Justum dedúxit, \red{et Oratio} Deus, qui beátum, \red{ut supra.}

\rubric{Deinde Com. Feriæ.}

\rubric{In I Nocturno Lectiones \black{De Virgínibus,} de Communi 1 loco.}

\nocturn{In II Nocturno}
\lectio{Lectio iv}
\Y{C}{unegúndis} Augústa, Germanórum nobilíssima, ab ineúnte ætáte præclára exhíbuit sanctitátis indícia, ac Deíparæ Vírginis, cui se totam addíxerat, patrocínio mundi illécebras, et fluxas carnis voluptátes ea morum severitáte coércuit, ut virginitátem quoque perpétuam cólere, atque angélicam in terris vitam ágere proposúerit. Verum paréntum jussis, qui illam in matrimónium despónderant, obtemperáre coácta, illud a cælésti sponso, qui páscitur inter lília, précibus efflagitávit, ut quem suscíperet virum, virginitátis suæ custódem inveníret. Henríco ítaque Baváriæ duci, cognoménto Pio, póstea Romanórum imperatóri, in matrimónium trádita, raro exémplo virginitátem ipso annuénte illibátam servávit: in cujus testimónium, cum aliquándo, instigánte humáni géneris hoste, suspício quædam contra ipsam fuísset exórta, nudis pédibus super ignítos vómeres incédens ill\'æsa, egrégium suæ innocéntiæ pr\'æbuit arguméntum, quod sanctus ipse Henrícus suprémo mortis témpori appropínquans, abundántius declarávit; accítis namque impérii procéribus, aliísque princípibus viris, ac sanctæ Vírginis consanguíneis: qualem mihi, inquit, eam assignástis, talem reddo: vírginem dedístis, vírginem reddo.

\RVViv

\lectio{Lectio v}
\y{S}{ingulári} erga ecclésias pietáte pr\'ædita, sanctum Henrícum cónjugem suum ad eas construéndas et locupletándas exhortatiónibus incitávit, ópibus adjúvit. Bambergénsem ítaque episcopátum própriis facultátibus fundátum, beáto Petro Apostolórum Príncipi, Roman\'æque Ecclésiæ vectigálem fecérunt, plúribus áliis templis et cœnóbiis cæsaréa plane munificéntia locupletátis. Cum pene ómnia, quæ habébat in terris, Christi ecclésiis, ac paupéribus distribuísset, seípsam quoque defúncto viro in Jesu Christi obséquium impéndit, ac sancti Patris Benedícti régulam proféssa, et regnum mundi et omnem ornátum s\'æculi ingénti ánimi demissióne contémnens, sanctum religiósæ vitæ propósitum usque ad mortem exactíssime custodívit.

\RVVv 

\lectio{Lectio vi}
\y{J}{ugi} ítaque virtútum exercítio, cum annos quíndecim in cœnóbio bonórum óperum méritis cumuláta transegísset, ad cæléstis Agni núptias, quas únice optáverat, invitáta sancto fine quiévit. Fámulæ suæ sanctitátem Deus íllico póstmodum crebris miráculis demonstrávit; nam ad ejus túmulum cæci visum, claudi gressum, surdi audítum, muti loquélam, aliísque multis ac diutúrnis affécti languóribus salútem retulérunt; sed illud inprímis nóbile, quod pulvis de sepúlchro ejus assúmptus sæpe in fruméntum convérti repértus sit. Quibus rite probátis, ab Innocéntio tértio Pontífice Máximo sanctárum vírginum número est adscrípta.

\RVVvi 

\rubric{In III Nocturno Homilia in Evangel. \black{Símile erit regnum cælórum decem virgínibus,} de Communi 1 loco.}

\rubric{IX Lectio de Homilia Feriæ}

\hora{Ad Laudes}

\VRVii 

\BV 

\pars{Oratio}
\y{D}{eus}, qui beátæ Cunegúndi Vírgini tuæ terréna despícere, et ad cæléstis impérii culmen anheláre tribuísti: concéde, qu\'æsumus; ut ejus imitatióne calcátis mundi illécebris, ad gáudia ætérna secúri perveníre valeámus. Per Dóminum nostrum.

\rubric{Et fit Com. Feriæ.}

\hora{In II Vesperis}

\VRVii 

\MiiV 

\rubric{Et fit Com. sequentis:}

\AiC 

\VRCi 

\pars{Oratio}
\yb{D}{eus}, qui inter regáles delícias et mundi illécebras sanctum Casimírum virtúte constántiæ roborásti: quǽsumus; ut ejus intercessióne fidéles tui terréna despíciant, et ad cæléstia semper aspírent.

}

{\setstretch{0.97}
\rubric{Deinde post Com. Feriæ Com. S. Lucii I Papæ et Mart.:}

\AiM 

\VRMi 

\pars{Oratio}
\yb{D}{eus} qui nos beáti Lucii Martyris tui atque Pontíficis annua solemnitáte lætíficas: concede propítius; ut cujus natalícia cólimus, de ejúsdem étiam protectione gaudeámus. Per Dóminum.

}

\die{Die 6 Martii}{S. Fridolini}{Confessoris}{Duplex}

{\setstretch{0.97}

\VRCi 

\MiC 

\pars{Oratio}
\y{D}{eus}, qui perfécte coram te ambulántibus protéctor es et merces magna nimis: da nobis beáti Fridolíni Abbátis intercessióne et exémplo temporália despícere, et ad te tota mentis intentióne festináre. Per Dóminum.

}

\rubric{Et fit Com. Ss. Perpetuæ et Felicitatis Mm.:}

\A Istárum est enim regnum cælórum, qui contempsérunt vitam mundi, et pervenérunt ad prǽmia regni, et lavérunt stolas suas in sánguine Agni.

\V Glória et honóre coronásti eas, Dómine.
\R Et constituísti eas super ópera mánuum tuárum.

\pars{Oratio}
\yb{D}{a} nobis, quǽsumus, Dómine, Deus noster, sanctárum Mártyrum tuárum Perpétuæ et Felicitátis palmas incessábili devotióne venerári: ut, quas digna mente non póssumus celebráre, humílibus saltem frequentémus obséquiis. \red{(}Per Dóminum.\red{)}

\rubric{Deinde Com. Feriæ.}

\rubric{In I Nocturno Lectiones \black{Beátus vir,} de Communi Conf. non Pont. 1 loco.}

\columnbreak
{\setstretch{0.98}
\nocturn{In II Nocturno}
\lectio{Lectio iv}
\Y{F}{ridolínus}, in Hibérnia nóbili génere ortus, jam a juventúte sciéntia ac pietáte elúxit. Sacerdótio insignítus et máxime egrégiis ánimi dótibus pr\'æditus, verbum Dei cum tanta eloquéntia et cum tanto ómnium appláusu prædicábat, ut præcípuis honóribus propter suam artis oratóriæ perítiam afficerétur. Ast celebritátem humilitáti postpónens, divítias et omnem s\'æculi pompam spernens, opes patérnas inter cognátos, páuperes et ecclésias divísit, sicque sponte pauper factus, pátriam relíquit et in Gálliam sese cóntulit.

\RVCiv 

\lectio{Lectio v}
\y{I}{bi} váriis in locis cum ingénti zelo verbum Dei annuntiándo, multos in virtúte firmávit, multos ad pœniténtiam vocávit, plúrimos ad veri Dei notítiam et cultum perdúxit. Tandem in Pictávii subúrbiis cónstitit, ubi monastérii sancti Hilárii eléctus fuit Abbas. In rudéribus ecclésiæ suæ ossa magni Hilárii detéxit, éaque in ecclésia, quam delápsam una cum monastério instaurávit, honorífice recóndidit.

\RVCv 

\lectio{Lectio vi}
\y{C}{aritátis} igne consúmptus Pictávium relíquit, plures Gálliæ províncias lustrávit, quasi tuba exáltans vocem suam, verbi Dei indeféssus præco. Vária in váriis locis templa et monastéria struxit. In Alsátia quoque et ípsamet civitáte Argentína, quam máxime Dei glóriæ et animárum salúti operátus fídei propagátor, demum in Séckingen, Rheni ínsula prope Basiléam, in fínibus Suévie , consédit, ubi in honórem sancti Hilárii templum et duplex monastérium, álterum pro mónachis et álterum pro moniálibus, constítuit, ea sapientíssime rexit, et ibi circa annum quingentésimum trigésimum octávum post Christum natum sanctíssime in Dómino obdormívit.

\RVCvi 

\nocturn{In III Nocturno}
\scriptura{Léctio sancti Evangélii secúndum Matt\'æum}
\lectiocap{Lectio vii}{Cap. 24, 42-47}
\y{I}{n} illo témpore: Dixit Petrus ad Jesum: Ecce nos relíquimus ómnia et secúti sumus te; quid ergo erit nobis? Et réliqua.

\scriptura{Homilía sancti Hilárii Epíscopi}
\ex{Can. 20.}
\Y{A}{póstoli} dictis Dómini hæc reddunt: reliquísse se ómnia et cum ipso esse. Quibus Dóminus, cum séderit in majestátis suæ sede, sessúros super sedes duódecim ac tótidem tribus Israël judicatúros spopóndit, omnibúsque qui univérsa relíquerint propter nomen ejus, fructum céntupli pr\'æmii reservátum, multos autem ex novíssimis primos futúros et ex primis futúros novíssimos. Multa sunt, quæ non sinunt nos símplici intelléctu dicta evangélica suscípere.

\RVCvii 

\lectio{Lectio viii}
\y{I}{nterpósitis} nonnúllis rebus, quæ ex natúra humáni sensus sibi contrária sunt, ratiónem qu\'ærere cæléstis intelligéntiæ admonémur. Apóstoli dicunt et sequi se Christum, et se ómnia reliquísse. Quómodo ígitur fiunt tristes, et quómodo métuunt, dicéntes salvum esse néminem posse? Namque et ab áliis fíeri póterat, si quid fecíssent ipsi. Deínde cum fecíssent, quare metus, vel unde suscéptus est? Additur étiam in responsióne Dómini, hæc apud hómines impossibília, possibília apud Deum. Numquid apud hómines impossibília erant, quæ et Apóstoli se fecísse gloriántur, et fecísse eos Dóminus agnóscit?

\RVCviii 

\rubric{IX Lectio de Homilia Feriæ.}

}

\hora{Ad Laudes}

\VRCii 

\BC 

\pars{Oratio}
\y{D}{eus}, qui perfécte coram te ambulántibus protéctor es et merces magna nimis: da nobis beáti Fridolíni Abbátis intercessióne et exémplo temporália despícere, et ad te tota mentis intentióne festináre. Per Dóminum.

\rubric{Et fit Com. Ss. Perpetuæ et Felicitatis Mm.:}

\A Istárum est enim regnum cælórum, qui contempsérunt vitam mundi, et pervenérunt ad prǽmia regni, et lavérunt stolas suas in sánguine Agni.

\V Glória et honóre coronásti eas, Dómine.
\R Et constituísti eas super ópera mánuum tuárum.

\pars{Oratio}
\yb{D}{a} nobis, quǽsumus, Dómine, Deus noster, sanctárum Mártyrum tuárum Perpétuæ et Felicitátis palmas incessábili devotióne venerári: ut, quas digna mente non póssumus celebráre, humílibus saltem frequentémus obséquiis. \red{(}Per Dóminum.\red{)}

\rubric{Deinde Com. Feriæ.}

\rubric{Vesperæ a Capitulo de sequenti, Com. præcedentis:}

\AiiiC 

\V Justum dedúxit \red{et Oratio} Deus, qui perfécte, \red{ut supra.}

\rubric{Deinde Com. Ss. Perpetuæ et Felicitatis, ut supra ad Laudes}

\rubric{Postea Com. Feriæ.}



\die{Die 15 Martii}{S. Clementis Mariæ Hofbauer}{Confessoris}{Duplex}

\VRCi 

\MiC 

\pars{Oratio}
\y{D}{eus}, qui beátum Cleméntem Maríam miro fídei róbore et invíctæ constántiæ virtúte decorásti: ejus méritis et exémplis fac nos, qu\'æsumus, ita fortes in fide et caritáte fervéntes; ut pr\'æmia consequámur ætérna. Per Dóminum nostrum.

\rubric{Et fit Com. Feriæ.}

\rubric{In I Nocturno Lectiones \black{Beátus vir,} de Communi Conf. non Pont. 1 loco.}

\columnbreak
{\setstretch{0.965}
\nocturn{In II Nocturno}
\lectio{Lectio iv}
\Y{C}{lemens} María Hófbauer Tassovítii in Morávia natus, a prima ætáte in schola christiánæ perfectiónis erudítus est; magístra potíssimum religiosíssima matre, quæ puérulum post patris óbitum, osténsa Christi Dómini e cruce pendéntis imágine, sic allocúta fertur: En fili, qui hinc in pósterum tibi futúrus est pater, cura ut per eam, quæ ipsi placet, vjam incédas. Adoléscens pistóriam artem dídicit; sed ad majóra se vocári præséntiens cœpit inter laborióse artis exercítium ad stúdium incúmbere litterárum. Bis Romam pedes ad límina Apostolórum peregrinátus est, bis étiam rerum cæléstium contemplatióni libérius vacatúrus in erémum secéssit. Verum Deo illum ad altióra vocánte, intermíssa litterárum stúdia in Vindobonénsi Universitáte repétiit navitérque prosecútus est; donec ex purióre fonte sacras disciplínas haustúrus Romam tértium remigrávit. Ibi in sancti Alfónsi. famíliam singulári Dei providéntia cooptátus est,quam ipsíus ópera in díssitas regiónes propagátum iri sanctus idem Parens divíno instínctu præsénsit ac palam prænuntiávit.

\RVCiv 


\lectio{Lectio v}
\y{A}{bsolúto} cum magno spíritus fervóre tirocínio ac présbyter factus, in septentrionáles Európæ províncias a Sede Apostólica missus est. Erécto Varsáviæ in Polónia primo suæ Congregatiónis domicílio, et plúribus sibi adjúnctis sóciis, incredíbile est, quam copióso ibi fructu sacerdotália múnera ultra vicénnium expléverit; nihil labóris recúsans, sive ut cathólicos ad bonos mores et pietátis cultum revocáret, sive ut hæréticos et Judæos ad veram fidem addúceret. Multa étiam ad religiósam juventútis institutiónem pr\'æstitit. Orphanórum sortem miserátus eréxit hospítium, ubi misérrimos quosque patérna caritáte alébat; visus subínde stipem eórum causa emendicáre, imo nec fáciem ab increpántibus et in se conspuéntibus avértere. Inter has autem curas semper coram Deo ámbulans, Deíparæ Vírgini, cujus Rosárium assídue recitábat, singuláriter devótus, rectus, simplex, húmilis, cunctis affábilis et benígnus, ómnibus velut sanctitátis exémplar júgiter prelúxit.

\RVCv 

\lectio{Lectio vi}
\y{D}{eo} oli confísus, plures institúti sui domus váriis in regiónibus eréxit. Ipse vero multas ob témporum iniquitátem acerbitátes perpéssus, et e Polónia tandem cum suis expúlsus, secéssit Vindobónam, ubi duódecim postrémis vitæ annis ad Dei glóriam strénue operári perseverávit. Præter sédulam enim Sanctimoniálium Vírginum curam totus in eo fuit, ut prædicatióne verbi Dei, sacraménti Pœniténtiæ administratióne quam plúrimas a salúte dévias ánimas Christo lucrifáceret. Quem caritátis ignem tum sacerdótibus, tum máxime seléctæ júvenum cohórti, quorum complúres suum Congregatióni nomen dedérunt, infúndere adlaborávit. Illud autem indeféssa sua ópera consecútus est, ut languescéntem pópuli fidem catholicúmque spíritum, non parum illa aetáte imminútum, corroboráret; talíque succéssu étiam in Archidiœcési Friburgénsi in paróchiis Triberg et Jestétten laborávit; dignus proptérea, quem Pjus séptimus virum apostólicum, Vindobonénsis cleri decus Ecclesi\'æque cólumen appelláret. Méritis tandem plenus et prodigiórum aliorúmque cæléstium charísmatum múnere illústris, septuagenárius obdormívit in Dómino, Idibus Mártiis, anno millésimo octingentésimo vigésimo. Eum Leo décimus tértius miráculis clarum Beatórum catálogo, Pjus vero décimus novis fulgéntem signis Sanctórum fastis accénsuit.

\RVCvi 

\rubric{In III Nocturno Lectiones de Homilia in Evang. \black{Nolite timére,} de Communi Conf. non Pont. 2 loco.}

\rubric{IX Lectio de Homilia Feriæ.}

\hora{Ad Laudes}

\VRCii 

\BC 

\pars{Oratio}
\y{D}{eus}, qui beátum Cleméntem Maríam miro fídei róbore et invíctæ constántiæ virtúte decorásti: ejus méritis et exémplis fac nos, qu\'æsumus, ita fortes in fide et caritáte fervéntes; ut pr\'æmia consequámur ætérna. Per Dóminum nostrum.

\rubric{Et fit Com. Feriæ.}

\hora{In II Vesperis}

\VRCiii 

}

\MiiC 

\rubric{Et fit Com. Feriæ.}



\mens{Festa Aprilis}
\dieii{Die 21 Aprilis}{S. Conradi a Parzham}{Confessoris}{Duplex}

\V Amávit eum Dóminus, et ornávit eum. \TPA
\R Stolam glóriæ índuit eum. \TPA 

\MiC \TPA

\pars{Oratio}
\y{D}{eus}, qui misericórdiæ tuæ jánuam fidélibus patére voluísti: te súpplices exorámus; ut, intercedénte beáto Conrádo Confessóre tuo, temporália subsídia nobis tríbuas et ætérna. Per Dóminum nostrum.

\rubric{Et fit Com. S. Anselmi Ep., Conf. et Eccl. Doct.}

\A O Doctor óptime, Ecclésiæ sanctæ lumen, beáte Ansélme, divínæ legis amátor, deprecáre pro nobis Fílium Dei. \TPA 

\V Justum dedúxit Dóminus per vias rectas. \TPA 
\R Et osténdit illi regnu Dei. \TPA 

\pars{Oratio}
\yb{D}{eus}, qui pópulo tuo ætérnæ salútis beátum Ansélmum minístrum tribuísti: præsta, quǽsumus; ut quem Doctórem vitæ habúimus in terris, intercessórem habére mereámur in cælis. Per Dóminum nostrum.

\rubric{In I Nocturno, si dicendæ sint de Communi, Lectiones \black{Beátus vir,} de Communi Conf. non Pontificis 1 loco.}

\nocturn{In II Nocturno}
\lectio{Lectio iv}
\Y{C}{onrádus}, in ruráli pr\'ædio ad óppidum Parzham in diœcési Passaviénsi, piis honestísque paréntibus natus, inde a púero modéstia et solitúdinis amóre, quanta futúrus esset sanctitáte, præmonstrávit. Mirus jam tum in eo elucébat devotiónis fervor, máxime cum in ecclésia oráret; quam longe sitam, neque itínere neque frígore frequentáre solébat. Summo erga beátam Vírginem amóre inflammátus, quotídie ejus Rosárium assídue iterábat, diebúsque festis pers\'æpe ad remotióra ipsíus Deíparæ sanctuária pergébat: ex quibus pedéstribus peregrinatiónibus, júgiter orándo suscéptis, domum remeábat sub vésperum fere semper jejúnus. Integra juventúte in agrórum cultára transácta, ut árctius Deo adhæréret eíque libérius famularétur, mundo valedícere státuit; quaprópter perámplo património abdicáto, annum agens primum supra trigésimum inter fratres láicos órdinis Minórum Capuccinórum adnumerári obtínuit.

\RVCiv 

\lectio{Lectio v}
\y{S}{tatim} ab emíssa professióne, ad sanctæ Annæ convéntum civitátis Vetœttíngæ missus fuit; locus autem célebris est et frequentíssimus inter cétera Germánicæ diciónis Mariána sanctuária, ádeo ut plura fidélium centéna míllia illuc síngulis annis convéniant. Ob tantum ad eándem civitátem concúrsum, perdiffícile in cœnóbio evádit janitóris offícium; quod Conrádus, cui hujúsmodi munus statim ab ejus advéntu fúerat demandátum, ad óbitum usque retínuit, et tamquam exémpli apostolátum explévit. Impiger in labóre, parcus in loquéndo, erga páuperes largus, in peregrínis excipiéndis juvandísque promptus, per annos quadragínta et ámplius ostiárii onus æquanímiter gessit, máximo incolárum et peregrinórum emoluménto, in ómnibus tum ánime tum córporis necessitátibus.

\RVCv 

{\setstretch{0.99}
\lectio{Lectio vi}
\y{V}{irtútibus} ómnibus præclárus, siléntium summópere adamávit. Per diem subsecíva témporis moménta insumébat in quodam tuguríolo, jánuæ propínquo, e quo sanctíssimam Eucharístiam invísere et adoráre póterat; de nocte vero horas subripiébat somno, quas in fratrum conditório vel in ecclésia oratióni dabat. Etsi consensuísset, quiéti numquam se trádidit, sed in labóribus piísque exercítiis ad extrémum usque perréxit. Revéra quadam die cum plúribus peregrinántium turmis satisfecísset, agnóvit córporis víribus prorsus destítui; ac tríduo post, undécimo Kaléndas Maji, obdormívit in Dómino, anno millésimo octingentésimo nonagésimo quarto, ætátis suæ septuagésimo quinto. Hunc Dei fámulum, heróicis virtútibus et miráculis clarum, Pjus Papa undécimus anno millésimo nongentésimo trigésimo inter Beátos réttulit; novísque miráculis fulgéntem, quadriénnio post, idem Summus Póntifex Sanctórum fastis solémniter adscrípsit.

\RVCvi 

\rubric{In III Nocturno Homilia in Evangel. \black{Sint lumbi,} de Communi 1 loco.}

\columnbreak
\pro{Pro S. Anselmo:}
\lectio{Lectio ix}
\yb{A}{nsélmus}, Augústæ Prætóriæ in fínibus Itáliæ, nobílibus et cathólicis paréntibus natus, adoléscens, pátria et bonis ómnibus derelíctis, in monastério Beccénsi órdinis sancti Benedícti emíssa regulári professióne, in lítteris et virtútibus assequéndis mirum in modum profécit. Régibus, epíscopis veneratióni fuit, et sancto Gregório séptimo étiam accéptus, qui tunc, persecutiónibus agitátus, lítteras amóris plenas ad eum dedit, se et Ecclésiam ejus oratiónibus comméndans. Defúncto Lanfránco archiepíscopo Cantuariénsi, ejus olim præceptóre, ad eiúsdem ecclésiæ régimen vocátus, verbo et exémplo, scriptis et concíliis celebrátis, prístinam pietátem et ecclesiásticam disciplínam redúxit. Sed cum mox Willélmus rex vi et minis jura Ecclésiæ usurpáre tentásset, ípseque invícte restitísset, bonórum direptiónem et exsílium passus, Romam ad Urbánum secúndum se cóntulit. A quo honorífice excéptus et summis láudibus ornátus, in Barénsi concílio Spíritum Sanctum étiam a Fílio procedéntem, contra Græcórum errórem, innúmeris Scripturárum et sanctórum Patrum testimóniis propugnávit. Post mortem Willélmi, ab Henríco rege ejus fratre in Angliam revocátus obdormívit in Dómino.

\Te

\hora{Ad Laudes}

\V Justum dedúxit Dóminus per vias rectas. \TPA 
\R Et osténdit illi regnu Dei. \TPA 

\BC \TPA

\pars{Oratio}
\y{D}{eus}, qui misericórdiæ tuæ jánuam fidélibus patére voluísti: te súpplices exorámus; ut, intercedénte beáto Conrádo Confessóre tuo, temporália subsídia nobis tríbuas et ætérna. Per Dóminum nostrum.

\rubric{Et fit Com. S. Anselmi:}

\AiiCP \TPA

\V Amávit eum Dóminus, et ornávit eum. \TPA
\R Stolam glóriæ índuit eum. \TPA 

\pars{Oratio}
\yb{D}{eus}, qui pópulo tuo ætérnæ salútis beátum Ansélmum minístrum tribuísti: præsta, quǽsumus; ut quem Doctórem vitæ habúimus in terris, intercessórem habére mereámur in cælis. Per Dóminum nostrum.

}

\hora{In II Vesperis}

\V Justum dedúxit Dóminus per vias rectas. \TPA 
\R Et osténdit illi regnu Dei. \TPA 

\MiiC \TPA

\rubric{Et fit Com. sequentis:}

\AiMTP 

\VRMTPi 

\pars{Oratio}
\yb{B}{eatórum} Mártyrum paritérque Pontíficum Sotéris et Caji nos, qu\'æsumus, Dómine, festa tueántur: et eórum comméndet orátio veneránda. \red{(}Per Dóminum nostrum.\red{)}

\rubric{Deinde Com. S. Anselmi:}

\A O Doctor óptime, Ecclésiæ sanctæ lumen, beáte Ansélme, divínæ legis amátor, deprecáre pro nobis Fílium Dei. \TPA 

\V Amávit eum \red{et Oratio} Deus, qui pópulo, \red{ut supra.}



\die{Die 23 Aprilis}{S. Georgii}{Martyris}{\\\rubric{In Diœcesi:} Semiduplex\\\rubric{In urbe Friburgensi:}\\Duplex majus}

{\setstretch{0.975}
\VRMTPi 

\MiMTP

\pars{Oratio}
\y{D}{eus}, qui nos beáti Geórgii Mártyris tui méritis et intercessióne lætíficas: concéde propítius; ut qui tua per eum benefícia póscimus, dono tuæ grátiæ consequámur. Per Dóminum nostrum.

\rubric{Et fit Com. præcedentis:}

\AiiiMTP 

\VRMTPiii 

\pars{Oratio}
\yb{B}{eatórum} Mártyrum, paritérque Pontíficum Sotéris et Caji nos, qu\'æsumus Dómine, festa tueántur: et eórum comméndet orátio veneránda. Per Dóminum nostrum.

\nocturn{In II Nocturno}
\lectio{Lectio iv}
\Y{G}{eórgjus} nobilibus paréntibus in Cappadócia natus, sancte ab illis et religióse educátus est. Iis vita functis, adhuc adoléscens tribúnus mílitum factus, et áliis insignítus honóribus, sub Diocletiáno C\'æsare militávit. Verum cum ediíctis imperiálibus persecutiónem in christiános cérneret ácrius desævíre, id indígne ferens, opportúnum quoque fore illud tempus exístimans ad salútem et ad martýrii glóriam, facultátibus suis in páuperes erogátis servísque libertáte donátis, zelo fídei ac religiónis accénsus, imperatórem adit Christúmque verum Deum líbere conféssus, ipsum de sua ímpia in christiános crudelitáte constánter accúsat.

\RVMTPiv

\lectio{Lectio v}
\y{A}{dmirátus} veheménter imperátor, quod Geórgjus præter omnem opiniónem christiánus esset, primo blandis illum verbis ac magnis pollicitatiónibus, deínde minis pœnarúmque terróribus a fídei senténtia conátur avértere. Sed cum nihil profíceret, nec ullo modo viri constántiam posset de pio et sancto propósito dimovére, jubet ipsum primo in cárcerem trudi ibíque diu afflígi: tum verbéribus cæsum, rota mucrónibus obarmáta laniári, deínde torréri sartágine variéque torquéri. Quam ille tormentórum vim ubi álacri ac forti ánimo tulísset, índeque incólumis mirabíliter evasísset, plúrimos suo exémplo ad fídei constántiam excitávit.

\RVMTPv 

}

\lectio{Lectio vi}
\y{I}{mpjus} autem persecútor, cum exímiam hujúsmodi Mártyris fortitúdinem non divínæ virtúti, sed mágicis ártibus adscríberet, ipsum várie tortum atque in senténtia nihilóminus persisténtem, secúri pércuti jubet. Geórgjus ígitur fórtiter data cervíce, martýrii coróna decorátus est Nono Kaléndas Maji, anno salútis humánæ ducentésimo nonagésimo. Córporis ejus relíquiæ divérsas in regiónes delátæ, máximo ab íncolis habéntur honóre. Caput Romæ in Diaconía sancti Geórgii ad Velábrum honorífice asservátur; et ipsíus Mártyris nomen non solum in Oriénte, ubi passus est, sed et in occíduis oris ubíque claríssimum est. Antiquiórem inprímis intra Galliárum æque ac Germániæ fines nóminis ejus veneratiónem testántur tum cœnóbia multa, tum innúmeræ fere parochiáles ecclésiæ, quæ passim sub ejus invocatióne, plúribus abhinc s\'æculis Deo dicátæ deprehendúntur.

\RVMTPvi 

\nocturn{In III Nocturno}
\scriptura{Léctio sancti Evangélii secúndum Joánnem}
\lectiocap{Lectio vii}{Cap. 15, 1-7}
\y{I}{n} illo témpore: 
Dixit Jesus discípulis suis:
Ego sum vitis vera: et Pater meus agrícola est.
Et réliqua.

\scriptura{De Homilía sancti Augustíni Epíscopi}
\ex{Tractatus 80 in Joannem}
\Y{D}{énique} cum de Patre, tamquam de agrícola dixísset, quod infructuósos pálmites tollat, fructuósos autem purget, ut plus áfferant, fructum, contínuo étiam se ipsum mundatórem pálmitum osténdens: Jam vos inquit, mundi estis propter sermónem, quem locútus sum vobis. Ecce, et ipse mundátor est pálmitum, quod est agrícolæ, non vitis offícium, qui étiam pálmites operários suos fecit: nam etsi non dant increméntum, impéndunt tamen áliquod adjuméntum, sed non de suo: quia sine me, inquit, nihil potéstis fácere.

\RVMTPvii 

\lectio{Lectio viii}
\y{A}{udi} étiam ipsos confiténtes: Quid autem est Apóllo? quid autem Paulus? Minístri, per quos credidístis, et unicuíque sicut Dóminus dedit. Ego plantávi, Apóllo rigávit. Et hoc ergo, sicut unicuíque Dóminus dedit. Non ítaque de suo. Jam vero quod séquitur: Sed Deus increméntum dedit; non per illos, sed per seípsum facit. Excédit hoc humánam humilitátem, excédit angélicam sublimitátem: nec omníno pértinet, nisi ad agrícolam Trinitátem.

\RVMTPviii 

\lectio{Lectio ix}
\y{J}{am} vos mundi estis: mundi scílicet, atque mundándi: neque enim nisi mundi essent, fructum ferre potuíssent. Et tamen omnem qui fert fructum, purgat agrícola, ut fructum plus áfferat. Fert fructum, quia múndus est: atque ut plus áfferat, purgátur adhuc. Quis enim est in hac vita sic mundus, ut non sit magis magísque mundándus? Ubi si dixérimus, quia peccátum non habémus, nos ipsos sedúcimus, et véritas in nobis non est. Mundet ítaque mundos, hoc est fructuósos, ut tanto sint fructuosióres, quanto fúerint mundióres

\Te 

\hora{Ad Laudes}

\VRMTPii 

\BMTP 

\columnbreak
\pars{Oratio}
\y{D}{eus}, qui nos beáti Geórgii Mártyris tui méritis et intercessióne lætíficas: concéde propítius; ut qui tua per eum benefícia póscimus, dono tuæ grátiæ consequámur. Per Dóminum nostrum.

\rubric{In Diœcesi Vesperæ de sequenti, Com. præcedentis:}

\AiiiMTP 

\V Pretiósa, \red{et Oratio} Deus, qui nos, \red{ut supra.}

\rubric{In urbe Friburgensi in Vesperis Com. sequentis:}

\AiMTP 

\VRMTPi 

\pars{Oratio}
\yb{D}{eus}, qui beátum Fidélem seráphico spíritus ardóre succénsum in veræ fídei propagatióne martýrii palma et gloriósis miráculis decoráre dignátus es: ejus, qu\'æsumus, méritis et intercessióne, ita nos per grátiam tuam in fide et caritáte confírma; ut in servítio tuo fidéles usque ad mortem inveníri mereámur. Per Dóminum nostrum.



\die{Die 24 Aprilis}{\rubric{In territorio Hohenzollerano:}\\S. Fidelis a Sigmaringa}{Martyris}{Duplex majus}

\rubric{Omnia ut in Breviario hac die.}



\die{Die 26 Aprilis}{S. Trudperti}{Martyris}{Duplex}

\AiMTP

\VRMTPi 

\pars{Oratio}
\y{P}{ræsta} qu\'æsumus omnípotens Deus: ut intercedénte beáto Trudpérto Mártyre tuo, et a cunctis adversitátibus liberémur in córpore, et a pravis cogitatiónibus mundémur in mente. Per Dóminum nostrum.

\rubric{Et fit Com. Ss. Cleti et Marcellini Pontificum et Martyrum:}

\AiiMTP

\VRMTPii

\pars{Oratio}
\yb{B}{eatórum} Mártyrum, paritérque Pontíficum Cleti et Marcellíni nos, Dómine, fóveat pretiósa conféssio: et pia júgiter intercéssio tueátur. Per Dóminum nostrum.

\nocturn{In II Nocturno}
\lectio{Lectio iv}
\Y{T}{rudpértus} génere Scotus regiáque stirpe proféctus, religiónis causa, Theodóro post Joánnem hoc nómine quartum Pontífice Máximo, Romam divíno amóre incénsus se cóntulit, ubi ómnia urbis loca sanctitáte conspícua circúmiens, flagrantíssimo pietátis afféctu venerabátur. Inde in pátriam rédiens, quam vitæ suæ ratiónem institúeret, eo in itínere sério cogitáre cœpit. Et vero magni illíus Patriárchæ Abrahæ exémplum sequi decérnens, ómnibus hujus s\'æculi divítiis, propínquis patrióque solo ob Christi amórem relíctis, in solitárium quemdam Brisgójæ locum contemplatióni aptíssimum, impetráta ab Othbérto Habsburgénsis famíliæ, dómino loci, licéntia habitándi, se recépit; ubi in hunc usque diem monastérium ejus nómine insignítum invísitur.

\RVMTPiv

\lectio{Lectio v}
\y{O}{thbérto}, a quo prædíctum locum dono accéperat, ob ejus singuláres virtútes carus mirum in modum fuit, quinímo auxíliis opibúsque ejus adjútus, monastérium ibídem constrúxit. Tanta autem humilitáte pr\'æditus dícitur, ut iis, qui in ædifício construéndo óperam dabant, ministráre non erubúerit. Oratióni instánter vacábat, castitátem perpétuo cóluit, jejúniis vigiliisque corpus maceráre non desistébat, mentem divínis contemplatiónibus assídue exércuit. Hoc præter cétera hábuit, ut Dei amóre repléri vehementíssime sitíret. Ecclésiam prætérea suis mánibus ac labóribus magnis ædificáre aggréssus ópere formáque notábili perfécit, quam Sanctórum Apostolórum Petri et Pauli nómine Martínus Constantiénsis epíscopus solémni ritu consecrávit.

\RVMTPv 

\lectio{Lectio vi}
\y{Q}{uibus} rebus felíciter conféctis, humáni géneris hostis sanctíssimum virum Trudpértum, insígnem Christi mílitem, frequéntius more suo ex insídiis aggréssus, váriis tentatiónibus oppugnábat; quas ille jejúnio et oratióne superábat, hisce pietátis armis invíctus. Advértens vero dæmon se in eo vincéndo óperam pérdere, tímuit ne, si diu víveret hic Christi athléta, majóres de ipso quotídie reportáret triúmphos; ídeo immísit in quorámdam hóminum scelestórum ánimos ejus interiméndi consílium, quo indúcti innocentíssimum virum per insídias interfecérunt sexto Kaléndas Maji, anno post Christum natum sexcentésimo quadragésimo tértio. Ejus corpus Othbértus honorífice (ut sanctum Mártyrem decébat) sepeliéndum curávit. Anno autem septingentésimo septuagésimo, cum ejus sanctíssimam vitam ac mirácula, qua cooperánte Dómino et ante et post martýrium multa édidit, diligénter Stéphanus tértius. Póntifex Máximus cognovísset, solémni ritu illum in sanctórum Mártyrum númerum rétulit.

\RVMTPvi 

\nocturn{In III Nocturno}
\scriptura{Léctio sancti Evangélii secúndum Joánnem}
\lectiocap{Lectio vii}{Cap. 15, 1-7}
\y{I}{n} illo témpore: 
Dixit Jesus discípulis suis:
Ego sum vitis vera: et Pater meus agrícola est.
Et réliqua.

\scriptura{De Homilía sancti Augustíni Epíscopi}
\ex{Tractatus 81 in Joannem}
\Y{V}{item} se dixit esse Jesus, et discípulos suos pálmites, et agrícolam Patrem: unde jam pridem, sicut potúimus, disputátum est. In hac autem lectióne, cum adhuc de seípso, qui est vitis, et de suis palmítibus, hoc est discipulis, loquerétur: Manéte, inquit, in me et ego in vobis: non eo modo illi in ipso, sicut ipse in illis. Utrúmque autem prodest non ipsi, sed illis: ita quippe in vite sunt pálmites, ut viti non cónferant, sed inde accípiant unde vivant; ita vero vitis in palmítibus, ut vitále aliméntum subminístret eis, non sumat ab eis.

\RVMTPvii

\lectio{Lectio viii}
\y{A}{c} per hoc, et manéntem in se habére Christum, et manére in Christo, discípulis prodest utrúmque, non Christo. Nam præcíso pálmite, potest de viva radíce álius pulluláre; qui autem præcísus est, sine radíce non potest vivere. Dénique adjúngit, et dicit: Sicut palmes non potest ferre fructum a semetípso, nisi mánserit in vite: sic nec vos, nisi in me manséritis. Magna grátiæ commendátio, fratres mei: corda ínstruit humílium, ora óbstruit superbórum.

\RVMTPviii 

{\setstretch{0.96}

\lectio{Lectio ix}
\y{E}{cce} cui, si audent, respóndeant, qui ignorántes Dei justítiam, et suam voléntes constitúere, justítiæ Dei non sunt subjécti. Ecce cui respóndeant, sibi placéntes, et ad bona ópera faciénda Deum sibi necessárium non putántes? Nonne huic resístunt veritáti, hómines mente corrúpti, réprobi circa fidem, qui respóndent, et loquúntur iniquitátem, dicéntes: A Deo habémus, quod hómines sumus; a nobis ipsis autem, quod justi sumus? Hæc sunt inánia præsumptiónis vestrae: et si est in vobis ullus sensus, horréte: qui enim a semetípso se fructum exístimat ferre, in vite non est; qui in vite non est, in Christo non est.

\Te 

\hora{Ad Laudes}

\VRMTPii 

\BMTP

\pars{Oratio}
\y{P}{ræsta} qu\'æsumus omnípotens Deus: ut intercedénte beáto Trudpérto Mártyre tuo, et a cunctis adversitátibus liberémur in córpore, et a pravis cogitatiónibus mundémur in mente. Per Dóminum nostrum.

\rubric{Et fit Commemoratio Ss. Cleti et Marcellini:}

\AiMTP 

\VRMTPi 

\pars{Oratio}
\yb{B}{eatórum} Mártyrum, paritérque Pontíficum Cleti et Marcellíni nos, Dómine, fóveat pretiósa conféssio: et pia júgiter intercéssio tueátur. Per Dóminum nostrum.

\rubric{Vesperæ de sequenti, Com. præcedentis.}

}



\die{Die 27 Aprilis}{S. Petri Canisii}{Confessoris et Ecclesiæ Doctoris}{Duplex II classis}

\rubric{Omnia de Communi Confessoris non Pontificis, præter ea, quæ in Breviario hac die habentur propria.}

\rubric{In I Vesperis fit tantum Commemoratio præcedentis:}

\AiiiMTP 

\V Pretiósa \red{et Oratio} Præsta, qu\'æsumus, \red{ut supra.}

\rubric{Lectiones I Nocturni \black{Sapiéntiam,} de Communi Doctorum.}

\rubric{In II Vesperis fit tantum Commemoratio sequentis.}

\end{multicols}

\pagebreak

\begin{multicols}{2}
\privdieii{Feria IV infra Hebdom. IV post Octavam Paschæ}{In Dedicatione Ecclesiæ Cathedralis}{}{Duplex I classis cum Octava communi}

\rubric{Omnia de Communi Dedicationis Ecclesiæ, præter sequentia:}

\lectio{Lectio vi}
\ex{(ex publicis documentis)}
\Y{T}{emplum} hoc, miríficum opus artis góthicæ, per tria fere médii ævi s\'æcula, tam regnántium quam cívium sédula cura, largitiónibus, labóribus exstrúctum, sub die quarta et quinta Decémbris, anni millésimi quingentésimi décimi tértii, fuit Deo dedicátum in honórem Dóminæ Nostræ in cælum Assúmptæ, Dedicatiónis autem Festum póstea celebrándum per Guliélmum Cardinálem, apostólicæ Sedis Legátum, Domínicæ quintæ post Pascha affíxum fúerat. Sed, témporum decúrsu, ob rerum vicissitúdines, suppréssa Diœcési Constantiénsi, in urbe Friburgénsi Archiepiscopális et Metropolitána Sedes erécta est, atque ita ista Ecclésia, nobilitáte et pulchritúdine in tota regióne princeps, facta est cathedrális: quam Románi Pontífices spirituálibus grátiis et privilégiis auxérunt. Benedíctus vero Papa décimus quintus ejus Dedicatiónis Anniversárium, ante diem Festum Ascensiónis Dómini Nostri, quotánnis in tota Diœcési státuit recoléndum. Grátias proínde omnipoténti Deo referámus, qui templa manufácta elégit et sanctificávit, ut nomen ejus esset ibi, eumque rogémus, ut omnes, qui huc deprecatúri convénerint, intercedénte beáta María Vírgine, consolatiónis grátiam consequántur.

\rubric{Infra Octavam et in die Octava Antiphonæ et Psalmi ad omnes Horas et Versus Nocturnorum de occurrenti Hebdomadæ die ut in Psalterio; reliqua ut in Festo præter Lectiones, quæ in 1 Nocturno dicuntur de Scriptura occurrenti cum suis Responsoriis de Tempore, in II et III in Communi singulis diebus assignantur propriæ.}

\die{Feria IV sequenti}{In Octava Dedicationis Ecclesiæ Cathedralis}{}{Duplex majus}
\chead{\trim{Feria IV infra Hebdom. V post Octavam Paschæ}{In Octava Dedicationis Ecclesiæ Cathedralis}}

{\setstretch{0.99}
\rubric{Pro utentibus Breviario antiquo Lectiones III Nocturni de Feria IV infra Octavam.}

\pro{Pro Vigilia Ascensionis:}
\lectio{Lectio ix}
\scriptura{Léctio sancti Evangélii secúndum Joánnem}
\lectiocap{Lectio vii}{Cap. 17, 1-11}
\y{I}{n} illo témpore: 
Sublevátis Jesus óculis in c\'ælum, dixit:
Pater, venit hora, clarífica Fílium tuum.
Et réliqua.

\scriptura{De Homilía sancti Augustíni Epíscopi}
\ex{Tractus 104 in Joannem, sub med.}
\yb{P}{óterat} Dóminus noster, unigénitus et coætérnus Patri, in forma servi, et ex forma servi, si hoc opus esset, oráre siléntio: sed ita se Patri exhibére vóluit precatórem, ut meminísset nostrum se esse doctórem. Proínde eam, quam fecit oratiónem pro nobis, notam fecit et nobis: quóniam tanti magístri non solum ad ipsos sermocinátio, sed étiam pro ipsis ad Patrem orátio, discipulórum est ædificátio: et si illórum, qui hæc dicta áderant auditúri, profécto et nostra, qui fuerámus conscrípta lectúri.

\Te 

\rubric{Ad Laudes fit Com. Vigiliæ Ascensionis:}

\A Pater, venit hora, clarífica Fílium tuum claritáte quam hábui, priúsquam mundus esset, apud te, allelúja.

\V In resurrectióne tua, Christe, allelúja.
\R Cæli et terra læténtur, allelúja.

\pars{Oratio}
\yb{D}{eus}, a quo bona cuncta procédunt, largíre supplícibus tuis: ut cogitémus, te inspiránte, quæ recta sunt; et, te gubernánte, éadem faciámus. Per Dóminum nostrum.

\rubric{Vesperæ de sequenti Festo Ascensionis sine ulla Commemoratione.}

}

\mensii{Festa Maji}
\dieii{Die 16 Maji}{S. Joannis Nepomuceni}{Presbyteri et Martyris}{Duplex majus}

\VRMTPi 

\MiMTP 

\pars{Oratio}
\y{D}{eus}, qui ob invíctum beáti Joánnis sacramentále siléntium, nova Ecclésiam tuam martýrii coróna decorásti: da nobis ejus intercessióne et exémplo, linguam caute custodíre; ac ómnia pótius mala, quam ánimæ detriméntum, in hoc s\'æculo toleráre. Per Dóminum nostrum.

\rubric{Et fit Com. præcedentis:}

\AiiiCA

\VRCiiiA

\pars{Oratio}
\yb{D}{eus}, qui, ad christiánam páuperum eruditiónem et ad juvéntam in via veritátis firmándam, sanctum Joánnem Baptístam Confessórem excitásti, et novam per eum in Ecclésia famíliam collegísti: concéde propítius; ut ejus intercessióne et exémplo, stúdio glóriæ tuæ in animárum salúte fervéntes, ejus in cælis corónæ partícipes fíeri valeámus. \red{(}Per Dóminum nostrum.\red{)}

\rubric{Deinde Com. S. Ubaldi Ep. et Conf.:}

\AiCPA 

\VRCPiA

\pars{Oratio}
\yb{A}{uxílium} tuum nobis, Dómine, quǽsumus, placátus impénde: et, intercessióne beáti Ubáldi Confessóris tui atque Pontíficis, contra omnes diáboli nequítias déxteram super nos tuæ propitiatiónis exténde. Per Dóminum nostrum.

\nocturn{In II Nocturno}
\lectio{Lectio iv}
\Y{J}{oánnes} Nepomúci Bohémiæ óppido, unde Nepomucéni cognómen duxit, a paréntibus ætáte provéctis, non sine futúræ sanctitátis præságio, flammis supra nascéntis domum mirabíliter collucéntibus, ortus est. Cum infans in gravem morbum incidísset, beátæ Vírginis ope, cui natum paréntes referébant accéptum, e vitæ perículo evásit incólumis. Egrégia índole, piáque institutióne cæléstibus indíciis obsequénte, inter sanctas religiosásque exercitatiónes puerítiam egit; nam ecclésiam frequénter adíre, ac sacerdótibus ad aras operántibus ministráre in delíciis habébat. Zatécii politióribus lítteris ad humanitátem informátus, Pragæ vero gravióribus disciplínis excúltus, philosóphiæ theológiæ, sacrorúmque cánonum magistérium et láuream eméruit. Sacerdótio initiátus, atque a Sciéntia Sanctórum ad lucra animárum rite comparátus, ministério verbi Dei se pénitus addíxit. Cum igitur in vítiis exstirpándis, et revocándis in vjam salútis errántibus, eloquéntia et pietáte úberes éderet fructus, inter canónicos metropolitánæ ecclésiæ Pragénsis cooptátus, mox sibi demandátam Evangélii coram rege Wencesláo quarto prædicándi provínciam suscépit, eo succéssu, ut Joánnis suásu multa rex fáceret, magnóque in honóre ejus virtútes habéret. Conspícuas tamen, quas ille óbtulit dignitátes, Dei servus, ne a divíni verbi præcónio avocarétur, constantíssime recusávit.

{\setstretch{0.98}

\RVMTPiv 

\lectio{Lectio v}
\y{R}{égiis} illum eleemósynis in páuperes erogándis præféctum Joánna regína consciéntiæ sibi moderatórem ascívit. Cum autem Wencesláus ab offício institutóque decessísset, atque in vítia præceps abriperétur, piæ; autem cónjugis obtestatiónes et mónita graváte ferret, conténdere ausus est, ut in sacramentáli judício sacerdóti crédita regínæ arcána sibi a Joánne panderéntur. At Dei miníster blandítiis primum, torméntis deínde et cárceris squalóre tentátus, nefárice regis cupiditáti fórtiter óbstitit. Furéntem tamen Wenceslái ánimum cum ab exsecrándo propósito nec humána, nec divína jura deterrérent, suprémum agónem, quem instáre sibi athléta Christi nóverat, pópulo in contióne de impendéntibus étiam regni calamitátibus admónito, non obscúre prænuntiávit. Mox Boleslávjam proféctus, ad beátæ Vírginis imáginem antíquo cultu célebrem, cæléste præsídium ad certándum bonum certámen effúsis précibus implorávit. Inde véspere reverténtem in pervigílio Domínicæ Ascensiónis rex e fenéstra conspicátus arcéssit; cumque veheméntius urgéret, et próximam in aquis, si obluctári pérgeret, submersiónem intentáret, Joánnes invícta constántia terróres minásque refutávit. Itaque regis império in Moldávam flumen Pragam intérfluens noctu dejéctus, illústrem martýrii corónam est consecútus.

\RVMTPv 

}

\lectio{Lectio vi}
\y{S}{acrílegum} fácinus clam patrátum, et Mártyris glóriam insígne prodígium divínitus patefécit. Ubi enim exánime corpus secúndo flúmine vehi cœpit, ardéntes faces aquis supernatántes et discurréntes apparuérunt. Quam ob rem ex aréna postrídie mane corpus elátum, canónici deinde, regis iram nihil timéntes, in metropolitánam ecclésiam solémni pompa intulérunt, et sepultáre mandárunt. Cum autem in dies invícti sacerdótis memória miráculis, et máxima fidélium eórum præcípue qui fama periclitántur, veneratióne crésceret, post annos demum ámplius trecéntos, in jurídica recognitióne córporis, quod sub humo támdiu jacúerat, lingua ejus incorrúpta et vívida repérta est: quæ sexto post anno judícibus a Sede apostólica delegátis exhíbita, novo prodígio reṕente intúmuit, et subobscúrum rubórem in purpúreum commutávit. His ítaque aliísque signis rite probátis, Benedíctus décimus tértius Póntifex Máximus die décima nona mensis Mártii, anno salútis millésimo septingentésimo vigésimo nono, primum hunc sacramentális sigílli assertórem, arcáni fidem sánguine obsignántem, sanctórum Mártyrum catálogo adscrípsit.

\RVMTPvi 

\nocturn{In III Nocturno}
\scriptura{Léctio sancti Evangélii secúndum Matth\'æum}
\lectiocap{Lectio vii}{Cap. 10, 26-32}
\y{I}{n} illo témpore: 
Dixit Jesus discípulis suis:
Nihil est opértum, quod non revelábitur; et occúltum, quod non sciétur.
Et réliqua.

\scriptura{De Homilía sancti Hilárii Epíscopi}
\ex{Comment. in Matthæum, can. 10, post medium}
\Y{D}{óminus} diem judícii osténdit, quæ abstrúsam voluntátis nostræ consciéntiam prodet: et ea quæ nunc occúlta exstimántur, luce cognitiónis públicæ déteget. Igitur non minas, non consília, non potestátes insectántium monet esse metuéndas: quia dies judícii nulla hæc fuísse, atque inánia revelábit. Et quod dico vobis in ténebris, dícite in lúmine: et quod in aure audítis, prædicáte super tecta. Non légimus Dóminum sólitum fuísse nóctibus sermocinári, et doctrínam in ténebris tradidísse: sed quia omnis sermo ejus carnálibus ténebræ sunt, et verbum ejus infidélibus nox est.


\RVMTPvii 

\lectio{Lectio viii}
\y{I}{taque} id quod a se dictum est, cum libertáte fídei et confessiónis vult esse loquéndum. Idcírco quæ in ténebris dicta sunt prædicári jussit in lúmine: ut quæ secréto áurium commíssa sunt, super tecta, id est, excélso loquéntium præcónio audiántur. Constánter enim Dei ingerénda cognítio est, et profúndum doctrínæ evangélicæ secrétum in lúmine prædicatiónis apostólicæ revelándum: non timéntes eos, quibus cum sit licéntia in córpora, tamen in ánimam jus nullum est: sed timéntes pótius Deum, cui perdéndæ in gehénna et ánimæ et córporis sit potéstas.

\RVMTPviii 

\pro{Pro S. Ubaldo:}
\lectio{Lectio ix}
\yb{U}{báldus}, Eugúbii in Umbria nóbili génere natus, a primis annis pietáte et lítteris egrégie est institútus; jamque adoléscens, ut uxórem dúceret, sæpe tentátus, nunquam tamen a propósito servándæ virginitátis recéssit. Sacérdos efféctus, patrimónium suum paupéribus et ecclésiis distríbuit. Canonicórum regulárium órdinis sancti Augustíni institútum suscípiens, illud in pátriam tránstulit. Ab Honório secúndo summo Pontífice ecclésiæ Eugubínæ invítus præfícitur, et episcopális consecratiónis múnere decorátur. Factus forma gregis ex ánimo, de consuéta vivéndi ratióne nihil ádmodum immutávit, et in omni virtútum génere enítuit. Diútius infirmitátibus afflíctus, Deo indesinénter grátias agébat. Cum multis annis ecclésiam sibi commíssam summa cum laude gubernásset, sanctis opéribus et miráculis clarus, quiévit in pace.

\Te 

\hora{Ad Laudes}

\VRMTPii 

\BMTP 

\pars{Oratio}
\y{D}{eus}, qui ob invíctum beáti Joánnis sacramentále siléntium, nova Ecclésiam tuam martýrii coróna decorásti: da nobis ejus intercessióne et exémplo, linguam caute custodíre; ac ómnia pótius mala, quam ánimæ detriméntum, in hoc s\'æculo toleráre. Per Dóminum nostrum.

\rubric{Et fit Com. S. Ubaldi:}

\AiiCPA 

\VRCPiiA

\pars{Oratio}
\yb{A}{uxílium} tuum nobis, Dómine, quǽsumus, placátus impénde: et, intercessióne beáti Ubáldi Confessóris tui atque Pontíficis, contra omnes diáboli nequítias déxteram super nos tuæ propitiatiónis exténde. Per Dóminum nostrum.

\hora{In II Vesperis}

\VRMTPiii 

\MiiMTP 

\columnbreak
\rubric{Et fit Com. sequentis:}

\AiCA 

\VRiCA 

\pars{Oratio}
\yb{D}{eus}, qui beátum Paschálem Confessórem tuum mirífica erga Córporis et Sánguinis tui sacra mystéria dilectióne decorásti: concéde propítius; ut, quam ille ex hoc divíno convívio spíritus percépit pinguédinem, eándem et nos percípere mereámur: Qui vivis.

\rubric{Deinde fit Com. S. Ubaldi:}

\AiiiCPA 

\V Justum dedúxit \red{et Oratio} Auxílium tuum, \red{ut supra.}



\mens{Festa Junii}
\dieii{Die 5 Junii}{S. Bonifatii}{Episcopi et Martyris}{Duplex II classis}

\rubric{Omnia ut in Communi Martyrum Tempore Paschali, præter ea, quæ in Breviario hac die habentur propria.}

\rubric{In I Vesperis fit Commemoratio præcedentis.}

\rubric{In I Nocturno Lectiones \black{A Miléto,} de Communi unius Martyris extra Tempore Paschale 1 loco cum Responsoriis Temporis Paschalis.}

\rubric{In II Vesperis fit Commemoratio sequentis.}

\end{multicols}

\ornamentvi

\end{document}
