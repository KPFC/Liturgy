\documentclass[fontsize=9.5pt,paper=A6,twoside,BCOR=1mm,DIV=21,headinclude]{scrarticle}
\usepackage{breviarium}
\begin{document}
\renewcommand{\section}{}
\thispagestyle{empty}
\begin{multicols}{2}
%\dieii{Sabbato ante Domin. ultimam Octobris}{B. Mariæ Virg.\\Protectricis Rubricarum}{}{}
\privdieii{}{Beatæ Mariæ Virginis\\Protectricis Rubricarum}{}{}

\lectiocap{Introitus}{Is. 63, 2-3}
\Y{Q}{uare} ergo rubrum est induméntum tuum, et vestiménta tua sicut calcántium in torculári? Tórcular calcávi solus, et de géntibus non est vir mecum.

\red{Ps. 118, 97} Quomódo diléxi legem tuam, Dómine: tota die meditátio mea est. \V Glória Patri.

\pars{Oratio}
\y{D}{eus}, qui humília réspicis et alta a longe cognóscis: da fámulis tuis humilitátem beátæ Maríæ semper Vírginis puro corde sectári; quæ áudiens verbum Dei custodívit illud et præcépta tua semper servávit. Per Dóminum nostrum.

\scriptura{Léctio libri Sapiéntiæ}
\ex{Prov. 6, 20-23}
\y{C}{onsérva}, fili mi, præcépta patris tui, et ne dimíttas legem matris tuæ. Liga ea in corde tuo júgiter, et circúmda gútturi tuo. Cum ambuláveris, gradiántur tecum; cum dormiéris, custódiant te: et evígilans lóquere cum eis. Quia mandátum lucérna est, et lex lux, et via vitæ increpátio disciplínæ.

\lectiocap{Graduale}{Prov. 8, 34-36}
\red{B}eatus homo qui audit me, et qui vígilat ad fores meas quotídie: Qui me invénerit, invéniet vitam, et háuriet salútem a Dómino. 
\V Qui autem in me peccáverit, lædet ánimam suam; omnes qui me odérunt díligunt mortem.

\red{A}llelúja, Allelúja. \V \red{Sap. 8, 32} Nunc ergo, fílii, áudite me :
beáti qui custódiunt vias meas. \red{A}llelúja.

\scriptura{\CPs Sequéntia sancti Evangélii secúndum Lucam}
\ex{Cap. 11, 27-28}
\Y{I}{n} illo témpore: Loquénte Jesu ad turbas, extóllens vocem quædam múlier de turba dixit illi : Beátus venter qui te portávit, et úbera quæ suxísti. At ille dixit : Quinímmo beáti, qui áudiunt verbum Dei et custódiunt illud.

\lectiocap{Offertorium}{Ps. 68, 10}
\red{Z}elus domus tuæ comédit me et oppróbria exprobrántium tibi cecidérunt super me.

\pars{Secreta}
%\y{T}{ua}, Dómine, propitiatióne, et beátæ Maríæ semper Vírginis intercessióne, ad perpétuam atque præséntem hæc oblátio nobis profíciat prosperaitátem et pacem. Per Dóminum nostrum.
\y{H}{ujus}, Dómine, sacraménti percéptio peccatórum nostrórum máculas abstérgat: et,
%gloriósæ beátæ Maríæ semper Vírginis intercessióne
ejus gloriósæ Vírginis intercessióne, quæ legem orándi simúlque credéndi semper tuétur,
per obœdiéntiæ viam ad supérna nos regna perdúcat. Per Dóminum.
%nostrum.

\rubric{Præfatio de B. Maria Virg. \black{Et te in Festivitáte.}}

\lectiocap{Communio}{Ps. 18, 8-9}
\red{L}ex Dómini immaculáta, convértens ánimas. Præcéptum Dómini lúcidum, illúminans óculos.

\pars{Postcommunio}
\y{H}{æc} nos commúnio, Dómine, purget a crímine: et, intercedénte beáta Vírgine Dei Genitríce María, cæléstis remédii fáciat esse consórtes. Per eúndem Dóminum nostrum.

\end{multicols}

%\ornamentvi
\vspace{\fill}

{
	\hspace{\fill}	\scriptsize{non approbatum}

}

\end{document}
