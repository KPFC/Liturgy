\documentclass[fontsize=9pt,paper=A6,twoside,BCOR=1mm,DIV=22,headinclude]{scrarticle}
\usepackage{breviarium}
\begin{document}
\titulum{Breviarium Monasticum}{In Festo Pentecostes}{Ad Matutinum}

%\vspace{-1.5em}

\begin{multicols}{2}
\dieii{Sabbato}{In Vigilia Pentecostes}{}{Semiduplex I classis}

\hora{Ad Matutinum}

\I Allelúja, Christum Dóminum ascendéntem in cælum, \red{*} Veníte, adorémus, allelúja. 
\red{Ps. 94} Veníte, exsultémus.

\pars{Hymnus}
\begin{hymnus}
\y{Æ}{térne} Rex altíssime,\\
\hspace{2.4em} Redémptor et fidélium,\\
Quo mors solúta déperit,\\
Datur triúmphum grátiæ.

\red{S}candens tribúnal déxteræ\\
Patris, potéstas ómnium\\
Colláta Jesu cǽlitus,\\
Quæ non erat humánitus:

\red{U}t trina rerum máchina,\\
Cæléstium, terréstrium,\\
Et inferórum cóndita,\\
Flectat genu jam súbdita.

\red{T}remunt vidéntes Angeli\\
Versam vicem mortálium:\\
Culpat caro, purgat caro,\\
Regnat Deus Dei caro.

\red{T}u esto nostrum gáudium,\\
Manens olýmpo prǽditum:\\
Mundi regis qui fábricam,\\
Mundána vincens gáudia.

\red{H}inc te precántes quǽsumus,\\
Ignósce culpis ómnibus,\\
Et corda sursum súbleva\\
Ad te supérna grátia.

\red{U}t, cum repénte cœ́peris\\
Clarére nube iúdicis,\\
Pœnas repéllas débitas,\\
Reddas corónas pérditas.

\red{G}lória tibi Dómine,\\
Qui scandis super sídera,\\
Cum Patre, et Sancto Spíritu,\\
In sempitérna sǽcula.
\red{A}men.
\end{hymnus}

\nocturn{In I Nocturno}

\A A summo cælo \f egréssio ejus, et occúrsus ejus usque ad summum ejus, allelúja.

\input{../PsalmiM/101Y.tex}

\input{../PsalmiM/102.tex}

\begin{psalmus}
\pars{Psalmus 103}

\y{B}{énedic}, ánima mea, Dómino: * Dómine, Deus meus, magnificátus es veheménter.

Confessiónem, et decórem induísti: * amíctus lúmine sicut vestiménto:

Exténdens cælum sicut pellem: * qui tegis aquis superióra ejus.

Qui ponis nubem ascénsum tuum: * qui ámbulas super pennas ventórum.

Qui facis ángelos tuos, spíritus: * et minístros tuos ignem uréntem.

Qui fundásti terram super stabilitátem suam: * non inclinábitur in sǽculum sǽculi.

Abýssus, sicut vestiméntum, amíctus ejus: * super montes stabunt aquæ.

Ab increpatióne tua fúgient: * a voce tonítrui tui formidábunt.

Ascéndunt montes: et descéndunt campi * in locum, quem fundásti eis.

Términum posuísti, quem non transgrediéntur: * neque converténtur operíre terram.

Qui emíttis fontes in convállibus: * inter médium móntium pertransíbunt aquæ.

Potábunt omnes béstiæ agri: * exspectábunt ónagri in siti sua.

Super ea vólucres cæli habitábunt: * de médio petrárum dabunt voces.

Rigans montes de superióribus suis: * de fructu óperum tuórum satiábitur terra:

Prodúcens fænum iuméntis, * et herbam servitúti hóminum:

Ut edúcas panem de terra: * et vinum lætíficet cor hóminis:

Ut exhílaret fáciem in óleo: * et panis cor hóminis confírmet.

Saturabúntur ligna campi, et cedri Líbani, quas plantávit: * illic pásseres nidificábunt.

Heródii domus dux est eórum: \f montes excélsi cervis: * petra refúgium herináciis.

Fecit lunam in témpora: * sol cognóvit occásum suum.

Posuísti ténebras, et facta est nox: * in ipsa pertransíbunt omnes béstiæ silvæ.

Cátuli leónum rugiéntes, ut rápiant, * et quǽrant a Deo escam sibi.

Ortus est sol, et congregáti sunt: * et in cubílibus suis collocabúntur.

Exíbit homo ad opus suum: * et ad operatiónem suam usque ad vésperum.

Quam magnificáta sunt ópera tua, Dómine! \f ómnia in sapiéntia fecísti: * impléta est terra possessióne tua.

\end{psalmus}


\begin{psalmus}
\pars{Divisio}

\y{H}{oc} mare magnum, et spatiósum mánibus: * illic reptília, quorum non est númerus.

Animália pusílla cum magnis: * illic naves pertransíbunt.

Draco iste, quem formásti ad illudéndum ei: * ómnia a te exspéctant ut des illis escam in témpore.

Dante te illis, cólligent: * aperiénte te manum tuam, ómnia implebúntur bonitáte.

Averténte autem te fáciem, turbabúntur: \f áuferes spíritum eórum, et defícient, * et in púlverem suum reverténtur.

Emíttes spíritum tuum, et creabúntur: * et renovábis fáciem terræ.

Sit glória Dómini in sǽculum: * lætábitur Dóminus in opéribus suis:

Qui réspicit terram, et facit eam trémere: * qui tangit montes, et fúmigant.

Cantábo Dómino in vita mea: * psallam Deo meo, quámdiu sum.

Iucúndum sit ei elóquium meum: * ego vero delectábor in Dómino.

Defíciant peccatóres a terra, et iníqui ita ut non sint: * bénedic, ánima mea, Dómino.

\end{psalmus}


\begin{psalmus}
\pars{Psalmus 104}

\y{C}{onfitémini} Dómino, et invocáte nomen ejus: * annuntiáte inter gentes ópera ejus.

Cantáte ei, et psállite ei: * narráte ómnia mirabília ejus.

Laudámini in nómine sancto ejus: * lætétur cor quæréntium Dóminum.

Quǽrite Dóminum, et confirmámini: * quǽrite fáciem ejus semper.

Mementóte mirabílium ejus, quæ fecit: * prodígia ejus, et judícia oris ejus.

Semen Ábraham, servi ejus: * fílii Jacob, elécti ejus.

Ipse Dóminus Deus noster: * in univérsa terra judícia ejus.

Memor fuit in sǽculum testaménti sui: * verbi, quod mandávit in mille generatiónes:

Quod dispósuit ad Ábraham: * et juraménti sui ad Isaac:

Et státuit illud Jacob in præcéptum: * et Israël in testaméntum ætérnum:

Dicens: Tibi dabo terram Chánaan, * funículum hereditátis vestræ.

Cum essent número brevi, * paucíssimi et íncolæ ejus:

Et pertransiérunt de gente in gentem, * et de regno ad pópulum álterum.

Non relíquit hóminem nocére eis: * et corrípuit pro eis reges.

Nolíte tángere christos meos: * et in prophétis meis nolíte malignári.

Et vocávit famem super terram: * et omne firmaméntum panis contrívit.

Misit ante eos virum: * in servum venúmdatus est Joseph.

Humiliavérunt in compédibus pedes ejus, \f ferrum pertránsiit ánimam ejus * donec veníret verbum ejus.

Elóquium Dómini inflammávit eum: \f misit rex, et solvit eum; * princeps populórum, et dimísit eum.

Constítuit eum dóminum domus suæ: * et príncipem omnis possessiónis suæ:

Ut erudíret príncipes ejus sicut semetípsum: * et senes ejus prudéntiam docéret.

\end{psalmus}


{\setstretch{0.98}
\begin{psalmus}
\pars{Divisio}

\y{E}{t} intrávit Israël in Ægýptum: * et Jacob áccola fuit in terra Cham.

Et auxit pópulum suum veheménter: * et firmávit eum super inimícos ejus.

Convértit cor eórum ut odírent pópulum ejus: * et dolum fácerent in servos ejus.

Misit Móysen, servum suum: * Aaron, quem elégit ipsum.

Pósuit in eis verba signórum suórum: * et prodigiórum in terra Cham.

Misit ténebras, et obscurávit: * et non exacerbávit sermónes suos.

Convértit aquas eórum in sánguinem: * et occídit pisces eórum.

Édidit terra eórum ranas: * in penetrálibus regum ipsórum.

Dixit, et venit cœnomyía: * et cínifes in ómnibus fínibus eórum.

Pósuit plúvias eórum grándinem: * ignem comburéntem in terra ipsórum.

Et percússit víneas eórum, et ficúlneas eórum: * et contrívit lignum fínium eórum.

Dixit, et venit locústa, et bruchus, * cujus non erat númerus:

Et comédit omne fænum in terra eórum: * et comédit omnem fructum terræ eórum.

Et percússit omne primogénitum in terra eórum: * primítias omnis labóris eórum.

Et edúxit eos cum argénto et auro: * et non erat in tríbubus eórum infírmus.

Lætáta est Ægýptus in profectióne eórum: * quia incúbuit timor eórum super eos.

Expándit nubem in protectiónem eórum: * et ignem ut lucéret eis per noctem.

Petiérunt, et venit cotúrnix: * et pane cæli saturávit eos.

Dirúpit petram et fluxérunt aquæ: * abiérunt in sicco flúmina;

Quóniam memor fuit verbi sancti sui: * quod hábuit ad Ábraham, púerum suum.

Et edúxit pópulum suum in exsultatióne, * et eléctos suos in lætítia.

Et dedit illis regiónes géntium: * et labóres populórum possedérunt:

Ut custódiant justificatiónes ejus, * et legem ejus requírant.

\end{psalmus}


\A A summo cælo egréssio ejus, et occúrsus ejus usque ad summum ejus, allelúja.

\V Ascéndo ad Patrem meum, et Patrem vestrum, allelúja.
\R Deum Deum, et Deum vestrum, allelúja.

\Evangelicalectio 

\scriptura{Léctio sancti Evangélii secúndum Joánnem}
\lectiocap{Lectio i}{Cap. 14, 15-21}
}
\y{I}{n} illo témpore: Dixit Jesus discípulis suis: Si dilígitis me, mandáta mea serváte. Et ego rogábo Patrem, et álium Paráclitum dabit vobis. Et réliqua.

\scriptura{Homilía sancti Augustíni Epíscopi}
\ex{Tract. 74 in Joann., n. 4 et 5; et 75, n. 1}
\Y{Q}{uod} ait, Rogábo Patrem, et álium Paráclitum dabit vobis: osténdit et seípsum esse Paráclitum. Paráclitus enim Latíne dícitur advocátus: et dictum est de Christo: Advocátum habémus ad Patrem, Jesum Christum justum. Sic autem mundum dixit non posse accípere Spíritum Sanctum, sicut étiam dictum est: Prudéntia carnis inimíca est Deo: legi enim Dei non est subiécta, nec enim potest: velut si dicámus: Iniustítia justítia esse non potest. Mundum quippe ait hoc loco, mundi signíficans dilectóres: quæ diléctio non est a Patre. Et ídeo dilectióni hujus mundi, de qua satágimus ut minuátur et consumátur in nobis, contrária est diléctio Dei, quæ diffúnditur in córdibus nostris per Spíritum Sanctum, qui datus est nobis.

\R Ego \f rogábo Patrem, et álium Paráclitum dabit vobis,
\red{*} Ut máneat vobíscum in ætérnum, Spíritum veritátis, allelúja.
\V Si enim non abíero, Paráclitus non véniet ad vos: si autem abíero, mittam eum ad vos.
\red{U}t máneat.

\lectio{Lectio ii}
\y{M}{undus} ergo eum accípere non potest, quia non videt eum, neque scit eum. Non enim habet invisíbiles óculos mundána diléctio, per quos vidéri Spíritus Sanctus potest, qui vidéri nisi invisibíliter non potest. Vos autem, inquit, cognoscétis eum: quia apud vos manébit, et in vobis erit. Erit in eis, ut máneat; non manébit, ut sit: prius est enim esse alícubi, quam manére. Sed ne putárent quod dictum est, Apud vos manébit; ita dictum, quemádmodum apud hóminem hospes visibíliter manére consuévit, expósuit quid díxerit: Apud vos manébit, cum adiúnxit et dixit, In vobis erit.

\R Ponis nubem \f ascénsum tuum, Dómine:
\red{*} Qui ámbulas super pennas ventórum, allelúja.
\V Confessiónem et decórem induísti, amíctus lumen sicut vestiméntum.
\red{Q}ui ámbulas.

\lectio{Lectio iii}
\y{E}{rgo} invisibíliter vidétur. Nec, si non sit in nobis, potest esse in nobis ejus sciéntia: sic enim a nobis vidétur in nobis et nostra consciéntia. Nam fáciem vidémus altérius, nostram vidére non póssumus: consciéntiam vero nostram vidémus, altérius non vidémus. Sed consciéntia nunquam est nisi in nobis: Spíritus autem Sanctus potest esse étiam sine nobis. Datur quippe ut sit et in nobis: sed vidéri et sciri, quemádmodum vidéndus et sciéndus est, non potest a nobis, si non sit in nobis. Post promissiónem Spíritus Sancti, ne quisquam putáret, quod ita eum Dóminus datúrus fúerit velut pro seípso, ut non et ipse cum eis esset futúrus, adiécit atque ait: Non relínquam vos órphanos, véniam ad vos. Quamvis ergo nos Fílius Dei suo Patri adoptáverit fílios, et eúndem Patrem nos volúerit habére per grátiam, qui ejus Pater est per natúram: tamen étiam ipse circa nos patérnum afféctum quodámmodo demónstrat, cum dicit: Non relínquam vos órphanos.

\R Si enim \f non abíero, Paráclitus non véniet ad vos: si autem abíero, mittam eum ad vos.
\red{*} Cum autem vénerit ille, docébit vos omnem veritátem, allelúja.
\V Non enim loquétur a semetípso: sed quæcúmque áudiet, loquétur: et quæ ventúra sunt, annuntiábit vobis.
\red{C}um.
\red{G}lória.
\red{C}um.

\nocturn{In II Nocturno}

\A Exaltáre, Dómine, \f in virtúte tua: cantábimus et psallémus, allelúja.

\begin{psalmus}
\pars{Psalmus 105}

\Y{C}{onfitémini} Dómino, quóniam bonus: * quóniam in sǽculum misericórdia ejus.

Quis loquétur poténtias Dómini, * audítas fáciet omnes laudes ejus?

Beáti, qui custódiunt judícium, * et fáciunt justítiam in omni témpore.

Meménto nostri, Dómine, in beneplácito pópuli tui: * vísita nos in salutári tuo:

Ad vidéndum in bonitáte electórum tuórum, \f ad lætándum in lætítia gentis tuæ: * ut laudéris cum hereditáte tua.

Peccávimus cum pátribus nostris: * iniúste égimus, iniquitátem fécimus.

Patres nostri in Ægýpto non intellexérunt mirabília tua: * non fuérunt mémores multitúdinis misericórdiæ tuæ.

Et irritavérunt ascendéntes in mare, * Mare Rubrum.

Et salvávit eos propter nomen suum: * ut notam fáceret poténtiam suam.

Et incrépuit Mare Rubrum, et exsiccátum est, * et dedúxit eos in abýssis sicut in desérto.

Et salvávit eos de manu odiéntium: * et redémit eos de manu inimíci.

Et opéruit aqua tribulántes eos: * unus ex eis non remánsit.

Et credidérunt verbis ejus: * et laudavérunt laudem ejus.

Cito fecérunt, oblíti sunt óperum ejus: * et non sustinuérunt consílium ejus.

Et concupiérunt concupiscéntiam in desérto: * et tentavérunt Deum in inaquóso.

Et dedit eis petitiónem ipsórum: * et misit saturitátem in ánimas eórum.

Et irritavérunt Móysen in castris: * Aaron, sanctum Dómini.

Apérta est terra, et deglutívit Dathan: * et opéruit super congregatiónem Abíron.

Et exársit ignis in synagóga eórum * flamma combússit peccatóres.

Et fecérunt vítulum in Horeb * et adoravérunt scúlptile.

Et mutavérunt glóriam suam * in similitúdinem vítuli comedéntis fænum.

Oblíti sunt Deum, qui salvávit eos, \f qui fecit magnália in Ægýpto, mirabília in terra Cham: * terribília in Mari Rubro.

Et dixit ut dispérderet eos: * si non Móyses, eléctus ejus, stetísset in confractióne in conspéctu ejus:

Ut avérteret iram ejus ne dispérderet eos: * et pro níhilo habuérunt terram desiderábilem:

Non credidérunt verbo ejus, \f et murmuravérunt in tabernáculis suis: * non exaudiérunt vocem Dómini.

Et elevávit manum suam super eos: * ut prostérneret eos in desérto:

Et ut deíceret semen eórum in natiónibus: * et dispérgeret eos in regiónibus.

Et initiáti sunt Beélphegor: * et comedérunt sacrifícia mortuórum.

Et irritavérunt eum in adinventiónibus suis: * et multiplicáta est in eis ruína.

Et stetit Phínees, et placávit: * et cessávit quassátio.

Et reputátum est ei in justítiam: * in generatiónem et generatiónem usque in sempitérnum.

\end{psalmus}


\begin{psalmus}
\pars{Divisio}

\y{E}{t} irritavérunt eum ad aquas contradictiónis: \f et vexátus est Móyses propter eos: * quia exacerbavérunt spíritum ejus.

Et distínxit in lábiis suis: * non disperdidérunt gentes, quas dixit Dóminus illis.

Et commísti sunt inter gentes, \f et didicérunt ópera eórum: et serviérunt sculptílibus eórum: * et factum est illis in scándalum.

Et immolavérunt fílios suos, * et fílias suas dæmóniis.

Et effudérunt sánguinem innocéntem: \f sánguinem filiórum suórum et filiárum suárum, * quas sacrificavérunt sculptílibus Chánaan.

Et infécta est terra in sanguínibus, \f et contamináta est in opéribus eórum: * et fornicáti sunt in adinventiónibus suis.

Et irátus est furóre Dóminus in pópulum suum: * et abominátus est hereditátem suam.

Et trádidit eos in manus géntium: * et domináti sunt eórum qui odérunt eos.

Et tribulavérunt eos inimíci eórum, \f et humiliáti sunt sub mánibus eórum: * sæpe liberávit eos.

Ipsi autem exacerbavérunt eum in consílio suo: * et humiliáti sunt in iniquitátibus suis.

Et vidit, cum tribularéntur: * et audívit oratiónem eórum.

Et memor fuit testaménti sui: * et pœnítuit eum secúndum multitúdinem misericórdiæ suæ.

Et dedit eos in misericórdias * in conspéctu ómnium qui céperant eos.

Salvos nos fac, Dómine, Deus noster: * et cóngrega nos de natiónibus:

Ut confiteámur nómini sancto tuo: * et gloriémur in laude tua.

Benedíctus Dóminus, Deus Israël, a sǽculo et usque in sǽculum: * et dicet omnis pópulus: Fiat, fiat.

\end{psalmus}


\begin{psalmus}
\pars{Psalmus 106}

\y{C}{onfitémini} Dómino quóniam bonus: * quóniam in sǽculum misericórdia ejus.

Dicant qui redémpti sunt a Dómino, \f quos redémit de manu inimíci: * et de regiónibus congregávit eos:

A solis ortu, et occásu: * ab aquilóne, et mari.

Erravérunt in solitúdine in inaquóso: * vjam civitátis habitáculi non invenérunt.

Esuriéntes, et sitiéntes: * ánima eórum in ipsis defécit.

Et clamavérunt ad Dóminum cum tribularéntur: * et de necessitátibus eórum erípuit eos.

Et dedúxit eos in vjam rectam: * ut irent in civitátem habitatiónis.

Confiteántur Dómino misericórdiæ ejus: * et mirabília ejus fíliis hóminum.

Quia satiávit ánimam inánem: * et ánimam esuriéntem satiávit bonis.

Sedéntes in ténebris, et umbra mortis: * vinctos in mendicitáte et ferro.

Quia exacerbavérunt elóquia Dei: * et consílium Altíssimi irritavérunt.

Et humiliátum est in labóribus cor eórum: * infirmáti sunt, nec fuit qui adjuváret.

Et clamavérunt ad Dóminum cum tribularéntur: * et de necessitátibus eórum liberávit eos.

Et edúxit eos de ténebris, et umbra mortis: * et víncula eórum disrúpit.

Confiteántur Dómino misericórdiæ ejus: * et mirabília ejus fíliis hóminum.

Quia contrívit portas ǽreas: * et vectes férreos confrégit.

Suscépit eos de via iniquitátis eórum: * propter iniustítias enim suas humiliáti sunt.

Omnem escam abomináta est ánima eórum: * et appropinquavérunt usque ad portas mortis.

Et clamavérunt ad Dóminum cum tribularéntur: * et de necessitátibus eórum liberávit eos.

Misit verbum suum, et sanávit eos: * et erípuit eos de interitiónibus eórum.

Confiteántur Dómino misericórdiæ ejus: * et mirabília ejus fíliis hóminum.

Et sacríficent sacrifícium laudis: * et annúntient ópera ejus in exsultatióne.

Qui descéndunt mare in návibus, * faciéntes operatiónem in aquis multis.

Ipsi vidérunt ópera Dómini, * et mirabília ejus in profúndo.

\end{psalmus}


\begin{psalmus}
\pars{Divisio}

\y{D}{ixit}, et stetit spíritus procéllæ: * et exaltáti sunt fluctus ejus.

Ascéndunt usque ad cælos, et descéndunt usque ad abýssos: * ánima eórum in malis tabescébat.

Turbáti sunt, et moti sunt sicut ébrius: * et omnis sapiéntia eórum devoráta est.

Et clamavérunt ad Dóminum cum tribularéntur: * et de necessitátibus eórum edúxit eos.

Et státuit procéllam ejus in auram: * et siluérunt fluctus ejus.

Et lætáti sunt quia siluérunt: * et dedúxit eos in portum voluntátis eórum.

Confiteántur Dómino misericórdiæ ejus: * et mirabília ejus fíliis hóminum.

Et exáltent eum in ecclésia plebis: * et in cáthedra seniórum laudent eum.

Pósuit flúmina in desértum: * et éxitus aquárum in sitim.

Terram fructíferam in salsúginem: * a malítia inhabitántium in ea.

Pósuit desértum in stagna aquárum: * et terram sine aqua in éxitus aquárum.

Et collocávit illic esuriéntes: * et constituérunt civitátem habitatiónis.

Et seminavérunt agros, et plantavérunt víneas: * et fecérunt fructum nativitátis.

Et benedíxit eis, et multiplicáti sunt nimis: * et iuménta eórum non minorávit.

Et pauci facti sunt: * et vexáti sunt a tribulatióne malórum, et dolóre.

Effúsa est contémptio super príncipes: * et erráre fecit eos in ínvio, et non in via.

Et adiúvit páuperem de inópia: * et pósuit sicut oves famílias.

Vidébunt recti, et lætabúntur: * et omnis iníquitas oppilábit os suum.

Quis sápiens et custódiet hæc? * et intélleget misericórdias Dómini.

\end{psalmus}


\input{../PsalmiM/107.tex}

\input{../PsalmiM/108.tex}

\A Exaltáre, Dómine, in virtúte tua: cantábimus et psallémus, allelúja.

\lectiocap{Capitulum}{1 Petr. 4, 9-10}
\y{H}{ospitáles} ínvicem sine murmuratióne: \f unusquísque, sicut accépit grátiam, in altérutrum illam administrántes, * sicut boni dispensatóres multifórmis grátiæ Dei.

\V Ascéndens Christus in altum, allelúja.
\R Captívam duxit captivitátem, allelúja.

\pars{Oratio}
\y{O}{mnípotens} sempitérne Deus: fac nos tibi semper et devótam gérere voluntátem; et majestáti tuæ sincéro corde servíre. Per Dóminum.



\privdie{}{Dominica Pentecostes}{}{Duplex I classis cum Octava privilegiata I ordinis}

{\setstretch{0.98}
\hora{Ad Matutinum}

\I Allelúja, Spíritus Dómini replévit orbem terrárum: \red{*} Veníte, adorémus, allelúja.
\red{Ps. 94} Veníte, exsultémus.

\pars{Hymnus}
\begin{hymnus}
\y{J}{am} Christus astra ascénderat,\\
\hspace{1.1em} Revérsus unde vénerat,\\
Promíssum Patris múnere,\\
Sanctum datúrus Spíritum.

\red{S}olémnis urgébat dies,\\
Quo mýstico septémplici\\
Orbis volútus sépties,\\
Signat beáta témpora.

\red{C}um lucis hora tértia\\
Repénte mundus íntonat,\\
Orántibus Apóstolis\\
Deum venísse núntiat.

\red{D}e Patris ergo lúmine\\
Decórus ignis almus est,\\
Qui fida Christi péctora\\
Calóre Verbi cómpleat.

\red{I}mpléta gaudent víscera,\\
Affláta Sancto Spíritu,\\
Vocésque divérsas íntonant,\\
Fantur Dei magnália.

\red{E}x omni gente cógniti,\\
Græcis, Latínis, Bárbaris,\\
Cunctísque admirántibus,\\
Linguis loquúntur ómnium.

\red{J}udǽa tunc incrédula,\\
Vesána torvo spíritu,\\
Ructáre musti crápulam,\\
Alúmnos Christi cóncrepat.

\red{S}ed signis et virtútibus\\
Occúrrit, et docet Petrus,\\
Falsa profári pérfidos,\\
Ioéle teste cómprobans.

\red{G}lória Patri Dómino,\\
Natóque, qui a mórtuis\\
Surréxit, ac Paráclito,\\
In sæculórum sǽcula.
\red{A}men.
\end{hymnus}

\nocturn{In I Nocturno}

\A Factus est \f repénte de cælo sonus adveniéntis spíritus veheméntis, allelúja, allelúja.
}

\input{../PsalmiM/1Y.tex}

\input{../PsalmiM/8.tex}

\input{../PsalmiM/18.tex}

\input{../PsalmiM/23.tex}

\input{../PsalmiM/26.tex}

\begin{psalmus}
    \pars{Psalmus 28}

    \y{A}{fférte} Dómino, fílii Dei: * afférte Dómino fílios aríetum.

    Afférte Dómino glóriam et honórem, \f afférte Dómino glóriam nómini ejus: * adoráte Dóminum in átrio sancto ejus.

    Vox Dómini super aquas, \f Deus majestátis intónuit: * Dóminus super aquas multas.

    Vox Dómini in virtúte: * vox Dómini in magnificéntia.

    Vox Dómini confringéntis cedros: * et confrínget Dóminus cedros Líbani:

    Et commínuet eas tamquam vítulum Líbani: * et diléctus quemádmodum fílius unicórnium.

    Vox Dómini intercidéntis flammam ignis: \f vox Dómini concutiéntis desértum: * et commovébit Dóminus desértum Cades.

    Vox Dómini præparántis cervos, et revelábit condénsa: * et in templo ejus omnes dicent glóriam.

    Dóminus dilúvium inhabitáre facit: * et sedébit Dóminus Rex in ætérnum.

    Dóminus virtútem pópulo suo dabit: * Dóminus benedícet pópulo suo in pace.

\end{psalmus}


\A Factus est repénte de cælo sonus adveniéntis spíritus veheméntis, allelúja, allelúja.

\V Spíritus Dómini replévit orbem terrárum, allelúja.
\R Et hoc quod cóntinet ómnia, sciéntiam habet vocis, allelúja.

\scriptura{De Actibus Apostolórum}
\lectiocap{Lectio i}{Cap. 2, 1-21}
\Y{C}{um} compleréntur dies Pentecóstes, erant omnes páriter in eódem loco: et factus est repénte de cælo sonus, tamquam adveniéntis spíritus veheméntis, et replévit totam domum ubi erant sedéntes. Et apparuérunt illis dispertítæ linguæ tamquam ignis, sedítque supra síngulos eórum: et repléti sunt omnes Spíritu Sancto, et cœpérunt loqui váriis linguis, prout Spíritus Sanctus dabat éloqui illis. Erant autem in Jerúsalem habitántes Judǽi, viri religiósi ex omni natióne quæ sub cælo est.

\R Cum compleréntur \f dies Pentecóstes, erant omnes páriter in eódem loco, allelúja: et súbito factus est sonus de cælo, allelúja,
\red{*} Tamquam spíritus veheméntis, et replévit totam domum, allelúja, allelúja.
\V Dum ergo essent in unum discípuli congregáti propter metum Judæórum, sonus repénte de cælo, venit super eos.
\red{T}amquam spíritus.

\lectio{Lectio ii}
\y{F}{acta} autem hac voce, convénit multitúdo, et mente confúsa est, quóniam audiébat unusquísque lingua sua illos loquéntes. Stupébant autem omnes, et mirabántur, dicéntes: Nonne ecce omnes isti qui loquúntur, Galilǽi sunt? et quómodo nos audívimus unusquísque linguam nostram in qua nati sumus? Parthi, et Medi, et Ælamítæ, et qui hábitant Mesopotámjam, Judǽam, et Cappadócjam, Pontum, et Asiam, Phrýgjam, et Pamphýljam, Ægýptum, et partes Líbyæ quæ est circa Cyrénen: et ádvenæ Románi, Judǽi quoque, et Prosélyti, Cretes, et Arabes: audívimus eos loquéntes nostris linguis magnália Dei.

\R Repléti sunt \f omnes Spíritu Sancto: et cœpérunt loqui, prout Spíritus Sanctus dabat éloqui illis:
\red{*} Et convénit multitúdo dicéntium, allelúja.
\V Loquebántur váriis linguis Apóstoli magnália Dei.
\red{E}t convénit.

\lectio{Lectio iii}
\y{S}{tupébant} autem omnes, et mirabántur ad ínvicem, dicéntes: Quidnam vult hoc esse? Alii autem irridéntes dicébant: Quia musto pleni sunt isti. Stans autem Petrus cum úndecim, levávit vocem suam, et locútus est eis: Viri Judǽi, et qui habitátis Jerúsalem univérsi, hoc vobis notum sit, et áuribus percípite verba mea. Non enim, sicut vos æstimátis, hi ébrii sunt, cum sit hora diéi tértia:

\R Jam non dicam \f vos servos, sed amícos meos; quia ómnia cognovístis, quæ operátus sum in médio vestri, allelúja:
\red{*} Accípite Spíritum Sanctum in vobis Paráclitum: ille est, quem Pater mittet vobis, allelúja.
\V Vos amíci mei estis, si fecéritis quæ ego præcípio vobis.
\red{A}ccípite.

\lectio{Lectio iv}
\y{S}{ed} hoc est quod dictum est per prophétam Ioël: Et erit in novíssimis diébus, dicit Dóminus, effúndam de Spíritu meo super omnem carnem: et prophetábunt fílii vestri et fíliæ vestræ, et iúvenes vestri visiónes vidébunt, et senióres vestri sómnia somniábunt. Et quidem super servos meos, et super ancíllas meas, in diébus illis effúndam de Spíritu meo, et prophetábunt: et dabo prodígia in cælo sursum, et signa in terra deórsum, sánguinem, et ignem, et vapórem fumi: sol convertétur in ténebras, et luna in sánguinem, ántequam véniat dies Dómini magnus et maniféstus. Et erit: omnis quicúmque invocáverit nomen Dómini, salvus erit.

\R Spíritus Sanctus, \f procédens a throno, Apostolórum péctora invisibíliter penetrávit novo sanctificatiónis signo:
\red{*} Ut in ore eórum ómnium génera nasceréntur linguárum, allelúja.
\V Advénit ignis divínus, non combúrens, sed illúminans, et tríbuit eis charísmatum dona.
\red{U}t in.
\red{G}lória Patri.
\red{U}t in.

\nocturn{In II Nocturno}

\A Confírma hoc, Deus, \f quod operátus es in nobis: a templo sancto tuo, quod est in Jerúsalem, allelúja, allelúja.

\input{../PsalmiM/32Y.tex}

\input{../PsalmiM/45.tex}

\input{../PsalmiM/46.tex}

\input{../PsalmiM/47.tex}

\input{../PsalmiM/95.tex}

\input{../PsalmiM/97.tex}

\A Confírma hoc, Deus, quod operátus es in nobis: a templo sancto tuo, quod est in Jerúsalem, allelúja, allelúja.

\V Spíritus Paráclitus, allelúja.
\R Docébit vos ómnia, allelúja.

\scriptura{Sermo Sancti Leónis Papæ}
\ex{Sermo LXXV; de Pentecoste I, cap. 1}
\lectio{Lectio v}
\Y{H}{odiérnam} solemnitátem, dilectíssimi, in precípuis festis esse venerándam, ómnium Catholicórum corda cognóscunt. Nec dúbium est, quanta huic diéi reveréntia debeátur, quem Spíritus Sanctus excellentíssimo sui múneris miráculo consecrávit.

\R Apparuérunt \f Apóstolis dispertítæ linguæ tamquam ignis, allelúja:
\red{*} Sedítque supra síngulos eórum Spíritus Sanctus, allelúja, allelúja.
\V Et cœpérunt loqui váriis linguis, prout Spíritus Sanctus dabat éloqui illis.
\red{S}edítque supra.

\lectio{Lectio vi}
\y{N}{am} ab illo die, quo Dóminus super omnem altitudinem cælórum ad déxteram Dei Patris concessúrus ascéndit, décimus iste est, qui ab eiúsdem resurrectióne quinquagésimus nobis in eo, a quo cœpit, illúxit magna mystéria véterum sacramentórum in se cóntinens et novórum; quibus manifestíssime declarátur, et grátiam prænuntiátam fuísse per legem et legem implétam esse per grátiam.

\R Loquebántur \f váriis linguis Apóstoli magnália Dei,
\red{*} Prout Spíritus Sanctus dabat éloqui illis, allelúja.
\V Repléti sunt omnes Spíritu Sancto, et cœpérunt loqui.
\red{P}rout.

{\setstretch{0.98}
\lectio{Lectio vii}
\y{S}{icut} enim Hebrǽo quondam pópulo ab Ægýptiis liberáto, quinquagésimo die post immolatiónem agni, Lex data est in monte Sina; ita post passiónem, qua verus Dei Agnus occísus est, quinquagésimo a resurrectióne ipsíus die in Apóstolos plebómque credéntium, Spíritus Sanctus illápsus est, ut fácile díligens Christiánus agnóscat, inítia véteris Testaménti evangélicis ministrásse princípiis, et ab eódem Spíritu cónditum fœdus secúndum, a quo primum fúerat institútum.

\R Disciplínam \f et sapiéntiam dócuit eos Dóminus, allelúja: firmávit in illis grátiam Spíritus sui,
\red{*} Et intelléctu implévit corda eórum, allelúja.
\V Repentíno namque sónitu Spíritus Sanctus super eos venit.
\red{E}t.

\lectio{Lectio viii}
\y{A}{b} hoc ígitur die tuba evangélicæ prædicatiónis intónuit: ab hoc die imbres charísmatum, flúmina benedictiónum, omne desértum et univérsam áridam rigavérunt: quóniam ad renovándam fáciem terræ Spíritus Dei ferebátur super aquas; et ad véteres ténebras abigéndas, novæ lucis fúlgura coruscábant, cum micántium splendóre linguárum, et verbum Dómini lúcidum, et elóquium conciperétur ignítum, cui ad creándum intelléctum, consummandúmque peccátum, et efficácia illuminándi, et vis inésset uréndi.

\R Ite \f in univérsum orbem, et prædicáte Evangélium, allelúja:
\red{*} Qui credíderit et baptizátus fúerit, salvus erit, allelúja, allelúja, allelúja.
\V In nómine meo dæmónia eiícient, linguis loquéntur novis, serpéntes tollent.
\red{Q}ui credíderit.
\red{G}lória.
\red{Q}ui credíderit.

\nocturn{In III Nocturno}

\A Emítte Spíritum tuum, \f et creabúntur: et renovábis fáciem terræ, allelúja, allelúja.
}

\input{../PsalmiM/Quis_est_isteY.tex}

\begin{psalmus}
    \lectiocap{Canticum Oseæ}{Oseaæ 6, 1-6}

		Veníte, et revertámur ad Dóminum, \f quia ipse cepit, et sanábit nos; * percútiet, et curábit nos.

		Vivificábit nos post duos dies; \f in die tértia suscitábit nos, * et vivémus in conspéctu ejus.

		Sciémus, sequemúrque, * ut cognoscámus Dóminum.

		Quasi dilúculum præparátus est egréssus ejus, * et véniet quasi imber nobis temporáneus et serótinus terræ.

		Quid fácjam tibi, Ephraim? quid fácjam tibi, Juda? \f misericórdia vestra quasi nubes matutína, * et quasi ros mane pertránsiens.

		Propter hoc dolávi in prophétis; \f occídi eos in verbis oris mei: * et judícia tua quasi lux egrediéntur.

		Quia misericórdiam vólui, et non sacrifícium; * et sciéntiam Dei plus quam holocáusta.

\end{psalmus}


\begin{psalmus}
\lectiocap{Canticum Sophoniæ}{Soph. 3, 8-13}

\y{E}{xspécta} me, dicit Dóminus, in die resurrectiónis meæ in futúrum: * quia judícium meum ut cóngregem gentes, et cólligam regna,

Et effúndam super eos indignatiónem meam, * omnem iram furóris mei:

In igne enim zeli mei * devorábitur omnis terra.

Quia tunc reddam pópulis lábium eléctum, \f ut ínvocent omnes in nómine Dómini, * et sérviant ei húmero uno.

Ultra flúmina Æthiópiæ, inde súpplices mei; * fílii dispersórum meórum déferent munus mihi.

In die illa non confundéris super cunctis adinventiónibus tuis, * quibus prævaricáta es in me:

Quia tunc áuferam de médio tui magníloquos supérbiæ tuæ, * et non adiícies exaltári ámplius in monte sancto meo.

Et derelínquam in médio tui pópulum páuperem et egénum: * et sperábunt in nómine Dómini.

Relíquiæ Israël non fácient iniquitátem, \f nec loquéntur mendácium, * et non inveniétur in ore eórum lingua dolósa:

Quóniam ipsi pascéntur, et accubábunt, * et non erit qui extérreat.

\end{psalmus}


\A Emítte Spíritum tuum, \f et creabúntur: et renovábis fáciem terræ, allelúja, allelúja.

\V Repléti sunt omnes Spíritu Sancto, allelúja.
\R Et cœpérunt loqui, allelúja.

\scriptura{Léctio sancti Evangélii secúndum Joánnem}
\lectiocap{Lectio ix}{Cap. 14}
\y{I}{n} illo témpore: Dixit Jesus discípulis suis: Si quis díligit me, sermónem meum servábit, et Pater meus díliget eum, et ad eum veniémus, et mansiónem apud eum faciémus. Et réliqua.

\scriptura{Homilía sancti Gregórii Papæ}
\ex{Hom. 30 in Evang., n. 1 et 2}
\Y{L}{ibet}, fratres caríssimi, evangélicæ verba lectiónis sub brevitáte transcúrrere, ut post diútius líceat in contemplatióne tantæ solemnitátis immorári. Hódie namque Spíritus Sanctus repentíno sónitu super discípulos venit, mentésque carnálium in sui amórem permutávit, et foris apparéntibus linguis ígneis, intus facta sunt corda flammántia; quia dum Deum in ignis visióne suscepérunt, per amórem suáviter arsérunt. Ipse namque Spíritus Sanctus amor est: unde et Joánnes dicit Deus cáritas est.

\R Advénit \f ignis divínus, non combúrens sed illúminans, non consúmens sed lucens: et invénit corda discipulórum receptácula munda:
\red{*} Et tríbuit eis charísmatum dona, allelúja, allelúja.
\V Invénit eos concórdes caritáte, et collustrávit eos inúndans grátia Deitátis.
\red{E}t tríbuit.

\lectio{Lectio x}
\y{Q}{ui} ergo mente íntegra Deum desíderat, profécto jam habet, quem amat. Neque enim quisquam posset Deum dilígere, si eum quem díligit, non habéret. Sed ecce, si unusquísque vestrum requirátur an díligat Deum: tota fidúcia et secúra mente respóndet, Díligo. In ipso autem lectiónis exórdio audístis quid Véritas dicit: Si quis díligit me, sermónem meum servábit. Probátio ergo dilectiónis, exhibítio est óperis. Hinc in epístola sua idem Joánnes dicit: Qui dicit: Díligo Deum, et mandáta ejus non custódit, mendax est.

\R Spíritus Sanctus \f replévit totam domum, ubi erant Apóstoli: et apparuérunt illis dispertítæ linguæ, tamquam ignis, sedítque supra síngulos eórum:
\red{*} Et repléti sunt omnes Spíritu Sancto, et cœpérunt loqui váriis linguis, prout Spíritus Sanctus dabat éloqui illis, allelúja, allelúja, allelúja.
\V Dum ergo essent in unum discípuli congregáti propter metum Judæórum, sonus repénte de cælo venit super eos.
\red{E}t repléti.

\lectio{Lectio xi}
\y{V}{ere} étenim Deum dilígimus et mandáta ejus custodímus, si nos a nostris voluptátibus coarctámus. Nam qui adhuc per illícita desidéria díffluit, profécto Deum non amat: quia ei in sua voluntáte contradícit. Et Pater meus díliget eum, et ad eum veniémus, et mansiónem apud eum faciémus. Pensáte, fratres caríssimi, quanta sit ista dígnitas, habére in cordis hospítio advéntum Dei.

\R Non vos \f me elegístis, sed ego elégi vos, et pósui vos:
\red{*} Ut eátis, et fructum afferátis, et fructus vester máneat, allelúja, allelúja.
\V Sicut misit me Pater, et ego mitto vos.
\red{U}t.

\lectio{Lectio xii}
\y{C}{erte}, si domum nostram quisquam dives aut prǽpotens amícus intráret, omni festinántia domus tota mundarétur, ne quid fortásse esset, quod óculos amíci intrántis offénderet. Tergat ergo sordes pravi óperis, qui Deo prǽparat domum mentis. Sed vidéte quid Véritas dicat: Veniémus, et mansiónem apud eum faciémus. In quorúmdam étenim corda venit, et mansiónem non facit: quia, per compunctiónem quidem, Dei respéctum percípiunt, sed tentatiónis témpore hoc ipsum quo compúncti fúerant, obliviscúntur; sicque ad perpetránda peccáta rédeunt, ac si hæc mínime planxíssent.

\R Spíritus Dómini \f replévit orbem terrárum:
\red{*} Et hoc quod cóntinet ómnia, sciéntiam habet vocis, allelúja, allelúja.
\V Omnium est enim ártifex, omnem habens virtútem, ómnia prospíciens.
\red{E}t.
\red{G}lória.
\red{E}t.

\scriptura{Sequéntia sancti Evangélii secúndum Joánnem}
\ex{Cap. 14, 23-31}
\Y{I}{n} illo témpore: Dixit Jesus discípulis suis: Si quis díligit me, sermónem meum servábit, et Pater meus díliget eum, et ad eum veniémus et mansiónem apud eum faciémus: qui non díligit me, sermónes meos non servat. Et sermónem quem audístis, non est meus: sed ejus, qui misit me, Patris. Hæc locútus sum vobis, apud vos manens. Paráclitus autem Spíritus Sanctus, quem mittet Pater in nómine meo, ille vos docébit ómnia et súggeret vobis ómnia, quæcúmque díxero vobis. Pacem relínquo vobis, pacem meam do vobis: non quómodo mundus dat, ego do vobis. Non turbétur cor vestrum neque formídet. Audístis, quia ego dixi vobis: Vado et vénio ad vos. Si diligerétis me, gauderétis útique, quia vado ad Patrem: quia Pater maior me est. Et nunc dixi vobis, priúsquam fiat: ut, cum factum fúerit, credátis. Iam non multa loquar vobíscum. Venit enim princeps mundi hujus, et in me non habet quidquam. Sed ut cognóscat mundus, quia díligo Patrem, et sicut mandátum dedit mihi Pater, sic fácio.

\columnbreak
\pars{Oratio}
\y{D}eus, qui hodiérna die corda fidélium Sancti Spíritus illustratióne docuísti: da nobis in eódem Spíritu recta sápere; et de ejus semper consolatióne gaudére. Per Dóminum … in unitáte ejúsdem Spíritus Sancti.
%\end{multicols}

%\pagebreak

%\begin{multicols}{2}
\privdie{Feria II}{Infra Octavam Pentecostes}{}{Duplex I classis}
\chead{\trim{}{Feria II Infra Octavam Pentecostes}}
\hora{Ad Matutinum}

\rubric{Invitatorium, Hymnus, Añæ, Psalmi, Cantica, \VV et \RR dicuntur ut in die Pentecostes.}

\nocturn{In I Nocturno}
\scriptura{De Actibus Apostolórum}
\lectiocap{Lectio i}{Cap. 19, 1-12}
\Y{F}{actum} est autem, cum Apóllo esset Corínthi, ut Paulus peragrátis superióribus pártibus veníret Ephesum, et inveníret quosdam discípulos: dixítque ad eos: Si Spíritum Sanctum accepístis credéntes? At illi dixérunt ad eum: Sed neque si Spíritus Sanctus est, audívimus. Ille vero ait: In quo ergo baptizáti estis? Qui dixérunt: In Joánnis baptísmate. 

\R Cum compleréntur \f dies Pentecóstes, erant omnes páriter in eódem loco, allelúja: et súbito factus est sonus de cælo, allelúja,
\red{*} Tamquam spíritus veheméntis, et replévit totam domum, allelúja, allelúja.
\V Dum ergo essent in unum discípuli congregáti propter metum Judæórum, sonus repénte de cælo, venit super eos.
\red{T}amquam spíritus.

\lectio{Lectio ii}
\y{D}{ixit} autem Paulus: Joánnes baptizávit baptísmo pœniténtiæ pópulum, dicens: In eum, qui ventúrus esset post ipsum, ut créderent, hoc est, in Jesum. His audítis, baptizáti sunt in nómine Dómini Jesu. Et cum imposuísset illis manus Paulus, venit Spíritus Sanctus super eos, et loquebántur linguis, et prophetábant. Erant autem omnes viri fere duódecim. 

\R Repléti sunt \f omnes Spíritu Sancto: et cœpérunt loqui, prout Spíritus Sanctus dabat éloqui illis:
\red{*} Et convénit multitúdo dicéntium, allelúja.
\V Loquebántur váriis linguis Apóstoli magnália Dei.
\red{E}t convénit.

\lectio{Lectio iii}
\y{I}{ntrogréssus} autem synagógam, cum fidúcia loquebátur per tres menses, dísputans, et suádens de regno Dei. Cum autem quidam induraréntur, et non créderent, maledicéntes vjam Dómini coram multitúdine, discédens ab eis, segregávit discípulos, quotídie dísputans in schola tyránni cujúsdam.

\R Iam non dicam \f vos servos, sed amícos meos; quia ómnia cognovístis, quæ operátus sum in médio vestri, allelúja:
\red{*} Accípite Spíritum Sanctum in vobis Paráclitum: ille est, quem Pater mittet vobis, allelúja.
\V Vos amíci mei estis, si fecéritis quæ ego præcípio vobis.
\red{A}ccípite.

\lectio{Lectio iv}
\y{H}{oc} autem factum est per biénnium, ita ut omnes, qui habitábant in Asia, audírent verbum Dómini, Jud\'æi atque Gentíles. Virtutésque non quáslibet faciébat Deus per manum Pauli: ita ut étiam super lánguidos deferréntur a córpore ejus sudária et semicínctia, et recedébant ab eis languóres, et spíritus nequam egrediebántur.

\R Spíritus Sanctus, \f procédens a throno, Apostolórum péctora invisibíliter penetrávit novo sanctificatiónis signo:
\red{*} Ut in ore eórum ómnium génera nasceréntur linguárum, allelúja.
\V Advénit ignis divínus, non combúrens, sed illúminans, et tríbuit eis charísmatum dona.
\red{U}t in.
\red{G}lória Patri.
\red{U}t in.

\nocturn{In II Nocturno}
\scriptura{Sermo sancti Augustíni Epíscopi}
\ex{Sermo 267 in die Pentecostes, cap. 1 et 2}
\lectio{Lectio v}
\Y{H}{odíerni} diéi solémnitas Dómini Dei magni, et magnæ grátiæ, quæ superfúsa est super nos, recordatiónem facit. Deo enim solémnitas celebrátur: ne, quod semel factum est, de memória deleátur. Solémnitas enim ab eo, quod solet in anno, nomen accépit: quómodo perénnitas flúminis dícitur, quia non siccátur æstáte, sed per totum annum fluit: ídeo perénne, id est, per annum: sic et solémne, quod solet in anno celebrári.

\R Apparuérunt \f Apóstolis dispertítæ linguæ tamquam ignis, allelúja:
\red{*} Sedítque supra síngulos eórum Spíritus Sanctus, allelúja, allelúja.
\V Et cœpérunt loqui váriis linguis, prout Spíritus Sanctus dabat éloqui illis.
\red{S}edítque supra.

\lectio{Lectio vi}
\y{C}{elebrámus} hódie advéntum Spíritus Sancti: Dóminus enim Spíritum de cælo misit, quem in terra promísit: et quia sic promíserat de cælo esse missúrum. Non postest ille veníre, ait, nisi ego abíero: dum autem abíero, mittam illum ad vos. Passus est, mórtuus est, resurréxit, ascéndit: restábat, ut impléret, quod promísit.

\R Loquebántur \f váriis linguis Apóstoli magnália Dei,
\red{*} Prout Spíritus Sanctus dabat éloqui illis, allelúja.
\V Repléti sunt omnes Spíritu Sancto, et cœpérunt loqui.
\red{P}rout.

\lectio{Lectio vii}
\y{H}{oc} exspectántes discípuli ejus, ánimæ, ut scriptum est, centum vigínti, decupláto número Apostolórum: duódecim enim elégit, et in centum vigínti Spíritum misit. Hoc ergo promíssum exspectántes, in una domo erant, orábant: quia desiderábant jam ipsa fide, quod ipsa oratióne, ipso spiritáli desidério: utres novi erant, vinum novum de cælo exspectabátur, et venit. Jam enim fúerat magnus botrus ille calcátus, et glorificátus. Légimus enim in Evangélio: Spíritus enim nondum erat datus, quia Jesus nondum fúerat glorificátus.

\R Disciplínam \f et sapiéntiam dócuit eos Dóminus, allelúja: firmávit in illis grátiam Spíritus sui,
\red{*} Et intelléctu implévit corda eórum, allelúja.
\V Repentíno namque sónitu Spíritus Sanctus super eos venit.
\red{E}t.

\lectio{Lectio viii}
\y{J}{am} quid respóndit, audístis, magnum miráculum. Omnes, qui áderant, unam linguam didícerant. Venit Spíritus Sanctus, impléti sunt, cœpérunt loqui linguis váriis ómnium géntium, quas non nóverant, nec didícerant; sed docébat ille, qui vénerat: intrávit, impléti sunt, fudit. Et tunc hoc erat signum, quicúmque accipiébat Spíritum Sanctum, súbito implétus Spíritu, linguis ómnium loquebátur, non illi solum centum vigínti.

\R Ite \f in univérsum orbem, et prædicáte Evangélium, allelúja:
\red{*} Qui credíderit et baptizátus fúerit, salvus erit, allelúja, allelúja, allelúja.
\V In nómine meo dæmónia eiícient, linguis loquéntur novis, serpéntes tollent.
\red{Q}ui credíderit.
\red{G}lória.
\red{Q}ui credíderit.

\nocturn{In III Nocturno}
\scriptura{Léctio sancti Evangélii secúndum Joánnem}
\lectiocap{Lectio ix}{Cap. 3}
\y{I}{n} illo témpore: Dixit Jesus Nicodémo: Sic Deus diléxit mundum, ut Fílium suum unigénitum daret: ut omnis, qui credit in eum, non péreat, sed hábeat vitam ætérnam. Et réliqua.

\scriptura{Homilía sancti Augustíni Epíscopi}
\ex{Tract. 12 in Joann., n. 12 et 13}
\Y{Q}{uantum} in médico est, sanáre venit ægrótum. Ipse se intérimit, qui præcépta médici observáre non vult. Venit Salvátor in mundum. Quare Salvátor dictus est mundi, nisi ut salvet mundum, non ut júdicet mundum? Salvári non vis ab ipso: ex te judicáberis. Et quid dicam, Judicáberis? Vide quid ait: Qui credit in eum, non judicátur. Qui autem non credit: quid dictúrum sperábas, nisi: Judicátur?

\R Advénit \f ignis divínus, non combúrens sed illúminans, non consúmens sed lucens: et invénit corda discipulórum receptácula munda:
\red{*} Et tríbuit eis charísmatum dona, allelúja, allelúja.
\V Invénit eos concórdes caritáte, et collustrávit eos inúndans grátia Deitátis.
\red{E}t tríbuit.

\lectio{Lectio x}
\y{Q}{uod} addit: Jam, inquit, judicátus est: nondum appáruit judícium, et jam factum est judícium. Novit enim Dóminus qui sunt ejus: novit qui permáneant ad corónam, qui permáneant ad flammam. Novit in área sua tríticum, novit et páleam: novit ségetem, novit et zizánia. Jam judicátus est, qui non credit. Quare judicátus? Quia non crédidit in nómine unigéniti Fílii Dei.

\R Spíritus Sanctus \f replévit totam domum, ubi erant Apóstoli: et apparuérunt illis dispertítæ linguæ, tamquam ignis, sedítque supra síngulos eórum:
\red{*} Et repléti sunt omnes Spíritu Sancto, et cœpérunt loqui váriis linguis, prout Spíritus Sanctus dabat éloqui illis, allelúja, allelúja, allelúja.
\V Dum ergo essent in unum discípuli congregáti propter metum Judæórum, sonus repénte de cælo venit super eos.
\red{E}t repléti.

\lectio{Lectio xi}
\y{H}{oc} est autem judícium: quia lux venit in mundum, et dilexérunt hómines magis ténebras, quam lucem: erant enim mala ópera eórum. Fratres mei, quorum ópera bona invénit Dóminus? Nullórum. Omnia ópera mala invénit. Quómodo ergo quidam fecérunt veritátem, et venérunt ad lucem? Et hoc enim séquitur: Qui autem facit veritátem, venit ad lucem.

\R Non vos \f me elegístis, sed ego elégi vos, et pósui vos:
\red{*} Ut eátis, et fructum afferátis, et fructus vester máneat, allelúja, allelúja.
\V Sicut misit me Pater, et ego mitto vos.
\red{U}t.

\lectio{Lectio xii}
\y{S}{ed} dilexérunt, inquit, ténebras magis quam lucem. Ibi pósuit vim. Multi enim dilexérunt peccáta sua, multi conféssi sunt peccáta sua: quia qui confitétur peccáta sua, et accúsat peccáta sua, jam cum Deo facit. Accúsat Deus peccáta tua: si et tu accúsas, conjúngeris Deo. Quasi duæ res sunt, homo et peccátor. Quod audis homo, Deus fecit: quod audis peccátor, ipse homo fecit. Dele, quod fecísti, ut Deus salvet, quod fecit. Opórtet ut óderis in te opus tuum, et ames in te opus Dei. Cum autem cœ́perit tibi displicére quod fecísti, inde incípiunt bona ópera tua, quia accúsas mala ópera tua. Inítium óperum bonórum, conféssio est óperum malórum.

\R Spíritus Dómini \f replévit orbem terrárum:
\red{*} Et hoc quod cóntinet ómnia, sciéntiam habet vocis, allelúja, allelúja.
\V Omnium est enim ártifex, omnem habens virtútem, ómnia prospíciens.
\red{E}t.
\red{G}lória.
\red{E}t.

\scriptura{Sequéntia sancti Evangélii secúndum Joánnem}
\ex{Cap. 3, 16-21}
\Y{I}{n} illo témpore: Dixit Jesus Nicodémo: Sic Deus diléxit mundum, ut Fílium suum unigénitum daret: ut omnis, qui credit in eum, non péreat, sed hábeat vitam ætérnam. Non enim misit Deus Fílium suum in mundum, ut júdicet mundum, sed ut salvétur mundus per ipsum. Qui credit in eum, non judicátur; qui autem non credit, jam judicátus est: quia non credit in nómine unigéniti Fílii Dei. Hoc est autem judícium: quia lux venit in mundum, et dilexérunt hómines magis ténebras quam lucem: erant enim eórum mala ópera. Omnis enim, qui male agit, odit lucem, et non venit ad lucem, ut non arguántur ópera ejus: qui autem facit veritátem, venit ad lucem, ut manifesténtur ópera ejus, quia in Deo sunt facta.

\pars{Oratio}
\y{D}{eus}, qui Apóstolis tuis Sanctum dedísti Spíritum: concéde plebi tuæ piæ petitiónis efféctum; ut, quibus dedísti fidem, largiáris et pacem. Per Dóminum … in unitáte ejúsdem Spíritus.
\ornamentviii


\end{multicols}

\ornamentvi

\end{document}
