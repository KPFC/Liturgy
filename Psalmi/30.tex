\begin{psalmus}
\pars{Psalmus 30}

\y{I}{n} te, Dómine, sperávi, non confúndar in ætérnum: * in justítia tua líbera me.

Inclína ad me aurem tuam, * accélera ut éruas me.

Esto mihi in Deum protectórem, et in domum refúgii: * ut salvum me fácias.

Quóniam fortitúdo mea, et refúgium meum es tu: * et propter nomen tuum dedúces me, et enútries me.

Edúces me de láqueo hoc, quem abscondérunt mihi: * quóniam tu es protéctor meus.

In manus tuas comméndo spíritum meum: * redemísti me, Dómine, Deus veritátis.

Odísti observántes vanitátes, * supervácue.

Ego autem in Dómino sperávi: * exsultábo, et lætábor in misericórdia tua.

Quóniam respexísti humilitátem meam, * salvásti de necessitátibus ánimam meam.

Nec conclusísti me in mánibus inimíci: * statuísti in loco spatióso pedes meos.

Miserére mei, Dómine, quóniam tríbulor: * conturbátus est in ira óculus meus, ánima mea, et venter meus:

Quóniam defécit in dolóre vita mea: * et anni mei in gemítibus.

Infirmáta est in paupertáte virtus mea: * et ossa mea conturbáta sunt.

Super omnes inimícos meos factus sum oppróbrium et vicínis meis valde: * et timor notis meis.

Qui vidébant me, foras fugérunt a me: * oblivióni datus sum, tamquam mórtuus a corde.

Factus sum tamquam vas pérditum: * quóniam audívi vituperatiónem multórum commorántium in circúitu.

In eo dum convenírent simul advérsum me, * accípere ánimam meam consiliáti sunt.

Ego autem in te sperávi, Dómine: * dixi: Deus meus es tu: in mánibus tuis sortes meæ.

Éripe me de manu inimicórum meórum, * et a persequéntibus me.

Illústra fáciem tuam super servum tuum, salvum me fac in misericórdia tua: * Dómine, non confúndar, quóniam invocávi te.

Erubéscant ímpii, et deducántur in inférnum: * muta fiant lábia dolósa.

Quæ loquúntur advérsus justum iniquitátem: * in supérbia, et in abusióne.

Quam magna multitúdo dulcédinis tuæ, Dómine, * quam abscondísti timéntibus te.

Perfecísti eis, qui sperant in te, * in conspéctu filiórum hóminum.

Abscóndes eos in abscóndito faciéi tuæ * a conturbatióne hóminum.

Próteges eos in tabernáculo tuo * a contradictióne linguárum.

Benedíctus Dóminus: * quóniam mirificávit misericórdiam suam mihi in civitáte muníta.

Ego autem dixi in excéssu mentis meæ: * Projéctus sum a fácie oculórum tuórum.

Ídeo exaudísti vocem oratiónis meæ, * dum clamárem ad te.

Dilígite Dóminum omnes sancti ejus: * quóniam veritátem requíret Dóminus, et retríbuet abundánter faciéntibus supérbjam.

Viríliter ágite, et confortétur cor vestrum, * omnes, qui sperátis in Dómino.

\end{psalmus}
