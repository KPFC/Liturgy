\begin{psalmus}
\pars{Psalmus 62}

\y{D}{eus}, Deus meus, * ad te de luce vígilo.

Sitívit in te ánima mea, * quam multiplíciter tibi caro mea.

In terra desérta, et ínvia, et inaquósa: \f sic in sancto appárui tibi, * ut vidérem virtútem tuam, et glóriam tuam.

Quóniam mélior est misericórdia tua super vitas: * lábia mea laudábunt te.

Sic benedícam te in vita mea: * et in nómine tuo levábo manus meas.

Sicut ádipe et pinguédine repleátur ánima mea: * et lábiis exsultatiónis laudábit os meum.

Si memor fui tui super stratum meum, \f in matutínis meditábor in te: * quia fuísti adiútor meus.

Et in velaménto alárum tuárum exsultábo, \f adhǽsit ánima mea post te: * me suscépit déxtera tua.

Ipsi vero in vanum quæsiérunt ánimam meam, \f introíbunt in inferióra terræ: * tradéntur in manus gládii, partes vúlpium erunt.

Rex vero lætábitur in Deo, \f laudabúntur omnes qui jurant in eo: * quia obstrúctum est os loquéntium iníqua.

\end{psalmus}
