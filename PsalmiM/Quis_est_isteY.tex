\begin{psalmus}
\lectiocap{Canticum Isaiæ}{Isaiæ 63, 1-5}

\Y{Q}{uis} est iste, qui venit de Edom, * tinctis véstibus de Bosra?

Iste formósus in stola sua, * grádiens in multitúdine fortitúdinis suæ.

Ego qui loquor justítiam, * et propugnátor sum ad salvándum.

Quare ergo rubrum est induméntum tuum, * et vestiménta tua sicut calcántium in torculári?

Tórcular calcávi solus, * et de géntibus non est vir mecum:

Calcávi eos in furóre meo, * et conculcávi eos in ira mea:

Et aspérsus est sanguis eorum super vestiménta mea, * et ómnia induménta mea inquinávi.

Dies enim ultiónis in corde meo; * annus redemptiónis meæ venit.

Circumspéxi, et non erat auxiliátor; * quæsívi, et non fuit qui adjuváret:

Et salvávit mihi bráchium meum, * et indignátio mea ipsa auxiliáta est mihi.

\end{psalmus}
