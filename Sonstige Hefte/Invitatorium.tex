\documentclass[fontsize=9pt,paper=A6,twoside,BCOR=1mm,DIV=22,headinclude]{scrarticle}
\usepackage{breviarium}
\begin{document}
\newcommand{\OZA}{\red{ (Osterzt.} Alleluja.\red{)} }

\image{Lithographie-Lamm-Evangelisten.jpg}

\vspace{4em}

\titulumii{Invitatorium}{für die einzelnen Tage des Jahres}{nach dem Breviarium Romanum von 1962}


\vfill
\pagebreak

%\begin{multicols}{2}
%\privdieii{}{Invitatorium}{}{für die einzelnen Tage des Jahres}
%\end{multicols}
%\vspace{-1em}
%\thispagestyle{empty}

%\dieii{}{Invitatorium}{}{}

\rubric{Es wird die Antiphon des Tages zwei Mal gebetet.}

\begin{paracol}{2}\pcb
\Y{V}{eníte}, exsultémus Dómino, jubilémus Deo, salutári nostro: præoccupémus fáciem ejus in confessióne, et in psalmis jubilémus ei.
\switchcolumn
\y{K}{ommt}, lasst uns jauchzen dem Herrn, lasst uns jubeln dem Gott unseres Heils: Laßt uns im voraus Sein Antlitz im Bekenntnis einnehmen und Ihn mit Psalmen umjubeln.
\end{paracol}
\rubric{Es wird die Antiphon einmal widerholt.}
\vspace{.5em}
\begin{paracol}{2}\pcb
	Quóniam Deus magnus Dóminus, et Rex magnus super omnes deos, quóniam non repéllet Dóminus plebem suam: quia in manu ejus sunt omnes fines terræ, et altitúdines móntium ipse cónspicit.
	\switchcolumn 
	Fürwahr Gott ist ein großer Herr, und ein großer König über alle Gottheiten. Denn der Herr wird Sein Volk nicht zurückstoßen: denn in Seiner Hand sind alle Enden der Erde, und auf die Höhen der Berge hat Er selbst sein Auge gerichtet.
\end{paracol}
\pcb
\rubric{Es wird der zweite Teil der Antiphon (ab dem Asterisk) wiederholt. Bei \black{Veníte, adorémus, et procidámus ante Deum} bzw. \black{Kommt, lasst uns anbeten und niederfallen vor Gott} kniet man sich hin.}
\vspace{.5em}
\begin{paracol}{2}\pcb
	Quóniam ipsíus est mare, et ipse fecit illud, et áridam fundavérunt manus ejus. \textit{Veníte, adorémus, et procidámus ante Deum:} plorémus coram Dómino, qui fecit nos, quia ipse est Dóminus, Deus noster; nos autem pópulus ejus, et oves páscuæ ejus.
	\switchcolumn 
	Denn sein ist das Meer, und er selbst hat es geschaffen, und das Land haben Seine Hände begründet. \textit{Kommt, lasst uns anbeten und niederfallen vor Gott:} Laßt uns weinen beim Herrn, der uns geschaffen hat, denn Er selbst ist der Herr, unser Gott; wir aber Sein Volk und die Schafe Seiner Weide.
\end{paracol}
\rubric{Es wird die ganze Antiphon wiederholt.}
\vspace{.4em}
\begin{paracol}{2}\pcb
	Hódie, si vocem ejus audiéritis, nolíte obduráre corda vestra, sicut in exacerbatióne secúndum diem tentatiónis in desérto: ubi tentavérunt me patres vestri, probavérunt et vidérunt ópera mea.
	\switchcolumn 
	Heute, wenn ihr doch Seine Stimme hörtet, verdunkelt nicht eure Herzen, so wie in der Erbitterung nach dem Tag der Versuchung in der Wüste: Wo Mich eure Väter versucht haben. Sie haben Meine Werke geprüft und gesehen.
\end{paracol}
\rubric{Es wird der zweite Teil der Antiphon wiederholt.}
\vspace{.4em}
\begin{paracol}{2}\pcb
	Quadragínta annis próximus fui generatióni huic, et dixi; Semper hi errant corde, ipsi vero non cognovérunt vias meas: quibus iurávi in ira mea; Si introíbunt in réquiem meam.
	\switchcolumn 
	Vierzig Jahre war Ich dieser Generation nahe und ich sprach: Immer irren sie im Herzen, wahrlich sie haben Meine Wege selbst nicht erkannt: Ihnen schwor ich in Meinem Zorn; wenngleich sie in Meine Ruhestätte eingezogen sind.
\end{paracol}
\rubric{Es wird die ganze Antiphon wiederholt}
\vspace{.4em}
\begin{paracol}{2}\pcb
	Glória Patri, et Fílio, et Spirítui Sancto.
	Sicut erat in princípio, et nunc, et semper, et in sǽcula sæculórum. Amen.
	\switchcolumn 
	Ehre sei dem Vater und dem Sohn, und dem Heiligen Geist.
	Wie es war im Anfang, so auch jetzt und alle Zeit und in Ewigkeit. Amen
\end{paracol}
\rubric{Es wird der erste Teil der Antiphon wiederholt, daraufhin wird noch einmal die ganze Antiphon wiederholt.}
\vfill

\pagebreak
\dieii{}{Temporale}{}{}
\proii{Invitatorien für Sonntage, Ferialtage und Feste im Weihnachts- oder Osterfestkreis}

%\proii{A Dominica I usque ad Sabbatum ante Dominicam III Adventus}
\proii{Vom 1. Adventssonntag bis zum Samstag vor dem 3. Adventssonntag}
\begin{paracol}{2}\pcb
	Regem ventúrum Dóminum, \red{*} Veníte adorémus.
	\switchcolumn
	Den König, den Herrn, der da kommen wird, \red{*} Kommt, lasset uns anbeten.
\end{paracol}
	
%\proii{A Dominica III usque ad diem ante Vigiljam Nativitatis Domini}
\proii{Vom 3. Adventssonntag bis zum Tag vor Heiligabend}
\begin{paracol}{2}\pcb
	Prope est jam Dóminus: \red{*} Veníte adorémus.
	\switchcolumn
	Schon ist nahe der Herr, \red{*} Kommt, lasset uns anbeten.
\end{paracol}

%\proii{In Vigilia Nativitatis Domini}
\proii{Am Heiligen Abend}
\begin{paracol}{2}\pcb
	Hódie sciétis, quia véniet Dóminus: \red{*} Et mane vidébitis glóriam ejus.
	\switchcolumn
	Heute sollt ihr wissen, dass der Herr kommen wird. \red{*} Und morgen werdet ihr Seine Herrlichkeit sehen.
\end{paracol}

%\proii{In Nativitate Domini}
\proii{Am Weihnachtstag und bis zum Tag vor Epiphanias, falls kein anderes Fest stattfindet.}
\begin{paracol}{2}\pcb
	Christus natus est nobis: \red{*} Veníte adorémus.
	\switchcolumn
	Christus ist uns geboren, \red{*} Kommt, lasset uns anbeten.
\end{paracol}

%\proii{S. Stephani Protomartyris}
\proii{Am Fest des Heiligen Protomärtyrers Stephanus (26. Dezember)}
\begin{paracol}{2}\pcb
	Christum natum, qui beátum hódie coronávit Stéphanum, \red{*} Veníte adorémus.
	\switchcolumn
	Christus, der uns geboren ist und der heute den heiligen Stephanus gekrönt hat, \red{*} Kommt, lasset uns anbeten.
\end{paracol}
	
%\proii{Ssmi Nominis Jesu}
\proii{Am Fest des Heiligsten Namen Jesu (Sonntag zwischen Neujahr und Epiphanie oder 2. Januar)}
\begin{paracol}{2}\pcb
	Admirábile nomen Jesu, quod est super omne nomen, \red{*} Veníte adorémus.
	\switchcolumn
	Den bewundernswerten Namen Jesu, der über allen Namen ist, \red{*} Kommt, lasset uns anbeten. 
\end{paracol}

\pagebreak
\proii{Am Fest der Erscheinung des Herrn (6. Januar)}
\begin{paracol}{2}\pcb
	Veníte adorémus eum: \red{*} quia ipse est Dóminus Deus noster.
	\switchcolumn
	Kommt, lasst uns Ihn anbeten, \red{*} denn Er selbst ist der Herr, unser Gott.
\end{paracol}

\proii{Vom Tag nach Epiphanias bis zum Fest der Taufe des Herrn (13. Januar)}
\begin{paracol}{2}\pcb
	Christus appáruit nobis \red{*} Veníte, adorémus.
	\switchcolumn
	Christus ist uns erschienen, \red{*} Kommt, lasset uns anbeten.
\end{paracol}

%\proii{S. Familiæ Jesu, Mariæ, Joseph}
\proii{Am Fest der Heiligen Familie (erster Sonntag nach Epiphanias)}
\begin{paracol}{2}\pcb
	Christum Dei Fílium, Maríæ et Joseph súbditum, \red{*} Veníte adorémus.
	\switchcolumn
	Christus den Sohn Gottes, der Maria und Joseph unterstellt war, \red{*} Kommt, lasset uns anbeten.
\end{paracol}

%\proii{In Dominicis per Annum}
\proii{An den Sonntagen nach Epiphanias und an den Sonntagen nach Pfingsten ab dem ersten Sonntag im Oktober}
\begin{paracol}{2}\pcb
	Adorémus Dóminum, \red{*} Quóniam ipse fecit nos.
	\switchcolumn
	Laßt uns den Herrn anbeten; \red{*} Denn Er hat uns geschaffen.
\end{paracol}

\proii{An den Sonntagen nach Pfingsten bis zum fünften Sonntag im September}
\begin{paracol}{2}\pcb
	Dóminum, qui fecit nos, \red{*} Veníte, adorémus.
	\switchcolumn
	Den Herrn, der uns geschaffen hat, \red{*} Kommt, lasset uns anbeten.
\end{paracol}

\proii{An Montagen nach Epiphanias und nach Pfingsten}
\begin{paracol}{2}\pcb
	Veníte \red{*} Exsultémus Dómino.
	\switchcolumn 
	Kommt, \red{*} Laßt uns jauchzen dem Herrn.
\end{paracol}
\rubric{Und es wird der entsprechende Vers im Psalm übersprungen, d. h. es folgt direkt nach der Wiederholung der Antiphon der Vers \black{Jubilémus Deo, salutári nostro} bzw. \black{Laßt uns jubeln dem Gott unseres Heils.}}
\vspace{.6em}

\proii{An Dienstagen nach Epiphanias und nach Pfingsten}
\begin{paracol}{2}\pcb
	Jubilémus Deo, \red{*} Salutári nostro.
	\switchcolumn 
	Laßt uns jubeln dem Gott, \red{*} unseren Heils.
\end{paracol}

\pagebreak
\proii{An Mittwochen nach Epiphanias und nach Pfingsten}
\begin{paracol}{2}\pcb
	Deum magnum Dóminum, \red{*} Veníte adorémus.
	\switchcolumn 
	Gott, den erhabenen Herrn, \red{*} Kommt, lasset uns anbeten.
\end{paracol}

\proii{An Donnerstagen nach Epiphanias und nach Pfingsten}
\begin{paracol}{2}\pcb
	Regem magnum Dóminum, \red{*} Veníte adorémus.
	\switchcolumn 
	Den König, den großen Herrn, \red{*} Kommt, lasset uns anbeten.
\end{paracol}

\proii{An Freitagen nach Epiphanias und nach Pfingsten}
\begin{paracol}{2}\pcb
	Dóminum, Deum nostrum, \red{*} Veníte adorémus.
	\switchcolumn 
	Den Herrn, unseren Gott, \red{*} Kommt, lasset uns anbeten.
\end{paracol}

\proii{An Samstagen nach Epiphanias und nach Pfingsten}
\begin{paracol}{2}\pcb
	Pópulus Dómini, et oves páscuæ ejus: \red{*} Veníte, adorémus.
	\switchcolumn 
	O Volk des Herrn und ihr Schafe seiner Weide: \red{*} Kommt, wir wollen Ihn anbeten.
\end{paracol}

\proii{In der Vorfastenzeit}
\begin{paracol}{2}\pcb
	Præoccupémus fáciem Dómini:~\red{*} Et in psalmis jubilémus ei.
	\switchcolumn
	Laßt uns im Voraus Sein Antlitz im Bekenntnis einnehmen: \red{*} Und Ihn mit Psalmen umjubeln.
\end{paracol}

\proii{In der Fastenzeit}
\begin{paracol}{2}\pcb
	Non sit vobis vanum mane súrgere ante lucem: \red{*} Quia promísit Dóminus corónam vigilántibus.
	\switchcolumn
	Laßt es euch nicht fruchtlos sein, frühzeitig vor dem Tageslichte aufzustehen; \red{*} Denn eine Krone hat der Herr verheißen denen, die wachen.
\end{paracol}

\proii{In der Passionszeit}
\begin{paracol}{2}\pcb
	Hódie, si vocem Dómini audiéritis, \red{*} Nolíte obduráre corda vestra.
	\switchcolumn 
	Heute, wenn ihr doch Seine Stimme hörtet, \red{*} verdunkelt nicht eure Herzen.
\end{paracol}
\rubric{Und es wird der entsprechende Vers im Psalm übersprungen, d. h. es folgt direkt nach der Wiederholung der Antiphon der Vers \black{Sicut in exacerbatióne secúndum diem tentatiónis in desérto} bzw. \black{So wie in der Erbitterung nach dem Tag der Versuchung in der Wüste.}}
\vspace{.6em}

\proii{In der Osterzeit}
\begin{paracol}{2}\pcb
	Surréxit Dóminus vere, \red{*} Allelúja.
	\switchcolumn 
	Der Herr ist wahrhaftig auferstanden, \red{*} Alleluja.
\end{paracol}

\proii{Ab Christi Himmelfahrt bis zur Pfingstvigil}
\begin{paracol}{2}\pcb
	Allelúja, Christum Dóminum ascendéntem in cælum, \red{*} Veníte adorémus, allelúja.
	\switchcolumn 
	Alleluja, den in den Himmel emporsteigenden Christus, den Herrn, \red{*} Kommt, lasset uns anbeten, alleluja.
\end{paracol}

\proii{An Pfingsten und in der Oktav}
\begin{paracol}{2}\pcb
	Allelúja, Spíritus Dómini replévit orbem terrárum: \red{*} Veníte, adorémus, allelúja.
	\switchcolumn 
	Alleluja, der Geist des Herrn erfüllt den Erdkreis: \red{*} Kommt, lasset uns anbeten, alleluja.
\end{paracol}

\proii{Am Dreifaltigkeitssonntag}
\begin{paracol}{2}\pcb
	Deum verum, unum in Trinitáte, et Trinitátem in Unitáte, \red{*} Veníte, adorémus.
	\switchcolumn 
	Den wahren Gott, der eins ist in der Dreifaltigkeit und Dreifaltig in der Einheit, \red{*} Kommt, lasset uns anbeten.
\end{paracol}

\proii{An Fronleichnam}
\begin{paracol}{2}\pcb
	Christum Regem adorémus dominántem géntibus: \red{*} Qui se manducántibus dat spíritus pinguédinem.
	\switchcolumn 
	Christus, den König lasst uns anbeten, der die Völkerscharen gebietet: \red{*} der sich den die Fülle des Geistes Speißenden gibt.
\end{paracol}

\proii{Am Fest des Heilisten Herzen Jesu}
\begin{paracol}{2}\pcb
	Cor Jesu amóre nostri vulnerátum \red{*} Veníte, adorémus.
	\switchcolumn 
	Das Herz Jesu, das in Liebe zu uns verletzt worden ist, \red{*} Kommt, lasst es uns anbeten.
\end{paracol}



\pagebreak
\dieii{}{Commune Sanctorum}{}{}
\proii{An Heiligenfesten werden folgende Invitatorien verwendet, wenn nicht ein anderes im Proprium angegeben ist.}

\proii{Für Apostel und Evangelisten}
\begin{paracol}{2}\pcb
	Regem Apostolórum Dóminum,~\red{*} Veníte, adorémus. \red{(Tempore Paschale:} Allelúja.\red{)}
	\switchcolumn 
	Den König der Apostel, den Herrn, \red{*} Kommt, lasset uns anbeten. \red{(In der Osterzeit:} Alleluja.\red{)}
\end{paracol}

\proii{Für einen oder mehrere Märtyrer außerhalb der Osterzeit}
\begin{paracol}{2}\pcb
	Regem Mártyrum Dóminum, \red{*} Veníte, adorémus.
	\switchcolumn 
	Den König der Märtyrer, den Herrn, \red{*} Kommt, lasset uns anbeten.
\end{paracol}

\proii{Für einen oder mehrere Märtyrer in der Osterzeit}
\begin{paracol}{2}\pcb
	Exsúltent in Dómino Sancti, \red{*} Allelúja.
	\switchcolumn 
	Es mögen jubeln im Herrn die Heiligen, \red{*} Alleluja.
\end{paracol}

\proii{Für Bekenner}
\begin{paracol}{2}\pcb
	Regem Confessórum Dóminum,~\red{*} Veníte, adorémus. \TPA
	\switchcolumn 
	Den König der Bekenner, den Herrn, \red{*} Kommt, lasset uns anbeten. \OZA
\end{paracol}

\proii{Für Jungfrauen}
\begin{paracol}{2}\pcb
	Regem Virgínum Dóminum, \red{*} Veníte, adorémus. \TPA
	\switchcolumn 
	Den König der Jungfrauen, den Herrn, \red{*} Kommt, lasset uns anbeten. \OZA
\end{paracol}

\proii{Für einzelne heilige Frauen, die keine Jungfrauen sind}
\begin{paracol}{2}\pcb
	Laudémus Deum nostrum \red{*} In confessióne beáæ \red{N.} \TPA
	\switchcolumn 
	Laßt uns unseren Gott loben, \red{*} für das Bekenntnis der heiligen \red{N.} \OZA
\end{paracol}
\rubric{Bei N. ist der jeweilige Name (im Genitiv) einzusetzen.}
\vfill

\pagebreak
\proii{Für mehrere heilige Frauen, die keine Jungfrauen sind}
\begin{paracol}{2}\pcb
	Laudémus Deum nostrum \red{*} In confessióne beatárum \red{N.} et \red{N.} \TPA
	\switchcolumn 
	Laßt uns unseren Gott loben, \red{*} für das Bekenntnis der heiligen \red{N.} und \red{N.} \OZA
\end{paracol}

\proii{Für das Kirchweihfest}
\begin{paracol}{2}\pcb
	Domum Dei decet sanctitúdo: \red{*} Sponsum ejus Christum adorémus in ea. \TPA
	\switchcolumn 
	Dem Haus Gottes gebührt Heiligkeit: \red{*} Laßt uns Christus seinen Bräutigam in ihm anbeten. \OZA
\end{paracol}

\proii{Für Feste der Seligen Jungfrau Maria}
\begin{paracol}{2}\pcb
	Sancta María, Dei Génitrix Virgo, \red{*} Intercéde pro nobis. \TPA
	\switchcolumn 
	Heilige Maria, Jungfrau, Mutter Gottes, \red{*} bitte für uns. \OZA
\end{paracol}

\proii{Für das Offizium der Heiligen Maria am Samstag}
\begin{paracol}{2}\pcb
	Ave, María, grátia plena: \red{*} Dóminus tecum. \TPA
	\switchcolumn 
	Gegrüßest seist du, Maria, voll der Gnade, \red{*} der Herr ist mit dir. \OZA
\end{paracol}

\proii{Für das Totenoffizium}
\begin{paracol}{2}\pcb
	Regem, cui ómnia vivunt, \red{*} Veníte adorémus. \red{(T. P.} Allelúja.\red{)}
	\switchcolumn 
	Den König, zu dem alles lebt, \red{*} Kommt, lasset uns anbeten. \OZA
\end{paracol}



\pagebreak
\dieii{}{Proprium Sanctorum}{}{}
\proii{Für Feste im Sanktorale, die ein eigenes Invitatorium haben.}

\proii{Am Fest der unbefleckten Empfängnis der Seligen Jungfrau Maria\\(8. Dezember)}
\begin{paracol}{2}\pcb
	Immaculátam Conceptiónem Vírginis Maríæ celebrémus: \red{*} Christum ejus Fílium adorémus Dóminum.
	\switchcolumn
	Die unbefleckte Empfängnis der Jungfrau Maria wollen lasst uns feiern, \red{*} Christus, ihren Sohn, lasst uns als den Herrn anbeten.
\end{paracol}

\proii{Am Fest der Bekehrung des Heiligen Apostels Paulus (25. Januar)}
\begin{paracol}{2}\pcb
	Laudémus Deum nostrum, \red{*} In conversióne Doctóris géntium.
	\switchcolumn 
	Laßt uns Gott loben, \red{*} für die Bekehrung des Doktors der Heiden.
\end{paracol}

\proii{Am Fest Mariä Lichtmess (2. Februar)}
\begin{paracol}{2}\pcb
	Ecce venit ad templum sanctum suum Dominátor Dóminus: \red{*} Gaude et lætáre, Sion, occúrrens Deo tuo.
	\switchcolumn
	Sieh, der Herr der Herrscher kommt zu seinem heiligen Tempel: \red{*} Freue dich und sei fröhlich Zion, deinem Gott entgegenschreitend.
\end{paracol}

\proii{Am Fest Unserer Lieben Frau von Lourdes (11. Februar)}

\vspace{-.3em}

\rubric{Wie am Fest der unbefleckten Empfängnis.}

\vspace{.3em}

\proii{Am Fest der Cathedra Petri (22. Februar)}
\begin{paracol}{2}\pcb
	Tu es pastor óvium, Princeps Apostolórum: \red{*} Tibi trádidit Deus claves regni cælórum.
	\switchcolumn 
	Du bist der Hüter der Schafe, der Fürst der Apostel: \red{*} Dir hat Gott die Schlüssel des Himmelreiches übergeben.
\end{paracol}

\proii{Am Fest der Heiligen Perpetua und Felicitas, Märtyrerinnen (6. März)}
\begin{paracol}{2}\pcb
	Laudémus Deum nostrum \red{*} In confessióne beatárum Perpétuæ et Felicitátis.
	\switchcolumn 
	Laßt uns unseren Gott loben, \red{*} für das Bekenntnis der heiligen Perpetua und Felicitas.
\end{paracol}

\pagebreak
\proii{Am Fest der Heiligen Francesca Romana, Witwe (9. März)}
\begin{paracol}{2}\pcb
	Laudémus Deum nostrum, \red{*} In confessióne beátæ Francíscæ.
	\switchcolumn 
	Laßt uns unseren Gott loben, \red{*} für das Bekenntnis der heiligen Franziska.
\end{paracol}

\proii{Am Fest des Heiligen Joseph (19. März)}
\begin{paracol}{2}\pcb
	Christum Dei Fílium, qui putári dignátus est fílius Joseph, \red{*} Veníte adorémus. \TPA
	\switchcolumn 
	Christus, den Sohn Gottes, der sich herabgelassen hat, als Sohn Josephs zu gelten, \red{*} Kommt, lasset uns anbeten. \OZA
\end{paracol}

\proii{An Maria Verkündigung (25. März)}
\begin{paracol}{2}\pcb
	Ave, María, grátia plena: \red{*} Dóminus tecum. \TPA
	\switchcolumn 
	Gegrüßest seist du, Maria, voll der Gnade, \red{*} der Herr ist mit dir. \OZA
\end{paracol}

\proii{Am Fest Josephs des Arbeiters (1. Mai)}
\begin{paracol}{2}\pcb
	Regem regum Dóminum, qui putári dignátus est fabri fílius, \red{*} Veníte, adorémus, allelúja.
	\switchcolumn 
	Den König der Könige den Herrn, der sich herabgelassen hat, als Sohn eines Handwerkers zu gelten, \red{*} Kommt, lasset uns anbeten.
\end{paracol}

\proii{Am Fest der Heiligen Monica, Witwe (4. Mai)}
\begin{paracol}{2}\pcb
	Laudémus Deum nostrum \red{*} In confessióne beátæ Mónicæ, allelúja. 
	\switchcolumn 
	Laßt uns unseren Gott loben, \red{*} für das Bekenntnis der Heiligen Monica.
\end{paracol}

\proii{An Maria Königin (31. Mai)}
\begin{paracol}{2}\pcb
	Christum Regem, qui suam coronávit Matrem, \red{*} Veníte, adorémus, allelúja.
	\switchcolumn 
	Christus den König, der seine Mutter gekrönt hat, \red{*} Kommt, lasset uns anbeten, alleluja.
\end{paracol}

\pagebreak
\proii{Am Fest der Heiligen Margarita, Witwe (10. Juni)}
\begin{paracol}{2}\pcb
	Laudémus Deum nostrum, \red{*} In confessióne beátæ Margarítæ. \TPA
	\switchcolumn 
	Laßt uns unseren Gott loben, \red{*} für das Bekenntnis der heiligen Margarita. \OZA
\end{paracol}

\proii{Am Fest der Geburt Johannes des Täufers (24. Juni)}
\begin{paracol}{2}\pcb
	Regem Præcursóris Dóminum, \red{*} Veníte, adorémus.
	\switchcolumn 
	Den König des Vorläufers, den Herrn, \red{*} Kommt, lasset uns anbeten.
\end{paracol}

\proii{Am Fest des Heiligsten Blutes unseres Herrn Jesus Christus (1. Juli)}
\begin{paracol}{2}\pcb
	Christum Dei Fílium, qui suo nos redémit sánguine, \red{*} Veníte, adorémus.
	\switchcolumn 
	Christus, den Sohn Gottes, der uns mit Seinem Blut losgekauft hat, \red{*} Kommt, lasset uns anbeten.
\end{paracol}

\proii{An Mariä Heimsuchung (2. Juli)}
\begin{paracol}{2}\pcb
	Visitatiónem Vírginis Maríæ celebrémus: \red{*} Christum ejus Fílium adorémus Dóminum.
	\switchcolumn
	Laßt uns die Heimsuchung der Jungfrau Maria feiern: \red{*} Christus ihren Sohn lasst uns als den Herrn anbeten.
\end{paracol}

\proii{Am Fest der Heiligen Elisabeth von Portugal, Witwe (8. Juli)}
\begin{paracol}{2}\pcb
	Laudémus Deum nostrum, \red{*} In confessióne beátæ Elísabeth.
	\switchcolumn 
	Laßt uns unseren Gott loben, \red{*} für das Bekenntnis der heiligen Elisabeth.
\end{paracol}

\proii{Am Fest der Heiligen Maria Magdalena (22. Juli)}
\begin{paracol}{2}\pcb
	Laudémus Deum nostrum \red{*} In conversióne Maríæ Magdalénæ.
	\switchcolumn
	Laßt uns unseren Gott loben, \red{*} für die Bekehrung der heiligen Maria Magdalena.
\end{paracol}

\proii{Am Fest der Heiligen Anna, Mutter der Seligen Jungfrau Maria (26. Juli)}
\begin{paracol}{2}\pcb
	Laudémus Deum nostrum, \red{*} In confessióne beátæ Anna.
	\switchcolumn 
	Laßt uns unseren Gott loben, \red{*} für das Bekenntnis der heiligen Anna.
\end{paracol}

\pagebreak
\proii{Am Fest der Verklärung unseres Herrn Jesus Christus (6. August)}
\begin{paracol}{2}\pcb
	Summum Regem glóriæ, \red{*} Christum adorémus.
	\switchcolumn
	Den höchsten König der Herrlichkeit, \red{*} Christus lasst uns anbeten.
\end{paracol}

\proii{Am Fest des Heiligen Märtyrers Laurentius (10. August)}
\begin{paracol}{2}\pcb
	Beátus Lauréntius Christi Martyr triúmphat coronátus in cælis: \red{*} Veníte, adorémus Dóminum.
	\switchcolumn 
	Der selige Märtyrer Christi Laurentius triumphiert gekrönt im Himmel: \red{*} Kommt, lasst uns den Herrn anbeten.
\end{paracol}

\proii{An Maria Himmelfahrt (15. August)}
\begin{paracol}{2}\pcb
	Veníte, adorémus Regem regum, \red{*} Cujus hódie ad æthéreum Virgo Mater assúmpta est cælum.
	\switchcolumn 
	Kommt, lasst uns anbeten den König der Könige, \red{*} dessen Mutter heute in den lichtreichen Himmel aufgenommen wurde.
\end{paracol}

\proii{Am Fest der Heiligen Johanna Franziska Fremiot von Chantal, Witwe (21. August)}
\begin{paracol}{2}\pcb
	Laudémus Deum nostrum, \red{*} In confessióne beátæ Joánnæ Francíscæ.
	\switchcolumn 
	Laßt uns unseren Gott loben, \red{*} für das Bekenntnis der heiligen Johanna Franziska.
\end{paracol}

\proii{Am Fest der Enthauptung Johannes des Täufers (29. August)}
\begin{paracol}{2}\pcb
	Regem Mártyrum Dóminum, \red{*} Veníte, adorémus.
	\switchcolumn 
	Den König der Märtyrer, den Herrn, \red{*} Kommt, lasset uns anbeten.
\end{paracol}

\proii{Am Fest der Geburt der Seligen Jungfrau Maria (8. September)}
\begin{paracol}{2}\pcb
	Nativitátem Vírginis Maríæ celebrémus: \red{*} Christum ejus Fílium adorémus Dóminum.
	\switchcolumn 
	Die Geburt der Jungfrau Maria lasst uns feiern: \red{*} Christus, ihren Sohn, lasst uns als den Herrn anbeten.
\end{paracol}

\pagebreak
\proii{An Kreuzerhöhung (14. September)}
\begin{paracol}{2}\pcb
	Christum Regem pro nobis in Cruce exaltátum, \red{*} Veníte, adorémus.
	\switchcolumn 
	Christus, den König, der für uns am Kreuz erhöht wurde, \red{*} Kommt, lasset uns anbeten.
\end{paracol}

\proii{Am Fest der sieben Schmerzen der Seligen Jungfrau Maria (15. September)}
\begin{paracol}{2}\pcb
	Stémus juxta crucem cum María Matre Jesu, \red{*} Cujus ánimam dolóris gládjus pertransívit.
	\switchcolumn 
	Laßt uns neben dem Kreuz mit Maria der Mutter Jesu stehen, \red{*} Deren Seele das Schwert des Leidens durchbohrt hat.
\end{paracol}

\proii{Am Fest des Heiligen Erzengels Michael (29. September)}
\begin{paracol}{2}\pcb
	Regem Archangelórum Dóminum, \red{*} Veníte, adorémus.
	\switchcolumn 
	Den König der Erzengel, den Herrn, \red{*} Kommt, lasset uns anbeten.
\end{paracol}

\proii{Am Fest der Heiligen Schutzengel (2. Oktober)}
\begin{paracol}{2}\pcb
	Regem Angelórum Dóminum, \red{*} Veníte, adorémus.
	\switchcolumn 
	Den König der Engel, den Herrn, \red{*} Kommt, lasset uns anbeten.
\end{paracol}

\proii{Am Fest des Rosenkranzes der Seligen Jungfrau Maria (7. Oktober)}
\begin{paracol}{2}\pcb
	Solemnitátem Rosárii Vírginis Maríæ celebrémus: \red{*} Christum ejus Fílium adorémus Dóminum.
	\switchcolumn 
	Die Feierlichkeit des Rosenkranzes der Jungfrau Maria lasst uns feiern: \red{*} Christus, ihren Sohn, lasst uns als den Herrn anbeten.
\end{paracol}

\proii{Am Fest der Mutterschaft der Seligen Jungfrau Maria (11. Oktober)}
\begin{paracol}{2}\pcb
	Maternitátem beátæ Vírginis Maríæ celebrémus: \red{*} Christum ejus Fílium adorémus Dóminum.
	\switchcolumn 
	Die Mutterschaft der Seligen Jungfrau Maria lasst uns feiern: \red{*} Christus, ihren Sohn, lasst uns als den Herrn anbeten.
\end{paracol}

\pagebreak
\proii{Am Fest der Heiligen Hedwig, Witwe (16. Oktober)}
\begin{paracol}{2}\pcb
	Laudémus Deum nostrum, \red{*} In confessióne beátæ Hedwígis.
	\switchcolumn 
	Laßt uns unseren Gott loben, \red{*} für das Bekenntnis der heiligen Hedwig.
\end{paracol}

\proii{Am Fest des Heiligen Erzengels Raphael (24. Oktober)}
\begin{paracol}{2}\pcb
	Regem Archangelórum Dóminum, \red{*} Veníte, adorémus.
	\switchcolumn 
	Den König der Erzengel, den Herrn, \red{*} Kommt, lasset uns anbeten.
\end{paracol}

\proii{Am Christköngigsfest (letzter Sonntag im Oktober)}
\begin{paracol}{2}\pcb
	Jesum Christum, Regem regum: \red{*} Veníte adorémus.
	\switchcolumn
	Jesus Christus, den König der Könige, \red{*} Kommt, lasset uns anbeten.
\end{paracol}

\proii{An Allerheiligen (1. November)}
\begin{paracol}{2}\pcb
	Regem regum Dóminum veníte adorémus: \red{*} Quia ipse est coróna Sanctórum ómnium.
	\switchcolumn 
	Kommt, lasst uns anbeten den König der Könige, den Herrn: \red{*} Denn er selbst ist die Krone aller Heiligen.
\end{paracol}

\proii{Am Fest des Heiligen Martin, Bischof und Bekenner (11. November)}
\begin{paracol}{2}\pcb
	Laudémus Deum nostrum, \red{*} In confessióne beáti Martíni.
	\switchcolumn 
	Laßt uns unseren Gott loben, \red{*} für das Bekenntnis des heiligen Martin.
\end{paracol}

\proii{Am Fest der Heiligen Elisabeth von Thüringen, Witwe (19. November)}
\begin{paracol}{2}\pcb
	Laudémus Deum nostrum, \red{*} In confessióne beátæ Elísabeth.
	\switchcolumn 
	Laßt uns unseren Gott loben, \red{*} für das Bekenntnis der heiligen Elisabeth.
\end{paracol}

\ornamentvi

\end{document}
