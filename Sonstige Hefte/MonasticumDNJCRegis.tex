\documentclass[fontsize=8pt,paper=A6,twoside,BCOR=1mm,DIV=22,headinclude]{scrarticle}
\usepackage{breviarium}
\begin{document}
%\titulum{Dominica ultima Octobris}{D. N. Jesu Christi Regis}{Duplex I classis}
\begin{multicols}{2}
\thispagestyle{empty}
\privdieii{Dominica ultima Octobris}{D. N. \red{J}esu \red{C}hristi \red{R}egis}{}{Duplex I classis}
{\setstretch{0.98}
\hora{In I Vesperis}

\vspace{-.3em}

\pars{Antiphona 1}

\vspace{-.5em}

\Yiv{P}{acíficus} \f vocábitur, et thronus ejus erit firmíssimus in perpétuum. \DixitDominus

\Av{2} Regnum ejus \f regnum sempitérnum est, et omnes reges sérvient ei et obédient.

\Av{3} Ecce Vir Oriens \f nomen ejus: sedébit et dominábitur, et loquétur pacem géntibus.

\Av{4} Ecce dedi te \f in lucem géntium, ut sis salus mea usque ad extrémum terræ.

\lectiocap{Capitulum}{Col 1:12-13}
\y{F}{ratres:} Grátias ágimus Deo Patri, qui dignos nos fecit in partem sortis sanctórum in lúmine, qui erípuit nos de potestáte tenebrárum, et tránstulit in regnum Fílii dilectiónis suæ.

\Rbr Afférte Dómino, \red{*} Famíliæ populórum.
\red{A}fférte.
\V Afférte Dómino glóriam et impérium.
\red{F}amíliæ.
\red{G}lória Patri.
\red{A}fférte.

\pars{Hymnus}

\vspace{-.4em}

}

{\setstretch{0.955}
\begin{hymnus}
	\y{T}{e} sæculórum Príncipem,\\
	\hspace{1.6em} Te, Christe, Regem géntium,\\
Te méntium, te córdium\\
Unum fatémur árbitrum.

\red{S}celésta turba clámitat:\\
Regnáre Christum nólumus:\\
Te nos ovántes ómnium\\
Regem suprémum dícimus.

\red{O} Christe, Princeps Pácifer,\\
Mentes rebélles súbice:\\
Tuóque amóre dévios,\\
Ovíle in unum cóngrega.

\red{A}d hoc cruénta ab árbore\\
Pendes apértis brácchiis,\\
Diráque fossum cúspide\\
Cor igne flagrans éxhibes.

\red{A}d hoc in aris ábderis\\
Vini dapísque imágine,\\
Fundens salútem fíliis\\
Transverberáto péctore.

\red{T}e natiónum Prǽsides\\
Honóre tollant público,\\
Colant magístri, iúdices,\\
Leges et artes éxprimant.

\red{S}ubmíssa regum fúlgeant\\
Tibi dicáta insígnia:\\
Mitíque sceptro pátriam\\
Domósque subde cívium.

\red{G}lória tibi, Dómine,\\
Qui sceptra mundi témperas,\\
Cum Patre, et almo Spíritu,\\
In sempitérna sǽcula. 
Amen.
\end{hymnus}

}

{\setstretch{0.97}
\rubric{Sic terminantur omnes Hymni usque ad Completorium sequentis diei inclusive.}

\V Data est mihi omnis potéstas.\\
\R In cælo et in terra.

\M Dabit illi \f Dóminus Deus sedem David, patris ejus: et regnábit in domo Jacob in ætérnum, et regni ejus non erit finis, allelúja.

\pars{Oratio}
\y{O}{mnípotens} sempitérne Deus, qui in dilécto Fílio tuo, universórum Rege, ómnia instauráre voluísti: concéde propítius; ut cunctæ famíliæ géntium, peccáti vúlnere disgregátæ, ejus suavíssimo subdántur império: Qui tecum vivit.

\rubric{Et fit Commemoratio Dominicæ occurrentis.}

\hora{Ad Matutinum}

\I Jesum Christum, Regem regum: \red{*} Veníte, adorémus. \Venite

}

\pars{Hymnus}
\vspace{-.3em}

\begin{hymnus}
	\y{Æ}{térna} Imago Altíssimi,\\
	\hspace{2.8em}Lumen, Deus, de Lúmine,\\
Tibi, Redémptor glória,\\
Honor, potéstas régia.

\red{T}u solus ante sǽcula\\
Spes atque centrum témporum,\\
Cui jure sceptrum géntium\\
Pater suprémum crédidit.

\red{T}u flos pudícæ Vírginis,\\
Nostræ caput propáginis,\\
Lapis cadúcus vértice\\
Ac mole terras óccupans.

\red{D}iro tyránno súbdita,\\
Damnáta stirps mortálium,\\
Per te refrégit víncula\\
Sibíque cælum víndicat.

\red{D}octor, Sacérdos, Légifer\\
Præfers notátum sánguine\\
In veste « Princeps príncipum\\
Regúmque Rex Altíssimus ».

\red{T}ibi voléntes súbdimur,\\
Qui jure cunctis ímperas:\\
Hæc cívium beátitas\\
Tuis subésse légibus.

\red{G}lória tibi, Dómine,\\
Qui sceptra mundi témperas,\\
Cum Patre, et almo Spíritu,\\
In sempitérna sǽcula.
Amen.
\end{hymnus}

\nocturn{In I Nocturno}

\A Ego autem \f constitútus sum Rex ab eo super Sion montem sanctum ejus, prǽdicans præcéptum ejus.

\input{../PsalmiM/2Y.tex}

{\setstretch{0.99}
\A Ego autem constitútus sum Rex ab eo super Sion montem sanctum ejus, prǽdicans præcéptum ejus.

\A Glória \f et honóre coronásti eum, Dómine: ómnia subjecísti sub pédibus ejus.

\input{../PsalmiM/8.tex}

\A Glória et honóre coronásti eum, Dómine: ómnia subjecísti sub pédibus ejus.

\A Elevámini, \f portæ æternáles, et introíbit Rex glóriæ.

\input{../PsalmiM/23.tex}

\A Elevámini, portæ æternáles, et introíbit Rex glóriæ.

\A Sedébit \f Dóminus Rex in ætérnum: Dóminus benedícet pópulo suo in pace.

\begin{psalmus}
    \pars{Psalmus 28}

    \y{A}{fférte} Dómino, fílii Dei: * afférte Dómino fílios aríetum.

    Afférte Dómino glóriam et honórem, \f afférte Dómino glóriam nómini ejus: * adoráte Dóminum in átrio sancto ejus.

    Vox Dómini super aquas, \f Deus majestátis intónuit: * Dóminus super aquas multas.

    Vox Dómini in virtúte: * vox Dómini in magnificéntia.

    Vox Dómini confringéntis cedros: * et confrínget Dóminus cedros Líbani:

    Et commínuet eas tamquam vítulum Líbani: * et diléctus quemádmodum fílius unicórnium.

    Vox Dómini intercidéntis flammam ignis: \f vox Dómini concutiéntis desértum: * et commovébit Dóminus desértum Cades.

    Vox Dómini præparántis cervos, et revelábit condénsa: * et in templo ejus omnes dicent glóriam.

    Dóminus dilúvium inhabitáre facit: * et sedébit Dóminus Rex in ætérnum.

    Dóminus virtútem pópulo suo dabit: * Dóminus benedícet pópulo suo in pace.

\end{psalmus}


\A Sedébit Dóminus Rex in ætérnum: Dóminus benedícet pópulo suo in pace.

\A Virga directiónis, \f virga regni tui: proptérea pópuli confitebúntur tibi in ætérnum, et in sǽculum sǽculi.

\input{../PsalmiM/44.tex}

\A Virga directiónis, virga regni tui: proptérea pópuli confitebúntur tibi in ætérnum, et in sǽculum sǽculi.

\A Psállite \f Regi nostro, psállite: quóniam Rex magnus super omnem terram.

\input{../PsalmiM/46.tex}

\A Psállite Regi nostro, psállite: quóniam Rex magnus super omnem terram.

}

\V Dáta est mihi omnis potéstas.
\R In cælo et in terra.

{\setstretch{0.98}
\scriptura{De Epistola beáti Pauli Apóstoli ad Colossénses}
\lectiocap{Lectio i}{Cap. 1, 3-23}
\Y{G}{rátias} agimus Deo, et Patri Dómini nostri Jesu Christi, semper pro vobis orántes, audiéntes fidem vestram in Christo Jesu, et dilectiónem quam habétis in sanctos omnes, propter spem, quæ repósita est vobis in cælis, quam audístis in verbo veritátis Evangélii, quod pervénit ad vos, sicut et in univérso mundo est, et fructíficat, et crescit, sicut in vobis, ex ea die, qua audístis, et cognovístis grátiam Dei in veritáte, sicut didicístis ab Epáphra, caríssimo consérvo nostro, qui est fidélis pro vobis miníster Christi Jesu, qui étiam manifestávit nobis dilectiónem vestram in spíritu.

\R Super sólium \f David et super regnum ejus sedébit in ætérnum:
\red{*} Et vocábitur nomen ejus Deus, Fortis, Princeps pacis.
\V Multiplicábitur ejus impérium, et pacis non erit finis.
\red{E}t vocábitur.

\lectio{Lectio ii}
\y{I}{deo} et nos ex qua die audívimus, non cessámus pro vobis orántes, et postulántes ut impleámini agnitióne voluntátis ejus, in omni sapiéntia et intelléctu spiritáli; ut ambulétis digne Deo per ómnia placéntes; in omni ópere bono fructificántes, et crescéntes in sciéntia Dei: in omni virtúte confortáti secúndum poténtiam claritátis ejus, in omni patiéntia et longanimitáte cum gáudio; grátias agéntes Deo Patri, qui dignos nos fecit in partem sortis sanctórum in lúmine, qui erípuit nos de potestáte tenebrárum, et tránstulit in regnum Fílii dilectiónis suæ, in quo habémus redemptiónem per sánguinem ejus, remissiónem peccatórum.

\R Aspiciébam \f in visu noctis, et ecce in núbibus cæli Fílius hóminis veniébat: et datum est ei regnum et honor:
\red{*} Et omnis pópulus, tribus et linguæ sérvient ei.
\V Potéstas ejus, potéstas ætérna, quæ non auferétur: et regnum ejus, quod non corrumpétur.
\red{E}t omnis.

}

\lectio{Lectio iii}
\y{Q}{ui} est imágo Dei invisíbilis, primogénitus omnis creatúræ; quóniam in ipso cóndita sunt univérsa in cælis et in terra, visibília, et invisibília, sive Throni, sive Dominatiónes, sive Principátus, sive Potestátes: ómnia per ipsum et in ipso creáta sunt: et ipse est ante omnes, et ómnia in ipso constant. Et ipse est caput córporis Ecclésiæ, qui est princípium, primogénitus ex mórtuis, ut sit in ómnibus ipse primátum tenens; quia in ipso complácuit omnem plenitúdinem inhabitáre, et per eum reconciliáre ómnia in ipsum, pacíficans per sánguinem crucis ejus, sive quæ in terris, sive quæ in cælis sunt.

\R Tu Béthlehem \f Ephrata, párvulus in míllibus Juda: ex te mihi egrediétur qui sit dominátor in Israël:
\red{*} Et erit iste Pax.
\V Egréssus ejus ab inítio, a diébus æternitátis: stabit, et pascet in fortitúdine Dómini.
\red{E}t erit.

\lectio{Lectio iv}
\y{E}{t} vos, cum essétis aliquándo alienáti, et inimíci sensu in opéribus malis; nunc autem reconciliávit in córpore carnis ejus per mortem, exhibére vos sanctos, et immaculátos, et irreprehensíbiles coram ipso; si tamen permanétis in fide fundáti, et stábiles, et immóbiles a spe evangélii, quod audístis, quod prædicátum est in univérsa creatúra quæ sub cælo est, cujus factus sum ego Paulus miníster.

\R Exsúlta \f satis, fília Sion; iúbila, fília Jerúsalem: ecce Rex tuus véniet tibi justus et Salvátor:
\red{*} Et loquétur pacem géntibus.
\V Potéstas ejus a mari usque ad mare: et a flumínibus usque ad fines terræ.
\red{E}t loquétur.
\red{G}lória Patri.
\red{E}t loquétur.

\nocturn{In II Nocturno}

\A Hic est Deus, \f Deus noster in ætérnum: ipse reget nos in sǽcula.

\columnbreak

{\setstretch{0.98}
\input{../PsalmiM/47Y.tex}

\A Hic est Deus, Deus noster in ætérnum: ipse reget nos in sǽcula.

\A Benedicéntur \f in ipso omnes tribus terræ; omnes Gentes magnificábunt eum.

\input{../PsalmiM/71.tex}

\A Benedicéntur in ipso omnes tribus terræ; omnes Gentes magnificábunt eum.

\A Dícite \f in géntibus quia Dóminus regnávit; ipse judicábit orbem terræ in æquitáte, et pópulos in veritáte sua.

\input{../PsalmiM/95.tex}


\A Dícite in géntibus quia Dóminus regnávit; ipse judicábit orbem terræ in æquitáte, et pópulos in veritáte sua.

\A Dóminus \f regnávit, exsúltet terra: læténtur ínsulæ multæ.

\input{../PsalmiM/96.tex}

\A Dóminus regnávit, exsúltet terra: læténtur ínsulæ multæ.

\A Jubiláte \f in conspéctu regis Dómini, quóniam venit judicáre terram.

\input{../PsalmiM/97.tex}

\A Jubiláte in conspéctu regis Dómini, quóniam venit judicáre terram.

\A Exaltáte \f Dóminum Deum nostrum, et adoráte in monte sancto ejus.

\input{../PsalmiM/98.tex}

\A Exaltáte Dóminum Deum nostrum, et adoráte in monte sancto ejus.

\V Afférte Dómino, famíliæ populórum.
\R Afférte Dómino glóriam et impérium.

}

{\setstretch{0.98}
\scriptura{Ex Lítteris Encýclicis Pii Papæ undécimi}
\lectio{Lectio v}
\ex{Litt. Encycl. \black{Quas primas} diei 11 Decembris 1925}
\Y{T}{ransláta} verbi significatióne Rex appellarétur Christus ob summum excelléntiæ gradum, quo inter omnes res creátas præstat atque éminet, jam diu communitérque usu venit. Ita enim fit, ut regnare is « in méntibus hóminum » dicátur non tam ob mentis áciem scientiǽque suæ amplitúdinem, quam quod ipse est Véritas, et veritátem ab eo mortáles hauríre atque obediénter accípere necésse est; « in voluntátibus » item « hóminum », quia non modo sanctitáti in eo voluntátis divínæ perfécta prorsus respóndet humánæ intégritas atque obtemperátio, sed étiam líberæ voluntáti nostræ id permotióne instinctúque suo súbicit, unde ad nobilíssima quæque exardescámus. « Córdium » dénique « Rex » Christus agnóscitur ob ejus « supereminéntem sciéntiæ caritátem » et mansuetúdinem benignitatémque ánimos alliciéntem: nec enim quemquam usque ádeo ab universitáte géntium, ut Christum Jesum, aut amári aliquándo cóntigit aut amátum iri in pósterum contínget.

\R Opórtet \f illum regnáre, quóniam ómnia subiécit Deus sub pédibus ejus:
\red{*} Ut sit Deus ómnia in ómnibus.
\V Cum subiécta fúerint illi ómnia, tunc et ipse Fílius subiéctus erit Patri.
\red{U}t sit.

\lectio{Lectio vi}
\y{V}{erum}, ut rem préssius ingrediámur, nemo non videt, nomen potestatémque regis, própria quidem verbi significatióne, Christo hómini vindicári oportére; nam, nisi quátenus homo est, a Patre « potestátem et honórem et regnum » accepísse dici nequit, quandóquidem Dei Verbum, cui éadem est cum Patre substántia, non potest ómnia cum Patre non habére commúnia, proptereáque ipsum in res creátas univérsas summum atque absolutíssimum impérium. Quo autem hæc Dómini nostri dígnitas et potéstas fundaménto consístat, apte Cyríllus Alexandrínus animadvértit: « Omnium, ut verbo dicam, creaturárum dominátum óbtinet, non per vim extórtum, nec aliúnde invéctum, sed esséntia sua et natúra; » scílicet ejus principátus illa nítitur unióne mirábili, quam hypostáticam appéllant.

\R Fecit nos \f regnum et sacerdótes Deo et Patri suo:
\red{*} Ipsi glória et impérium, in sǽcula sæculórum.
\V Ipse est primogénitus mortuórum, et princeps regum terræ.
\red{I}psi glória.

}

\lectio{Lectio vii}
\y{U}{nde} conséquitur, non modo ut Christus ab Angelis et homínibus Deus sit adorándus, sed étiam ut ejus império Hóminis, Angeli et hómines páreant et subjécti sint: nempe ut vel solo hypostáticæ uniónis nómine Christus potestátem in univérsas creatúras obtíneat. Jamvéro, ut hujus vim et natúram principátus paucis declarémus, dícere vix áttinet tríplici eum potestáte continéri, qua si carúerit, principátus vix intellégitur. Id ipsum deprómpta atque alláta ex sacris Lítteris de universáli Redemptóris nostri império testimónia plus quam satis signíficant, atque est cathólica fide credéndum, Christum Jesum homínibus datum esse útique Redemptórem cui fidant, at una simul legislatórem cui obédiant. Ipsum autem evangélia non tam leges condidísse narrant, quam leges condéntem indúcunt: quæ quidem præcépta quicúmque servárint, iídem a divíno Magístro, álias áliis verbis, et suam in eum caritátem probatúri et in dilectióne ejus mansúri dicúntur. Judiciáriam vero potestátem sibi a Patre attribútam ipse Jesus Judǽis, de Sábbati requiéte per mirábilem débilis hóminis sanatiónem violáta criminántibus, denúntiat: « Neque enim Pater iúdicat quemquam, sed omne judícium dedit Fílio ». In quo id étiam comprehénditur (quóniam res a judício disiúngi nequit) ut prǽmia et pœnas homínibus adhuc vivéntibus jure suo déferat. At prætérea potéstas illa, quam exsecutiónis vocant, Christo adjudicánda est, útpote cujus império parére omnes necésse sit, et ea quidem denuntiáta contumácibus irrogatióne suppliciórum, quæ nemo possit effúgere.

{\setstretch{0.97}
\R Factum est \f regnum hujus mundi Dómini nostri et Christi ejus:
\red{*} Et regnábit in sǽcula sæculórum.
\V Adorábunt in conspéctu ejus univérsæ famíliæ géntium; quóniam Dómini est regnum.
\red{E}t regnábit.

\lectio{Lectio viii}
\y{V}{erúmtamen} eiúsmodi regnum præcípuo quodam modo et spirituále esse et ad spirituália pertinére, cum ea, quæ ex Bíbliis supra protúlimus, verba planíssime osténdunt, tum Christus Dóminus sua agéndi ratióne confírmat.
Síquidem, non una data occasióne, cum Judæi, immo vel ipsi Apóstoli, per errórem censérent, fore ut Messías pópulum in libertátem vindicáret regnúmque Israël restitúeret, vanam ipse ac spem adímere et convéllere; rex a circumfúsa admirántium multitúdine renuntiándus, et nomen et honórem fugiéndo latendóque detrectáre; coram Præside Románo edícere, regnum suum « de hoc mundo » non esse.
Quod quidem regnum tale in evangéliis propónitur, in quod hómines pœniténtiam agéndo íngredi vero néqueant nisi per fidem et baptísmum, Oppónitur únice regno Sátanæ et potestáti tenebrárum, et ab ásseclis póstulat, non solum ut, abalienáto a divítiis rebúsque terrénis ánimo, morum prǽferant lenitátem et esúriant sitiántque justítiam, sed étiam ut semetípsos ábnegent et crucem suam tollant. Cum autem Christus et Ecclésiam Redémptor sánguine suo acquisíverit et Sacérdos se ipse pro peccátis hóstiam obtúlerit perpetuóque ófferat, cui non videátur régium ipsum munus utriúsque illíus natúram múneris indúere ac participáre? Túrpiter, ceteróquin, erret, qui a Christo hómine rerum civílium quarúmlibet impérium abiúdicet, cum is a Patre jus in res creátas absolutíssimum sic obtíneat, ut ómnia in suo arbítrio sint pósita.
Itaque, auctoritáte Nostra Apostólica, festum Dómni Nostri Jesu Christi Regis institúimus, quotánnis, postrémo mensis Octóbris Domínico die, qui scílicet Omnium Sanctórum celelbritátem próxime antećedit, ubíque terrárum agéndum. Item præcípimus, ut eo ipso die géneris humáni Sacratíssimo Cordi Jesu dedicátio quotánnis renovétur.

\R Decem córnua \f quæ vidísti, decem reges sunt. Hi cum Agno pugnábunt, et Agnus vincet illos:
\red{*} Quóniam Dóminus dominórum est, et Rex regum.
\V Regnávit Dóminus Deus noster omnípotens: gaudeámus et exsultémus, et demus glóriam ei.
\red{Q}uóniam.
\red{G}lória Patri.
\red{Q}uóniam.

}

\nocturn{In III Nocturno}

\A Christo \f datus est principátus et honor regni: omnis pópulus et tribus et linguæ sérvient ei in ætérnum.

\input{../PsalmiM/Benedictus_es_DomineY.tex}

\begin{psalmus}
\lectiocap{Canticum Annæ}{1 Reg. 2, 1-5}

\y{E}{xsultávit} cor meum in Dómino, * et exaltátum est cornu meum in Deo meo.

Dilatátum est os meum super inimícos meos: * quia lætáta sum in salutári tuo.

Non est sanctus, ut est Dóminus: \f neque enim est álius extra te, * et non est fortis sicut Deus noster.

Nolíte multiplicáre loqui sublímia, * gloriántes:

Recédant vétera de ore vestro: \f quia Deus scientiárum, Dóminus est, * et ipsi præparántur cogitatiónes.

Arcus fórtium superátus est, * et infírmi accíncti sunt róbore.

Repléti prius, pro pánibus se locavérunt: * et famélici saturáti sunt.

Donec stérilis péperit plúrimos: * et quæ multos habébat fílios, infirmáta est.

\end{psalmus}


\begin{psalmus}
\lectiocap{Canticum}{Ibidem, 6-10}

\y{D}{óminus} mortíficat et vivíficat, * dedúcit ad ínferos et redúcit.

Dóminus páuperem facit et ditat, * humíliat et súblevat.

Súscitat de púlvere egénum, * et de stércore élevat páuperem:

Ut sédeat cum princípibus, * et sólium glóriæ téneat.

Dómini enim sunt cárdines terræ, * et pósuit super eos orbem.

Pedes sanctórum suórum servábit, \f et ímpii in ténebris conticéscent: * quia non in fortitúdine sua roborábitur vir.

Dóminum formidábunt adversárii ejus: * et super ipsos in cælis tonábit:

Dóminus judicábit fines terræ, \f et dabit impérium regi suo, * et sublimábit cornu Christi sui.

\end{psalmus}


\A Christo datus est principátus et honor regni: omnis pópulus et tribus et linguæ sérvient ei in ætérnum.

\V Adorábunt eum omnes reges terræ.
\R Omnes gentes sérvient ei.

\scriptura{Léctio sancti Evangélii secúndum Joánnem}
\lectiocap{Lectio ix}{Cap. 18}
\y{I}{n} illo témpore: Dixit Pilátus ad Jesum: Tu es Rex Judæórum? Respóndit Jesus: A temetípso hoc dicis, an álii dixérunt tibi de me? Et réliqua.

\scriptura{Homilía sancti Augustíni Epíscopi}
\ex{Tract. 51 in Joan. 12-13; Tract 117 in Joan. 19-21}
\Y{Q}{uid} magnum fuit Regi sæculórum Regem fíeri hóminum? Non enim Rex Israël Christus ad exigéndum tribútum vel exércitum ferro armándum hostésque visibíliter debellándos; sed Rex Israël, quod mentes regat, quod in ætérnum cónsulat, quod in regnum cælórum credéntes, sperántes amantésque perdúcat. Dei ergo Fílius æquális Patri, Verbum per quod facta sunt ómnia, quod Rex esse vóluit Israël, dignátio est, non promótio; miseratiónis indícium est, non potestátis augméntum. Qui enim appellátus est in terra Rex Judæórum, in cælis est Dóminus Angelórum.
Sed Judæórum tantum Rex est Christus, an et géntium? Immo et géntium. Cum enim dixísset in prophetía: Ego autem sum constitútus Rex ab eo super Sion montem sanctum ejus, prǽdicans præcéptum Dómini: ne propter montem Sion solis Judǽis eum regem quisquam díceret constitútum, contínuo subiécit: Dóminus dixit ad me: Fílius meus es tu, ego hódie génui te; póstula a me et dabo tibi gentes hereditátem tuam et possessiónem tuam términos terræ.

\R Tua est \f poténtia, tuum regnum, Dómine: tu es super omnes gentes:
\red{*} Da pacem, Dómine, in diébus nostris.
\V Creátor ómnium, Deus, terríbilis et fortis, justus et miséricors.
\red{D}a pacem.

{\setstretch{0.98}
\lectio{Lectio x}
\ex{Tract. 115 in Joan. 18-36}
\y{R}{espóndit} Jesus: Regnum meum non est de hoc mundo. Si ex hoc mundo esset regnum meum, minístri mei útique decertárent, ut non tráderer Judǽis; nunc autem regnum meum non est hinc. Hoc est quod bonus Magíster scire nos vóluit; sed prius nobis demonstránda de regno ejus opínio, sive géntium, sive Judæórum a quibus id Pilátus audíerat: quasi proptérea morte fuísset plecténdus, quod illícitum affectáverit regnum, vel quóniam solent regnatúris invidére regnántes, et vidélicet cavéndum erat ne ejus regnum sive Románis, sive Judǽis esset advérsum.

\R Ecce \f apparébit Dóminus super nubem cándidam,
\red{*} Et cum eo Sanctórum míllia: et habébit in vestiménto, et in fémore suo scriptum: † Rex regum, et Dóminus dominántium.
\V Apparébit in finem, et non mentiétur; si moram fécerit, exspécta eum, quia véniens véniet.
\red{E}t cum.

\lectio{Lectio xi}
\y{P}{óterat} autem Dóminus quod ait, Regnum meum non est de hoc mundo, ad primam interrogatiónem prǽsidis respondére, ubi ei dixit, Tu es rex Judæórum? sed eum vicíssim intérrogans, utrum hoc a semetípso díceret, an audísset ab áliis, illo respondénte osténdere vóluit hoc sibi apud illum fuísse a Judǽis velut crimen obiéctum: patefáciens nobis cogitatiónes hóminum, quas ipse nóverat, quóniam vanæ sunt; eísque post responsiónem Piláti, jam Judǽis et géntibus opportúnius aptiúsque respóndens, Regnum meum non est de hoc mundo.

\R Non est \f símilis tui, Dómine, magnus nomen tuum in fortitúdine,
\red{*} Quis non timébit te, o Rex géntium?
\V Tuum est enim decus inter cunctos sapiéntes géntium: Tu Deus vivens et Rex sempitérnus
\red{Q}uis.

}

{\setstretch{0.98}

\rubric{Lectio XII de Homilia Dominicæ occurrentis, cum \R suo.}

\scriptura{Sequéntia sancti Evangélii secúndum Joánnem}
\ex{Cap. 18, 33-37}
\Y{I}{n} illo témpore: Dixit Pilátus ad Jesum: Tu es Rex Judæórum? Respóndit Jesus: A temetípso hoc dicis, an álii dixérunt tibi de me? Respóndit Pilátus: Numquid ego Judǽus sum? Gens tua et pontífices tradidérunt te mihi: quid fecísti? Respóndit Jesus: Regnum meum non est de hoc mundo. Si ex hoc mundo esset regnum meum, minístri mei útique decertárent, ut non tráderer Judǽis: nunc autem regnum meum non est hinc. Dixit ítaque ei Pilátus: Ergo Rex es tu? Respóndit Jesus: Tu dicis, quia Rex sum ego. Ego in hoc natus sum et ad hoc veni in mundum, ut testimónium perhíbeam veritáti: omnis, qui est ex veritáte, audit vocem meam.

\hora{Ad Laudes}

\etper

\Av{1} Suscitábit \f Deus cæli regnum quod commínuet et consúmet univérsa regna, et ipsum stabit in ætérnum. \Dominusregnavit

\Av{2} Dedit ei Dóminus \f potestátem et honórem et regnum; et omnes pópuli, tribus et linguæ ipsi sérvient.

\Av{3} Exíbunt aquæ vivæ \f de Jerúsalem; et erit Dóminus Rex super omnem terram.

\Av{4} Magnificábitur \f usque ad términos terræ, et erit iste pax.

\Av{5} Gens et regnum \f quod non servíerit tibi, períbit: et gentes solitúdine vastabúntur.

\capitulum{Coloss. 1, 12-13}
\y{F}{ratres}: Grátias ágimus Deo Patri, qui dignos nos fecit in partem sortis sanctórum in lúmine, qui erípuit nos de potestáte tenebrárum, et tránstulit in regnum Fílii dilectiónis suæ.

\Rbr Data est mihi \red{*} Omnis potéstas.
\red{D}ata est.
\V In cælo et in terra.
\red{O}mnis potéstas.
\red{G}lória Patri.
\red{D}ata est.

\pars{Hymnus}
\begin{hymnus}
	\y{V}{exílla} Christus ínclita\\
	\hspace{2em}Late triúmphans éxplicat:\\
Gentes adéste súpplices,\\
Regíque regum pláudite.

\red{N}on Ille regna cládibus:\\
Non vi metúque súbdidit\\
Alto levátus stípite,\\
Amóre traxit ómnia.

\red{O} ter beáta cívitas\\
Cui rite Christus ímperat,\\
Quæ jussa pergit éxsequi\\
Edícta mundo cǽlitus!

\red{N}on arma flagrant ímpia,\\
Pax usque firmat fœ́dera,\\
Arrídet et concórdia,\\
Tutus stat ordo cívicus.

\red{S}ervat fides connúbia,\\
Juvénta pubet íntegra,\\
Pudíca florent límina\\
Domésticis virtútibus.

\red{O}ptáta nobis spléndeat\\
Lux ista, Rex dulcíssime:\\
Te, pace adépta cándida,\\
Adóret orbis súbditus.

\red{G}lória tibi, Dómine,\\
Qui sceptra mundi témperas,\\
\hspace{-.5em}Cum Patre, et almo Spíritu,\\
In sempitérna sǽcula.
Amen.
\end{hymnus}

}

{\setstretch{0.96}
\V Multiplicábitur ejus impérium.
\R Et pacis non erit finis.

\B Fecit nos Deo \f et Patri suo regnum, primogénitus mortuórum, et Princeps regum terræ, allelúja.

\pars{Oratio}
\y{O}{mnípotens} sempitérne Deus, qui in dilécto Fílio tuo, universórum Rege, ómnia instauráre voluísti: concéde propítius; ut cunctæ famíliæ géntium, peccáti vúlnere disgregátæ, ejus suavíssimo subdántur império: Qui tecum vivit.

\rubric{Et fit Commemoratio Dominicæ occurrentis.}

\hora{Ad Tertiam}

\rubric{Capitulum ut supra ad Laudes.}

\V Afférte Dómino, famíliæ populórum.
\R Afférte Dómino glóriam et impérium.

}

\hora{Ad Sextam}

\capitulum{Coloss. 1, 16-17}
\y{O}{mnia} per ipsum et in ipso creáta sunt, et ipse est ante omnes, et ómnia in ipso constant. Et ipse est caput córporis Ecclésiæ, qui est princípium, primogénitus ex mórtuis, ut sit in ómnibus ipse primátum tenens.

\V Adorábunt eum omnes reges terræ.
\R Omnes gentes sérvient ei.

\hora{Ad Nonam}

\capitulum{Coloss. 1, 19-20}
\y{I}{n} ipso complácuit omnem plenitúdinem inhabitáre, et per eum reconciliáre ómnia in ipsum, pacíficans per sánguinem crucis ejus sive quæ in terris, sive quæ in cælis sunt, in Christo Jesu Dómino nostro.

\V Multiplicábitur ejus impérium.
\R Et pacis non erit finis.

\hora{In II Vesperis}

\rubric{Omnia ut in I Vesperis, præter sequentia:}

\V Multiplicábitur ejus impérium.
\R Et pacis non erit finis.

\M Habet in vestiménto \f et in fémore suo scriptum: Rex regum, et Dóminus dominántium. Ipsi glória et impérium, in sǽcula sæculórum.

\rubric{Et fit Commemoratio Dominicæ occurrentis.}

\rubric{Si ultima Dominica occurrat die 31 Octobris, Vesperæ dicuntur de sequenti Festo Omnium Sanctorum cum Commemoratione Festi D. N. Jesu Christi Regis et Dominicæ.}


\end{multicols}

%\ornamentvi

\end{document}
