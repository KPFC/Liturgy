\documentclass[fontsize=7.9pt,paper=A6,twoside,BCOR=1mm,DIV=24,headinclude]{scrarticle}
\usepackage{breviarium}
\usepackage{Commune}
\usepackage{CommuneM}
\begin{document}
%\titulum{Dominica ultima Octobris}{D. N. Jesu Christi Regis}{Duplex I classis}

\null 

\vspace{-7em}

\begin{multicols}{2}
\thispagestyle{empty}

\privdieii{Sabbato post Octavam Ss. Corporis Christi}{In Festo Immaculati Cordis B. M. V.}{}{Duplex II classis}

{\setstretch{0.96}

\rubric{Omnia ut in Communi Festorum B. M. V., præter sequentia.}

\hora{In I Vesperis}

\VRBMVi

\M Exsultávit cor meum \f in Dómino, et exaltátum est cornu meum in Deo meo, quia lætáta sum in salutári tuo.

\pars{Oratio}
\y{O}{mnípotens} sempitérne Deus, qui in Corde beátæ Maríæ Vírginis dignum Spíritus Sancti habitáculum præparásti: concéde propítius; ut eiúsdem immaculáti Cordis festivitátem devóta mente recoléntes, secúndum Cor tuum vívere valeámus. Per Dóminum … in unitáte ejúsdem Spíritus Sancti.


\hora{Ad Matutinum}

\rubric{In I Nocturno Lectiones \black{Ego sapiéntia.}}

\nocturn{In II Nocturno}
\scriptura{Sermo sancti Bernardíni Senénsis}
\lectio{Lectio v}
\ex{Ex sermone 9 de visitatione}
\Y{Q}{uis} mortálium, nisi divíno tutus oráculo, de vera Dei et hóminis Genetríce quidquam módicum, sive grande præsúmat incircumcísis, immo pollútis lábiis nomináre, quam Pater ante sǽcula Deus perpétuam prædestinávit in Vírginem, digníssimam Fílius elégit in Matrem, Spíritus Sanctus omnis grátiæ domicílium præparávit? Quibus verbis ego homúnculus sensus altíssimos virgínei Cordis, sanctíssimo ore prolátos, éfferam, quibus non súfficit lingua ómnium Angelórum? Dóminus enim ait: Bonus homo de bono thesáuro cordis profert bona; quod verbum potest étiam esse thesáurus.

\mRVBMVv

\lectio{Lectio vi}
\y{Q}{uis} inter puros hómines mélior homo potest excogitári, quam illa, quæ méruit éffici Mater Dei, quæ novem ménsibus in corde et in útero suo ipsum Deum hospitáta est? Quis thesáurus mélior, quam ipse divínus amor quo fornáceum cor Vírginis ardens erat?
De hoc ígitur Corde quasi de fornáce divíni ardóris Virgo beáta prótulit verba bona, id est, verba ardentíssimæ caritátis. Sicut enim a vase summo et óptimo vino pleno, non potest exíre nisi óptimum vinum; aut sicut a fornáce summi ardóris non egréditur nisi incéndium fervens; sic quippe a Christi Matre exíre non pótuit verbum, nisi summi summéque divíni amóris atque ardóris.

\mRVBMVvi

\lectio{Lectio vii}
\y{S}{apiéntis} quoque dóminæ et matrónæ est pauca verba, sólida tamen atque sententiósa habére; proínde septem vícibus quasi septem verba tantum miræ senténtiæ et virtútis a Christi benedictíssima Matre legúntur dicta, ut mýstice ostendátur ipsam fuísse plenam grátia septifórmi. Cum Angelo bis tantúmmodo est locúta. Cum Elísabeth bis étiam. Cum Fílio étiam bis, semel in templo, secúndo in núptiis. Cum minístris semel. Et in his ómnibus semper valde parum locúta est; excépto quod in laude Dei et gratiárum actióne se ámplius dilatávit, scílicet, quum ait: Magníficat ánima mea Dóminum. Ubi non cum hómine, sed cum Deo locúta fuit. Hæc septem verba secúndum septem amóris procéssus et actus sub miro gradu et órdine sunt proláta; quasi sint septem flammæ fornácei Cordis ejus.

\mRVBMVvii

\lectio{Lectio viii}
\Yii{C}{ultum} litúrgicum, quo Cordi Immaculáto Vírginis Maríæ débitus tribúitur honor, cuíque plures viri sancti ac mulíeres viam parárunt, ipsa Apostólica Sedes primum approbávit ineúnte sǽculo undevicésimo, cum Pius Papa séptimus festum Puríssimi Cordis Maríæ Vírginis instítuit, ab ómnibus diœcésibus et religiósis famíliis, quæ id petiíssent, pie sanctéque agéndum: quod póstmodum Pius Papa nonus Offício ac Missa própria auxit. Ardens autem stúdium atque optátum, jam sǽculo décimo séptimo exórtum et in dies invaléscens, ut nempe ejúsmodi festum, majóri solemnitáte donátum, totíus Ecclésiæ commúne efficerétur, Summus Póntifex Pius duodécimus benígne excípiens anno millésimo nongentésimo quadragésimo secúndo, bello atrocíssimo per orbem fere totum ingravescénte, infínitas populórum ærúmnas míserans, pro sua in Matrem cæléstem pietáte ac fidúcia genus hóminum univérsum illíus Cordi benigníssimo obsecratióne solémni eníxe commendávit, atque in honórem ejúsdem Immaculáti Cordis festum cum Offício et Missa própriis in perpétuum ubíque celebrándum indíxit.

\mRVBMVviii

\nocturn{In III Nocturno}
\scriptura{Léctio sancti Evangélii secúndum Joánnem}
\lectiocap{Lectio ix}{Cap. 19}
\y{I}{n} illo témpore: Stábant juxta crucem Jesu Mater ejus, et soror Matris ejus María Cléophæ, et María Magdaléne. Et réliqua.

\scriptura{Homilía sancti Robérti Bellarmíno Epíscopi}
\ex{De septem verbis Christi in Cruce, cap. 12}
\Y{O}{nus} et jugum impósitum a Dómino sancto Joánni, ut Vírginis Matris curam géreret, vere fuit jugum suáve et onus leve. Quis enim non libentíssime cohabitáret Matri illi, quæ Verbum incarnátum in útero novem ménsibus portávit, et illi totos trigínta annos devotíssime dulcissiméque cohabitávit? Quis non invídeat dilécto Dómini, qui in abséntia Fílii Dei præséntiam obtínuit Matris Dei? Sed, nisi fallor, póssumus et nos a benignitáte Verbi nostri causa incarnáti et ex dilectióne nímia nostri causa crucifíxi, précibus impetráre, ut dicat et nobis: Ecce Mater tua; et Matri suæ de nobis dicat: Ecce fílius tuus.

\mRVBMVix

\lectio{Lectio x}
\y{N}{on} est avárus pius Dóminus gratiárum, dúmmodo ad thronum grátiæ ejus cum fide et fidúcia et non ficto corde, sed vero et sincéro accedámus. Qui nos coherédes esse vóluit regni Patris sui, non dedignábitur certe nos coherédes habére amóris Matris suæ.
Sed nec ipsa Virgo benigníssima gravábitur multitúdine filiórum, cum sinum amplíssimum hábeat et valde cúpiat, nullum períre ex his, quos Fílius suus tam pretióso sánguine et tam pretiósa morte redémit.

\mRVBMVx

\lectio{Lectio xi}
\y{A}{deámus} ergo cum fidúcia ad thronum grátiæ Christi, et supplíciter nec sine lácrimis ab eo petámus, ut de unoquóque nostrum Matri suæ dicat: Ecce fílius tuus; et unicuíque nostrum de Matre sua dicat: Ecce Mater tua.
Quam bene nobis erit sub præsídio tantæ Matris? Quis nos detráhere audébit de sinu ejus? Quæ nos tentátio superáre póterit, confidéntes in patrocínio Matris Dei et nostræ? neque nos primi érimus in tanti consecutióne benefícii.

\mRVBMVxi

\lectio{Lectio xii}
\y{M}{ulti} nos præcessérunt; multi, inquam, ad singuláre et plane matérnum patrocínium tantæ Vírginis accessérunt, et nemo confúsus aut tristis dimíssus est, sed omnes hílares et gaudéntes, freti patrocínio tantæ Matris. De qua enim scriptum est: Ipsa cónteret caput tuum, in ea confídunt, se quoque fidénter ambulatúros super áspidem et basilíscum, et conculcatúros leónem et dracónem. Neque enim vidétur posse períre is, de quo dictum sit Vírgini a Christo: Ecce fílius tuus, dúmmodo et ipse non surda aure áudiat, quod ei Christus díxerit: Ecce Mater tua.

\mRVBMVxii

\scriptura{Sequéntia sancti Evangélii secúndum Joánnem}
\ex{Cap. 19, 25-27}
\Y{I}{n} illo témpore: Stabant juxta crucem Jesu mater ejus, et soror matris ejus María Cléophæ, et María Magdaléne. Cum vidísset ergo Jesus matrem, et discípulum stantem, quem diligébat, dicit matri suæ: Múlier, ecce fílius tuus. Deinde dicit discípulo: Ecce mater tua. Et ex illa hora accépit eam discípulus in sua.

\hora{Ad Laudes}

\VRBMVii 

\B O beáta Virgo \f María: tu grátiæ Mater, tu spes mundi, exáudi nos fílios tuos clamántes ad te.

\pars{Oratio}
\y{O}{mnípotens} sempitérne Deus, qui in Corde beátæ Maríæ Vírginis dignum Spíritus Sancti habitáculum præparásti: concéde propítius; ut eiúsdem immaculáti Cordis festivitátem devóta mente recoléntes, secúndum Cor tuum vívere valeámus. Per Dóminum … in unitáte ejúsdem Spíritus Sancti.

\hora{In II Vesperis}

\VRBMViii 

\M Exsultávit cor meum \f in Dómino, et exaltátum est cornu meum in Deo meo, quia lætáta sum in salutári tuo.

}

\end{multicols}

%\ornamentvi

\end{document}
