\documentclass[fontsize=9pt,paper=A6,twoside,BCOR=1mm,DIV=22,headinclude]{scrarticle}
\usepackage{breviarium}
\begin{document}
\titulumalter{}{Regula}{Sancti P. Benedicti}
\begin{multicols}{2}
\mensii{Incipit Prologus}
%{Regulæ Monaste\/riorum}{}

\rubric{1. Ian \hfill 2. Mai \hfill 1. Sept.}

\Y{O}{bsculta}, o fili, præcepta magistri, et inclina aurem cordis tui et admonitionem pii patris libenter excipe et efficaciter conple, ut ad eum per oboedientiæ laborem redeas, a quo per inoboedientiæ desidiam recesseras. Ad te ergo nunc mihi sermo dirigitur, quisquis abrenuntians propriis voluntatibus, Domino Christo vero Regi militaturus oboedientiæ fortissima atque præclara arma sumis.

In primis, ut quidquid agendum inchoas bonum, ab eo perfici instantissima oratione deposcas, ut qui nos iam in filiorum dignatus est numero conputare, non debet aliquando de malis actibus nostris contristari. Ita enim ei omni tempore de bonis suis in nobis parendum est ut non solum iratus pater suos non aliquando filios exheredet, sed nec ut metuendus dominus inritatus a malis nostris, ut nequissimos servos perpetuam tradat ad poenam qui eum sequi noluerint ad gloriam.

\rubric{2. Ian. \hfill 3. Mai \hfill 2. Sept.}

\y{E}{xurgamus} ergo tandem aliquando excitante nos Scriptura ac dicente: Hora est iam nos de somno surgere, et apertis oculis nostris ad deificum lumen adtonitis auribus audiamus divina cotidie clamans quid nos admonet vox dicens: Hodie si vocem eius audieritis, nolite obdurare corda vestra. Et iterum: Qui habet aures audiendi audiat, quid Spiritus dicat ecclesiis. Et quid dicit? Venite, filii, audite me; timorem Dei docebo vos. Currite dum lumen vitæ habetis, ne tenebræ mortis vos conprehendant.

\rubric{3. Ian. \hfill 4. Mai \hfill 3. Sept.}

\y{E}{t} quærens Dominus in multitudine populi cui hæc clamat operarium suum iterum dicit: Quis est homo qui vult vitam et cupit videre dies bonos? Quod si tu audiens respondeas: Ego, dicit tibi Deus: Si vis habere veram et perpetuam vitam, prohibe linguam tuam a malo et labia tua ne loquantur dolum; deverte a malo et fac bonum, inquire pacem et sequere eam. Et cum hæc feceritis, oculi mei super vos et aures meas ad preces vestras, et antequam me invocetis, dicam vobis: Ecce adsum. Quid dulcius ab hac voce Domini invitantis nos, fratres carissimi? Ecce pietate sua demonstrat nobis Dominus viam vitæ.

\rubric{4. Ian. \hfill 5. Mai \hfill 4. Sept.}

\y{S}{uccinctis} ergo fide vel observantia bonorum actuum lumbis nostris, per ducatum Evangelii pergamus itinera eius, ut mereamur eum qui nos vocavit in regnum suum videre. In cujus regni tabernaculo si volumus habitare, nisi illuc bonis actibus curritur, minime pervenitur. Sed interrogemus cum Propheta Dominum dicentes ei: Domine, quis habitabit in tabernaculo tuo, aut quis requiescet in monte sancto tuo? Post hanc interrogationem, fratres, audiamus Dominum respondentem et ostendentem nobis viam ipsius tabernaculi, dicens: Qui ingreditur sine macula et operatur iustitiam; qui loquitur veritatem in corde suo, qui non egit dolum in lingua sua; qui non fecit proximo suo malum, qui obprobrium non accepit adversus proximum suum; qui malignum diabulum aliqua suadentem sibi cum ipsa suasione sua a conspectibus cordis sui respuens deduxit ad nihilum, et parvulos cogitatos eius tenuit et adlisit ad Christum; qui timentes Dominum de bona observantia sua non se reddunt elatos, sed ipsa in se bona non a se posse, sed a Domino fieri existimantes, operantem in se Dominum magnificant, illud cum Propheta dicentes: Non nobis, Domine, non nobis, sed nomini tuo da gloriam; sicut nec Paulus Apostolus de prædicatione sua sibi aliquid inputavit dicens: Gloria Dei sum id quod sum; et iterum ipse dicit: Qui gloriatur, in Domino glorietur.

\rubric{5. Ian. \hfill 6. Mai \hfill 5. Sept.}

\y{U}{nde} et Dominus in Evangelio ait: Qui audit verba mea hæc et facit ea, similabo eum viro sapienti qui ædificavit domum suam super petram; venerunt flumina, flaverunt venti, et inpegerunt in domum illam, et non cecidit, quia fundata erat super petram. Hæc conplens Dominus expectat nos cotidie his suis sanctis monitis factis nos respondere debere. Ideo nobis propter emendationem malorum huius vitæ dies ad indutias relaxantur, dicente Apostolo: An nescis quia patientia Dei ad pænitentiam te adducit? Nam pius Dominus dicit: Nolo mortem peccatoris, sed convertatur et vivat.

\rubric{6. Ian. \hfill 7. Mai \hfill 6. Sept.}

\y{C}{um} ergo interrogassemus Dominum, fratres, de habitatore tabernaculi eius, audivimus habitandi præceptum; sed si conpleamus habitatoris officium, erimus heredes regni cælorum. Conpleamus habitatoris officium. Ergo præparanda sunt corda nostra et corpora sanctæ præceptorum oboedientiæ militanda, et quod minus habet in nos natura possibile, rogemus Dominum, ut gratiæ suæ iubeat nobis adiutorium ministrare. Et si, fugientes gehennæ poenas, ad vitam volumus pervenire perpetuam, dum adhuc vacat et in hoc corpore sumus et hæc omnia per hanc lucis vitam vacat implere, currendum et agendum est modo quod in perpetuo nobis expediat.

\rubric{7. Ian. \hfill 8. Mai \hfill 7. Sept.}

\y{C}{onstituenda} est ergo nobis dominici scola servitii. In qua institutione nihil asperum, nihil grave nos constituturos speramus; sed et si quid paululum restrictius, dictante æquitatis ratione, propter emendationem vitiorum vel conservationem caritatis processerit, non ilico pavore perterritus refugias viam salutis, quæ non est nisi angusto initio incipienda. Processu vero conversationis et fidei, dilatato corde inenarrabili dilectionis dulcedine curritur via mandatorum Dei, ut ab ipsius numquam magisterio discedentes, in eius doctrinam usque ad mortem in monasterio perseverantes passionibus Christi per patientiam participemur, ut et regno eius mereamur esse consortes. Amen.

\caput{\uppercase{Explicit Prologus}}{}

\mensii{Incipit Textus Regulæ}

\caputii{Caput I}{De generibus monachorum}

\rubric{8. Ian. \hfill 9. Mai \hfill 8. Sept.}

\Y{M}{onachorum} quattuor esse genera, manifestum est. Primum coenobitarum, hoc est monasteriale, militans sub regula vel abbate. Deinde secundum genus est anachoritarum id est heremitarum, horum qui non conversationis fervore novicio, sed monasterii probatione diuturna, qui didicerunt contra diabulum multorum solacio iam docti pugnare, et bene extructi fraterna ex acie ad singularem pugnam heremi, securi iam sine consolatione alterius, sola manu vel brachio contra vitia carnis vel cogitationum, Deo auxiliante, pugnare sufficiunt.

\rubric{9. Ian. \hfill 10. Mai \hfill 9. Sept.}

\y{T}{ertium} vero monachorum teterrimum genus est sarabaitarum, qui nulla regula adprobati, experienta magistra, sicut aurum fornacis, sed in plumbi natura molliti, adhuc operibus servantes sæculo fidem, mentiri Deo per tonsuram noscuntur. Qui bini aut terni aut certe singuli sine pastore, non dominicis sed suis inclusi ovilibus, pro lege eis est desideriorum voluptas, cum quidquid putaverint vel elegerint, hoc dicunt sanctum, et quod noluerint, hoc putant non licere. Quartum vero genus est monachorum quod nominatur girovagum, qui tota vita sua per diversas provincias ternis aut quaternis diebus per diversorum cellas hospitantur, semper vagi et numquam stabiles, et propriis voluntatibus et guilæ inlecebris servientes, et per omnia deteriores sarabaitis. De quorum omnium horum miserrima conversatione melius est silere quam loqui. His ergo omissis, ad coenobitarum fortissimum genus disponendum, adiuvante Domino, veniamus.

\caput{Caput II}{Qualis debeat esse abbas}

\diesiii{10. Ian.}{11. Mai}{10. Sept.}
\Y{A}{bbas} qui præesse dignus est monasterio semper meminere debet quod dicitur et nomen maioris factis implere. Christi enim agere vices in monasterio creditur, quando ipsius vocatur pronomine, dicente apostolo: Accepistis spiritum adoptionis filiorum, in quo clamamus: Abba, Pater. Ideoque abbas nihil extra præceptum Domini quod sit debet aut docere aut constituere vel iubere, sed iussio eius vel doctrina fermentum divinæ iustitiæ in discipulorum mentibus conspargatur, memor semper abbas quia doctrinæ suæ vel discipulorum oboedientiæ, utrarumque rerum, in tremendo iudicio Dei facienda erit discussio. Sciatque abbas culpæ pastotis incumbere quidquid in ovibus paterfamilias utilitatis minus potuerit invenire. Tantumdem iterum erit ut, si inquieto vel inoboedienti gregi pastoris fuerit omnis diligentia adtributa et morbidis earum actibus universa fuerit cura exhibita, pastor eorum in iudicio Domini absolutus dicat cum Propheta Domino: Iustitiam tuam non abscondi in corde meo, veritatem tuam et salutare tuum dixi; ipsi autem contemnentes spreverunt me, et tunc demum inoboetientibus cursæ suæ ovibus poena sit eis prævalens ipsa mors.

\diesiii{11. Ian.}{12. Mai}{11. Sept.}

\y{E}{rgo}, cum aliquis suscipit nomen abbatis, duplici debet doctrina suis præesse discipulis, id est omnia bona et sancta factis amplius quam verbis ostendat, ut capacibus discipulis mandata Domini verbis proponere, duris corde vero et simplicioribus factis suis divina præcepta monstrare. Omnia vero quæ discipulis docuerit esse contraria, in suis factis indicet non agenda, ne aliis prædicans ipse reprobus inveniatur, ne quando illi dicat Deus precanti: quare tu enarras iustitias meas et adsumis testamentum meum per os tuum? tu vero odisti disciplinam et proiecisti sermones meos post te, et:qui in fratris tui oculo festucam videbas, in tuo trabem non vidisti.

\diesiii{12. Ian.}{13. Mai}{12. Sept.}

\y{N}{on} ab eo persona in monasterio discernatur. Non unus plus ametur quam alius, nisi quem in bonis actibus aut oboedientia invenerit meliorem. Non convertenti ex servitio præponatur ingenuus, nisi alia rationabilis causa existat. Quod si ita, iustitia dictante, abbati visum fuerit, et de cuiuslibet ordine id faciet; sin alias, propria teneant loca, quia: Sive servus sive liber, omnes in Christo unum sumus et sub uno Domino æqualem servitutis militiam baiulamus, quia: Non est apud Deum personarum acceptio. Solummodo in hac parte apud ipsum discernimur, si meliores ab aliis in operibus bonis et humiles inveniamur. Ergo æqualis sit ab eo omnibus caritas, una præ beatur in omnibus secundum merita disciplina.

\diesiii{13. Ian.}{14. Mai}{13. Sept.}

\y{I}{n} doctrina sua namque abbas apostolicam debet illam semper formam servare in qua dicit: Argue, obsecra, increpa, id est, miscens temporibus tempora, terroribus blandimenta, dirum magistri, pium patris ostendat affectum, id est indisciplinatos et inquietos debet durius arguere, oboedientes autem et mites et patientes, ut in melius proficiant obsecrare, neglegentes et contemnentes ut increpat et corripiat admonemus. Neque dissimulet peccata delinquentiump; sed ut, mox ut coeperint oriri, radicitus ea ut prævalet amputet, memor periculi Heli sacerdotis de Silo. Et honestiores quidem atque intellegibiles animos prima vel secunda admonitione verbis corripiat, inprobos autem et duros ac superbos vel inoboedientes verberum vel corporis castigatio in ipso initio peccati coerceat, sciens scriptum: Stultus verbis non corrigitur, et iterum: Percute filium tuum virga et liberabis animam eius a morte.

\diesiii{14. Ian.}{15. Mai}{14. Sept.}

\y{M}{eminere} debet semper abbas quod est, meminere quod dicitur, et scire quia cui plus committitur, plus ab eo exigitur. Sciatque quam difficilem et arduam rem suscipit, regere animas et multorum servire moribus, et alium quidem blandimentis, alium vero increpationibus, alium suasionibus; et secundum unuscuiusque qualitatem vel intellegentiam, ita se omnibus conformet et aptet ut non solum detrimenta gregis sibi commissi non patiatur, verum in augmentatione boni gregis gaudeat.

\diesiii{15. Ian.}{16. Mai}{15. Sept.}

\y{A}{nte} omnia, ne dissimulans aut parvipendens salutem animarum sibi commissarum, ne plus gerat sollicitudinem de rebus transitoriis et terrenis atque caducis, sed semper cogitet quia animas suscepit regendas, de quibus et rationem redditurus est. Et ne causetur de minori forte substantia , meminerit scriptum: Primum quærite regnum Dei et iustitiam eius, et hæc omnia adicientur vobis, et iterum: Nihil deest timentibus eum. Sciatque quia qui suscipit animas regendas paret se ad rationem reddendam. Et quantum sub cura sua fratrum se habere scierit numerum, agnoscat pro certo quia in die iudicii ipsarum omnium animarum est redditurus Domino rationem, sine dubio addita et suæ animæ. Et ita, timens semper futuram discussionem pastoris de creditis ovibus, cum de aliis ratiociniis cavet, redditur de suis sollicitus, et cum de monitionibus suis emendationem aliis subministrat, ipse efficitur a vitiis emendatus. 

\caput{Caput III}{De adhibendis ad consilium fratribus}
\diesiii{16. Ian.}{17. Mai}{16. Sept.}
\Y{Q}{uotiens} aliqua præcipua sunt in monasterio, convocet abbas omnem congregationem et dicat ipse unde agitur. Et audiens consilium fratrum tractet apud se et quod utilius iudicaverit faciat. Ideo autem omnes ad consilium vocari diximus, quia sæpe iuniori Dominus revelat quod melius est. Sic autem dent fratres consilium cum omni humilitatis subiectione, et non præsumant procaciter defendere quod eis visum fuerit; et magis in abbatis pendat arbitrio, ut quod salubrius esse iudicaverit, ei cuncti oboediant. Sed sicut discipulos convenit oboedire magistro, ita et ipsum provide et iuste condecet cuncta disponere.

\diesiii{17. Ian.}{18. Mai}{17. Sept.}
\y{I}{n} omnibus igitur omnes magistram sequentur regulam, neque ab ea temere declinetur a quoquam. Nullus in monasterio proprii sequatur cordis voluntatem. Neque præsumat quisquam cum abbate suo proterve aut foris monasterium contendere. Quod si præsumpserit, regulari disciplinæ subiaceat. Ipse tamen abbas cum timore Dei et observatione regulæ omnia faciat, sciens se procul dubio de omnibus iudiciis suis æquissimo iudici Deo rationem redditurum. Si qua vero minora agenda sunt in monasterii utilitatibus, seniorum tantum utatur consilio, sicut scriptum est: Omnia fac cum consilio, et post factum non pæniteberis. 

\caput{Caput IV}{Quæ sunt instrumenta bonorum operum}
\diesiii{18. Ian.}{19. Mai}{18. Sept.}
\begin{psalmus}
\Y{I}{n} primis Dominum Deum diligere ex toto corde, tota anima, tota virtute. 

Deinde proximum tamquam seipsum. 

Deinde non occidere. 

Non adulterare. 

Non facere futum.

Non concupiscere. 

Non falsum testimonium dicere. 

Honorare omnes homines. 

Et quod sibi quis fieri non vult, alio ne faciat. 

Abnegare semetipsum sibi ut sequatur Christum. 

Corpus castigare. 

Delicias non amplecti. 

Ieiunium amare. 

Pauperes recreare.

Nudum vestire. 

Infirmum visitare. 

Mortuum sepelire.

In tribulatione subvenire.

Dolentem consolari. 

Sæculi actibus se facere alienum. 

Nihil amori Christi præponere.
\end{psalmus}

\diesiii{19. Ian.}{20. Mai}{19. Sept.}
\begin{psalmus}
\y{I}{ram} non perficere.

\hspace{.00001em} Iracundiæ tempus non reservare.

Dolum in corde non tenere.

Pacem falsam non dare.

Caritatem non derelinquere.

Non iurare ne forte periuret.

Veritatem ex corde et ore proferre.

Malum pro malo non reddere.

Iniuriam non facere, sed et factas patienter sufferre.

Inimicos diligere.

Maledicentes se non remaledicere, sed magis benedicere.

Persecutionem pro iustitia sustinere.

Non esse superbum.

Non vinolentum.

Non multum edacem.

Non somnulentum.

Non pigrum.

Non murmuriosum.

Non detractorem.

Spem suam Deo committere.

Bonum aliquid in se cum viderit, Deo adplicet, non sibi.

Malum vero semper a se factum sciat et sibi reputet.
\end{psalmus}

\diesiii{20. Ian.}{21. Mai}{20. Sept.}
\begin{psalmus}
\y{D}{iem} iudicii timere.

\hspace{1.1em} Gehennam expavescere.

Vitam æternam omni concupiscentia spiritali desiderare.

Mortem cotidie ante oculos suspectam habere. 

Actus vitæ suæ omni hora custodire. 

In omni loco Deum se respicere pro certo scire. 

Cogitationes malas cordi suo advenientes mox ad Christum adlidere et seniori spiritali patefacere.

Os suum a malo vel pravo eloquio custodire. 

Multum loqui non amare. 

Verba vana aut risui apta non loqui. 

Risum multum aut excussum non amare. 

Lectiones sanctas libenter audire.

Orationi frequenter incumbere. 

Mala sua præterita cum lacrimis vel gemitu cotidie in oratione Deo confiteri. 

De ipsis malis de cetero emendare. 

Desideria carnis non efficere. 

Voluntatem propriam odire. 

Præceptis abbatis in omnibus oboedire, etiam si ipse aliter – quod absit – agat, memores illud dominicum præceptum: Quæ dicunt facite, quæ autem faciunt facere nolite. 

Non velle dici sanctum antequam sit, sed prius esse quod verius dicatur.
\end{psalmus}

\diesiii{21. Ian.}{22. Mai}{21. Sept.}
\begin{psalmus}
\y{P}{ræcepta} Dei factis cotidie adimplere. 

Castitatem amare. 

Nullum odire. 

Zelum non habere. 

Invidiam non exercere. 

Contentionem non amare. 

Elationem fugere. 

Et seniores venerare. 

Iuniores diligere. 

In Christi amore pro inimicis orare. 

Cum discordante ante solis occasum in pacem redire. 

Et de Dei misericordia numquam desperare.
\end{psalmus}

Ecce hæc sunt instrumenta artis spiritalis. Quæ cum fuerint a nobis die noctuque incessabiliter adimpleta et in die iudicii reconsignata, illa mercis nobis a Domino reconpensabitur quam ipse promisit: Quod oculus non vidit nec auris audivit, quæ præparavit Deus his qui diligunt illum. Officina vero ubi hæc omnia diligenter operemur claustra sunt monasterii et stabilitas in congregatione. 

\caput{Caput V}{De obœdentia}
\diesiii{22. Ian.}{23. Mai}{22. Sept.}
\Y{P}{rimus} humilitatis gradus est oboedientia sine mora. Hæc convenit his qui nihil sibi a Christo carius aliquid existimant. Propter servitium sanctum quod professi sunt seu propter metum gehennæ vel gloriam vitæ æternæ, mox aliquid imperatum a maiore fuerit, ac si divinitus imperetur, moram pati nesciant in faciendo. De quibus Dominus dicit: Obauditu auris oboedivit mihi. Et item dicit doctoribus: Qui vos audit me audit.

Ergo hii tales, relinquentes statim quæ sua sunt et voluntatem propriam deserentes, mox exoccupatis manibus et quod agebant imperfectum relinquentes, vicino oboedentiæ pede iubentis vocem factis sequuntur, et veluti uno momento prædicta magistri iussio et perfecta discipuli opera, in velocitate timoris Dei, ambæ res communiter citius explicantur. Quibus ad vitam æternam gradiendi amor incumbit, ideo angustam viam arripiunt, unde Dominus dicit: Angusta via est quæ ducit ad vitam, ut non suo arbitrio viventes et desideriis suis et voluptatibus oboedientes, sed ambulantes alieno iudicio et imperio, in coenobiis degentes abbatem sibi præesse desiderant. Sine dubio hii tales illam Domini imitantur sententiam qua dicit: Non veni facere voluntatem meam, sed eius qui misit me.

\diesiii{23. Ian.}{24. Mai}{23. Sept.}
\y{S}{ed} hæc ipsa oboedientia tunc acceptabilis erit Deo et dulcis hominibus, si quod iubetur non trepide, non tarde, non tepide, aut cum murmurio vel cum responso nolentis efficiatur, quia oboedientia quæ maioribus præbetur Deo exhibetur; ipse enim dixit: Qui vos audit me audit. Et cum bono animo a discipulis præberi oportet, quia hilarem datorem diligit Deus. Nam, cum malo animo si oboedit discipulus et non solum ore, sed etiam in corde si murmuraverit, etiam si impleat iussionem, tamen acceptum iam non erit Deo, qui cor eius respicit murmurantem. Et pro tali facto nullam consequitur gratiam, immo poenam murmurantium incurrit, si non cum satisfactione emendaverit. 

\caput{Caput VI}{De taciturnitate}
\diesiii{23. Ian.}{24. Mai}{23. Sept.}
\Y{F}{aciamus} quod ait Propheta: Dixi: custodiam vias meas, ut non delinquam in lingua mea. Posui ori meo custodiam, obmutui et humiliatus sum et silui a bonis. Hic ostendit Propheta, si a bonis eloquiis interdum propter taciturnitatem debet tacere, quanto magis a malis verbis propter poenam peccati debet cessari. Ergo quamvis de bonis et sanctis et ædificationum eloquiis perfectis discipulis propter taciturnitatis gravitatem rara loquendi concedatur licentia, quia scriptum est multiloquio non effugies peccatum, et alibi: Mors et vita in manibus linguæ. Nam loqui et docere magistrum condecet, tacere et audire discipulum convenit.

Et ideo, si qua requirenda sunt a priore, cum omni humilitate et subiectione reverentiæ requirantur. Scurrilitates vero vel verba otiosa et risum moventia æterna clusura in omnibus locis damnamus et ad talia eloquia aperire os non permittimus. 

\caput{Caput VII}{De humilitate}
{\setstretch{0.96}
\diesiii{24. Ian.}{25. Mai}{24. Sept.}
\Y{C}{lamat} nobis Scriptura divina, fratres, dicens: Omnis qui se exaltat humiliabitur et qui se humiliat exaltabitur. Cum hæc ergo dicit, ostendit nobis omnem exaltationem genus esse superbiæ. Quod se cavere Propheta indicat dicens: Domine, non est exaltatum cor meum neque elati sunt oculi mei, neque ambulavi in magnis neque in mirabilibus super me. Sed quid, si non humiliter sentiebam, si exaltavi animam meam, sicut ablactatum super matrem suam, ita retribues in animam suam.

\diesiii{26. Ian.}{27. Mai}{26. Sept.}
\y{U}{nde} fratres, si summæ humilitatis volumus culmen adtingere et ad exaltationem illam cælestem ad quam per præsentis vitæ humilitatem ascenditur, volumus velociter pervenire, actibus nostris ascendentibus scala illa erigenda est quæ in somnio Iacob apparuit, per quam ei descendentes et ascendentes angeli monstrabantur. Non aliud sine dubio descensus ille et ascensus a nobis intelligitur nisi exaltatione descendere et humilitate ascendere. Scala vero ipsa erecta nostra est vita in sæculo, quæ humiliato corde a Domino erigatur ad cælum. Latera enim eius scalæ dicimus nostrum esse corpus et animam, in qua latera diversos gradus humilitatis vel disciplinæ evocatio divina ascendendo inseruit. 

\diesiii{27. Jan.}{28. Mai}{27. Sept.}
\y{P}{rimus} itaque humilitatis gradus est, si timorem Dei sibi ante oculos semper ponens, oblivionem omnimo fugiat et semper sit memor omnia quæ præcepit Deus, ut qualiter et contemnentes Deum gehenna de peccatis incendat et vita æterna quæ timentibus Deum præparata est, animo suo semper revolvat. Et custodiens se omni hora a peccatis et vitiis, id est cogitationum, linguæ, manuum, pedum vel voluntatis propriæ sed et desideria , æstimet se homo de cælis a Deo semper respici omni hora et facta sua omni loco ab aspectu Divinitatis videri et ab angelis omni hora renuntiari. Demonstrans nobis hoc Propheta, cum in cogitationibus nostris ita Deum semper præsentem ostendit dicens: Scrutans corda et renes Deus; et item: Dominus novit cogitationes hominum; et item dicit: Intellexisti cogitationes meas a longe; et: Quia cogitatio hominis confitebitur tibi. Nam ut sollicitus sit circa cogitationes suas perversas, dicat semper utilis frater in corde suo: Tunc ero immaculatus coram eo si observavero me ab iniquitate mea.

%{\setstretch{1.005}
\diesiii{28. Jan.}{29. Mai}{28. Sept.}
\y{V}{oluntatem} vero propriam ita facere prohibemur cum dicit Scriptura nobis: Et a voluntatibus tuis avertere. Et item rogamus Deum in oratione ut fiat illius voluntas in nobis.

Docemur ergo merito nostram non facere voluntatem cum cavemus illud quod dicit Scriptura: Sunt viæ quæ putantur ab hominibus rectæ, quarum finis usque ad profundum inferni demergit, et cum item pavemus illud quod de neglegentibus dictum est: Corrupti sunt et abominabiles facti sunt in voluntatibus suis. In desideriis vero carnis ita nobis Deum credamus semper esse præsentem, cum dicit Propheta Domino: Ante te est omne desiderium meum.

}

\diesiii{29. Jan.}{30. Mai}{29. Sept.}
\y{C}{avendum} ergo ideo malum desiderium, quia mors secus introitum dilectationis posita est. Unde Scriptura præcepit dicens: Post concupiscentias tuas non eas. Ergo si oculi Domini speculantur bonos et malos et Dominus de cælo semper respicit super filios hominum, ut videat si est intellegens aut requirens Deum, et si ab angelis nobis deputatis cotidie die noctuque Domino factorum nostrorum opera nuntiantur, cavendum est ergo omni hora, fratres, sicut dicit in psalmo Propheta, ne nos declinantes in malo et inutiles factos aliqua hora aspiciat Deus et, parcendo nobis in hoc tempore, quia pius est et expectat nos converti in melius, ne dicat nobis in futuro: Hæc fecisti et tacui.

{\setstretch{1.01}
\diesiii{30. Jan.}{31. Mai}{30. Sept.}
\y{S}{ecundus} humilitatis gradus est, si propriam quis non amans voluntatem desideria sua non delectetur implere, sed vocem illam Domini factis imitemur dicentis: Non veni facere voluntatem meam, sed eius qui me misit. Item dicit Scriptura: Voluntas habet poenam et necessitas parit coronam.

}

\diesiii{31. Jan.}{1. Iun.}{1. Oct.}
\y{T}{ertius} humilitatis gradus est, ut quis pro Dei amore omni oboedientia se subdat maiori, imitans Dominum, de quo dicit Apostolus: Factus oboediens usque ad mortem.

\diesiii{1. Febr.}{2. Iun.}{2. Oct.}
\y{Q}{uartus} humulitatis gradus est, si in ipsa oboedientia duris et contrariis rebus vel etiam quibuslibet inrogatis iniuriis, tacite conscientia patientiam amplectatur et sustinens non lassescat vel discedat, dicente Scriptura: Qui perseveraverit usque in finem, hic salvus erit. Item: Confortetur cor tuum et sustine Dominum. Et ostendens fidelem pro Domino universa etiam contraria sustinere debere, dicit ex persona sufferentium: Propter te morte adficimur tota die, æstimati sumus ut oves occisionis. Et securi de spe retributionis divinæ subsecuntur gaudentes et dicentes: Sed in his omnibus superamus propter eum qui dilexit nos. Et item alio loco Scriptura: Probasti nos, Deus, igne nos examinasti sicut igne examinatur argentum; induxisti nos in laqueum; posuisti tribulationes in dorso nostro. Et ut ostendat sub priore debere nos esse, subsequitur dicens: Inposuisti homines super capita nostra. Sed et præceptum Domini in adversis et iniuriis per patientiam adimplentes, qui percussi in maxillam præbent et aliam, auferenti tunicam dimittunt et pallium, angarizati militario vadunt duo, cum Paulo apostolo falsos fratres sustinent et persecutionem sustinent, et maledicentes se benedicent.

\diesiii{2. Febr.}{3. Iun.}{3. Oct.}
\y{Q}{uintus} humilitatis gradus est, si omnes cogitationes malas cordi suo advenientes vel mala a se absconse commissa per humilem confessionem abbatem non celaverit suum. Hortans nos de hac re Scriptura dicens: Revela ad Dominum viam tuam et spera in eum. Et item dicit: Confitemini Domino quoniam bonus, quoniam in æternum misericordia eius. Et item Propheta: Delictum meum cognitum tibi feci et iniustitias meas non operui. Dixi: pronuntiabo adversum me iniustitias meas Domino, et tu remisisti impietatem cordis mei.

\diesiii{3. Febr.}{4. Iun.}{4. Oct.}
\y{S}{extus} humilitatis gradus est, si omni vilitate vel extremitate contentus sit monachus, et ad omnia quæ sibi iniunguntur velut operarium malum se iudicet et indignum, dicens sibi cum Propheta: Ad nihilum redactus sum et nescivi; ut iumentum factus sum apud te et ego semper tecum.

\diesiii{4. Febr.}{5. Iun.}{5. Oct.}
\y{S}{eptimus} humilitatis gradus est: si omnibus se inferiorem et viliorem non solum sua lingua pronuntiet, sed etiam intimo cordis credat affectu, humilians se et dicens cum Propheta: Ego autem sum vermis et non homo, obprobrium hominum et abiectio plebis. Exaltatus sum et humiliatus et confusus. Et item: Bonum mihi quod humiliasti me, et discam mandata tua.

\diesiii{5. Febr.}{6. Iun.}{6. Oct.}
\y{O}{ctavus} humilitatis gradus est, si nihil agat monachus, nisi quod communis monasterii regula vel maiorum cohortatur exempla.

\diesiii{6. Febr.}{7. Iun.}{7. Oct.}
\y{N}{onus} humilitatis gradus est, si linguam ad loquendum prohibeat monachus et taciturnitatem habens, usque ad interrogationem non loquatur, monstrante Scriptura quia in multoloquio non effugitur peccatum, et quia vir linguosus non dirigitur super terram.

\diesiii{7. Febr.}{8. Iun.}{8. Oct.}
\y{D}{ecimus} humilitatis gradus est, si non sit facilis ac promptus in risu, qui scriptum est: Stultus in risu exaltat vocem suam.

\diesiii{8. Febr.}{9. Iun.}{9. Oct.}
\y{U}{ndecimus} humilitatis gradus est, si cum loquitur monachus, leniter et sine risu, humiliter cum gravitate vel pauca verba et rationabilia loquatur, et non sit clamosus in voce, sicut scriptum est: Sapiens verbis innotescit paucis.

\diesiii{9. Febr.}{10. Iun.}{10. Oct.}
\y{D}{uodecimus} humilitatis gradus est, si non solum corde monachus, sed etiam ipso corpore humilitatem videntibus se semper indicet, id est Opere Dei, in oratorio, in monasterio, in horto, in via, in agro vel ubicumque sedens, ambulans vel stans, inclinato sit semper capite, defixis in terram aspectibus, reum se omni hora de peccatis suis æstimans iam se tremendo iudicio repræsentari æstimet, dicens sibi in corde semper illud, quod publicanus ille evangelicus fixis in terram oculis dixit: Domine, non sum dignus, ego peccator, levare oculos meos ad cælos. Et item cum Propheta: Incurvatus sum et humiliatus sum usquequaque.

Ergo, his omnibus humilitatis gradibus ascensis, monachus mox ad caritatem Dei perveniet illam quæ perfecta foris mittit timorem, per quam universa quæ prius non sine formidine observabat absque ullo labore velut naturaliter ex consuetudine incipiet custodire, non iam timore gehennæ, sed amore Christi et consuetudine ipsa bona et dilectatione virtutum. Quæ Dominus iam in operarium suum mundum a vitiis et peccatis Spiritu Sancto dignabitur demonstrare. 

\caputii{Caput VIII}{De officiis divinis in noctibus}
\diesiii{10. Febr.}{11. Iun.}{11. Oct.}
\Y{H}{iemis} tempore, id est a kalendas novembres usque in Pascha, iuxta considerationem rationis, octava hora noctis surgendum est, ut modice amplius de media nocte pausetur et iam digesti surgant. Quod vero restat post Vigilias a fratribus qui psalterii vel lectionum aliquid indigent, meditationi inserviatur. A Pascha autem usque ad supradictas novembres sic temperetur hora, ut Vigiliarum agenda parvissimo intervallo, quo fratres ad necessaria naturæ exeant, mox Matutini qui incipiente luce agendi sunt, subsequantur. 

\caput{Caput IX}{Quanti psalmi dicendi sunt nocturni}
{\setstretch{0.96}
\diesiii{11. Febr.}{12. Iun.}{12. Oct.}
\Y{H}{iemis} tempore suprascripto, in primis versu tertio dicendum: Domine, labia mea aperies, et os meum adnuntiabit laudem tuam. Cum subiungendus est tertius psalmus et Gloria. Post hunc, psalmum nonagesimum quartum cum antefana, aut certe decantandum. Inde sequatur ambrosianum, deinde sex psalmi cum antefanas. Quibus dictis, dicto versu, benedicat abbas et, sedentibus omnibus in scamnis, legantur vicissim a fratribus in codice super analogium tres lectiones, inter quas et tria responsoria cantentur. Duo responsoria sine Gloria dicantur; post tertiam vero lectionem, qui cantat dicat Gloriam. Quam dum incipit cantor dicere, mox omnes de sedilia sua surgant ob honorem et reverentiam sanctæ Trinitatis. Codices autem legantur in Vigiliis divinæ auctoritatis tam Veteris Testamenti quam Novi, sed et expositiones earum, quæ a nominatis et orthodoxis catholicis Patribus factæ sunt. Post has vero tres lectiones cum responsoria sua, sequantur reliqui sex psalmi cum Alleluia canendi. Post hos, lectio Apostoli sequatur ex corde recitanda, et versus, et supplicatio litaniæ, id est Quirie eleison. Et sic finiantur Vigiliæ nocturnæ. 

}

\caputii{Caput X}{Qualiter æstatis tempore agatur nocturna laus}
\diesiii{12. Febr.}{13. Iun.}{13. Oct.}
\Y{A}{} Pascha autem usque ad kalendas novembres, omnis ut supra dictum est psalmodiæ quantitas teneatur excepto quod lectiones in codice propter brevitatem noctium legantur, sed pro ipsis tribus lectionibus una de Veteri Testamento memoriter dicatur, quam brevis responsorius subsequatur. Et reliqua omnia, ut dictum est, impleantur, id est ut numquam minus a duodecim psalmorum quantitate ad Vigilias nocturnas dicantur, exceptis tertio et nonagesimo quarto psalmo. 

\caput{Caput XI}{Qualiter diebus dominicis vigiliæ agantur}
\diesiii{13. Febr.}{14. Iun.}{14. Oct.}
\Y{D}{ominico} die temperius surgatur ad Vigilias. In quibus Vigiliis teneatur mensura, id est, modulatis ut supra disposuimus sex psalmis et versu, residentibus cunctis disposite et per ordinem in subselliis, legantur in codice, ut supra diximus, quattuor lectiones cum responsoriis suis. Ubi tantum in quarto responsorio dicatur a cantante Gloria; quam dum incipit, mox omnes cum reverentia surgant. Post quibus lectionibus sequantur ex ordine alii sex psalmi cum antefanas sicut anteriores, et versu. Post quibus iterum legantur aliæ quattuor lectiones cum responsoriis suis, ordine quo supra. Post quibus dicantur tria cantica Prophetarum, quas instituerit abbas; quæ cantica cum Alleluia psallantur. Dicto etiam versu et benedicente abbate, legantur aliæ quattuor lectiones de Novo Testamento, ordine quo supra. Post quartum autem responsorium incipiat abbas hymnum Te Deum laudamus. Quo perdicto, legat abbas lectionem de Evangelia, cum honore et timore stantibus omnibus. Qua perlecta, respondeant omnes Amen, et subsequatur mox abbas hymnum Te decet laus, et data benedictione incipiant Matutinos. Qui ordo Vigiliarum omni tempore tam æstatis quam hiemis æqualiter in die dominico teneatur, nisi forte – quod absit – tardius surgant, aliquid de lectionibus breviandum est aut responsoriis. Quod tamen omnino caveatur ne proveniat; quod si contigerit, digne inde satisfaciat Deo in oratorio per cuius evenerit neglectum. 

\caput{Caput XII}{Quomodo matutinorum sollemnitas agatur}
\diesiii{14. Febr.}{15. Iun.}{15. Oct.}
\Y{I}{n} Matutinis dominico die, in primis dicatur sexagesimus sextus psalmus, sine antefana, in directum. Post quem dicatur quinquagesimus cum Alleluia. Post quem dicatur centisemus septimus decimus et sexagesimus secundus. Inde Benectiones et Laudes, lectionem de Apocalipsis una ex corde et responsorium, ambrosianum, versu, canticum de Evangelia, litania, et conpletum est. 

\caputii{Caput XIII}{Privatis diebus qualiter agantur matutini}
{\setstretch{0.95}
\diesiii{15. Febr.}{16. Iun.}{16. Oct.}
\Y{D}{iebus} autem privatis Matutinorum sollemnitas ita agatur, id est, ut sexagesimus sextus psalmus sine antefana, subtrahendo modice, sicut Dominica, ut omnes occurant ad quinquagesimum, qui cum antefana dicatur. Post quem alii duo psalmi dicantur secundum consuetudinem, id est: secunda feria quintus et tricesimus quintus; tertia feria quadragesimus secundus et quinquagesimus sextus; quarta feria sexagesimum tertium et sexagesimum quartum; quinta feria octogesimum septimum et octogesimum nonum; sexta feria septuagesimum quintum et nonagesimum primum; sabbatorum autem centesimum quadragesimum secundum et canticum Deuteronomium, qui dividatur in duas Glorias. Nam ceteris diebus canticum unumquemque die suo ex Prophetis, sicut psallit Ecclesia romana, dicantur. Post hæc sequantur Laudes; deinde lectio una Apostoli memoriter recitanda, responsorium, ambrosianum, versu, canticum de Evangelia, litania et conpletum est.

\diesiii{16. Febr.}{17. Iun.}{17. Oct.}
\y{P}{lane} agenda matutina vel vespertina non transeat aliquando, nisi in ultimo per ordinem oratio dominica, omnibus audientibus, dicatur a priore propter scandalorum spinas quæ oriri solent, ut conventi per ipsius orationis sponsionem qua dicunt: Dimitte nobis sicut et nos dimittimus, purgent se ab huiusmodi vitio. Ceteris vero agendis ultima pars eius orationis dicatur, ut ab omnibus respondeatur: Sed libera nos a malo. 

}

\caput{Caput XIV}{In nataliciis sanctorum qualiter agantur vigiliæ}
{\setstretch{0.94}
\diesiii{17. Febr.}{18. Iun.}{18. Oct.}
\Y{I}{n} sanctorum vero festivitatibus vel omnibus sollemnitatibus sicut diximus dominico die agendum, ita agatur, excepto quod psalmi aut antefanæ vel lectiones ad ipsum diem pertinentes dicantur; modus autem suprascriptus teneatur. 

}

\caput{Caput XV}{Alleluia quibus temporibus dicatur}
{\setstretch{0.94}
\diesiii{18. Febr.}{19. Iun.}{19. Oct.}
\Y{A}{} sanctum Pascha usque Pentecosten sine intermissione dicatur Alleluia, tam in psalmis quam in responsoriis. A Pentecosten autem usque caput quadragesimæ, omnibus noctis, cum sex posterioribus psalmis tantum ad Nocturnos dicatur. Omni vero Dominica extra quadragesima cantica, Matutinos, Prima, Tertia, Sexta Nonaque cum Alleluia dicatur, Vespera vero iam antefana. Responsoria vero numquam dicantur cum Alleluia, nisi a Pascha usque Pentecosten. 

}

\caput{Caput XVI}{Qualiter divina opera per diem agantur}
{\setstretch{0.96}
\diesiii{19. Febr.}{20. Iun.}{20. Oct.}
\Y{U}{t} ait Propheta: Septies in die laudem dixi tibi. Qui septenarius sacratus numerus a nobis sic implebitur, si Matutino, Primæ, Tertiæ, Sextæ, Nonæ, Vesperæ Conpletoriique tempore nostræ servitutis officia persolvamus, quia de his diurnis Horis dixit: Septies in die laudem dixi tibi. Nam de nocturnis Vigiliis idem ipse Propheta ait: Media nocte surgebam ad confitendum tibi. Ergo his temporibus referamus laudes Creatori nostro super iudicia iustitiæ suæ, id est Matutinis, Prima, Tertia, Sexta, Nona, Vespera, Conpletorios, et nocte surgamus ad confitendum ei. 

}

\caput{Caput XVII}{Quot psalmi per easdem horas dicendi sunt}
{\setstretch{0.96}
\diesiii{20. Febr.}{21. Iun.}{21. Oct.}
\Y{I}{am} de Nocturnis vel Matutinis digessimus ordinem psalmodiæ; nunc de sequentibus Horis videamus. Prima hora dicantur psalmi tres singillatim et non sub una Gloria, hymnum eiusdem Horæ post versum Deus, in adiutorium, antequam psalmi incipiantur. Post expletionem vero trium psalmorum recitetur lectio una, versu et Quirie eleison et missas. Tertia vero, Sexta et Nona item eo ordine celebretur oratio, id est versu, hymnos earundem Horarum, ternos psalmos, lectionem et versu, Quirie eleison et missas. Si maior congregatio fuerit, cum antefanas, si vero minor, in directum psallantur.

Vespertina autem sinaxis quattuor psalmis cum antefanas terminetur. Post quibus psalmis lectio recitanda est: inde responsorium, ambrosianum, versu, canticum de Evangelia, litania, et oratione dominica fiant missæ. Conpletorios autem trium psalmorum dictione terminentur; qui psalmi directanei sine antefana dicendi sunt. Post quos hymnum eiusdem Horæ, lectionem unam, versu, Quirie eleison, et benedictione missæ fiant. 

}

\caput{Caput XVIII}{Quo ordine ipsi psalmi dicendi sunt}
\diesiii{21. Febr.}{22. Iun.}{22. Oct.}
\Y{I}{n} primis dicatur versu: Deus, in adiutorium meum intende; Domine, ad adiuvandum me festina, Gloria, inde hymnum unuscuiusque Horæ. deinde Prima Hora, Dominica, dicenda quattuor capitula psalmi centesimi octavi decimi. Reliquis vero Horis, id est Tertia, Sexta vel Nona, terna capitula suprascripti psalmi centesimi octavi decimi dicantur. Ad Primam autem secundæ feriæ dicantur tres psalmi, id est primus, secundus et sextus. Et ita per singulos dies ad Primam, usque Dominica, dicantur per ordinem terni psalmi usque nonum decimum psalmum, ita sane, ut nonus psalmus et septimus decimus partiantur in binos. Et sic fit, ut ad Vigilias Dominica semper a vicesimo incipiatur.

\diesiii{22. Febr.}{23. Iun.}{23. Oct.}
\y{A}{d} Tertiam vero, Sextam Nonamque secundæ feriæ novem capitula quæ residua sunt de centesimo octavo decimo, ipsa terna per easdem Horas dicantur. Expenso ergo psalmo centesimo ocatvo decimo duobus diebus, id est Dominico et secunda feria, tertia feria iam ad Tertiam, Sextam vel Nonam psallantur terni psalmi a centesimo nono decimo usque centesimo vicesimo septimo, id est psalmi novem. Quique psalmi semper usque Dominica per easdem Horas itidem repetantur, hymnorum nihilominus, lectionum vel versuum dispositionem uniformem cunctis diebus servatam. Et ita scilicet semper Dominica a centesimo octavo decimo incipietur.

\diesiii{23. Febr.}{24. Iun.}{24. Oct.}
\y{V}{espera} autem cotidie quattuor psalmorum modulatione canatur. Qui psalmi incipiantur a centesimo nono usque centesimo quadragesimo septimo, exceptis his qui in diversis Horis ex eis sequestrantur, id est a centesimo septimo decimo usque centesimo vicesimo septimo et centesimo tricesimo tertio et centesimo quadragesimo secundo; reliqui omnes in Vespera dicendi sunt. Et quia minus veniunt tres psalmi, ideo dividendi sunt qui ex numero suprascripto fortiores inveniuntur, id est centesimum tricesimum octavum et centesimum quadragesimum tertium et centesimum quadragesimum quartum; centesimus vero sextus decimus, quia parvus est, cum centesimo quinto decimo coniungatur. Digesto ergo ordine psalmorum vespertinorum, reliquia, id est lectionem, responsum, hymnum, versum vel canticum, sicut supra taxavimus impleatur. Ad Conpletorios vero cotidie idem psalmi repetantur, id est quartum, nonagesimum et centesimum tricesimum tertium.

\diesiii{(24. Febr. si fuerit bissextilis; alias iungitur præcedenti)}{25. Iun.}{25. Oct.}
\y{D}{isposito} ordine psalmodiæ diurnæ, reliqui omnes psalmi qui supersunt æqualiter dividantur in septem noctium Vigilias, partiendoscilicet qui inter eos prolixiores sunt psalmi et duodecim per unamquamque constituens noctem. Hoc præcipue commonentes ut, si sui forte hæc distributio psalmorum displicuerit, ordinet si melius aliter iudicaverit, dum omnimodis id adtendat, ut omni ebdomada psalterium ex integro numero centum quinquaginta psalmorum psallantur, et dominico die semper a caput reprendatur ad Vigilias. Quia nimis inertem devotionis suæ servitium ostendunt monachi qui minus a psalterio cum canticis consuetudinariis per septimanæ circulum psallunt, dum quando legamus sanctos Patres nostros uno die hoc strenue implesse, quod nos tepidi utinam septimana integra persolvamus. 

\caput{Caput XIX}{De disciplina psallendi}
{\setstretch{1.01}
\diesiii{24. (25.) Febr.}{26. Iun.}{26. Oct.}
\Y{U}{bique} credimus divinam esse præsentiam et oculos Domini in omni loco speculari bonos et malos, maxime tamen hoc sine aliqua dubitatione credamus, cum ad opus divinum adsistimus. Ideo semper memores simus quod ait Propheta: Servite Domino in timore, et iterum: Psallite sapienter, et: In conspectu angelorum psallam tibi. Ergo consideremus qualiter oporteat in conspectu Divinitatis et angelorum eius esse, et sic stemus ad psallendum, ut mens nostra concordet voci nostræ. 

}

\caputii{Caput XX}{De reverentia orationis}
\diesiii{25. (26.) Febr.}{27. Iun.}{27. Oct.}
\Y{S}{i}, cum hominibus potentibus volumus aliqua suggerere, non præsumimus nisi cum humilitate et reverentia, quanto magis Domino Deo universorum cum omni humilitate et puritatis devotione supplicandum est. Et non in multiloquio, sed in puritate cordis et conpunctione lacrimarum nos exaudiri sciamus. Et ideo brevis debet esse et pura oratio, nisi forte ex affectu inspirationis divinæ gratiæ protendatur. In conventu tamen omnino brevietur oratio, et facto signo a priore pariter surgant.

\caput{Caput XXI}{De decanis monasterii}
\diesiii{26. (27.) Febr.}{28. Iun.}{28. Oct.}
\Y{S}{i} maior fuerit congregatio, elegantur de ipsis fratres boni testimonii et sanctæ conversationis, et constituantur decani, qui sollicitudinem gerant super decanias suas in omnibus secundum mandata Dei et præcepta abbatis sui. Qui decani tales elegantur in quibus securus abbas partiat onera sua. Et non elegantur per ordinem, sed secundum vitæ meritum et sapientiæ doctrinam. Quique decani, si ex eis aliqua forte qui inflatus superbia repertus fuerit reprehensibilis, correptus semel et iterum atque tertio si emendare noluerit, deiciatur, et alter in loco eius qui dignus est subrogetur. Et de præposito eadem constituimus. 

\caput{Caput XXII}{Quomodo dormiant monachi}
\diesiii{27. (28.) Febr.}{29. Iun.}{29. Oct.}
\Y{S}{inguli} per singula lecta dormiant. Lectisternia pro modo conversationis secundum dispensationem abbatis sui accipiant. Si potest fieri omnes in uno loco dormiant; sin autem multitudo non sinit, deni aut viceni cum senioribus qui super eos solliciti sint, pausent. Candela iugiter in eadem cella ardeat usque mane. Vestiti dormiant et cincti cingulis aut funibus, ut cultellos suos ad latus suum non habeant dum dormiunt, ne forte per somnium vulnerent dormientem; et ut parati sint monachi semper et, facto signo absque mora surgentes, festinent invicem se prævenire ad opus Dei, cum omni tamen gravitate et modestia. Adulescentiores fratres iuxta se non habeant lectos, sed permixti cum senioribus. Surgentes vero ad opus Dei invicem se moderate cohortentur propter somnulentorum excusationes. 

\caput{Caput XXIII}{De excommunicatione culparum}
\diesiii{28. (29.) Febr.}{30. Iun.}{30. Oct.}
\Y{S}{i} quis frater contumax aut inoboediens aut superbus aut murmurans vel in aliquo contrarius sanctæ regulæ et præceptis seniorum suorum contemptor repertus fuerit, hic secundum Domini nostri præceptum admoneatur semel et secundo secrete a senioribus suis. Si non emendaverit, obiurgetur publice coram omnibus. Si vero neque sic correxerit, si intelligit qualis poena sit, excommunicationi subiaceat; sin autem inprobus est, vindictæ corporali subdatur. 

\caput{Caput XXIV}{Qualis debet esse modus excommunicationis}
\diesiii{1. Mart.}{1. Iul.}{31. Oct.}
\Y{S}{ecundum} modum culpæ, et excommunicationis vel disciplinæ mensura debet extendi. Qui culparum modus in abbatis pendat iudicio. Si quis tamen frater in levioribus culpis invenitur, a mensæ participatione privetur. Privati autem a mensæ consortio ista erit ratio, ut in oratorio psalmum aut antefanam non imponat, neque lectionem recitet, usque ad satisfactionem. Refectionem autem cibi post fratrum refectionem solus accipiat, ut, si verbi gratia fratres reficiunt sexta hora, ille frater nona, si fratres nona, ille vespera, usque dum satisfactione congrua veniam consequatur. 

\caputii{Caput XXV}{De gravioribus culpis}
\diesiii{2. Mart.}{2. Iul.}{1. Nov.}
\Y{I}{s} autem frater qui gravioribus culpæ noxa tenetur, suspendatur a mensa, simul ab oratorio. Nullus ei fratrum in nullo iungatur consortio nec in conloquio. Solus sit ad opus sibi iniunctum, persistens in pænitentiæ luctu, sciens illam terribilem Apostoli sententiam dicentis: Traditum eiusmodi hominem in interitum carnis, ut spiritus salvus sit in diem Domini. Cibi autem refectionem solus percipiat, mensura vel hora qua præviderit abbas ei conpetere; nec a quoquam benedicatur transeunte nec cibum quod ei datur. 

\caput{Caput XXVI}{De his qui sine iussione iungunt se excommunicatis}
\diesiii{3. Mart.}{3. Iul.}{2. Nov.}
\Y{S}{i} quis frater præsumpserit sine iussione abbatis fratri excommunicato quolibet modo se iungere aut loqui cum eo vel mandatum ei dirigere, similem sortiatur excommunicationis vindictam. 

\caput{Caput XXVII}{Qualiter debeat abbas sollicitus esse circa excommunicatos}
\diesiii{4. Mart.}{4. Iul.}{3. Nov.}
\Y{O}{mni} sollicitudine curam gerat abbas circa delinquentes fratres, quia: Non est opus sanis medicus, sed male habentibus. Et ideo uti debet omni modo ut sapiens medicus: inmittere senpectas, id est seniores sapientes fratres, qui quasi secrete consolentur fratrem fluctuantem et provocent ad humilitatis satisfactionem et consolentur eum ne abundantiori tristitia absorbeatur, sed, sicut ait item Apostolus: Confirmetur in eo caritas, et oretur pro eo ab omnibus.

Magnopere enim debet sollicitudinem gerere abbas et omni sagacitate et industria currere, ne aliquam de ovibus sibi creditis perdat. Noverit enim se infirmarum curam suscepisse animarum, non super sanas tyrannidem. Et metuat Prophetæ comminationem per quam dicit Deus: Quod crassum videbatis, et quod debile erat proiciebatis. Et Pastoris boni pium imitetur exemplum, qui, relictis nonaginta novem ovibus in montibus, abiit unam ovem quæ erraverat quærere. Cuius infirmati in tantum conpassus est, ut eam in sacris humeris suis dignaretur inponere et sic reportare ad gregem.

\caput{Caput XXVIII}{De his qui sæpius correpti emendare noluerint}
\diesiii{5. Mart.}{5. Iul.}{4. Nov.}
\Y{S}{i} quis frater frequenter correptus pro qualibet culpa, si etiam excommunicatus non emendaverit, acrior ei accedat correptio, id est ut verberum vindicta in eum procedant. Quod si nec ita correxerit, aut forte – quod absit – in superbia elatus etiam defendere voluerit opera sua, tunc abbas faciat quod sapiens medicus: si exhibuit fomenta, si unguenta adhortationum, si medicamina Scripturarum divinarum, si ad ultimum ustionem excommunicationis vel plagarum virgæ, et iam si viderit nihil suam prævalere industriam, adhibeat etiam, quod maius est, suam et omnium fratrum pro eo orationem, ut Dominus qui omnia potest operetur salutem circa infirmum fratrem. Quod si nec isto modo sanatus fuerit, tunc iam utatur abbas ferro abscisionis, ut ait Apostolus: Auferte malum ex vobis; et iterum: Infidelis si discedit, discedat, ne una ovis morbida omnem gregem contagiet. 

\caput{Caput XXIX}{Si debeant fratres exeuntes de monasterio item recipi}
\diesiii{6. Mart.}{6. Iul.}{5. Nov.}
\Y{F}{rater} qui proprio vitio egreditur de monasterio, si reverti voluerit, spondeat prius omnem emendationem pro quo egressus est, et sic in ultimo gradu recipiatur, ut ex hoc eius humilitas conprobetur. Quod si denuo exierit, usque tertio ita recipiatur, iam postea sciens omnem sibi reversionis aditum denegari. 

\caput{Caput XXX}{De pueris minori ætate qualiter corripiantur}
\diesiii{7. Mart.}{7. Iul.}{6. Nov.}
\Y{O}{mnis} ætas vel intellectus proprias debet habere mensuras. Ideoque quotiens pueri vel adulescentiores ætate, aut qui minus intellegere possunt, quanta poena sit excommunicationis, hii tales dum delinquunt, aut ieiuniis nimiis affligantur aut acris verberibus coerceantur, ut sanentur. 

\caput{Caput XXXI}{De cellarario monasterii qualis sit}
\diesiii{8. Mart.}{8. Iul.}{7. Nov.}
\Y{C}{ellararius} monasterii elegatur de congregatione sapiens, maturis moribus, sobrius, non multum edax, non elatus, non turbulentus, non iniuriosus, non tardus, non prodigus, sed timens Deum; qui omni congregationi sit sicut pater. Curam gerat de omnibus. Sine iussione abbatis nihil faciat. Quæ iubentur custodiat. Fratres non contristet. Si quis frater ab eo foret aliqua inrationabiliter postulat, non spernendo eum contristet, sed rationabiliter cum humilitate male petenti deneget. Animam suam custodiat, memor semper illud apostolicum, quia: Qui bene ministraverit, gradum bonum sibi adquirit. Infirmorum, infantum, hospitum pauperumque cum omni sollicitudine curam gerat, sciens sine dubio, quia pro his omnibus in die iudicii rationem redditurus est. Omnia vasa monasterii cunctamque substantiam ac si altaris vasa sacrata conspiciat. Nihil ducat neglegendum. Neque avaritiæ studeat neque prodigus sit et stirpator substantiæ monasterii, sed omnia mensurate faciat et secundum iussionem abbatis.

\diesiii{9. Mart.}{9. Iul.}{8. Nov.}
\y{H}{umilitatem} ante omnia habeat, et cui substantia non est quod tribuatur, sermo responsionis porrigatur bonus, ut scriptum est: Sermo bonus super datum optimum. Omnia quæ ei iniunxerit abbas, ipsa habeat sub cura sua; a quibus eum prohibuerit, non præsumat. Fratribus constitutam annonam sine aliquo tyfo vel mora offerat, ut non scandalizentur, memor divini eloquii, quid mereatur qui scandalizaverit unum de pusillis. Si congregatio maior fuerit, solacia ei dentur, a quibus adiutus et ipse æquo animo impleat officium sibi commissum. Horis conpetentibus et dentur quæ danda sunt et petantur quæ petanda sunt, nemo perturbetur neque contristetur in domo Dei.

\caput{Caput XXXII}{De ferramentis vel rebus monasterii}
\diesiii{10. Mart.}{10. Iul.}{9. Nov.}
\Y{S}{ubstantia} monasterii in ferramentis vel vestibus seu quibuslibet rebus prævideat abbas fratres de quorum vita et moribus securus sit; et eis singula, ut utile iudicaverit, consignet custodienda atque recolligenda. Ex quibus abbas brevem teneat, ut dum sibi in ipsa adsignata fratres vicissim succedunt, sciat quid dat aut quid recipit. Si quis autem sordide aut neglegenter res monasterii tractaverit, corripiatur; si non emendaverit, disciplinæ regulari subiaceat. 

\caput{Caput XXXIII}{Si quid debeant monachi proprium habere}
\diesiii{11. Mart.}{11. Iul.}{10. Nov.}
\Y{P}{ræcipue} hoc vitium radicitus amputandum est de monasterio ne quis præsumat aliquid dare aut accipere sine iussione abbatis, neque aliquid habere proprium, nullam omnimo rem, neque codicem, neque tabulas, neque grafium, sed nihil omnimo, quippe quibus nec corpora sua nec voluntates licet habere in propria voluntate; omnia vero necessaria a patre sperare monasterii, nec quicquam liceat habere quod abbas non dederit aut permiserit. Omniaque omnibus sint communia, ut scriptum est, ne quisquam suum aliquid dicat vel præsumat. Quod si quisquam huic nequissimo vitio deprehensus fuerit delectari, admoneatur semel et iterum; si non emendaverit, correptioni subiaceat. 

\caput{XXXIV}{Si omnes æqualiter debeant necessaria accipere}
\diesiii{12. Mart.}{12. Iul.}{11. Nov.}
\Y{S}{icut} scriptum est: Dividebatur singulis prout cuique opus erat. Ubi non dicimus ut personarum – quod absit – acceptio sit, sed infirmitatum consideratio; ubi qui minus indiget, agat Deo gratias et non contristetur, qui vero plus indiget, humilietur pro infirmitate, non extollatur pro misericordia; et ita omnia membra erunt in pace. Ante omnia, ne murmurationis malum pro qualicumque causa in aliquo qualicumque verbo vel significatione appareat. Quod si deprehensus fuerit, districtiori disciplinæ subdatur. 


\end{multicols}

\ornamentvi

\end{document}
